Рівняння 
\begin{equation}
    \label{eq:2.1}
    \Sum_{i, j = 1}^n a_{i j}(x) \cdot u_{x_i x_j} + \Phi(x, u, \nabla u) = 0
\end{equation}
у кожній фіксованій точці $x_0$ можна звести до канонічного вигляду неособливим лінійним перетворенням $\xi = B^T x$, де $B$ -- така матриця, що перетворення $y = B \eta$ приводить квадратичну форму
\begin{equation}
    \label{eq:2.2}
    \Sum_{i, j = 1}^n a_{i j}(x_0) \cdot y_i \cdot y_j
\end{equation}
до канонічного вигляду. (Нагадаємо, що довільну квадратичну форму можна звести до канонічного вигляду, наприклад методом виділення повних квадратів.)
