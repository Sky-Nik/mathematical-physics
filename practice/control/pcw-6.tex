\documentclass[a5paper, 12pt]{article}
\usepackage[utf8]{inputenc}
\usepackage[T1, T2A]{fontenc}
\usepackage[english, ukrainian]{babel}
\usepackage[margin=.25in]{geometry}
\usepackage{amsmath, amssymb}

\newenvironment{system*}{%
  \begin{equation*}%
    \left\{%
      \begin{aligned}%
}{%
      \end{aligned}%
    \right.%
  \end{equation*}%
}

\newcommand{\RR}{\mathbb{R}}
\newcommand{\Int}{\displaystyle\int\limits}

\setlength{\parindent}{0in}

\newcommand*\diff{\mathop{}\!\mathrm{d}}

\title{{\Huge МАТЕМАТИЧНА ФІЗИКА}}
\author{Скибицький Нікіта}
\date{\today}

\pagestyle{empty}

\begin{document}
%
\section*{Потенціали}%
%
\begin{enumerate}
	\item Діріхле (І роду) --- заряд іншого знаку, нульовий заряд.
	\item Неймана (ІІ роду) --- заряд того ж знаку, нульовий потік.
\end{enumerate}%
%
\begin{enumerate}
	\item Границя є площиною --- заряд такої ж абсолютної величини.
	\item Границя є сферою --- величина заряду в $R/r_0$ разів більша.
\end{enumerate}%
%
У сферичні координати функція Гріна переписується через $\gamma = \angle P_0 O P$.%
%
\section*{Теплопровідність ($\RR^1$)}%
%
\begin{system*}
	& a^2 \cdot u_{xx} - u_t = - f(x, t), \quad t > 0, \quad x \in \RR^1, \\
	& u(x, 0) = u_0(x).
\end{system*}%
%
\begin{equation*}
	u(x, t) = \Int_0^t \Int_{-\infty}^\infty f(\xi, \tau) \varepsilon(x - \xi, t - \tau) \diff \xi \diff \tau + \Int_{-\infty}^\infty \varepsilon(x - \xi, t) u_0(\xi) \diff \xi.
\end{equation*}%
%
\begin{equation*}
	\varepsilon(x,t)=\frac{1}{2a\sqrt{\pi t}} \cdot \exp\left\{ - \frac{|x|^2}{4a^2t} \right\}, \quad x \in \RR^1
\end{equation*}%
%
\section*{Струна}%
%
\begin{enumerate}
	\item Кінець закріплений --- продовжуємо непарним чином.
	\item Кінець вільний --- продовжуємо парним чином.
\end{enumerate}%
%
\begin{system*}
	& a^2 u_{xx} - u_{tt} = - f(x, t), \quad t > 0, \quad x \in \RR^1, \\
	& u(x, 0) = u_0(x), \\
	& u_t(x, 0) = v_0(x).
\end{system*}%
%
\begin{equation*}
	\begin{aligned}
		u(x, t) &= \frac{u_0(x - at) + u_0(x + at)}{2} + \frac{1}{2a} \Int_{x - at}^{x + at} v_0(\xi) \diff \xi + \\
		&\quad + \frac{1}{2a} \Int_0^t \Int_{x - a(t - \tau)}^{x + a(t - \tau)} f(\xi, \tau) \diff \xi \diff \tau.
	\end{aligned}
\end{equation*}%
%
\end{document}
