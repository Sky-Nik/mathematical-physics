\section{Домашнє завдання за 9/5}

\begin{problem}
    За допомогою методу послідовних наближень знайти розв'язки наступних інтегральних рівнянь:
    \begin{enumerate}
        \item $\phi(x) = 1 + \dfrac{1}{20} \Int_0^2 (1 + x) t^2 \phi(t) dt$, $\phi_0(x) = 1$.
        \item $\phi(x) = \sin x + \Int_{-1}^1 (\cos x\sin t + \cos 2x \sin 2t) \phi(t) dt$, $\phi_0(x) = \sin x$.
        \item $\phi(x) = x + 1 - \Int_0^x (x - t) \phi(t) dt$, $\phi_0(x) = 1$.
    \end{enumerate}
\end{problem}

\begin{solution}
    \begin{enumerate}
        \item Рівняння Фредгольма, тому перевіримо умови збіжності методу послідовних наближень: 
        \begin{equation*}
            \begin{split}
                \lambda = \dfrac{1}{20} = 0.05 < \dfrac{1}{B} &= \left(\Int_0^2 \Int_0^2 (1 + x)^2 t^4 dt dx\right)^{-1/2} = \left(\left.\dfrac{t^5}{5}\right|_0^2 \cdot \Int_0^2 (1 + x)^2 dx\right)^{-1/2} = \\ 
                &= \left(\dfrac{32}{5} \cdot \Int_0^2 (1 + x)^2 dx\right)^{-1/2} = \left(\dfrac{32}{5} \cdot \Int_1^3 x^2 dx\right)^{-1/2} = \left(\dfrac{32}{5} \cdot \left.\dfrac{x^3}{3}\right|_1^3\right)^{-1/2} = \\
                &= \left(\dfrac{32}{5} \cdot \dfrac{26}{3}\right)^{-1/2} \approx 0.134271.
            \end{split}
        \end{equation*}
        Далі,
        \begin{equation*}
            \begin{split}
            \phi_1(x) &= \dfrac{1}{20} \Int_0^2 (1 + x) t^2 dt + 1 = \dfrac{1 + x}{20} \Int_0^2 t^2 dt + 1 = \dfrac{1 + x}{20} \cdot \left.\dfrac{t^3}{3}\right|_0^2 + 1 = 1 + \dfrac{2}{15}(x + 1). \\
            \phi_2(x) &= \dfrac{1}{20} \Int_0^2 (1 + x) t^2 \left(1 + \dfrac{2}{15}(t + 1)\right) dt + 1 = 1 + \dfrac{2}{15}(x + 1) + \dfrac{1}{20} \Int_0^2 (1 + x) t^2 \dfrac{2}{15}(t + 1) dt = \\ 
            &= 1 + \dfrac{2}{15}(x + 1) + \dfrac{2}{45}(x + 1).
            \end{split}
        \end{equation*}
        Далі вже зрозуміло, що 
        \[
            \phi(x) = 1 + \dfrac{2(x + 1)}{15} + \dfrac{2(x + 1)}{3\cdot 15} + \dfrac{2(x + 1)}{3^2\cdot 15} + \ldots = 1 + \dfrac{3}{2} \cdot \dfrac{2(x+1)}{15} = \dfrac{6+x}{5}.
        \]
        Справді, у цьому можна пересвідчитися застосувавши метод математичної індукції.
        \item Рівняння Фредгольма, тому перевіримо умови збіжності методу послідовних наближень (для обчислення інтегралу нижче було використано систему комп'ютерної алгебри Maple): 
        \[
            \begin{split}
                \lambda = 1 > \dfrac{1}{B} &= \left(\Int_{-1}^1 \Int_{-1}^1 (\cos x \sin t + \cos 2x \sin 2t)^2 dt dx\right)^{-1/2} \approx 0.561725.
            \end{split}
        \]
        Як бачимо, умови збіжності не виконуються, тому це рівняння не можна розв'язувати методом послідовних наближень.
        \item Рівняння Вольтерри, тому немає потреби перевіряти умови збіжності, просто шукаємо 
        \begin{equation*}
            \begin{split}
                \phi_1(x) &= -\Int_0^x (x-t) dt + x + 1 = 1 + x - \dfrac{x^2}{2}. \\
                \phi_2(x) &= -\Int_0^x (x-t) \left(1 + t - \dfrac{t^2}{2}\right) + x + 1 = 1 + x - \dfrac{x^2}{2} -\Int_0^x (x-t) \left(t - \dfrac{t^2}{2}\right) = \\
                &= 1 + x - \dfrac{x^2}{2} - \dfrac{x^3}{6} + \dfrac{x^4}{24}
            \end{split}
        \end{equation*}
        Нескладно зрозуміти, що $\phi_n(x)$ -- перші $2n+1$ додаток із суми $\Sum_{n=0}^\infty (-1)^{[n/2]} \dfrac{x^n}{n!}$, звідки $\phi(x) = e^{ix}$.
    \end{enumerate}
\end{solution}

\begin{problem}
    Показати, що метод послідовних наближень збігається для довільної $f \in L_2(a, b)$ при $|\lambda| < \dfrac1B$ і $K(x, y) \in L_2((a, b) \times (a, b))$, де стала $B$: 
    \[
        \Int_a^b \Int_a^b |K(x, t)|^2 dx dt = B^2 < \infty.
    \]
\end{problem}

\begin{solution}
    Простір $L_2(a, b)$ повний з метрикою $\rho(f, g) = \sqrt{\Int_a^b (f(x) - g(x))^2 dx}$, тому для збіжності послідовності $\{\phi_n\}_{n = 0}^\infty$  достатньо показати її фундаментальність.\\
    
    Зробимо ми це показавши, що $\rho(\phi_{n + 1} - \phi_n) \le q \rho(\phi_n - \phi_{n - 1})$, де $q < 1$. Для зручності позначимо $\psi_n(x) = \phi_{n + 1}(x) - \phi_n(x)$, тоді маємо
    \begin{equation*}
        \begin{split}
            \psi_{n + 1}(x) = \phi_{n + 1}(x) - \phi_n(x) &= \left(\lambda \Int_a^b K(x, t) \phi_n (t) dt + f(x)\right) - \left(\lambda \Int_a^b K(x, t) \phi_{n - 1} (t) dt + f(x)\right) = \\ 
            &= \lambda \Int_a^b K(x, t) (\phi_n(t) - \phi_{n - 1}(t)) dt = \lambda \Int_a^b K(x, t) \psi_n(t) dt.
        \end{split}
    \end{equation*}
    
    Взявши норми, отримаємо 
    \[
        \|\psi_{n+1}\| = |\lambda| \cdot \left\|\Int_a^b K(x, t) \psi_n(t) dt \right\| \le |\lambda| \cdot \left\|\Int_a^b K(x, t) dt\right\| \cdot \|\psi_n\| = |\lambda| \cdot \|\textbf{K}\| \cdot \|\psi_n\|. 
    \] 
    
    Таким чином, залишилося показати, що $\|\textbf{K}\| \le B$. Справді, 
    \begin{equation*}
        \begin{split}
            \|\textbf{K}\| = \sqrt{\Int_a^b \left(\Int_a^b K(x, t) dt\right)^2 dx} &\lor \sqrt{\Int_a^b \Int_a^b |K(x, t)|^2 dt dx} = B \\
            \Int_a^b \left(\Int_a^b K(x, t) dt\right)^2 dx &\lor \Int_a^b \Int_a^b |K(x, t)|^2 dt dx \\
            \left(\Int_a^b K(x, t) dt\right)^2 &\le \Int_a^b |K(x, t)|^2 dt,
        \end{split}    
    \end{equation*}
    а це вже нерівність Коші-Буняковського (для $f = f$ і $g = \textbf{1}$).
\end{solution}

\begin{problem}[Владимиров, 5.11]
    Знайти резольвенту інтегрального рівняння Вольтерри з ядром:
    \begin{enumerate}
        \item $K(x, y) = 1$;
        \item $K(x, y) = x - y$;
    \end{enumerate}
\end{problem}

\begin{solution}
    \begin{enumerate}
        \item Будемо шукати повторні ядра:
        \begin{equation*}
            \begin{split}
                K_1(x, y) &= 1 \\
                K_2(x, y) &= \Int_y^x K(x, z) \cdot K(z, y) dz = \Int_y^x dz = x - y \\
                K_3(x, y) &= \Int_y^x K(x, z) \cdot K_2(z, y) dz = \Int_y^x (z - y) dz = \dfrac{(x - y)^2}{2} 
            \end{split}
        \end{equation*}
        Далі вже зрозуміло, що $K_n(x, y) = \dfrac{(x - y)^n}{n!}$, а $R(x, y) = \Sum_{n=0}^\infty \lambda^n (x-y)^n = e^{\lambda(x - y)}$.\\
        
        Справді, у цьому можна пересвідчитися застосувавши метод математичної індукції.
        \item Будемо шукати повторні ядра:
        \begin{equation*}
            \begin{split}
                K_1(x, y) &= x - y \\
                K_2(x, y) &= \Int_y^x K(x, z) \cdot K(z, y) dz = \Int_y^x (x - z) \cdot (z - y) dz = \Int_y^x (xz - xy - z^2 + zy) dz = \\
                &= x\left.\dfrac{z^2}{2}\right|_y^x - xy\, z\bigg|_y^x - \left.\dfrac{z^3}{3}\right|_y^x + y\left.\dfrac{z^2}{2}\right|_y^x = \dfrac{(x-y)^3}{3!} \\
                K_3(x, y) &= \Int_y^x K(x, z) \cdot K(z, y) dz = \Int_y^x (x - z) \cdot \dfrac{(z - y)^3}{3!} dz = \cdots = \dfrac{(x-y)^5}{5!}
            \end{split}
        \end{equation*}
        Далі вже зрозуміло, що $K_n(x, y) = \dfrac{(x - y)^{2n-1}}{(2n-1)!}$, а 
        \[
            R(x, y) = \Sum_{n=1}^\infty \lambda^{n-1}\dfrac{(x - y)^{2n-1}}{(2n-1)!} = \dfrac{1}{\sqrt{\lambda}} \Sum_{n=1}^\infty \dfrac{\left(\sqrt{\lambda}(x - y)\right)^{2n-1}}{(2n-1)!} = \dfrac{1}{\sqrt{\lambda}}\sinh\left(\sqrt{\lambda}(x - y)\right).
        \]
        
        Справді, у цьому можна пересвідчитися застосувавши метод математичної індукції.
    \end{enumerate}
\end{solution}