% cls && pdflatex pract-1-9.tex && cls && pdflatex pract-1-9.tex && start pract-1-9.pdf
\documentclass[a4paper, 12pt]{article}
\usepackage[T2A,T1]{fontenc}
\usepackage[utf8]{inputenc}
\usepackage[english, ukrainian]{babel}
\usepackage{amsmath, amssymb}
\usepackage[top = 2 cm, left = 1 cm, right = 1 cm, bottom = 2 cm]{geometry} 

\allowdisplaybreaks
\setlength\parindent{0pt}

\usepackage{xcolor}
\usepackage{hyperref}
\hypersetup{unicode=true,colorlinks=true,linktoc=all,linkcolor=violet}

\usepackage{amsthm}
\theoremstyle{definition}
\newtheorem*{example*}{Приклад}
\newtheorem*{solution*}{Розв'язок}

\newcommand{\RR}{\mathbb{R}}
\newcommand{\CC}{\mathbb{C}}
\newcommand{\Max}{\displaystyle\max\limits}
\newcommand{\Sup}{\displaystyle\sup\limits}
\newcommand{\Sum}{\displaystyle\sum\limits}
\newcommand{\Int}{\displaystyle\int\limits}
\newcommand{\Iint}{\displaystyle\iint\limits}
\newcommand{\Lim}{\displaystyle\lim\limits}
\DeclareMathOperator{\const}{const}

\newcommand*\diff{\mathop{}\!\mathrm{d}}

\renewcommand{\bf}[1]{\textbf{#1}}
\renewcommand{\epsilon}{\varepsilon}
\renewcommand{\phi}{\varphi}

\newenvironment{system}{\begin{equation} \left\{\begin{aligned}}{\end{aligned} \right. \end{equation}}
\newenvironment{system*}{\begin{equation*} \left\{\begin{aligned}}{\end{aligned} \right. \end{equation*}}

\usepackage{fancyhdr}
\pagestyle{fancy}
\lhead{Рівняння математичної фізики}
\rhead{Практичні заняття}
\cfoot{\thepage}

\begin{document}

У ваших руках конспект практичних занять що містить приклади розв'язування типових практичних задач з нормативного курсу ``Рівняння математичної фізики'', що читається на третьому курсі спеціальності прикладна математика факультету комп'ютерних наук та кібернетики Київського національного університету імені Тараса Шевченка. \\

Приклади розв'язань у компактній формі відображають матеріал курсу, допомагають сформувати загальне уявлення про предмет вивчення, правильно зорієнтуватися в даній галузі знань. Конспект практичних занять з названої дисципліни сприятиме більш успішному вивченню дисципліни, причому більшою мірою для студентів заочної форми, екстернату, дистанційного та індивідуального навчання. \\

Комп'ютерний набір та верстка -- Скибицький Нікіта Максимович.

\tableofcontents \newpage

\section{Розв'язати інтегральне рівняння методом послідовних наближень}
\begin{example*}
	Методом послідовних наближень знайти розв’язок інтегрального рівняння \[\phi(x) = x + \lambda \Int_0^1 (xt)^2 \phi(t) \diff t.\]
\end{example*}

\begin{solution*}
	$M = 1$, $V = 1$. \\

	Побудуємо повторні ядра \[ K_{(1)}(x, t) = x^2t^2, \quad K_2(x, t) = \Int_0^1 x^2 z^4 t^2 \diff z = \dfrac{x^2t^2}{5}, \quad K_{(p)}(x, t) = \dfrac{1}{5^{p - 2}} \Int_0^1 x^2 z^4 t^2 \diff z = \dfrac{x^2t^2}{5^{p - 1}}. \]
	
	Резольвента має вигляд \[\mathcal{R}(x, t, \lambda) = x^2 t^2 \left(1 + \dfrac{\lambda}{5} + \dfrac{\lambda^2}{5^2} + \ldots + \dfrac{\lambda^p}{5^p} + \ldots \right) = \dfrac{5x^2t^2}{5 - \lambda}, \quad |\lambda| < 5. \]

	Розв’язок інтегрального рівняння має вигляд: \[ \phi(x) + x + \Int_0^1 \dfrac{5x^2t^3}{5 - \lambda} \diff t = x + \dfrac{5x^2}{4(5 - \lambda)}. \]
\end{solution*}

\newpage

\section{Розв'язати інтегральне рівняння зведенням до СЛАР}

\begin{example*}
	Знайти розв’язок інтегрального рівняння \[ \phi(x) = \lambda \Int_0^\pi \sin(x - y) \phi(y) \diff y + \cos(x). \]
\end{example*}

\begin{solution*}
	\[ \phi(x) = \lambda \sin(x) \Int_0^\pi \cos(y) \phi(y) \diff y - \lambda \cos(x) \Int_0^\pi \sin(y) \phi(y) \diff y + \cos(x). \]

	Позначимо \[ c_1 = \Int_0^\pi \cos(y) \phi(y) \diff y, \quad c_2 = \Int_0^\pi \sin(y) \phi(y) \diff y. \]

	\[ \phi(x) = \lambda (c_1 \sin(x) - c_2 \cos(x)) + \cos(x). \]
	
	Підставляючи останню рівність в попередні отримаємо систем рівнянь:
	\begin{system*}
		c_1 &= \Int_0^\pi \cos(y) (\lambda c_1 \sin(y) - \lambda c_2 \cos(y) + \cos(y)) \diff y, \\
		c_2 &= \Int_0^\pi \sin(y) (\lambda c_1 \sin(y) - \lambda c_2 \cos(y) + \cos(y)) \diff y.
	\end{system*}
	
	Після обчислення інтегралів:
	\begin{system*}
		c_1 + \dfrac{\lambda \pi}{2} c_2 &= \dfrac{\pi}{2}, \\
		- \dfrac{\lambda\pi}{2} c_1 + c_2 &= 0.
	\end{system*}

	Визначник цієї системи
	\[ D(\lambda) = \begin{vmatrix} 1 & \dfrac{\lambda\pi}{2} \\ -\dfrac{\lambda\pi}{2} & 1 \end{vmatrix} = 1 + \left( \dfrac{\lambda \pi}{2} \right)^2 \ne 0. \]
	
	За правилом Крамера маємо
	\[ c_1 = \dfrac{2 \pi}{4 + (\lambda \pi)^2}, \quad c_2 = \dfrac{\lambda \pi^2}{4 + (\lambda \pi)^2}. \]
	
	Таким чином розв’язок має вигляд
	\[ \phi(x) = \dfrac{2 \lambda \pi \sin(x) + 4 \cos (x)}{4 + (\lambda \pi)^2}. \]
\end{solution*}

\newpage

\section{Знайти характеристичні числа та власні функції інтегрального оператору}

\begin{example*}
	Знайти характеристичні числа та власні функції інтегрального оператора \[ \phi(x0 = \lambda \Int_0^1 \left( \left( \dfrac{x}{t} \right)^{2/5} + \left( \dfrac{t}{x} \right)^{2/5} \right) \phi(t) \diff t. \]
\end{example*}

\begin{solution*}
	\[ \phi(x) = \lambda x^{2 / 5} \Int_0^1 t^{-2/5} \phi(t) \diff t + \lambda x^{-2/5} \Int_0^1 t^{2/5} \phi(t) \diff t. \]

	Позначимо \[ c_1 = \Int_0^1 t^{-2/5} \phi(t) \diff t, \quad c_2 = \Int_0^1 t^{2/5} \phi(t) \diff t. \]

	\[ \phi(x) = \lambda c_1 x^{2/5} + \lambda c_2 x^{-2/5}. \]
	
	\begin{system*}
		c_1 &= \Int_0^1 t^{-2/5} (\lambda c_1 t^{2/5} + \lambda c_2 t^{-2/5}) \diff t, \\
		c_2 &= \Int_0^1 t^{2/5} (\lambda c_1 t^{2/5} + \lambda c_2 t^{-2/5}) \diff t.
	\end{system*}
	
	\begin{system*}
		(1 - \lambda) c_1 - 5 \lambda c_2 &= 0, \\
		-\dfrac{5\lambda}{9} c_1 + (1 - \lambda) c_2 &= 0.
	\end{system*}
	
	\[ D(\lambda) = \begin{vmatrix} 1 - \lambda & - 5 \lambda \\ - \dfrac{5\lambda}{9} & 1 - \lambda \end{vmatrix} = (1 - \lambda)^2 - \dfrac{25\lambda^2}{9} = 0. \]
	
	\[ \lambda_1 = \dfrac{3}{8}, \quad \lambda_2 = - \dfrac{3}{2}. \]
	
	З системи однорідних рівнянь при $\lambda = \lambda_1 = \dfrac{3}{8}$ маємо $c_1 = 3 c_2$. Тоді маємо власну функцію $\phi_1(x) = 3 x^{2 / 5} + x^{-2 / 5}$. \\

	При $\lambda = \lambda_2 = - \dfrac{3}{2}$ маємо $c_1 = - 3 c_2$. Маємо другу власну функцію $\phi_2(x) = - 3 x^{2 / 5} + x^{-2 / 5}$.
\end{solution*}

\newpage

\section{Розв'язати інтегральне рівняння при всіх значеннях параметрів}

\begin{example*}
	Знайти розв’язок інтегрального рівняння при всіх значеннях параметрів $\lambda$, $a$, $b$, $c$: \[\phi(x) = \lambda \Int_{-1}^1 (\sqrt[3]{x} + \sqrt[3]{y}) \phi(y) \diff y + ax^2 + bx + c. \]
\end{example*}

\begin{solution*}
	Запишемо рівняння у вигляді: \[\phi(x) = \lambda \sqrt[3]{x} \Int_{-1}^1 \phi(y) \diff y + \lambda \Int_{-1}^1 \sqrt[3]{y} \phi(y) \diff y + ax^2 + bx + c. \]

	Введемо позначення: \[ c_1 = \Int_{-1}^1 \phi(y) \diff y, \quad c_2 = \Int_{-1}^1 \sqrt[3]{y} \phi(y) \diff y, \]
	та запишемо розв’язок у вигляді: $\phi(x) = \lambda \sqrt[3]{x} c_1 + \lambda c_2 + ax^2 + bx + c$. \\

	Для визначення констант отримаємо СЛАР:
	\begin{system*}
		c_1 - 2 \lambda c_2 &= \dfrac{2a}{3} + 2 c, \\
		- \dfrac{6\lambda}{5} c_1 + c_2 &= \dfrac{6b}{7}.
	\end{system*}

	Визначник системи дорівнює \[\begin{vmatrix} 1 & - 2 \lambda \\ - \dfrac{6\lambda}{5} & 1 \end{vmatrix} = 1 - \dfrac{12\lambda^2}{5}.\]

	Характеристичні числа ядра $\lambda_1 = \dfrac{1}{2} \sqrt{\dfrac{5}{3}}$, $\lambda_2 = - \dfrac{1}{2} \sqrt{\dfrac{5}{3}}$. \\

	Нехай $\lambda \ne \lambda_1$, $\lambda \ne \lambda_2$. Тоді розв'язок існує та єдиний для будь-якого вільного члена і має вигляд \[ \phi(x) = \dfrac{5 \lambda (14 a + 30 \lambda b + 42 c)}{21 (5 - 12 \lambda^2)} \sqrt[3]{x} + \dfrac{28 \lambda a + 84 \lambda c + 30 b}{7(5 - 12 \lambda^2)} + ax^2 + bx + c. \]

	Нехай $\lambda = \lambda_1 = \dfrac{1}{2} \sqrt{\dfrac{5}{3}}$. Тоді система рівнянь має вигляд:
	\begin{system*}
		c_1 - \sqrt{\dfrac{5}{3}} c_2 &= \dfrac{2a}{3} + 2 c, \\
		c_1 - \sqrt{\dfrac{5}{3}} c_2 &= - \sqrt{\dfrac{5}{3}} \dfrac{6b}{7}.
	\end{system*}

	Ранги розширеної і основної матриці співпадатимуть якщо має місце рівність $\dfrac{2a}{3} + 2c = - \sqrt{\dfrac{5}{3}} \dfrac{6}{7} b$, $(*)$. \\

	При виконанні цієї умови розв'язок існує $c_2 = c_2$, $c_1 = \sqrt{\dfrac{5}{3}} c_2 + \dfrac{2a}{3} + 2c$. \\

	Таким чином розв'язок можна записати \[ \phi(x) = \dfrac{1}{2} \sqrt{\dfrac{5}{3}} \sqrt[3]{x} \left( \sqrt{\dfrac{5}{3}} c_2 + \dfrac{2a}{3} + 2c \right) + \dfrac{1}{2} \sqrt{\dfrac{5}{3}} c_2 + ax^2 + bx + x. \]

	Якщо $\lambda = \lambda_1 = \dfrac{1}{2} \sqrt{\dfrac{5}{3}}$, а умова $(*)$ не виконується, то розв'язків не існує. \\

	Нехай $\lambda = \lambda_2 = - \dfrac{1}{2} \sqrt{\dfrac{5}{3}}$. Після підстановки цього значення отримаємо СЛАР
	\begin{system*}
		c_1 + \sqrt{\dfrac{5}{3}} c_2 &= \dfrac{2a}{3} + 2 c, \\
		c_1 + \sqrt{\dfrac{5}{3}} c_2 &= \sqrt{\dfrac{5}{3}} \dfrac{6b}{7}.
	\end{system*}

	Остання система має розв'язок при умові, $\dfrac{2a}{3} + 2c = \sqrt{\dfrac{5}{3}} \dfrac{6}{7} b$, $(**)$. \\

	При виконанні умови $(**)$, розв’язок існує $c_2 = c_2$, $c_1 = - \sqrt{\dfrac{5}{3}} c_2 + \dfrac{2a}{3} + 2c$. \\

	Розв'язок інтегрального рівняння можна записати: \[ \phi(x) = \dfrac{1}{2} \sqrt{\dfrac{5}{3}} \sqrt[3]{x} \left( -\sqrt{\dfrac{5}{3}} c_2 + \dfrac{2a}{3} + 2c \right) + \dfrac{1}{2} \sqrt{\dfrac{5}{3}} c_2 + ax^2 + bx + c. \]
\end{solution*}

\newpage

\section{Знайти значення параметрів для яких інтегральне рівняння має розв'язок для будь-якого значення $\lambda$}

\begin{example*}
	Знайти ті значення параметрів $a$, $b$ для яких інтегральне рівняння \[\phi(x) = \lambda \Int_{-1}^1 \left( xy - \dfrac{1}{3} \right) \phi(y) \diff y + ax^2 - bx + 1 \] має розв'язок для будь-якого значення $\lambda$.
\end{example*}

\begin{solution*}
	Знайдемо характеристичні числа та власні функції спряженого однорідного рівняння (ядро ермітове). \[ \phi(x) = \lambda x \Int_{-1}^1 y \phi(y) \diff y - \dfrac{\lambda}{3} \Int_{-1}^1 \phi(y) \diff y = \lambda x c_1 - \dfrac{\lambda}{3} c_2. \]
	
	\begin{system*}
		c_1 &= \Int_{-1}^1 y \phi(y) \diff y = \Int_{-1}^1 y \left(\lambda y c_1 - \dfrac{\lambda}{3} c_2 \right) \diff y = \dfrac{2 \lambda}{3} c_1, \\
		c_2 &= \Int_{-1}^1 \phi(y) \diff y = \Int_{-1}^1 \left(\lambda y c_1 - \dfrac{\lambda}{3} c_2 \right) \diff y = - \dfrac{2 \lambda}{3} c_2.
	\end{system*}

	\[ D(\lambda) = \begin{vmatrix} 1 - \dfrac{2\lambda}{3} & 0 \\ 0 & 1 + \dfrac{2\lambda}{3} \end{vmatrix} = 0.\]

	\[ \lambda_1 = \dfrac{3}{2}, \quad \lambda_2 = - \dfrac{3}{2}.\]
	
	\[ \phi_1(x) = x, \quad \phi_2(x) = 1.\]
	
	Умови ортогональності:
	\begin{system*}
		\Int_{-1}^1 (ax^2 - bx + 1) x \diff x = - \dfrac{2b}{3} &= 0, \\
		\Int_{-1}^1 (ax^2 - bx + 1) \diff x = \dfrac{2a}{3} + 2 &= 0.
	\end{system*}
	\[ a = -3, \quad b = 0. \]
\end{solution*}

\newpage

\section{Розв’язати інтегральне рівняння за формулою Шмідта}

\begin{example*}
    Знайти розв’язок інтегрального рівняння \[ \phi(x) = \lambda \Int_0^1 K(x, y) \phi(y) \diff y + x,\] де \[ K(x,y)=\begin{cases}x(y-1), & 0 \le x \le y \le 1 \\ y(x-1), & 0\le y\le x\le1\end{cases}. \]
\end{example*}

\begin{solution*}
    Розв’язок будемо шукати за формулою Шмідта. Знайдемо характеристичні числа та власні функції ермітового ядра. Запишемо однорідне рівняння \[ \phi(x) = \lambda \Int_0^x y(x-1)\phi(y)\diff y + \lambda\Int_x^1x(y-1)\phi(y)\diff y.\]
    
    Продиференціюємо рівняння: \[ \phi'(x) = \lambda \Int_0^x y\phi(y)\diff y + \lambda x(x-1)\phi(x) + \lambda\Int_x^1(y-1)\phi(y)\diff y - \lambda x(x-1)\phi(x).\]
    
    Обчислимо другу похідну: \[ \phi''(x) = \lambda x\phi(x)-\lambda(x-1)\phi(x).\]

    Або після спрощення $\phi'' = \lambda \phi$. Доповнимо диференціальне рівняння другого порядку граничними умовами вигляду (\ref{eq:5.2}), (\ref{eq:5.3}). Легко бачити що \[ \phi(0) = \lambda \Int_0^0 y(0-1)\phi(y)\diff y + \lambda\Int_0^10(y-1)\phi(y)\diff y=0. \]

    Аналогічно \[ \phi(1) = \lambda\Int_0^1y(1-1)\phi(y)\diff y + \lambda\Int_1^11(y-1)\phi(y)\diff y=0.\]

    Таким чином отримаємо задачу Штурма-Ліувілля:
    \begin{system*}
        & \phi'' = \lambda \phi, \quad 0 < x < 1, \\
        & \phi(0) = \phi(1) = 0.
    \end{system*}

    Для знаходження власних чисел і власних функцій розглянемо можливі значення параметру $\lambda$:
    \begin{enumerate}
        \item $\lambda > 0$, $\phi(x)=c_1\sinh(\sqrt{\lambda}x)+c_2\cosh(\sqrt{\lambda}x)$. \\

        Враховуючи граничні умови, маємо систему рівнянь 
        \begin{system*}
            & c_1 \cdot 0 + c_2 = 0, \\
            & c_1 \sinh(\sqrt{\lambda}) + c_2 \cosh(\sqrt{\lambda}) = 0.
        \end{system*}

        Визначник цієї системи повинен дорівнювати нулю. \[ D(\lambda) = \begin{vmatrix} 0 & 1 \\ \sinh(\sqrt{\lambda}) & \cosh(\sqrt{\lambda}) \end{vmatrix} = - \sinh(\sqrt{\lambda}) = 0. \]

        Єдиним розв’язком цього рівняння є $\lambda = 0$, яке не задовольняє, бо $\lambda > 0$. Це означає, що система рівнянь має тривіальний розв’язок і будь-яке $\lambda > 0$ не є власним числом.

        \item $\lambda = 0$, $\phi(x) = c_1 x + c_2$. З граничних умов маємо, що $c_1 = c_2 = 0$. Тобто $\lambda=0$ не є власним числом.

        \item $\lambda < 0$, $\phi(x) = c_1\sin(\sqrt{-\lambda}x)+c_2\cos(\sqrt{-\lambda}x)$. \\

        Враховуючи граничні умови, маємо систему рівнянь
        \begin{system*}
            & c_1 \cdot 0 + c_2 = 0, \\
            & c_1 \sin(\sqrt{-\lambda}) + c_2 \cos(\sqrt{-\lambda}) = 0.
        \end{system*}

        Визначник цієї системи прирівняємо до нуля \[ D(\lambda) = \begin{vmatrix} 0 & 1 \\ \sin(\sqrt{-\lambda}) & \cos(\sqrt{-\lambda}) \end{vmatrix} = -\sin(\sqrt{-\lambda}) = 0. \]

        Це рівняння має зліченну множину розв’язків $\lambda_k = - (\pi k)^2$, $k = 1, 2, \ldots$. Система лінійних алгебраїчних рівнянь має розв’язок $c_2=0$, $c_1=c_1$. \\ 

        Таким чином нормовані власні функції задачі Штурма-Ліувілля мають вигляд $\phi_k(x) = \sqrt{2} \sin(k \pi x)$. \\

        Порахуємо коефіцієнти Фур’є \[f_n = (f, \phi_n) = \sqrt{2} \Int_0^1 x \sin(\pi nx) \diff x = \sqrt{2} \dfrac{(-1)^n}{\pi n}. \]

        Згідно до формули Шмідта розв’язок рівняння при $\lambda \ne \lambda_k$ має вигляд: \[ \phi(x) = x - 2 \lambda \Sum_{k=1}^\infty \dfrac{(-1)^{k+1}\sin(\pi k x)}{((\pi k)^2+\lambda)\pi k}.\]

        При $\lambda = \lambda_k$ розв’язок не існує, оскільки не виконана умова ортогональності вільного члена до власної функції.
    \end{enumerate}
\end{solution*}

\newpage

\section{Звести задачу Штурма-Ліувілля до інтегрального рівняння}

\begin{example*}
	Звести задачу Штурма-Ліувілля до інтегрального рівняння з ермітовим неперервним ядром:
	\begin{system*}
		& \bf{L}y \equiv -(1+e^x)y''-e^xy'=\lambda x^2y, \quad 0 < x < 1, \\
		& y(0) - 2y'(0) = y'(1) = 0.
	\end{system*}
\end{example*}

\begin{solution*}
	Побудуємо функцію Гріна оператора $\bf{L}$. Розглянемо задачі Коші:
	\begin{system*}
		& -(1+e^x)v_i''-e^xv_i'=0, \quad i=1,2,\\
		& v_1(0)-2v_1'(0)=v_2'(1)=0.
	\end{system*}

	Загальний розв’язок диференціального рівняння $-(1+e^x)y''-e^xy'=0$ має вигляд $c_1(x-\ln(1+e^x))+c_2$. Тоді розв’язки задач Коші:
	\[ v_1(x) = a(x - \ln(1+e^x)+1+\ln2), \quad a = const, \quad v_2(x) = b, \quad b = const. \]

	Обчислимо визначник Вронського \[\begin{vmatrix} v_1(x) & v_2(x) \\ v_1'(x) & v_2'(x) \end{vmatrix} = \dfrac{ab}{1+e^x}. \]

	Перевіримо тотожність Ліувілля $p(x)w(x) = ab = const$. Запишемо функцію Гріна за формулою:
	\[ G(x, \xi) = - \begin{cases} (x-\ln(1+e^x)+1+\ln2), & 0\le x \le \xi \le 1, \\ (\xi-\ln(1+e^\xi)+1+\ln2), & 0\le \xi \le i \le 1. \end{cases} \]

	Запишемо інтегральне рівняння \[y(x) = \lambda \Int_0^1 G(x,\xi)\xi^2y(\xi)\diff\xi. \]

	Симетризуємо ядро інтегрального рівняння, помножимо обидві частини на $x$: \[ x \cdot y(x) = \lambda \Int_0^1 x \cdot \xi \cdot G(x, \xi) \cdot \xi \cdot y(\xi) \diff \xi. \]

	Введемо позначення: \[ \omega(x) = xy(x), \quad G_1(x, \xi) = x\xi G(x,\xi).\]

	Отримаємо однорідне інтегральне рівняння Фредгольма другого роду з симетричним ядром: \[\omega(x) = \lambda \Int_0^1 G_1(x, \xi) \omega(\xi) \diff \xi.\]
\end{solution*}

\newpage

\section{Визначити тип рівняння і звести до канонічної форми, дві змінні}

\begin{example*}
	Визначити тип рівняння і привести його до канонічної форми запису $y^2 u_{xx} - x^2 u_{yy} = 0$.
\end{example*}

\begin{solution*}
	Складемо характеристичне рівняння \[ \dfrac{\diff y}{\diff x} = \dfrac{0\pm\sqrt{0+(xy)^2}}{y^2}, \quad \dfrac{\diff y}{\diff x} = \pm\dfrac{x}{y}. \]

	Оскільки обидві характеристики є дійсними, то рівняння має гіперболічний тип. Останнє рівняння можна записати у вигляді: \[y \diff y = \pm x \diff x.\]

	Загальні інтеграли цього рівняння мають вигляд: \[y^2 \pm x^2 = \const.\]

	Для отримання першої канонічної форми запису необхідно обрати нові змінні \[\xi = x^2 + y^2,\quad \eta = x^2 - y^2. \]

	Для отримання другої канонічної форми запису оберемо змінні \[ \alpha = \dfrac{\xi + \eta}{2} = x^2, \quad \beta = \dfrac{\xi-\eta}{2} = y^2, \]

	обчислимо похідні 
	\[ u_x = u_\alpha 2 x + u_\beta \cdot 0, \quad u_y = u_\alpha \cdot 0 + u_\beta 2 y, \]
	\[ u_{xx} = u_{\alpha\alpha} 4x^2 + 2 u_{\alpha\beta} \cdot 0 + u_{\beta\beta} \cdot 0 + u_\alpha 2 + u_\beta \cdot 0, \]
	\[ u_{yy} = u_{\alpha\alpha} \cdot 0 + 2 u_{\alpha\beta} \cdot 0 + u_{\beta\beta} 
	4y^2 + u_\alpha \cdot 0 + u_\beta 2. \]

	Підставимо знайдені похідні в рівняння:
	\[ y^2(4x^2u_{\alpha\alpha} + 2u_\alpha) - x^2(4y^2u_{\beta\beta} + 2u_\beta) = 0, \]
	\[ u_{\alpha\alpha} - u_{\beta\beta} + \dfrac{1}{2x^2y^2}(u_\alpha - u_\beta) = 0, \]
	або остаточно маємо:
	\[ u_{\alpha\alpha} - u_{\beta\beta} + \dfrac{1}{2\alpha\beta}(u_\alpha - u_\beta) = 0. \]
\end{solution*}

\newpage

\section{Визначити тип рівняння і звести до канонічної форми, три змінні}

\begin{example*}
	Визначити тип рівняння і привести його до канонічної форми запису $u_{xx} + 2 u_{xy} - 2 u_{xz} + 2 u_{yy} + 6 u_{zz} = 0$.
\end{example*}

\begin{solution*}
	Поставимо у відповідність головній частині рівняння квадратичну форму: \[\lambda_1^2+2\lambda_1\lambda_2-2\lambda_1\lambda_3+2\lambda_2^2+6\lambda_3^2. \]

	Методом виділення повних квадратів приведемо квадратичну форму до канонічної форми запису
	\begin{multline*}
		\lambda_1^2+2\lambda_1\lambda_2-2\lambda_1\lambda_3+2\lambda_2^2+6\lambda_3^2 = (\lambda_1^2+2\lambda_1\lambda_2-2\lambda_1\lambda_3+\lambda_2^2+\lambda_3^2 - 2 \lambda_2\lambda_3)^2 + \\ 
		+ \lambda_2^2 + 2\lambda_2\lambda_3 + 5\lambda_3^2 = (\lambda_1+\lambda_2-\lambda_3)^2 + (\lambda_2^2 + 2\lambda_2\lambda_3+\lambda_3^2) + 4\lambda_3^2 = \\
		= (\lambda_1+\lambda_2-\lambda_3)^2 + (\lambda_2 + \lambda_3)^2 + (2\lambda_3)^2.
	\end{multline*}

	Вводимо нові незалежні змінні: \[ \begin{matrix} \mu_1 = \lambda_1 + \lambda_2 - \lambda_3, \\ \mu_2 = \lambda_2 + \lambda_3, \\ \mu_3 = 2 \lambda_3 \end{matrix} \]

	Отримаємо канонічну форму запису для квадратичної форми: $\mu_1^2 + \mu_2^2 + \mu_3^2$. \\

	Оскільки при квадраті кожної змінної коефіцієнт дорівнює 1, то квадратична форма та рівняння мають еліптичний тип. \\

	Запишемо тепер заміну змінних, яка приведе рівняння до канонічної форми запису. Обчислимо матрицю оберненого лінійного перетворення:
	\[ \begin{matrix} \lambda_1 = \mu_1 - \mu_2 + \mu_3 \\ \lambda_2 = \mu_2 - \mu_3/2 \\ \lambda_3 = \mu_3 / 2 \end{matrix} \quad B = \begin{pmatrix} 1 & -1 & 1 \\ 0 & 1 & -1/2 \\ 0 & 0 & 1/2 \end{pmatrix}, \quad \lambda = B \mu. \]

	Обчислимо транспоновану матрицю \[ B^T = \begin{pmatrix} 1 & 0 & 0 \\ -1 & 1 & 0 \\ 1 & -1/2 & 1/2 \end{pmatrix}. \]

	Для диференціального рівняння маємо нові змінні: $\vec \xi = B^T \vec x$. Або в координатній формі: \[ \left\{ \begin{matrix} \xi = x \\ \eta = y - x \\ \zeta = x - y / 2 + z / 2 \end{matrix} \right. \]

	У новій системі координат головна частина диференціального рівняння буде мати канонічну форму запису еліптичного рівняння $u_{\xi\xi} + u_{\eta\eta} + u_{\zeta\zeta} = 0$.
\end{solution*}

\end{document}