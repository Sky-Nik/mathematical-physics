\documentclass[a4paper, 12pt]{article}
\usepackage[utf8]{inputenc}
\usepackage[english, ukrainian]{babel}

\usepackage{amsmath, amssymb}
\usepackage{multicol}
\usepackage{graphicx}
\usepackage{float}

\allowdisplaybreaks
\setlength\parindent{0pt}
\numberwithin{equation}{subsection}

\usepackage{hyperref}
\hypersetup{unicode=true,colorlinks=true,linktoc=all,linkcolor=red}

\numberwithin{equation}{subsection}

\renewcommand{\bf}[1]{\textbf{#1}}
\renewcommand{\it}[1]{\textit{#1}}
\newcommand{\bb}[1]{\mathbb{#1}}
\renewcommand{\cal}[1]{\mathcal{#1}}

\renewcommand{\epsilon}{\varepsilon}
\renewcommand{\phi}{\varphi}

\DeclareMathOperator{\diam}{diam}
\DeclareMathOperator{\rang}{rang}
\DeclareMathOperator{\const}{const}

\newenvironment{system}{%
  \begin{equation}%
    \left\{%
      \begin{aligned}%
}{%
      \end{aligned}%
    \right.%
  \end{equation}%
}
\newenvironment{system*}{%
  \begin{equation*}%
    \left\{%
      \begin{aligned}%
}{%
      \end{aligned}%
    \right.%
  \end{equation*}%
}

\makeatletter
\newcommand*{\relrelbarsep}{.386ex}
\newcommand*{\relrelbar}{%
  \mathrel{%
    \mathpalette\@relrelbar\relrelbarsep%
  }%
}
\newcommand*{\@relrelbar}[2]{%
  \raise#2\hbox to 0pt{$\m@th#1\relbar$\hss}%
  \lower#2\hbox{$\m@th#1\relbar$}%
}
\providecommand*{\rightrightarrowsfill@}{%
  \arrowfill@\relrelbar\relrelbar\rightrightarrows%
}
\providecommand*{\leftleftarrowsfill@}{%
  \arrowfill@\leftleftarrows\relrelbar\relrelbar%
}
\providecommand*{\xrightrightarrows}[2][]{%
  \ext@arrow 0359\rightrightarrowsfill@{#1}{#2}%
}
\providecommand*{\xleftleftarrows}[2][]{%
  \ext@arrow 3095\leftleftarrowsfill@{#1}{#2}%
}
\makeatother

\newcommand{\NN}{\mathbb{N}}
\newcommand{\ZZ}{\mathbb{Z}}
\newcommand{\QQ}{\mathbb{Q}}
\newcommand{\RR}{\mathbb{R}}
\newcommand{\CC}{\mathbb{C}}

\newcommand{\Max}{\displaystyle\max\limits}
\newcommand{\Sup}{\displaystyle\sup\limits}
\newcommand{\Sum}{\displaystyle\sum\limits}
\newcommand{\Int}{\displaystyle\int\limits}
\newcommand{\Iint}{\displaystyle\iint\limits}
\newcommand{\Lim}{\displaystyle\lim\limits}

\newcommand*\diff{\mathop{}\!\mathrm{d}}

\newcommand*\rfrac[2]{{}^{#1}\!/_{\!#2}}


\title{{\Huge МАТЕМАТИЧНА ФІЗИКА}}
\author{Скибицький Нікіта}
\date{\today}

\usepackage{amsthm}
\usepackage[dvipsnames]{xcolor}
\usepackage{thmtools}
\usepackage[framemethod=TikZ]{mdframed}

\theoremstyle{definition}
\mdfdefinestyle{mdbluebox}{%
	roundcorner = 10pt,
	linewidth=1pt,
	skipabove=12pt,
	innerbottommargin=9pt,
	skipbelow=2pt,
	nobreak=true,
	linecolor=blue,
	backgroundcolor=TealBlue!5,
}
\declaretheoremstyle[
	headfont=\sffamily\bfseries\color{MidnightBlue},
	mdframed={style=mdbluebox},
	headpunct={\\[3pt]},
	postheadspace={0pt}
]{thmbluebox}

\mdfdefinestyle{mdredbox}{%
	linewidth=0.5pt,
	skipabove=12pt,
	frametitleaboveskip=5pt,
	frametitlebelowskip=0pt,
	skipbelow=2pt,
	frametitlefont=\bfseries,
	innertopmargin=4pt,
	innerbottommargin=8pt,
	nobreak=true,
	linecolor=RawSienna,
	backgroundcolor=Salmon!5,
}
\declaretheoremstyle[
	headfont=\bfseries\color{RawSienna},
	mdframed={style=mdredbox},
	headpunct={\\[3pt]},
	postheadspace={0pt},
]{thmredbox}

\declaretheorem[style=thmbluebox,name=Теорема,numberwithin=subsubsection]{theorem}
\declaretheorem[style=thmbluebox,name=Лема,numberwithin=subsubsection]{lemma}
\declaretheorem[style=thmbluebox,name=Твердження,numberwithin=subsubsection]{proposition}
\declaretheorem[style=thmbluebox,name=Принцип,numberwithin=subsubsection]{th_principle}
\declaretheorem[style=thmbluebox,name=Закон,numberwithin=subsubsection]{law}
\declaretheorem[style=thmbluebox,name=Закон,numbered=no]{law*}
\declaretheorem[style=thmbluebox,name=Формула,numberwithin=subsubsection]{th_formula}
\declaretheorem[style=thmbluebox,name=Рівняння,numberwithin=subsubsection]{th_equation}
\declaretheorem[style=thmbluebox,name=Умова,numberwithin=subsubsection]{th_condition}
\declaretheorem[style=thmbluebox,name=Наслідок,numberwithin=subsubsection]{corollary}

\declaretheorem[style=thmredbox,name=Приклад,numberwithin=subsubsection]{example}
\declaretheorem[style=thmredbox,name=Приклади,sibling=example]{examples}

\declaretheorem[style=thmredbox,name=Властивість,numberwithin=subsubsection]{property}
\declaretheorem[style=thmredbox,name=Властивості,sibling=property]{properties}

\mdfdefinestyle{mdgreenbox}{%
	skipabove=8pt,
	linewidth=2pt,
	rightline=false,
	leftline=true,
	topline=false,
	bottomline=false,
	linecolor=ForestGreen,
	backgroundcolor=ForestGreen!5,
}
\declaretheoremstyle[
	headfont=\bfseries\sffamily\color{ForestGreen!70!black},
	bodyfont=\normalfont,
	spaceabove=2pt,
	spacebelow=1pt,
	mdframed={style=mdgreenbox},
	headpunct={ --- },
]{thmgreenbox}

\mdfdefinestyle{mdblackbox}{%
	skipabove=8pt,
	linewidth=3pt,
	rightline=false,
	leftline=true,
	topline=false,
	bottomline=false,
	linecolor=black,
	backgroundcolor=RedViolet!5!gray!5,
}
\declaretheoremstyle[
	headfont=\bfseries,
	bodyfont=\normalfont\small,
	spaceabove=0pt,
	spacebelow=0pt,
	mdframed={style=mdblackbox}
]{thmblackbox}

\declaretheorem[name=Вправа,numberwithin=subsubsection,style=thmblackbox]{exercise}
\declaretheorem[name=Зауваження,numberwithin=subsubsection,style=thmgreenbox]{remark}
\declaretheorem[name=Визначення,numberwithin=subsubsection,style=thmblackbox]{definition}

\newtheorem{problem}{Задача}[subsection]
\newtheorem{sproblem}[problem]{Задача}
\newtheorem{dproblem}[problem]{Задача}
\renewcommand{\thesproblem}{\theproblem$^{\star}$}
\renewcommand{\thedproblem}{\theproblem$^{\dagger}$}
\newcommand{\listhack}{$\empty$\vspace{-2em}} 

\theoremstyle{remark}
\newtheorem*{solution}{Розв'язок}


\begin{document}

\tableofcontents

\setcounter{section}{4}
\setcounter{subsection}{3}
\setcounter{subsubsection}{4}
\setcounter{theorem}{21}
\setcounter{equation}{55}

\subsubsection{Функція Гріна граничних задач оператора теплопровідності}

Будемо розглядати граничні задачі для рівняння теплопровідності:

\begin{system}
	& a^2 \Delta u(x, t) - \frac{\partial u(x,t)}{\partial t} = - F(x, t), \quad x \in \Omega, t > 0, \\
	& u(x, 0) = u_0(x), \\
	& \left. \ell_i u(x, t) \right|_{x \in S} = f(x, t), \quad i = 1, 2, 3.
\end{system}

Тут 
\begin{align}
	\left. \ell_1 u(x, t) \right|_{x \in S} &= \left. u(x, t) \right|_{x \in S}, \\
	\left. \ell_2 u(x, t) \right|_{x \in S} &= \left. \frac{\partial u(x, t)}{\partial n} \right|_{x \in S}, \\
	\left. \ell_3 u(x, t) \right|_{x \in S} &= \left. \frac{\partial u(x, t)}{\partial n} + \alpha(x, t) \cdot u(x, t) \right|_{x \in S}
\end{align}
--- оператори граничних умов першого, другого, або третього роду.

\begin{definition}[функції Гріна граничної задачі теплопровідності]
	Функцію $E_i (x, \xi, t - \tau)$ будемо називати \textit{функцією Гріна першої, другої або третьої граничної задачі рівняння теплопровідності} в області $\Omega$ з границею $S$ для $t > 0$, якщо вона є розв'язком настуної граничної задачі:
	\begin{system}
		& a^2 \Delta_x E_i (x, \xi, t - \tau) - \frac{\partial E_i(x, \xi, t - \tau)}{\partial t} = - \delta(x - \xi, t - \tau), \quad x, \xi \in \Omega, \quad t > 0 \\
		& \left. E_i(x, \xi, t - \tau) \right|_{t - \tau \le 0} = 0, \\
		& \left. \ell_i E_i (x, \xi, t - \tau) \right|_{x \in S} = 0, \quad i = 1, 2, 3.
	\end{system}
\end{definition}

Еквівалетнне визначення можна надати у вигляді
\begin{definition}[функції Гріна граничної задачі теплопровідності]
	Функцію $E_i (x, \xi, t - \tau)$ будемо називати \textit{функцією Гріна першої, другої або третьої граничної задачі рівняння теплопровідності} в області $\Omega$ з границею $S$ для $t > 0$, якщо вона може бути представлена у вигляді
	\begin{equation}
		E_i(x, \xi, t - \tau) = \epsilon(x - \xi, t - \tau) + \omega_i(x, \xi, t - \tau),
	\end{equation}
	де  перший доданок є фундаментальним розв'язком оператора теплопровідності, а другий  є розв'язком наступної граничної задачі
	\begin{system}
		& a^2 \Delta_x \omega_i (x, \xi, t - \tau) - \frac{\partial \omega_i(x, \xi, t - \tau)}{\partial t} = - \delta(x - \xi, t - \tau), \quad x, \xi \in \Omega, \quad t > 0 \\
		& \left. \omega_i(x, \xi, t - \tau) \right|_{t - \tau \le 0} = 0, \\
		& \left. \ell_i \omega_i (x, \xi, t - \tau) \right|_{x \in S} = -\left.\ell_i \epsilon_i(x - \xi, t - \tau)\right|_{x \in S} \quad i = 1, 2, 3.
	\end{system}
\end{definition}

Вивчимо 
\begin{properties}[функції Гріна оператора теплопровідності]
	\begin{enumerate}
		\item Легко бачити, що функція Гріна граничних задач рівняння теплопровідності з аргументами $E_i(x, \xi, -t)$ задовольняє спряженому диференціальному рівнянню
		\begin{equation}
			a^2 \Delta_x E_i(x, \xi, -t) + \frac{\partial E_i(x, \xi, - t)}{\partial t} = - \delta(x - \xi) \delta (t - \tau), \quad x, \xi \in \Omega, \quad t > 0.
		\end{equation}
		\item Функція Гріна є також симетричною функцією своїх перших двох аргументів.
	\end{enumerate}
\end{properties}
	
\begin{proof}
	Доведемо другу властивість. Запишемо співвідношення, яким задовольняє функція Гріна:
	\begin{align}
		a^2 \Delta_x E_i(x, \xi, t - \tau_1) + \frac{\partial E_i(x, \xi, t - \tau_1)}{\partial t} &= - \delta(x - \xi) \delta (t - \tau_1), \quad x, \xi \in \Omega, \\
		a^2 \Delta_x E_i(x, \eta, \tau_2 - t) + \frac{\partial E_i(x, \eta, \tau_2 - t)}{\partial t} &= - \delta(x - \eta) \delta (t - \tau_2), \quad x, \eta \in \Omega.
	\end{align}
	Перше рівняння помножимо на $E_i(x, \xi, \tau_2 - t)$, друге рівняння помножимо на $E_i(x, \xi, t - \tau_1)$, віднімемо від першого друге і проінтегруємо по $x \in \Omega$ і по $-\infty < t < \tau$:
	% \begin{equation}
		% \begin{aligned}
			a^2 \Iiint_\Omega
		\end{aligned}
	\end{equation}
\end{proof}

\end{document}