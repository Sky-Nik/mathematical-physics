\subsubsection{Нерівність Пуанкаре---Фрідріхса}

\begin{proposition}
    Розглянемо обмежену область $\Omega$ і покажемо, що для будь-якої функції $u(x) \in \overset{\circ}{W} {}_2^1(\Omega)$ виконується нерівність Пуанкаре---Фрідріхса:
    \begin{equation}
        \label{eq:5.1.18}
        \int_\Omega u^2(x) \diff x \le C_\Omega^2 \int_\Omega \sum_{i = 1}^n u_{x_i}^2(x) \diff x.
    \end{equation}
\end{proposition}

\begin{proof}
    Покажемо справедливість цієї нерівності для будь-якої функції $u(x)$ з $C_0^\infty(\Omega)$. Будемо вважати, що область $\Omega$ можна заключити у паралелепіпед $\Pi = \{x: 0 \le x_i \le \ell_i, i = \overline{1..n}\}$. За межами області $\Omega$ продовжимо функцію $u(x)$ нульовим значенням, тобто $u(x) \equiv 0, x \in \Omega'$. Припустимо, що серед усіх сторін паралелепіпеда сторона $\ell_1$ є найменшою. Позначимо $x_1' = (x_2, \ldots, x_n) \subset \Pi_1 = \{x_1', 0 \le x_i \le \ell_i, i = \overline{2..n}\}$, а $x_1$ --- перша координата точки $x$. \medskip

    Запишемо очевидну рівність
    \begin{equation*}
        u(x_1, x_1') = \int_0^{x_1} \frac{\partial u(y_1, x_1')}{\partial y_1} \diff y_1.
    \end{equation*}

    Піднесемо обидві частини рівності до квадрату, проінтегруємо по паралелепіпеду $\Pi$ і оцінимо праву частину використовуючи нерівність Коші---Буняковського:

    \begin{align*}
        \int_\Pi u^2(x) \diff x
        &= \int_0^{\ell_1} \diff x_1 \int_{\Pi_1} \left( \int_0^{x_1} \frac{\partial u(y_1, x_1')}{\partial y_1} \diff y_1 \right)^2 \diff x_1' \le \\
        &\le \int_0^{\ell_1} \diff x_1 \int_{\Pi_1} \left(  x_1 \int_0^{\ell_1} \left( \frac{\partial u(y_1, x_1')}{\partial y_1} \diff y_1 \right)^2 \right) \diff x_1' = \\
        &= \frac{\ell_1^2}{2} \int_\Pi \left( \frac{\partial u}{\partial x_1} \right)^2 \diff x \le \\
        &\le \frac{\ell_1^2}{2} \int_{\Pi} \sum_{i = 1}^n \left( \frac{\partial u}{\partial x_i} \right)^2 \diff x.
    \end{align*}

    Тобто нерівність \eqref{eq:5.1.18} виконується з константою $C_\Omega = \frac{\ell_1}{\sqrt{2}}$. \medskip

    З нерівності \eqref{eq:5.1.18}, отриманої для функцій $C_0^\infty(\Omega)$ шляхом використання процедури поповнення, отримаємо, що ця нерівність є справедливою для функції $u(x) \in \overset{\circ}{W} {}_2^1(\Omega)$.
\end{proof}

\begin{remark}
    Зокрема, нерівність \eqref{eq:5.1.18} дозволяє ввести в просторі $\overset{\circ}{W} {}_2^1(\Omega)$ еквівалентну норму.
\end{remark}

\begin{proposition}
    Покажемо, що $\|u\|_2^2 = \int_\Omega \sum_{i = 1}^n \left( \frac{\partial u}{\partial x_i} \right)^2 \diff x$ є еквівалентною нормою в просторі $\overset{\circ}{W} {}_2^1(\Omega)$. Тобто, покажемо, що знайдуться такі додатні константи $C_1$, $C_2$, що $C_1 \|u\|_{W_2^1(\Omega)} \le \|u\|_2 \le C_2 \|u\|_{W_2^1(\Omega)}$.
\end{proposition}

\begin{proof}
    Дійсно,
    \begin{align*}
        \|u\|_2^2
        &= \left( \int_\Omega \sum_{i = 1}^n \left( \frac{\partial u}{\partial x_i} \right)^2 \diff x \right) \le \\
        &\le \left( \int_\Omega \sum_{i = 1}^n \left( \frac{\partial u}{\partial x_i} \right)^2 \diff x \right) + \int_\Omega u^2(x) \diff x = \\
        &= \|u\|_{W_2^1(\Omega)}^2,
    \end{align*}
    тобто $C_2 = 1$. \medskip
    
    Також виконується нерівність
    \begin{align*}
        \|u\|_{W_2^1(\Omega)}^2
        &= \left( \int_\Omega \sum_{i = 1}^n \left( \frac{\partial u}{\partial x_i} \right)^2 \diff x \right) + \int_\Omega u^2(x) \diff x \le \\
        &\le (1 + C_\Omega^2) \left( \int_\Omega \sum_{i = 1}^n \left( \frac{\partial u}{\partial x_i1} \right)^2 \diff x \right) = \\
        &= (1 + C_\Omega^2) \|u\|_2^2.
    \end{align*}

    З останньої нерівності випливає, що $C_1 = \frac{1}{\sqrt{1 + C_\Omega^2}}$.
\end{proof}

\begin{theorem}[Релліха, про компактність обмеженої множини]
    Будь-яка обмежена множина в просторі $W_2^1(\Omega)$ компактна в $L_2(\Omega)$.
\end{theorem}

\begin{remark}
    Тобто, якщо $M$ --- така нескінчена множина функцій, що $\forall u \in M$: $\|u\|_{W_2^1(\Omega)} \le C$, то з $M$ можна виділити нескінчену підмножину, яка збігається в $L_2(\Omega)$ до деякого елементу з цього простору. 
\end{remark}

\subsubsection{Еквівалентні нормування у просторах \texorpdfstring{$\overset{\circ}{W} {}_2^1(\Omega)$}{W21oOmega} та \texorpdfstring{$W_2^1(\Omega)$}{W21Omega}}

Нехай в області $\Omega$ з границею $S \in C^1$ задана дійсна неперервна в $\overline{\Omega}$ симетрична матриця $P(x) = (p_{i,j}(x))_{i,j=\overline{1..n}}$, функція $a(x) \in C(\Omega)$, а на границі $S$ функція $\sigma(x) \in C(S)$. Визначимо на $W_2^1(\Omega)$ ермітову білінійну форму 
\begin{equation}
    \label{eq:5.1.19}
    \begin{aligned}
        W(f, g)
        &= \int_\Omega \left( \sum_{i,j=1}^n p_{i,j}(x) f_{x_i}(x) \overline{g}_{x_j}(x) + a(x) f(x) \overline{g}(x) \right) \diff x + \\
        &\quad + \int_S \sigma(x) f(x) \overline{g}(x) \diff S.
    \end{aligned}
\end{equation}

\begin{theorem}[про еквівалентність норм у просторі $W_2^1(\Omega)$]
    Якщо матриця $P(x)$ додатно-визначена, тобто для кожного комплексного вектора $\xi = (\xi_1, \ldots, \xi_n)$ і для усіх $x \in \overline{\Omega}$: $\sum_{i,j=1}^n p_{i,j} \xi_i \overline{\xi}_j \ge \gamma \|\xi\|^2$, де $\gamma$ --- певна додатна стала, а функції $a(x)$ і $\sigma(x)$ невід'ємні у своїх областях визначення, то білінійна форма \eqref{eq:5.1.19} визначає в $W_2^1(\Omega)$ скалярний добуток, еквівалентний скалярному добутку
    \begin{equation}
        \label{eq:5.1.20}
        (f, g)_{W_2^1(\Omega)} = \int_\Omega ((\nabla f(x), \nabla g(x)) + f(x) \overline{g}(x)) \diff x.
    \end{equation}
\end{theorem}

\begin{remark}
    Це фактично означає, що існують такі додатні константи $C_1$, $C_2$ що виконується нерівність
    \begin{equation}
        \label{eq:5.1.21}
        C_2^2 \|f\|_{W_2^1(\Omega)}^2 \le W(f, f) \le C_1^2 \|f\|_{W_2^1(\Omega)}^2.
    \end{equation}
\end{remark}

\begin{remark}
    Таким чином, ця теорема дозволяє ввести у просторі $W_2^1(\Omega)$ норму
    \begin{equation}
        \label{eq:5.1.22}
        \|f\|_\star^2 = W(f, f),
    \end{equation}
    еквівалентну звичайній нормі в цьому просторі.
\end{remark}

\begin{proof}
    Для доведення теореми необхідно встановити справедливість двосторонньої нерівності \eqref{eq:5.1.21}. Зауважимо, що в виразі \eqref{eq:5.1.19} кожен з трьох доданків для $W(f,f)$ невід'ємний. Оскільки
    \begin{align*}
        \int_\Omega \left( \sum_{i,j=1}^n p_{i,j} f_{x_i} \overline{f}_{x_j} \right) \diff x
        &\le p_0 \int_\Omega \sum_{i,j=1}^n \left|f_{x_i}\right| \cdot \left|\overline{f}_{x_j}\right| \diff x \le \\
        &\le p_0 n \int_\Omega |\nabla f|^2 \diff x \le \\
        &\le p_0 n \|f\|_{W_2^1(\Omega)}^2, \quad p_0 = \max_{1\le i,j\le n} \|p_{i,j}\|_{C(\Omega)}, \\
        \int_\Omega a(x) |f|^2 \diff x
        &\le a_1 \|f\|_{L_2(\Omega)}^2, \quad a_1 = \|a\|_{C(\Omega)}, \\
        \int_S \sigma(x) |f|^2 \diff S
        &\le \sigma_1 \|f\|_{L_2(S)}^2 \le \\
        &\le C^2 \sigma_1 \|f\|_{W_2^1(\Omega)}^2, \quad \sigma_1 = \|\sigma\|_{C(S)}.
    \end{align*}

    З цих нерівностей випливає, що права нерівність \eqref{eq:5.1.21} виконується зі сталою $C_1^2 = p_0 n + a_1 + \sigma_1 C^2$. \medskip

    Покажемо справедливість лівої нерівності \eqref{eq:5.1.21}, а саме покажемо що виконується нерівність $\|f\|_{W_2^1(\Omega)}^2 \le \frac{1}{C_2^2} W(f, f)$. \medskip

    Припустимо зворотне, що відповідної константи $C_2$ не існує. Тоді для довільного цілого $m \ge 1$ знайдеться така функція $f_m(x) \in W_2^1(\Omega)$, що $\|f_m\|_{W_2^1(\Omega)}^2 \ge m W(f_m, f_m)$, тобто знайдеться \[g_m(x) = \frac{f_m(x)}{\|f_m\|_{W_2^1(\Omega)}},\quad \|g_m\|_{W_2^1(\Omega)} = 1,\] така, що
    \begin{align*}
        W(g_m, g_m)
        &= \int_\Omega \left( \sum_{i,j=1}^np_{i,h} g_{m, x_i} \overline{g}_{m, x_j} + a(x) |g_m|^2 \right) \diff x + \\
        &\quad + \int_S \sigma(x) |g_m|^2 \diff S \le \frac{1}{m}.     
    \end{align*}

    Звідси випливає, що кожне з доданків останньої нерівності не перевищує $1/m$ і тому виконуються нерівності:
    \begin{equation}
        \label{eq:5.1.23}
        \int_\Omega a|g_m|^2 \diff x < \frac{1}{m}, \quad \int_S \sigma |g_m|^2 \diff S < \frac{1}{m}, \quad \int_\Omega |\nabla g_m|^2 \diff x < \frac{1}{m \gamma}.
    \end{equation}

    З \eqref{eq:5.1.23} випливає, що послідовність $g_m$, $m = 1, 2, \ldots$ обмежена в $W_2^1(\Omega)$, тому з неї можна обрати фундаментальну в $L_2(\Omega)$ послідовність; нехай це є послідовність $g_m$, $m = 1, 2, \ldots$, тобто $\|g_m - g_p\|_{L_2(\Omega)} \to 0$, $m, p \to \infty$. Розглянемо
    \begin{align*}
        \|g_m - g_p\|_{W_2^1(\Omega)}^2
        &= \|g_m - g_p\|_{L_2(\Omega)}^2 + \| |\nabla (g_m - g_p)| \|_{L_2(\Omega)}^2 \le \\
        &\le \|g_m - g_p\|_{L_2(\Omega)}^2 + 2 \| |\nabla g_m| \|_{L_2(\Omega)}^2 + 2 \| |\nabla g_p| \|_{L_2(\Omega)}^2 \le \\
        &\le \|g_m - g_p\|_{L_2(\Omega)}^2 + \frac{2}{m \gamma} + \frac{2}{p \gamma} \xrightarrow[p, m \to \infty]{} 0.
    \end{align*}

    Таким чином послідовність $g_m$, $m = 1, 2, \ldots$ фундаментальна в просторі $W_2^1(\Omega)$ і тому збігається по нормі цього простору до деякого елементу $g \in W_2^1(\Omega)$. \medskip 

    Переходимо до границі при $m \to \infty$, отримаємо
    \begin{equation}
        \label{eq:5.1.24}
        \|g\|_{W_2^1(\Omega)} = 1, \quad \int_\Omega |\nabla g|^2 \diff x = 0, \quad \int_\Omega a |g|^2 \diff x = 0, \quad \int_S \sigma |g|^2 \diff S = 0.
    \end{equation}

    З перших двох рівностей випливає, що $g(x) = \const$, $x \in \Omega$, звідси маємо $g(x) = \frac{1}{\sqrt{|\Omega|}}$, $x \in \Omega$ таким чином маємо протиріччя з двома останніми рівностями \eqref{eq:5.1.24} і теорема доведена.
\end{proof}

\subsection{Узагальнені розв'язки задачі Діріхле еліптичних рівнянь в просторі \texorpdfstring{$W_2^1(\Omega)}$}{W21Omega}}

Будемо розглядати лінійне еліптичне рівняння другого порядку зі змінними коефіцієнтами:
\begin{equation}
    \label{eq:5.2.1}
    L u = \nabla \cdot (p(x) \nabla u) + a(x) u = f(x) + \um_{i = 1}^n \frac{\partial f_i(x)}{\partial x_i}.
\end{equation}

Головна частина рівняння \eqref{eq:5.2.1} $\nabla \cdot (p(x) \nabla u)$ допускає узагальнення вигляду
\begin{equation}
    \label{eq:5.2.2}
    \sum_{i, j = 1}^n \frac{\partial}{\partial x_i} \left( p_{i,j}(x) \frac{\partial u}{\partial x_j} \right), \quad p_{i,j}(x) = p_{j,i}(x).
\end{equation}

Для рівняння \eqref{eq:5.2.1} будемо припускати, що 
\begin{align}
    \label{eq:5.2.3}
    \nu &\le p(x) \le \mu, \quad \nu, \mu > 0, \\
    \label{eq:5.2.3'}
    a_1 &\le a(x) \le a_2.
\end{align}

Для головної частини \eqref{eq:5.2.2} умова \eqref{eq:5.2.3} трансформується в умову
\begin{equation}
    \label{eq:5.2.3''}
    \nu \xi^2 \le \sum_{i,j=1}^n p_{i,j} \xi_i \xi_j \le \mu \xi^2, \quad \xi^2 = \sum_{i = 1}^n \xi_i^2.
\end{equation}

Для рівняння \eqref{eq:5.2.1} будемо розглядати три основні граничні задачі:
\begin{itemize}
    \item з умовами Діріхле
    \begin{equation}
        \label{eq:5.2.4}
        \left. u \right|_S = \phi(x), 
    \end{equation}
    \item Неймана
    \begin{equation}
        \label{eq:5.2.5}
        \left. \frac{\partial u}{\partial N} \right|_S = \phi(x), 
    \end{equation}
    \item або Ньютона
    \begin{equation}
        \label{eq:5.2.6}
        \left. \frac{\partial u}{\partial N} + \sigma(x) u \right|_S = \phi(x).
    \end{equation}
\end{itemize}

Де $\frac{\partial u}{\partial N} = p(x) \frac{\partial u}{\partial n}$, для головної частини \eqref{eq:5.2.2} $\frac{\partial u}{\partial N} = \sum_{i,j=1}^n p_{i,j}(x) u_{x_j} \cos (n, x_i)$, а $n$ --- одинична зовнішня нормаль до поверхні $S$. \medskip

Усі перераховані задачі можуть бути зведені до задач з однорідним граничними умовами, тобто такими умовами, що $\phi(x) \equiv 0$. \meskip

Для цього необхідно записати граничну задачу відносно нової функції $v(x) = u(x) - \Phi(x)$, де функція $\Phi(x)$ задовольняє відповідній граничній умові та належить простору $W_2^1(\Omega)$. \medskip

Відносно $f$ та $f_i$, $i = \overline{1..n}$ будемо рахувати, що $\|f\|_{L_2(\Omega)}, \|f_i\|_{L_2(\Omega)} < \infty$. \medskip

Розглянемо граничну задачу для рівняння \eqref{eq:5.2.1} з однорідними граничними умовами
\begin{equation}
    \label{eq:5.2.4'}
    \left. u \right_S = 0.
\end{equation}

Введемо білінійну форму:
\begin{equation}
    \label{eq:5.2.7}
    L(u, \eta) := \int_\Omega \left( \sum_{i=1}^n p(x) u_{x_i} \eta_{x_i} - a(x)u\eta \right) \diff x = \int_\Omega \left( -f\eta + \sum_{i=1}^n f_i \eta_{x_i} \right) \diff x.
\end{equation}

Рівність \eqref{eq:5.2.7} можна отримати з рівняння \eqref{eq:5.2.1} шляхом множення його на функцію $\eta$, інтегрування добутку по області $\Omega$, застосування формули інтегрування за частинами та використанням умови \eqref{eq:5.2.4'}.

\begin{definition}
    Функцію $u(x)$ з простору $\overset{\circ}{W} {}_2^1$ будемо називати \emph{узагальненим розв'язком} задачі Діріхле \eqref{eq:5.2.1}, \eqref{eq:5.2.4'}, якщо для кожного елементу $\eta \in \overset{\circ}{W} {}_2^1(\Omega)$ виконується інтегральна тотожність \eqref{eq:5.2.7}.
\end{definition}

Інтегральна тотожність \eqref{eq:5.2.7} має зміст для більш широкого класу функцій ніж гранична задача \eqref{eq:5.2.1}, \eqref{eq:5.2.4'}. Цілком зрозуміло, що якщо обрати функцію $\eta(x), u(x) \in C_\infty^0(\Omega)$, то шляхом інтегрування за частинами можемо перейти від інтегральної тотожності до диференціального рівняння \eqref{eq:5.2.1}. \medskip

В той же час інтегральна тотожність \eqref{eq:5.2.7} має зміст для функцій $u, \eta \in \overset{\circ}{W} {}_2^1$. Таким чином з використанням інтегральної тотожності \eqref{eq:5.2.7} ми розширили поняття розв'язку граничної задачі Діріхле. \medskip

Отримаємо для узагальненого розв'язку енергетичну тотожність, яка дозволить нам довести єдиність узагальненого розв'язку задачі Діріхле. \medskip

Розглянемо квадратичну форму і запишемо нерівності:

% TODO
