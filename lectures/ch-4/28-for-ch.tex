\setcounter{section}{4}

\section{Дослідження узагальнених розв'язків граничних задач}

\subsection{Математичний апарат дослідження узагальнених розв'язків граничних задач}

\subsubsection{Допоміжні результати}

При дослідженні властивостей узагальнених розв'язків граничних задач ми будемо користуватися апаратом функціонального аналізу, тому нагадаємо деякі важливі поняття.

\begin{definition}
    Нехай $E$ --- лінійний нормований простів, тоді будемо говорити, що сукупність елементів $E' \subset E$ є \emph{всюди щільною} в $E$, якщо будь-який елемент $E$ можна представити як границю в нормі $E$ елементів з $E'$.
\end{definition}

\begin{definition}
    Якщо в $E$ існує злічена всюди щільна множина, то $E$ називають \emph{сепарабельним} простором.
\end{definition}

\begin{definition}
    Простір $E$ називають \emph{повним}, якщо для будь-якої фундаментальної в $E$ послідовності існує граничний елемент, що належить $E$.
\end{definition}

\begin{definition}
    Повний лінійний нормований простір називається \emph{банаховим} простором.
\end{definition}

\begin{definition}
    Банаховий простір $H$ в якому для будь-яких двох елементів $u, v \in H$ визначений скалярний добуток $(u, v)_H$ з усіма властивостями скалярного добутку називають \emph{гільбертовим} простором.
\end{definition}

\begin{definition}
    Послідовність $(u_n) \in H$ будемо називати такою, що \emph{слабко збігається} в $H$ до елементу $u$, якщо $(u_n - u, v)_H \xrightarrow[n \to \infty]{} 0$, $\forall v \in H$. 
\end{definition}

\begin{definition}
    Множина $M$ банахового (гільбертового) простору $B$ називається \emph{компактною}, якщо будь-яка нескінчена підмножина $M$ містить збіжну в просторі $B$ (під)послідовність.
\end{definition}

\begin{definition}
    \emph{Лінійним функціоналом} $\ell$ у гільбертовому просторі $H$ будемо називати лінійну неперервну числову функцію $\ell(u)$, визначену для кожного $u \in H$.
\end{definition}

\begin{definition}
    Функціонал є \emph{неперервним} якщо числова послідовність $\ell(u_n) \xrightarrow[n \to \infty]{} \ell(u)$ для $u_n \xrightarrow{H} u$.
\end{definition}

\begin{proposition}
    Будь-який неперервний лінійний функціонал є також і обмеженим ($|\ell(u)| \le C \|u\|_H$), і навпаки.
\end{proposition}

\begin{theorem}[Фрідьєша Ріса---Фішера про представлення лінійного неперервного функціоналу в гільбертовому просторі]
    Будь-який неперервний лінійний функціонал $\ell$, заданий у гільбертовому просторі $H$ може бути представлений у вигляді скалярного добутку $\ell(u) = (u,v)_H$, при цьому $v$ визначається для кожного функціоналу $\ell$ єдиним чином. Окрім цього, $\|\ell\| = \|v\|_H$.
\end{theorem}

\begin{definition}
    Лінійний оператор $A$, який діє з банахового простору $B_1$ у банахів простів $B_2$, називається \emph{обмеженим}, якщо існує така константа $C$, що для будь-якого елементу $f \in D(A) \subset B_1$ виконується нерівність $\|A f\|_{B_2} \le C \|f\|_{B_1}$.
\end{definition}

\begin{definition}
    Лінійний оператор $A$, який діє з банахового простору $B_1$ у банахів простів $B_2$, називається \emph{неперервним на елементі} $f \in D(A) \subset B_1$, якщо для будь-якої послідовності елементів $D(A) \ni f_n \xrightarrow[n \to \infty]{} f \in D(A)$ (тут збіжність по нормі банохового простору $B_1$), числова послідовність $\|Af_n\|_{B_2}$ збігається до елемента $\|Af\|_{B_2}$.
\end{definition}

\begin{definition}
    Якщо оператор $A$ неперервний на кожному елементі банахового простору $B_1$, його називають \emph{неперервним}.
\end{definition}

\begin{proposition}
    Для того щоб лінійний оператор $A$ був неперервним необхідно і достатньо щоб цей оператор був обмеженим.
\end{proposition}

\begin{definition}
    Нехай $D(A^\star) \subset H$, підмножина елементів гільбертового простору $H$, з наступними властивостями: для будь-якого $g \in D(A^\star)$ існує такий елемент $h \in H$, що для усіх $а \in D(A)$ виконується рівність $(A f, g) = (f, h)$. \medskip

    Тоді кажуть, що на множині $D(A^\star)$ \emph{заданий} оператор $A^\star$, який ставить у відповідність елементу $g \in D(A^\star)$ елемент $h = А^\star g$. Оператор $A^\star$ називається \emph{спряженим} до оператора $A$.
\end{definition}

\begin{proposition}
    Для основного і спряженого оператора виконується $(A f, g) = (f, A^\star g)$.
\end{proposition}

\begin{definition}
    Якщо $A = A^\star$, то оператор $A$ називається \emph{самоспряженим}.
\end{definition}

\begin{definition}
    Будемо називати лінійний оператор $A$, який діє  з гільбертового простору $H$ у себе, \emph{цілком неперервним} (eng. \emph{compact operator}), якщо він переводить будь-яку обмежену множину елементів $H$ у компактну множину елементів.
\end{definition}

\begin{definition}
    Оператор $A$, який діє у гільбертовому просторі називається \emph{скінченно-вимірним} (\emph{$n$-вимірним}), якщо він відображає гільбертов простір $H$ у його (\emph{$n$-вимірний}) підпростір.
\end{definition}

\begin{theorem}[про представлення цілком неперервного оператора]
    Для того щоб заданий на сепарабельному гільбертовому просторі $H$ лінійний обмежений оператор $A$ із $H$ в $H$ був цілком неперервним необхідно і достатньо щоб для довільного $\epsilon > 0$ існувало ціле число $n(\epsilon)$ і такі лінійні оператори $A_1$ та $A_2$, $A_1$ --- $n$-вимірний), а $A_2 \le \epsilon$, що $A = A_1 + A_2$. 
\end{theorem}

\subsubsection{Матричне представлення лінійного обмеженого оператора}

Нехай $A$ --- лінійний обмежений оператор, який діє з сепарабельного гільбертового простору $H$ в $H$ нехай $D_A = H$, $\{e_i\}_{i = 1}^\infty$ --- ортонормований базис $H$. Розглянемо нескінченну матрицю з елементами
\begin{equation}
    \label{eq:5.1.1}
    a_{i,j} = (A e_i, e_j) = (e_i, A^\star e_j), \quad i, j = 1, 2, \ldots
\end{equation}

\begin{definition}
    Будемо називати її \emph{матричним представленням} оператора $A$ по базису $\{e_i\}_{i = 1}^\infty$.
\end{definition}

Оскільки $a_{i,j} = (A e_i, e_j)$ --- коефіцієнт Фур'є елемента $A e_i$, то з рівності Парсеваля---Стєклова будемо мати, що
\begin{equation}
    \label{eq:5.1.2}
    \sum_{j = 1}^\infty |a_{i, j}|^2 = \|A e_i\|^2 \le \|A\|^2 = \|A^\star\|^2.
\end{equation}

Візьмемо довільний елемент $f \in H$ і нехай $f = \sum_{i = 1}^\infty f_i e_i$ --- його розвинення в ряд Фур'є. Оскільки $A f \in H$, то його коефіцієнти Фур'є мають вигляд
\begin{equation}
    \label{eq:5.1.3}
    (A f)_j = (Af, e_j) = \left( A \sum_{i = 1}^\infty f_i e_i, e_j \right) = \sum_{i = 1}^\infty f_i (A e_i, e_j) = \sum_{i = 1^\infty} f_i a_{i,j}, \quad j = 1, 2, \ldots
\end{equation}

Ряд в правій частині збігається абсолютно, оскільки його загальний член мажорується загальним членом збіжного ряду $\frac{1}{2} (|f_i|^2 + |a_{i,j}|^2)$ (за нерівністю Коші між середнім арифметичним і середнім геометричним). Підставляючи
значення коефіцієнтів Фур'є в ряд Фур'є $A f = \sum_{j = 1}^\infty (A f)_j e_j$, отримаємо
\begin{equation}
    \label{eq:5.1.4}
    A f = \sum_{j = 1}^\infty \left( a_{i,j} f_i \right) e_j.
\end{equation}

Таким чином, для довільного елементу $f \in H$ елемент $A f \in H$ може бути знайдений за формулою \eqref{eq:5.1.4} лише з використанням матриці $a_{i, j}$.

\begin{proposition}
    Матричне представлення спряженого оператора $A^\star$ в базисі $\{e_i\}_{i = 1}^\infty$ записується через матрицюз елементами
    \begin{equation*}
        a_{i, j}^\star = (A^\star e_i, e_j) = (e_i, A e_j) = \bar a_{j, i}, \quad i, j = 1, 2, \ldots
    \end{equation*}
    тобто через транспоновану і комплексно-спряжену.
\end{proposition}

\subsubsection{Стискаючий лінійний оператор}

Результати викладені в цьому параграфі справедливі для будь-якого банахового простору, але ми розглянемо лише випадок коли лінійний оператор діє в сеперебельному гільбертовому просторі $H$. 

\begin{definition}
    Будемо називати лінійний оператор $A$ що діє з $H$ в $H$ \emph{стискаючим} якщо $\|A\|_H < 1$.
\end{definition}

\begin{lemma}
    Якщо $A$ лінійний стискаючий оператор з $H$ в $H$, то існує заданий на $H$ оператор $(I - A)^{-1}$ з $H$ в $H$ такий що $\|(I - A)^{-1} \| \le \frac{1}{1 - \|A\|}$.
\end{lemma}

\begin{proof}
    Розглянемо рівняння
    \begin{equation}
        \label{eq:5.1.5}
        (I - A) f = g,
    \end{equation}
    і покажемо, що для довільного $g \in H$ єдиним розв'язком є розв'язок, що представляється збіжним в $H$ рядом
    \begin{equation}
        \label{eq:5.1.6}
        f = \sum_{k = 0}^\infty A^k g,
    \end{equation}
    де $A^0 = I$. \medskip

    Цей ряд збіжний, оскільки $H$ -- повний простір, а часткові суми ряду  $f_m = \sum_{k = 0}^m A^k g$ утворюють фундаментальну послідовність: при $p > m$:
    \begin{equation*}
        \|f_p - f_m\| = \|A^p g + \ldots + A^{m + 1} g\| \le \|A^p g\| + \ldots + \|A^{m + 1} g\| \le \|g\| \left( \|A\|^{m + 1} + \ldots \right) = \|g\| \cdot \frac{\|A\|^{m + 1}}{1 - \|A\|} \to 0,
    \end{equation*}
    при $m, p \to \infty$. \medskip

    Елемент $f \in H$ є розв'язком рівняння \eqref{eq:5.1.5} оскільки
    \begin{equation*}
        (I - A) f = (g + A g + A^2 g + \ldots) - (A g + A^2 g + \ldots) = g.    
    \end{equation*}

    Цей розв'язок єдиний. Дійсно, якщо припустити, що існують два елементи $f_1$ та $f_2$, то $f = f_1 - f_2$ задовольняє однорідному рівнянню $(I - A) f = 0$. Тому має місце оцінка $\|f\| = \|A f\| \le \|A\| \|f\|$, звідки або $\|A\| \ge 1$ і отримали суперечність, або $f = 0$ і $f_1 = f_2$. \medskip

    Тобто оператор $(I - A)^{-1}$ існує, визначений на усьому просторі $H$. Оскільки
    \begin{equation*}
        \|(I - A)^{-1} g\| = \|g + A g + \ldots + A^m g + \ldots| \le \|g\| (1 + \|A\| + \ldots) \le \frac{g}{1 - \|A\|}
    \end{equation*}
    для усіх $g \in H$, то оператор $(I - A)^{-1}$ є обмеженим. 
\end{proof}

\begin{remark}
    В припущеннях цієї леми існує також обмежений оператор $(I - A^\star)^{-1}$, оскільки $\|A\| = \|A^\star\|$. При цьому $(I - A^\star)^{-1} = \left( (I - A)^{-1} \right)^\star$.
\end{remark}

\subsubsection{Представлення операторного рівняння з цілком неперервним оператором у скінчено вимірному вигляді}

З теореми про представлення цілком неперервного оператора в сепарабельному гільбертовому просторі, операторне рівняння $(I - \lambda A) f = g$ можна представити у вигляді
\begin{equation}
    \label{eq:5.1.7}
    (I - \lambda A_2) f - \lambda A_1 f = g,
\end{equation}
де оператор $A_1$ --- $n$-вимірний, а $|\lambda| \|A_2\| \le \epsilon < 1$. Позначимо $(I - \lambda A_2) f = h$. За лемою~1 оператор $(I - \lambda_2 A)$ має на $H$ заданий обмежений оператор $(I - \lambda A_2)^{-1}$, при цьому $f = (I - \lambda A_2)^{-1} h$. \medskip

Таким чином операторне рівняння \eqref{eq:5.1.7} перепишеться у вигляді
\begin{equation}
    \label{eq:5.1.8}
    h - \lambda A_1 (I - \lambda A_2)^{-1} h = g.
\end{equation}

Для спряженого рівняння
\begin{equation}
    \label{eq:5.1.5'}
    (I - \bar \lambda A^\star) f^\star = g^\star
\end{equation}
можна записати представлення
\begin{equation}
    \label{eq:5.1.7'}
    (I - \bar \lambda A_2^\star) f^\star - \bar \lambda A_1^\star f^\star = g^\star.
\end{equation}

Застосуємо до останнього рівняння оператор $\left( I - \bar \lambda A_2^\star \right)^{-1}$, отримаємо еквівалентне рівняння:
\begin{equation}
    \label{eq:5.1.8'}
    f^\star - \bar \lambda \left( I - \bar \lambda A_2^\star \right)^{-1} A_1^\star f^\star = z^\star, \quad x^\star = \left( I - \bar \lambda A_2^\star \right)^{-1} g^\star.
\end{equation}

При цьому оператор $\left( I - \bar \lambda A_2^\star \right)^{-1} A_1^\star$ буде спряженим до оператора $A_1 (I - \lambda A_2)^{-1}$. Враховуючи скінченновимірність оператора $A_1$ скінченновимірним буде також оператор $A_1 (I - \lambda A_2)^{-1}$, тобто його матричне представлення може бути записане через елементи базису $\{e_1, e_2, \ldots, e_n\}$ і при цьому
\begin{equation*}
    \sum_{i = 1}^\infty |a_{i,j}|^2 = \left\| A_1 (I - \lambda A_2)^{-1} \right\|^2.
\end{equation*}

Рівняння \eqref{eq:5.1.8} можна представити у вигляді $\sum_j h_j e_j - \lambda \sum_j \sum_i h_i a_{i, j} e_j = \sum_j g_h e_j$, яке еквівалентне алгебраїчній системі рівнянь для коефіцієнтів Фур'є $h_1, h_2, \ldots, h_m, \ldots$ елементу $h$:
\begin{equation*}
    h_j - \lambda \sum_{i = 1}^\infty a_{i, j} h_i = g_j, \quad j \le n, \quad h_j = g_j, \quad j > n.
\end{equation*}

Оскільки $j > n$: $h_j = g_j$, тобто відомі, то остання система зводиться до системи алгебраїчних рівнянь
\begin{equation}
    \label{eq:5.1.9}
    h_j - \lambda \sum_{i = 1}^n a_{i, j} h_i = g_j + \lambda \sum_{i = n + 1}^\infty a_{i, j} g_j, \quad j \le n.
\end{equation}

Аналогічним чином спряжене рівняння \eqref{eq:5.1.8'} можна замінити системою лінійних алгебраїчних рівнянь для знаходження коефіцієнтів Фур'є $f_j^\star$, $j = 1, 2, \ldots$ елемента $f^\star$ через коефіцієнти Фур'є $z_j^\star$, $j = 1, 2, \ldots$ елемента $z^\star = (I - \bar \lambda A_2^\star)^{-1} g^\star$. При цьому для $f_j^\star$, $j \le n$ отримаємо лінійну алгебраїчну систему
\begin{equation}
    \label{eq:5.1.9'}
    f_j^\star - \bar \lambda \sum_{i = 1}^n \bar a_{j, i} f_j^\star = z_j^\star, \quad j = 1, 2, \ldots, n.
\end{equation}

При цьому
\begin{equation}
    \label{eq:5.1.10}
    f_j^\star = z_j^\star = \bar \lambda \sum_{i = 1}^n \bar a_{j, i} f_j^\star, \quad j > n.
\end{equation}

\subsubsection{Теореми Фредгольма для операторного рівняння}

Розглянемо операторні рівняння другого роду з лінійним цілком неперервним оператором $A$, який діє з гільбертового простору $H$ у гільбертів простір $H$:
\begin{equation}
    \label{eq:5.1.11}
    f - \lambda A f = g,
\end{equation}
спряжене рівняння[
\begin{equation}
    \label{eq:5.1.12}
    f^\star - \lambda A^\star - f^\star = g^\star
\end{equation}
та однорідні рівняння:
\begin{equation}
    \label{eq:5.1.11'}
    f - \lambda A f = 0,
\end{equation}
\begin{equation}
    \label{eq:5.1.12'}
    f^\star - \lambda A^\star - f^\star = 0,
\end{equation}
де $\lambda$ --- комплекснозначний параметр. \medskip

Для вказаних рівнянь має місце \textbf{теореми Фредгольма}, які узагальнюють відомі нам теореми Фредгольма для інтегральних рівняння Фредгольма другогороду (інтегральний оператор з неперервним та полярним ядром є цілком неперервним). 

\begin{theorem}[перша теорема Фредгольма для операторного рівняння]
    Якщо однорідне рівняння \eqref{eq:5.1.11'} має при даному значені параметру $\lambda$ лише тривіальний розв'язок, то спряжене рівняння \eqref{eq:5.1.12'} теж має лише тривіальний розв'язок, а неоднорідне рівняння \eqref{eq:5.1.11} і спряжене до нього неоднорідне рівняння \eqref{eq:5.1.10} мають єдині розв'язки для будь-яких $g, g^\star \in H$.
\end{theorem}

\begin{proof}
    Розглянемо системи лінійних алгебраїчних рівнянь \eqref{eq:5.1.9} та \eqref{eq:5.1.9'}, матриці цих систем є ермітово спряженими, таким чином модулі визначників цих матриць співпадають. Якщо для одна з цих систем має розв'язок при довільному вільному члені (тобто визначник відмінний від нуля) то і спряжене рівняння має розв'язок для довільного вільного члена. \medskip

    Тоді, зокрема, розв'язки однорідного рівняння і спряженого рівняння тільки тривіальні. \medskip

    Якщо одне з однорідних рівнянь має лише тривіальний розв'язок (тобто відповідний визначник не дорівнює нулю) то спряжене однорідне рівняння має лише тривіальний розв'язок, при цьому неоднорідні системи \eqref{eq:5.1.9} та \eqref{eq:5.1.9'} мають єдині розв'язки для довільних вільних членів. Ту саму властивість мають рівняння \eqref{eq:5.1.7} та \eqref{eq:5.1.7'}. Дійсно, нехай рівняння \eqref{eq:5.1.7} або \eqref{eq:5.1.7'} має розв'язок для довільного вільного члена $g \in H$ (або $g^\star \in H$). Або те ж саме рівняння \eqref{eq:5.1.8} та \eqref{eq:5.1.8'} має розв'язок для довільного вільного члена $g \in H$ (або $z^\star \in H$), зокрема і для $g \in H$ (або $z^\star \in H$), які натягнуті на елементи базису $e_1, e_2, \ldots, e_n$. \medskip

    Таким чином система алгебраїчних рівнянь \eqref{eq:5.1.9} або \eqref{eq:5.1.9'} мають розв'язки для довільного вільного члена, тому визначник цих систем відмінний від нуля, а значить відповідні однорідні системи рівнянь мають лише тривіальні розв'язки. \medskip

    Зворотньо, нехай одне з однорідних рівнянь \eqref{eq:5.1.11'} або \eqref{eq:5.1.12'} має лише тривіальний розв’язок, тоді відповідна однорідна система лінійних алгебраїчних рівнянь для \eqref{eq:5.1.9}, \eqref{eq:5.1.9'} мають лише тривіальні розв'язки. Визначники обох систем відмінні від нуля, тобто неоднорідні системи \eqref{eq:5.1.9}, \eqref{eq:5.1.9'} мають єдині розв'язки для довільних вільних членів, а тоді єдині розв'язки мають рівняння \eqref{eq:5.1.11}, \eqref{eq:5.1.12} для довільних вільних членів з $H$.
\end{proof}

\begin{theorem}[друга теорема Фредгольма для операторного рівняння]
    Однорідне рівняння \eqref{eq:5.1.11'} може мати нетривіальний розв'язок лише для зліченної множини значень параметру $\lambda = \lambda_k$, $k = 1, 2, \ldots$, кожне з яких має скінченну кратність. Спряжене рівняння \eqref{eq:5.1.12'} має нетривіальний розв'язок тоді і лише тоді, коли параметр $λ = \bar \lambda_k$, $k = 1, 2, \ldots$, при цьому кратність $\lambda_k$ і $\bar \lambda_k$ співпадає.
\end{theorem}

\begin{proof}
    Для довільного значення параметра $\lambda$: $\rang (\textbf{E} - \lambda \textbf{A}) = \rang (\textbf{E} - \bar \lambda \textbf{A}^\star) = q \le n$, де $\textbf{A} = (a_{i,j})_{i,j=1}^n$, $\textbf{A}^\star = (\bar a_{j,i})_{i,j=1}^n$. Таким чином однорідні для $\eqref{eq:5.1.9}$ та $\eqref{eq:5.1.9'}$ системи рівнянь мають однакову кількість лінійно незалежних розв'язків, що рівна $n - q$. Для випадку, коли $q < n$ число $\lambda$ будемо визначати як характеристичне число, число $r = n - q$ є кратність характеристичного числа $\lambda$, а відповідні розв'язки ($n$-вимірні вектори) $\textbf{f}^{(1)}, \textbf{f}^{(2)}, \ldots, \textbf{f}^{(r)}$ як власні вектори числа $\lambda$. При цьому спряжене однорідне рівняння для \eqref{eq:5.1.9'} має характеристичне число $\lambda$ і власні вектори $\textbf{f}^{\star(1)}, \textbf{f}^{\star(2)}, \ldots, \textbf{f}^{\star(r)}$.
\end{proof}

\begin{theorem}[третя теорема Фредгольма для операторного рівняння]
    Рівняння \eqref{eq:5.1.11} при $\lambda = \lambda_k$, кратності $r_k$ має розв'язок тоді і лише тоді, коли вільний член $g$ ортогональний до усіх розв'язків спряженого однорідного рівняння $f_{k + j, j = 0..r_k-1}^\star$, тобто $(g, f_{k+j}^\star)_H = 0$, $j=0..r_k-1$. Тоді розв'язок неєдиний і визначається з точністю до лінійної оболонки натягнутої на систему власних функцій характеристичного числа $\lambda_k$, тобто $f = f_0 + \sum_{j = 0}^{r_k - 1} c_j f_{k + j}$, де $f_0$ --- частинний розв'язок неоднорідного рівняння \eqref{eq:5.1.11}.
\end{theorem}

\begin{proof}
    Для випадку, коли параметр $\lambda$ є характеристичним числом, то для існування розв'язку неоднорідної системи лінійних алгебраїчних рівнянь \eqref{eq:5.1.9} згідно теореми Кронекера---Капеллі необхідно і достатньо щоб вільний член системи \eqref{eq:5.1.9} був ортогональним до усіх розв'язків спряженого однорідного рівняння. Тобто
    \begin{equation}
        \label{eq:5.1.13}
        0 = \sum_{j = 1}^n \left( g_j + \lambda \sum_{i = n + 1}^\infty a_{i, j} g_i \right) \textbf{f}_j^{\star(s)} = \sum_{j = 1}^n g_j \bar \textbf{f}_j^{\star (s)} +\sym_{j = n + 1}^\infty g_j \bar \textbf{f}_j^{\star(s)} = (g, \textbf{f}^\star).
    \end{equation}

    Розв'язок рівняння \eqref{eq:5.1.11} очевидно може бути записаний у вигляді $f^{(0)} + \sum_{s = 1}^r c_s f^{(s)}$, де $f^{(0)}$ --- довільний розв’язок неоднорідного рівняння, $c_s$, $s = 1, \ldots, r$ --- довільні константи.
\end{proof}

\subsubsection{Простір $W_2^k(\Omega)$ та його властивості}

Нехай $\Omega$ --- обмежена область з границею $S$, яка задовольняє умові Ліпшиця. Нехай $C^\infty(\bar \Omega)$ --- % лінійна
множина функцій неперервних з усіма їх похідними в області $\bar \Omega$. Позначимо через $C_0^\infty(\Omega)$ % лінійну
множину усіх функцій з $C^\infty(\bar \Omega)$, які мають компактний носій в області $\Omega$, тобто усі функції з множини $C_0^\infty (\bar \Omega)$ тотожно перетворюються в нуль в околі границі $S$ разом з усіма своїми похідними. \medskip

\begin{proposition}
    Для функцій з класів $C_0^\infty(\Omega)$ та $C^\infty(\bar \Omega)$ можна ввести скалярний добуток
    \begin{equation*}
        (u, v)_{L_2(\Omega)} = \int_\Omega u(x) \bar v(x) \diff x,
    \end{equation*}
    який породжує норму
    \begin{equation*}
        \|u\|_{L_2(\Omega)}^2 = \int_\Omega u(x) \bar u(x) \diff x.
    \end{equation*}
\end{proposition}

\begin{proposition}
    Для функцій з класів $C_0^\infty(\Omega)$ та $C^\infty(\bar \Omega)$ можна ввести скалярний добуток і в іншій
    спосіб: нехай $k$ --- ціле невід'ємне число. Тоді
    \begin{equation}
        \label{eq:5.1.14}
        (u, v)_{W_2^k(\Omega)} = \sum_{|i| \le k} \int_\Omega D^i u D^i v \diff x,
    \end{equation}
    де 
    \begin{equation}
        \label{eq:5.1.15}
        D^i u = \frac{\partial^{i_1 + i_2 + \ldots + i_n} u}{\partial x_1^{i_1} \partial x_2^{i_2} \ldots \partial x_n^{i_n}}.
    \end{equation}
\end{proposition}

\begin{proposition}
    Скалярний добуток \eqref{eq:5.1.14} породжує норму 
    \begin{equation}
        \label{eq:5.1.14'}
        \|u\|_{W_2^k(\Omega)}^2 = (u, u)_{W_2^k(\Omega)} = \sum_{|i| \le k} \int_\Omega D^i u D^i u \diff x.
    \end{equation}
\end{proposition}

Ця функція відповідає усім аксіомам норми і таким чином простір $C^\infty(\bar \Omega)$ із введеною нормою $\|u\|_{W_2^k(\Omega)}$ стає нормованим простором, який ми будемо позначати $S_2^k(\Omega)$. Збіжність у цьому просторі означає, що, якщо $\|u_n -  u\|_{W_2^k(\Omega)} \xrightarrow[n \to \infty]{} 0$, то $\|D^i u_n - D^i u\|_{L_2(\Omega)} \xrightarrow[n \to \infty]{} 0$, $\forall |i| \le k$, тобто в просторі $S_2^k(\Omega)$ по нормі $L_2(\Omega)$ збігається послідовність функцій та послідовність усіх частинних похідних до $k$-го порядку включно. \medskip

Особливістю нормованого простору $S_2^k(\Omega)$ є його неповнота, тобто не будь-яка фундаментальна по нормі простору $S_2^k(\Omega)$ послідовність буде мати границю, що належить цьому простору. \medskip

Важливим для подальшого буде наступна властивість похідних функцій з класу $S_2^k(\Omega)$:
\begin{equation}
    \label{eq:5.1.16}
    \int_\Omega D^i u(x) \psi(x) \diff x = (-1)^{|i|} \int_\Omega u(x) D^i \psi(x) \diff x, \quad \forall u \in S_2^k(\Omega), \psi \in C_0^\infty(\Omega).
\end{equation} 

Для побудови повного простору з нормою $W_2^k(\Omega)$ будемо використовувати стандартну процедуру поповнення. \medskip

У випадку $k = 0$ повним простором, який містить в собі $S_2^0(\Omega)$ та щільну % лінійну
множину $C^\infty(\bar \Omega)$ є простір $L_2(\Omega)$. \medskip

Для випадку $k > 0$ скалярний добуток введений в $S_2^k(\Omega)$ не може бути розширений на усі елементи простору $L_2(\Omega)$, оскільки не усі елементи цього простору мають похідні в якому-небудь розумному сенсі. \medskip

Розглянемо послідовність $\{u_n\}$ фундаментальну в $S_2^k(\Omega)$. Тобто $\|u_n - u_m\|_{W_2^k(\Omega)} \xrightarrown, m \to \infty]{} 0$, або $\|D^i u_n - D^i u_m\|_{L_2(\Omega)} \xrightarrow[n, m \to \infty]{} 0$, $|i| \le k$. Можливі два випадки:
\begin{enumerate}
    \item Послідовність $\{u_n\}$ збігається до деякого елементу $u$ в $S_2^k(\Omega)$, тобто $\|u_n - u\|_{W_2^k(\Omega)} \xrightarrow[n \to \infty]{} 0$ і елемент $u \in S_2^k(\Omega)$, тобто $u \in C^\infty(\Omega)$. При цьому послідовність $\{D^i u_n\} \xrightarrow D^i u$ по нормі $L_2(\Omega)$ для $|i| \le k$.
    \item Послідовність $\{u_n\}$ збігається до деякого елементу $u$ по нормі $S_2^k(\Omega)$, але елемент $u \not\in S_2^k(\Omega)$, тобто $u \not\in c^\infty(\Omega)$. В цьому випадку $u$ (як граничний елемент фундаментальної послідовності) будемо додавати до простору $S_2^k(\Omega)$.
\end{enumerate}

Якщо до простору $S_2^k(\Omega)$ додати границі усіх фундаментальних по нормі $W_2^k(\Omega)$ послідовностей, то ми отримаємо повний нормований простір який позначатимемо $W_2^k(\Omega)$. \medskip

Покажемо, що будь-який елемент $u \in W_2^k(\Omega)$ має похідні до $k$-го порядку включно Дійсно, якщо $u \in C^\infty(\Omega)$, то ця функція має похідні будь-якого порядку в класичному розумінні. Якщо $u$ є границею деякої фундаментальної послідовності, тобто $u = \lim_{n \to \infty} u_n$, то послідовність $\{D^i u_n\}$, як вже зазаначалося, фундаментальна по нормі $L_2(\Omega)$, оскільки простір $L_2(\Omega)$ є повним, то послідовність $\{D^i u_n\}$ збігається по нормі $L_2(\Omega)$ до деякого елементу $v^{(i)} \in L_2(\Omega)$.

\begin{definition}
    Саме функцію $v^{(i)}$ і можна вважати узагальненою похідною функції $u \in W_2^k(\Omega)$. Тобто $\im_{n \to \infty} D^i u_n = v^{(i)}$ в $L_2(\Omega)$. 
\end{definition}

\begin{proposition}
    Покажемо, що функція $v^{(i)}$ задовольняє інтегральній тотожності \eqref{eq:5.1.16}.
\end{proposition}

\begin{proof}
    Справді,
    \begin{equation*}
        \int_\Omega D^i u_n \psi(x) \diff x = (-1)^{|i|} \int_\Omega u_n D^i \psi(x) \diff x,
    \end{equation*}
    перейдемо до границі при $n \to \infty$ в лівій і правій частині:
    \begin{equation}
        \label{eq:5.1.17}
        \int_\Omega v^{(i)} \psi(x) \diff x = (-1)^{|i|} \int_\Omega u D^i \psi(x) \diff x.
    \end{equation}

    Остання рівність і показує, що $v^{(i)}$ можна вважати похідною $D^i u$. 
\end{proof}

\begin{remark}
    Можна показати що така похідна єдина.
\end{remark}

\begin{remark}
    Аналогічно простору$W_2^k(\Omega)$ можна побудувати простір $\overset{\circ}{W}_2^k(\Omega)$, який будується за допомогою процедури поповнення простора $C_0^\infty(\Omega)$.
\end{remark}
