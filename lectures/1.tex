% \documentclass[a4paper, 12pt]{article}
\usepackage[utf8]{inputenc}
\usepackage[english, ukrainian]{babel}

\usepackage{amsmath, amssymb}
\usepackage{multicol}
\usepackage{graphicx}
\usepackage{float}

\allowdisplaybreaks
\setlength\parindent{0pt}
\numberwithin{equation}{subsection}

\usepackage{hyperref}
\hypersetup{unicode=true,colorlinks=true,linktoc=all,linkcolor=red}

\numberwithin{equation}{subsection}

\renewcommand{\bf}[1]{\textbf{#1}}
\renewcommand{\it}[1]{\textit{#1}}
\newcommand{\bb}[1]{\mathbb{#1}}
\renewcommand{\cal}[1]{\mathcal{#1}}

\renewcommand{\epsilon}{\varepsilon}
\renewcommand{\phi}{\varphi}

\DeclareMathOperator{\diam}{diam}
\DeclareMathOperator{\rang}{rang}
\DeclareMathOperator{\const}{const}

\newenvironment{system}{%
  \begin{equation}%
    \left\{%
      \begin{aligned}%
}{%
      \end{aligned}%
    \right.%
  \end{equation}%
}
\newenvironment{system*}{%
  \begin{equation*}%
    \left\{%
      \begin{aligned}%
}{%
      \end{aligned}%
    \right.%
  \end{equation*}%
}

\makeatletter
\newcommand*{\relrelbarsep}{.386ex}
\newcommand*{\relrelbar}{%
  \mathrel{%
    \mathpalette\@relrelbar\relrelbarsep%
  }%
}
\newcommand*{\@relrelbar}[2]{%
  \raise#2\hbox to 0pt{$\m@th#1\relbar$\hss}%
  \lower#2\hbox{$\m@th#1\relbar$}%
}
\providecommand*{\rightrightarrowsfill@}{%
  \arrowfill@\relrelbar\relrelbar\rightrightarrows%
}
\providecommand*{\leftleftarrowsfill@}{%
  \arrowfill@\leftleftarrows\relrelbar\relrelbar%
}
\providecommand*{\xrightrightarrows}[2][]{%
  \ext@arrow 0359\rightrightarrowsfill@{#1}{#2}%
}
\providecommand*{\xleftleftarrows}[2][]{%
  \ext@arrow 3095\leftleftarrowsfill@{#1}{#2}%
}
\makeatother

\newcommand{\NN}{\mathbb{N}}
\newcommand{\ZZ}{\mathbb{Z}}
\newcommand{\QQ}{\mathbb{Q}}
\newcommand{\RR}{\mathbb{R}}
\newcommand{\CC}{\mathbb{C}}

\newcommand{\Max}{\displaystyle\max\limits}
\newcommand{\Sup}{\displaystyle\sup\limits}
\newcommand{\Sum}{\displaystyle\sum\limits}
\newcommand{\Int}{\displaystyle\int\limits}
\newcommand{\Iint}{\displaystyle\iint\limits}
\newcommand{\Lim}{\displaystyle\lim\limits}

\newcommand*\diff{\mathop{}\!\mathrm{d}}

\newcommand*\rfrac[2]{{}^{#1}\!/_{\!#2}}


% \title{{\Huge МАТЕМАТИЧНА ФІЗИКА}}
% \author{Скибицький Нікіта}
% \date{\today}

% \usepackage{amsthm}
\usepackage[dvipsnames]{xcolor}
\usepackage{thmtools}
\usepackage[framemethod=TikZ]{mdframed}

\theoremstyle{definition}
\mdfdefinestyle{mdbluebox}{%
	roundcorner = 10pt,
	linewidth=1pt,
	skipabove=12pt,
	innerbottommargin=9pt,
	skipbelow=2pt,
	nobreak=true,
	linecolor=blue,
	backgroundcolor=TealBlue!5,
}
\declaretheoremstyle[
	headfont=\sffamily\bfseries\color{MidnightBlue},
	mdframed={style=mdbluebox},
	headpunct={\\[3pt]},
	postheadspace={0pt}
]{thmbluebox}

\mdfdefinestyle{mdredbox}{%
	linewidth=0.5pt,
	skipabove=12pt,
	frametitleaboveskip=5pt,
	frametitlebelowskip=0pt,
	skipbelow=2pt,
	frametitlefont=\bfseries,
	innertopmargin=4pt,
	innerbottommargin=8pt,
	nobreak=true,
	linecolor=RawSienna,
	backgroundcolor=Salmon!5,
}
\declaretheoremstyle[
	headfont=\bfseries\color{RawSienna},
	mdframed={style=mdredbox},
	headpunct={\\[3pt]},
	postheadspace={0pt},
]{thmredbox}

\declaretheorem[style=thmbluebox,name=Теорема,numberwithin=subsubsection]{theorem}
\declaretheorem[style=thmbluebox,name=Лема,numberwithin=subsubsection]{lemma}
\declaretheorem[style=thmbluebox,name=Твердження,numberwithin=subsubsection]{proposition}
\declaretheorem[style=thmbluebox,name=Принцип,numberwithin=subsubsection]{th_principle}
\declaretheorem[style=thmbluebox,name=Закон,numberwithin=subsubsection]{law}
\declaretheorem[style=thmbluebox,name=Закон,numbered=no]{law*}
\declaretheorem[style=thmbluebox,name=Формула,numberwithin=subsubsection]{th_formula}
\declaretheorem[style=thmbluebox,name=Рівняння,numberwithin=subsubsection]{th_equation}
\declaretheorem[style=thmbluebox,name=Умова,numberwithin=subsubsection]{th_condition}
\declaretheorem[style=thmbluebox,name=Наслідок,numberwithin=subsubsection]{corollary}

\declaretheorem[style=thmredbox,name=Приклад,numberwithin=subsubsection]{example}
\declaretheorem[style=thmredbox,name=Приклади,sibling=example]{examples}

\declaretheorem[style=thmredbox,name=Властивість,numberwithin=subsubsection]{property}
\declaretheorem[style=thmredbox,name=Властивості,sibling=property]{properties}

\mdfdefinestyle{mdgreenbox}{%
	skipabove=8pt,
	linewidth=2pt,
	rightline=false,
	leftline=true,
	topline=false,
	bottomline=false,
	linecolor=ForestGreen,
	backgroundcolor=ForestGreen!5,
}
\declaretheoremstyle[
	headfont=\bfseries\sffamily\color{ForestGreen!70!black},
	bodyfont=\normalfont,
	spaceabove=2pt,
	spacebelow=1pt,
	mdframed={style=mdgreenbox},
	headpunct={ --- },
]{thmgreenbox}

\mdfdefinestyle{mdblackbox}{%
	skipabove=8pt,
	linewidth=3pt,
	rightline=false,
	leftline=true,
	topline=false,
	bottomline=false,
	linecolor=black,
	backgroundcolor=RedViolet!5!gray!5,
}
\declaretheoremstyle[
	headfont=\bfseries,
	bodyfont=\normalfont\small,
	spaceabove=0pt,
	spacebelow=0pt,
	mdframed={style=mdblackbox}
]{thmblackbox}

\declaretheorem[name=Вправа,numberwithin=subsubsection,style=thmblackbox]{exercise}
\declaretheorem[name=Зауваження,numberwithin=subsubsection,style=thmgreenbox]{remark}
\declaretheorem[name=Визначення,numberwithin=subsubsection,style=thmblackbox]{definition}

\newtheorem{problem}{Задача}[subsection]
\newtheorem{sproblem}[problem]{Задача}
\newtheorem{dproblem}[problem]{Задача}
\renewcommand{\thesproblem}{\theproblem$^{\star}$}
\renewcommand{\thedproblem}{\theproblem$^{\dagger}$}
\newcommand{\listhack}{$\empty$\vspace{-2em}} 

\theoremstyle{remark}
\newtheorem*{solution}{Розв'язок}


% \begin{document}

% \tableofcontents

\section{Вступ}

\subsection{Предмет і методи математичної фізики}

Сучасні технології дослідження реального світу доволі інтенсивно використовують методи математичного моделювання, зокрема ці методи широко використовуються тоді, коли дослідження реального (фізичного) об'єкту є неможливими, або надто дорогими. Вже традиційними стали моделювання властивостей таких фізичних об'єктів:
\begin{itemize}
	\item температурні поля і теплові потоки;
	\item електричні, магнітні та електромагнітні поля;
	\item концентрація речовини в розчинах, розплавах або сумішах;
	\item напруження і деформації в пружних твердих тілах;
	\item параметри рідини або газу, який рухається (обтікає) деяке тіло;
	\item перенос різних субстанцій потоками рідин або газу та інші.
\end{itemize}

Характерною особливістю усіх математичних моделей, що описують перелічені та багато інших процесів є те, що параметри, які представляють інтерес для дослідника є функціями точки простору $\bf{x} = (x_1, x_2, x_3)$ та часу $t$, а самі співвідношення з яких ці характеристики обчислюються є диференціальними рівняннями в частинних похідних зі спеціальними додатковими умовами (крайовими умовами), які дозволяють виділяти однозначний розв'язок. \medskip

Таким чином можна сказати, що основними об'єктами дослідження предмету математична фізика є крайові задачі для рівнянь в частинних похідних, які моделюють певні фізичні процеси. \medskip

Процес дослідження реального об'єкту фізичного світу можна представити за наступною схемою:
\begin{enumerate}
	\item Побудова математичної моделі реального процесу у вигляді диференціального рівняння або системи диференціальних рівнянь в частинних похідних, доповнення диференціального рівняння в частинних похідних граничними умовами.
	\item Дослідження властивостей сформульованої крайової задачі з точки зору її коректності. Коректність постановки задачі передбачає виконання наступних умов:
	\begin{itemize}
		\item Розв'язок крайової задачі існує;
		\item Розв'язок єдиний;
		\item Розв'язок неперервним чином залежить від вхідних даних.
	\end{itemize}
	\item Знаходження розв'язку крайової задачі: 
	\begin{itemize}
		\item точного для найбільш простих задач;
		\item або наближеного для переважної більшості задач.
	\end{itemize}
\end{enumerate}

Треба відмітити, що усі перелічені пункти дослідження окрім побудови наближених методів знаходження розв'язків відносяться до предмету дисципліни Математична фізика. \medskip

Для дослідження задач математичної фізики використовуються математичний апарат наступних розділів математики:
\begin{itemize}
	\item математичний аналіз;
	\item лінійна алгебра;
	\item диференціальні рівняння;
	\item теорія функцій комплексної змінної;
	\item функціональний аналіз;
\end{itemize}

При побудові математичних моделей використовуються знання з елементарної фізики. \medskip

Наведемо приклад доволі простої і в той же час цілком реальної математичної моделі розповсюдження тепла в стрижні.

\begin{example}[моделі розповсюдження тепла в стрижні]
	Нехай ми маємо однорідний стрижень з теплоізольованою боковою поверхнею і наступними фізичними параметрами:
	\begin{itemize}
		\item $\rho$ --- густина матеріалу;
		\item $S$ --- площа поперечного перерізу;
		\item $k$ --- коефіцієнт теплопровідності;
		\item $c$ --- коефіцієнт теплоємності;
		\item $L$ --- довжина стрижня.
	\end{itemize}

	Позначимо $u(x, t)$ --- температуру стрижня в точці $x$ в момент часу $t$, $u_0(x)$ --- температуру стрижня у точці $x$ в початковий момент часу $t = 0$. \medskip

	Припустимо, що на лівому кінці стрижня температура змінюється за заданим законом $\phi(t)$, а правий кінець стрижня теплоізольований.
\end{example}

В таких припущеннях математична модель може бути записана у вигляді наступної граничної задачі:

\begin{itemize}
	\item диференціальне рівняння:
	\begin{equation}
		c \rho \cdot \dfrac{\partial u(x, t)}{\partial t} = k \cdot \dfrac{\partial^2 u(x, t)}{\partial x^2}, \quad 0 < x < L, \quad t > 0
	\end{equation}
	\item граничні умови на кінцях стрижня:
	\begin{equation}
		u(0, u) = \phi(t), \quad  \dfrac{\partial u(L, t)}{\partial x} = 0
	\end{equation}
	\item початкова умова:
	\begin{equation}
		u(x, 0) = u_0(t)
	\end{equation}
\end{itemize}

\newpage

\section{Інтегральні рівняння}

\subsubsection{Основні поняття}

\begin{definition}
	\it{Інтегральні рівняння} --- рівняння, що містять невідому функцію під знаком інтегралу.
\end{definition}

Багато задач математичної фізики зводяться до лінійних інтегральних рівнянь наступних двох виглядів.

\begin{definition}[інтегрального рівняння Фредгольма II роду]
	\it{Інтегральним рівнянням Фредгольма ІІ роду} називається рівняння вигляду
	\begin{equation}
		\phi(x) = \lambda \Int_G K(x, y) \phi(y) \diff y + f(x).
	\end{equation}
\end{definition}

Тут $\lambda$ --- комплексний параметр, $\lambda \in \CC$ (відомий або невідомий), $G$ --- область інтегрування, $G \subseteq \RR^n$, $\overline G$ --- замкнена та обмежена.

\begin{definition}[інтегрального рівняння Фредгольма I роду]
	\it{Інтегральним рівнянням Фредгольма І роду} називається рівняння вигляду
	\begin{equation}
		\Int_G K(x, y) \phi(y) \diff y = f(x).
	\end{equation}
\end{definition}

\begin{definition}[ядра інтегрального рівняння]
	\it{Ядром} інтегральних рівнянь наведених вище називається функція $K(x, y) \in C\left(\overline G\times\overline G\right)$.
\end{definition}

\begin{definition}[вільного члена інтегрального рівняння]
	\it{Вільним \allowbreak членом} інтегральних рівнянь наведених вище називається функція $f(x) \in C\left(\overline G\right)$.
\end{definition}

\begin{definition}[однорідного рівняння Фредгольма II роду]
	Інтегра\-льне рівняння Фредгольма II роду при $f(x) \equiv 0$ називається \it{однорідним}:
	\begin{equation}
		\phi(x) = \lambda \Int_G K(x, y) \phi(y) \diff y.
	\end{equation}
\end{definition}

\begin{definition}[інтегрального оператора]
	Зрозуміло, що кожному ядру $K(x, y)$ відповідає \it{інтегральний оператор} $\bf{K}$ який визначається наступним чнном:
	\begin{equation}
		\bf{K}: \phi(x) \mapsto (\bf{K} \phi)(x) = \Int_G K(x, y) \phi(y) \diff y.
	\end{equation}
\end{definition}

\begin{remark}
    Будемо записувати інтегральні рівняння скорочено в операторній формі:
    \begin{align}
    	\phi &= \lambda \bf{K} \phi + f, \\
    	\bf{K} \phi &= f, \\
    	\phi &= \lambda \bf{K} \phi.
    \end{align}
\end{remark}

\begin{definition}[спряженого (союзного ядра)]
	\it{Спряженим (сюзним) ядром} називається функція
	\begin{equation}
		K^\star (x, y) = \overline{K}(y, x).
	\end{equation}
\end{definition}

\begin{definition}[спряженого (союзного) рівняння]
	Інтегральне рівняння
	\begin{equation}
		\psi(x) = \overline \lambda \Int_G K^\star (x, y) \psi(y) \diff y + g(x)
	\end{equation}
	називається \it{спряженим (союзним)} до відповідного інтегрального рівняння Фредгольма ІІ роду.
\end{definition}

\begin{remark}
    Операторна форма рівнянь останніх двох рівнянь:
    \begin{align}
    	\psi &= \overline \lambda \bf{K}^\star \psi + g, \\
    	\psi &= \overline \lambda \bf{K}^\star \psi.
    \end{align}
\end{remark}

\begin{definition}[характеристичних (власних) чисел ядра]
	Комплексні $\lambda$, при яких однорідне інтегральне рівняння Фредгольма ІІ роду має нетривіальні розв'язки, називаються \it{характеристичними (власними) числами} ядра $K(x, y)$.
\end{definition}

\begin{definition}[власних функцій ядра]
	Розв'язки, які відповідають власним числам, називаються \it{власними функціями} ядра.
\end{definition}

\begin{definition}[кратності характеристичного числа]
	Кількість лі\-ній\-но-незалежних власних функцій називається \it{кратністю характеристичного числа}.
\end{definition}

\subsection{Метод послідовних наближень}

\subsubsection{Метод послідовних наближень для неперервного ядра}

Нагадаємо кілька визначень:
\begin{definition}[норми у $C\left(\overline G\right)$]
	\it{Нормою} у банаховому просторі неперервних функцій $C\left(\overline G\right)$ називається
	\begin{equation}
		\|f\|_{C\left(\overline G\right)} = \Max_{x \in \overline G} |f(x)|.
	\end{equation}
\end{definition}

\begin{definition}[норми у $L_2(G)$]
	\it{Нормою} у гільбертовому просторі інтегровних з квадратом функцій $L_2(G)$ називається
	\begin{equation}
		\|f\|_{L_2(G)} = \left( \Int_G |f(x)|^2 \diff x \right)^{1/2}.
	\end{equation}
\end{definition}

\begin{definition}[скалярного добутку у $L_2(G)$]
	\it{Скалярноим добутком} у просторі $L_2(G)$ наизвається
	\begin{equation}
		(f, g)_{L_2(G)} = \Int_G f(x) \bar g(x) \diff x.
	\end{equation}
\end{definition}

\begin{lemma} 
	\label{lemma:2.1.4}
	Інтегральний оператор $\bf{K}$ з неперервним ядром $K(x, y)$ петворює множини функцій $C\left(\overline G\right) \xrightarrow{\bf{K}} C\left(\overline G\right)$, $L_2(G) \xrightarrow{\bf{K}} L_2(G)$, $L_2(G) \xrightarrow{\bf{K}} C\left(\overline G\right)$ обмежений та мають місце нерівності:
	\begin{align}
		\| \bf{K} \phi \|_{C(G)} &\le M V \| \phi \|_{C(G)}, \\
		\| \bf{K} \phi \|_{L_2(G)} &\le M V \| \phi \|_{L_2(G)}, \\
		\| \bf{K} \phi \|_{C(G)} &\le M \sqrt{V} \| \phi \|_{L_2(G)},
	\end{align}

	де
	\begin{equation}
		M = \Max_{(x, y) \in G \times G} |K(x, y)|, \quad V = \Int_G \diff y.
	\end{equation}
\end{lemma}

\begin{remark}
	Мається на увазі, що довільна функція $\phi$ з множини функцій $C\left(\overline G\right)$ під дією інтегрального оператора $\bf{K}$ переходить у функцію $\bf{K}\phi$ з множини функцій $C\left(\overline G\right)$, і так далі.
\end{remark}

\begin{proof}
	Нехай $\phi \in L_2(G)$. Тоді $\phi$ --- абсолютно інтегровна функція на $G$ і, оскільки ядро $K(x, y)$ неперервне на $G \times G$, функція $(\bf{K}\phi)(x)$ неперервна на $G$. Тому оператор $\bf{K}$ переводить $L_2(G)$ в $C\left(\overline G\right)$ і, з врахуванням нерівності Коші-Буняковського, обмежений. Доведемо нерівності:
	\begin{enumerate}
		\item $\| \bf{K} \phi \|_{C(G)} \le M V \| \phi \|_{C(G)}$:
		\begin{equation}
			\begin{aligned}
				\| \bf{K} \phi \|_{C\left(\overline G\right)} &= \Max_{x \in \overline G} \left| \Int_G K(x, y) \phi(y) \diff y \right| \le \\
				&\le \Max_{x \in \overline G} \Int_G \left( |K(x, y)| \cdot |\phi(y)| \right) \diff y \le \\
				&\le \Max_{x \in \overline G} \left( \Max_{y \in \overline G} |K(x, y)| \cdot \Max_{y \in \overline G} |\phi(y)| \cdot \Int_G \diff y \right) \le \\
				&\le \Max_{(x, y) \in \overline G \times \overline G} |K(x, y)| \cdot \Max_{y \in \overline G} |\phi(y)| \cdot \Int_G \diff y = \\
				&= M V \|\phi\|_{C\left(\overline G\right)}.
			\end{aligned}
		\end{equation}
		\item $\| \bf{K} \phi \|_{L_2(G)} \le M V \| \phi \|_{L_2(G)}$:
		\begin{equation}
			\begin{aligned}
				\left( \| \bf{K} \phi \|_{L_2(G)} \right)^2 &= \Int_G \left| \Int_G K(x, y) \phi(y) \diff y \right|^2 \diff x \le \\
				&\le \Int_G \left| \Max_{y \in \overline G} |K(x, y)| \cdot \Int_G \phi(y) \diff y \right|^2 \diff x \le \\
				&\le \left( \Max_{(x, y) \in \overline G \times \overline G} |K(x, y)| \right)^2 \cdot \left| \Int_G \phi(y) \diff y \right|^2 \cdot \Int_G \diff x \le \\
				&\le (M \| \phi\|_{L_2(G)} V)^2.
			\end{aligned}
		\end{equation}
		\item $\| \bf{K} \phi \|_{C(G)} \le M \sqrt{V} \| \phi \|_{L_2(G)}$:
		\begin{equation}
			\begin{aligned}
				\| \bf{K} \phi \|_{C\left(\overline G\right)} &= \Max_{x \in \overline G} |(\bf{K} \phi) (x)| = \\
				&= \Max_{x \in \overline G} \left| \Int_G K(x, y) \phi(y) \diff y \right| \le \\
				&\le \Max_{x \in \overline G} \sqrt{\Int_G |K(x, y)|^2 \diff y} \cdot \sqrt{\Int_G |\phi(y)|^2 \diff y} \le \\
				&\le M \sqrt{V} \|\phi\|_{L_2(G)}.
			\end{aligned}
		\end{equation}
	\end{enumerate}
\end{proof}

Розв'язок інтегрального рівняння другого роду записаного у операторній формі будемо шукати методом послідовних наближень, тобто запустимо наступний ітераційний процес:
\begin{equation}
	\phi_0 = f, \quad \phi_1 = \lambda \bf{K} \phi_0 + f, \quad \phi_2 = \lambda \bf{K} \phi_1 + f, \quad \ldots, \quad \phi_{n + 1} = \lambda \bf{K} \phi_n + f.
\end{equation}

Тоді можна записати
\begin{equation}
	\phi_{n + 1} = \Sum_{i = 0}^{n + 1} \lambda^i \bf{K}^i f,
\end{equation}
де
$\bf{K}^{i + 1} = \bf{K} (\bf{K}^i)$. \medskip

Хочеться за розв'язок взяти функцію
\begin{equation}
	\phi_\infty = \Lim_{n \to \infty} \phi_{n}.
\end{equation}

Це підводить нас до
\begin{definition}[ряду Неймана]
	\it{Рядом Неймана} оператора $\bf{K}$ називається
	\begin{equation}
		\Sum_{i = 0}^\infty \lambda^i \bf{K}^i f = \Lim_{n \to \infty} \phi_{n}.
	\end{equation}
\end{definition}

Дослідимо збіжність ряду Неймана:
\begin{equation}
	\begin{aligned}
		\left\| \Sum_{i = 0}^\infty \lambda^i \bf{K}^i f \right\|_{C\left(\overline G\right)} &\le \Sum_{i = 0}^\infty |\lambda^i| \cdot \| \bf{K}^i f \|_{C\left(\overline G\right)} \le \\
		&\le \Sum_{i = 0}^\infty |\lambda^i| \cdot (MV)^i \cdot \| f \|_{C\left(\overline G\right)} = \dfrac{\|f\|_{C\left(\overline G\right)}}{1 - |\lambda| \cdot MV}.
	\end{aligned}
\end{equation}

Справді,
\begin{equation}
	\| \bf{K} \phi\|_{C\left(\overline G\right)} \le MV \|\phi\|_{C\left(\overline G\right)},
\end{equation}
тому
\begin{equation}
	\|\bf{K}^2\phi\|_{C\left(\overline G\right)} \le (MV)^2 \|\phi\|_{C\left(\overline G\right)}
\end{equation} 
і, взагалі кажучи,
\begin{equation}
	\|\bf{K}^i\phi\|_{C\left(\overline G\right)} \le (MV)^i \|\phi\|_{C\left(\overline G\right)}.
\end{equation}

Отже ми довели наступне
\begin{proposition}[про умову збіжності методу послідовних наближень]
	Ряд Неймана збігається рівномірно при 	
	\begin{equation}
		|\lambda| < \dfrac{1}{MV}.
	\end{equation} 
\end{proposition}

\begin{lemma}[про єдиність розв'язку за умови збіжності методу послідовних наближень]
	При виконанні умови збіжності методу послідовних наближень інтегральне рівняння ІІ роду має єдиний розв'язок.
\end{lemma}

\begin{proof}
	Дійсно припустимо, що їх два:
	\begin{equation}
		\begin{aligned}
			\phi^{(1)} &= \lambda \bf{K} \phi^{(1)} + f, \\
			\phi^{(2)} &= \lambda \bf{K} \phi^{(2)} + f.
		\end{aligned}
	\end{equation}

	Тоді можемо розглянути їхню різницю
	\begin{equation}
		\phi^{(0)} = \phi^{(1)} - \phi^{(2)}.
	\end{equation}

	Вона буде задовольняти лоднорідному рівнянню:
	\begin{equation}
		\phi^{(0)} = \lambda \bf{K} \phi^{(0)}.
	\end{equation}

	Обчислимо норму Чебишева:
	\begin{equation}
		|\lambda| \cdot \|\bf{K} \phi^{(0)}\|_{C\left(\overline G\right)} = \| \phi^{(0)} \|_{C\left(\overline G\right)}.
	\end{equation}

	Застосовуючи нерівність з леми до лівої частини отримуємо
	\begin{equation}
		\| \phi^{(0)} \|_{C\left(\overline G\right)} \le |\lambda| \cdot MV \cdot \|\phi^{(0)}\|_{C\left(\overline G\right)}.
	\end{equation}

	Звідси безпосередньо випливає
	\begin{equation}
		(1 - |\lambda| \cdot MV) \cdot \|\phi^{(0)}\|_{C\left(\overline G\right)} \le 0.
	\end{equation}

	Звідси маємо, що $\|\phi^{(0)}\|_{C\left(\overline G\right)} = 0$.
\end{proof}

Таким чином доведена
\begin{theorem}[про існування розв'язку інтегрального рівняння Фредгольма з неперервним
ядром для малих значень параметру]
	Будь-яке інтегральне рівняння Фредгольма другого роду з неперервним ядром $K(x, y)$ при умові $|\lambda| < 1 / MV$ має єдиний розв'язок $\phi$ в класі неперервних функцій $C\left(\overline G\right)$ для будь-якого неперервного вільного члена $f$. Цей роз\-в'я\-зок може бути знайдений у вигляді ряду Неймана.
\end{theorem}

\allowbreak

\subsubsection{Повторні ядра}

\begin{proposition}[про перенос інтегрального оператору через кому у скалярному добутку]
	$\forall f, g \in L_2(G)$ має місце рівність 
	\begin{equation}
		(\bf{K}f,g)_{L_2(G)} = (f, \bf{K}^\star g)_{L_2(G)}.
	\end{equation}
\end{proposition}

\begin{proof}
	Якщо $f, g \in L_2(G)$ то, за лемою \ref{lemma:2.1.4}, маємо $\bf{K}f, \bf{K}^\star g \in L_2(G)$, і тому
	\begin{equation}
		\begin{aligned}
			(\bf{K}f, g) &= \Int_G (\bf{K}f)(\overline g(x)) \diff x = \\
			&= \Int_G \left( \Int_G K(x, y) f(y) \diff y\right) \overline{g}(x) \diff x = \\
			&= \Int_G f(y) \left( \Int_G K(x, y) \overline g(x) \diff x\right) \diff y = \\
			&= \Int_G f(y) \cdot (\bf{K}^\star  g)(y) \diff y = \\
			&= (f, \bf{K}^\star g).
		\end{aligned}
	\end{equation}
\end{proof}

\begin{lemma}[про композицію інтегральних операторів]
	Якщо $\bf{K}_1$, $\bf{K}_2$ --- інтегральні оператори що мають неперервні ядра $K_1(x, y)$ і $K_2(x, y)$ відповідно, то оператор $\bf{K}_3 = \bf{K}_2 \bf{K}_1$ також інтегральний оператор з неперервним ядром
	\begin{equation}
		K_3(x, z) = \Int_G K_2(x, y) K_1(y, z) \diff y.
	\end{equation}
\end{lemma}

\begin{remark}
	При цьому справедлива формула: $(\bf{K}_2 \bf{K}_1)^\star = \bf{K}_1^\star \bf{K}_2^\star$.
\end{remark}

\begin{proof}
	Нехай $K_1(x, y)$, $K_2(x, y)$ --- ядра інтегральних операторів $\bf{K}_1$, $\bf{K}_2$. Розглянемо $\bf{K}_3 = \bf{K}_2 \bf{K}_1$:
	\begin{equation}
		\begin{aligned}
			(\bf{K}_3 f)(x) &= (\bf{K}_2\bf{K}_1f)(x) = \\
			&= \Int_G K_2(x, y) \left( \Int_G K_1(y, z) f(z) \diff z \right) \diff y = \\
			&= \Int_G \left( \Int_G K_2(x, y) K_1(y, z) \diff y\right) f(z) \diff z = \\
			&= \Int_G K_3(x, z) f(z) \diff z.
		\end{aligned}
	\end{equation}
	
	Тобто
	\begin{equation}
		\Int_G K_2(x, y) K_1(y, z) \diff y
	\end{equation}
	--- ядро оператора $\bf{K}_2\bf{K}_1$. \medskip

	Згідно правила переносу інтегрального оператора через кому у скалярному добутку для всіх $f, g \in L_2(G)$ отримуємо $(f, \bf{K}_3^\star g - \bf{K}_1^\star  \bf{K}_2^\star  g) = 0$, звідки випливає, що $\bf{K}_3^\star  = \bf{K}_1^\star  \bf{K}_2^\star $.
\end{proof}

Із доведеної леми випливає, що оператори $\bf{K}^n = \bf{K} (\bf{K}^{n - 1}) = (\bf{K}^{n - 1})\bf{K}$ --- інтегральні та їх ядра $K_{(n)}(x, y)$ --- неперервні та задовольняють рекурентним співвідношенням:
\begin{equation}
	K_{(1)}(x, y) = K(x, y), \quad \ldots, \quad K_{(n)}(x, y) = \Int_G K(x, \xi) K_{(n - 1)}(\xi, y) \diff \xi
\end{equation}

\begin{definition}[повторних (ітерованих) ядер]
	При цьому інтегра\-льні ядра $K_{(n)}(x, y)$ називаються \it{повторними (ітерованими)}.
\end{definition}

\begin{remark}
    Операторна форма:
    \begin{equation}
        \bf{K}f = \Int_G K(x, y) f(y) \diff y, \quad \ldots, \quad \bf{K}^n f = \Int_G K_{(n)}(x, y)f(y) \diff y.
    \end{equation}
\end{remark}

\subsubsection{Резольвента інтегрального оператора}

Пригадаємо представлення розв'язку інтегрального рівняння II роду у вигляді ряду Неймана. Виконаємо перетворення
\begin{equation}
	\begin{aligned}
		\phi(x) &= f(x) + \lambda \Sum_{i = 1}^\infty \lambda^{i - 1} (\bf{K}^i f) x = \\
		&= f(x) + \Sum_{i = 1}^\infty \lambda^{i - 1} K_{(i)} (x, y) f(y) \diff y = \\
		&= f(x) + \lambda \Int_G \left( \Sum_{i = 1}^\infty \lambda^{i - 1} K_{(i)} (x, y) \right) f(y) \diff y = \\
		&= f(x) + \lambda \Int_G \mathcal{R}(x, y, \lambda) f(y) \diff y,
	\end{aligned}
\end{equation}
при $|\lambda| < 1 / MV$.

\begin{definition}[резольвенти інтегрального оператора]
	Функція
	\begin{equation}
		\cal{R}(x, y, \lambda) = \Sum_{i = 1}^\infty \lambda^{i - 1} K_{(i)} (x, y)
	\end{equation}
	називається \it{резольвентою} інтегрального оператора $K(x, y)$.
\end{definition}

\begin{remark}
    Операторна форма запису розв'язку рівняння Фредгольма через резольвенту ядра має вигляд:
    \begin{equation}
    	\phi = f + \lambda \bf{R} f.
    \end{equation}
\end{remark}

\begin{proposition}
	\label{proposition:2.1.17}
	Мають місце операторні рівності:
	\begin{align}
		\phi &= (E + \lambda \bf{R})f, \\
		(E - \lambda \bf{K})\phi &= f, \\
		\phi &= (E - \lambda \bf{K})^{-1}f.
	\end{align}
\end{proposition}

\begin{exercise}
	Доведіть попереднє твердження.
\end{exercise}

Таким чином маємо
\begin{equation}
	E + \lambda \bf{R} = (E - \lambda \bf{K})^{-1}, \quad |\lambda| < \dfrac{1}{MV}.
\end{equation}

Зважуючи на формулу розв'язку рівняння Фредгольма через резольвенту, має місце
\begin{theorem}[про існування розв'язку інтегрального рівняння Фредгольма з неперервним
ядром для малих значенням параметру]
	Будь-яке інтегральне рівняння Фредгольма другого роду з неперервним ядром $K(x, y)$ при умові $|\lambda| < 1 / MV$ має єдиний розв'язок $\phi$ в класі неперервних функцій $C\left(\overline G\right)$ для будь-якого неперервного вільного члена $f$. Цей розв'язок може бути знайдений у вигляді $f + \lambda \bf{R} f$ за допомогою резольвенти $\bf{R}$.
\end{theorem}

\newpage

\begin{example}
	Методом послідовних наближень знайти розв'язок інтегрального рівняння \begin{equation*}\phi(x) = x + \lambda \Int_0^1 (xt)^2 \phi(t) \diff t.\end{equation*}
\end{example}

\begin{solution}
	Перш за все зауважимо, що $M = 1$ і $V = 1$. \medskip

	Побудуємо повторні ядра 
	\begin{equation*}
		\begin{aligned} 
			K_{(1)}(x, t) &= x^2t^2, \\
			K_2(x, t) &= \Int_0^1 x^2 z^4 t^2 \diff z = \dfrac{x^2t^2}{5}, \\ 
			K_{(p)}(x, t) &= \dfrac{1}{5^{p - 2}} \Int_0^1 x^2 z^4 t^2 \diff z = \dfrac{x^2t^2}{5^{p - 1}}.
		\end{aligned}
	\end{equation*}
		
	Резольвента має вигляд 
	\begin{equation*}
		\cal{R}(x, t, \lambda) = x^2 t^2 \left(1 + \frac{\lambda}{5} + \frac{\lambda^2}{5^2} + \ldots + \frac{\lambda^p}{5^p} + \ldots \right) = \frac{5x^2t^2}{5 - \lambda}, \quad |\lambda| < 5.
	\end{equation*}

	Розв'язок інтегрального рівняння має вигляд: 
	\begin{equation*}
		\phi(x) + x + \Int_0^1 \dfrac{5x^2t^3}{5 - \lambda} \diff t = x + \dfrac{5x^2}{4(5 - \lambda)}.
	\end{equation*}
\end{solution}

% \end{document}