\documentclass[a4paper, 12pt]{article}
\usepackage[utf8]{inputenc}
\usepackage[english, ukrainian]{babel}

\usepackage{amsmath, amssymb}
\usepackage{multicol}
\usepackage{graphicx}
\usepackage{float}

\allowdisplaybreaks
\setlength\parindent{0pt}
\numberwithin{equation}{subsection}

\usepackage{hyperref}
\hypersetup{unicode=true,colorlinks=true,linktoc=all,linkcolor=red}

\numberwithin{equation}{subsection}

\renewcommand{\bf}[1]{\textbf{#1}}
\renewcommand{\it}[1]{\textit{#1}}
\newcommand{\bb}[1]{\mathbb{#1}}
\renewcommand{\cal}[1]{\mathcal{#1}}

\renewcommand{\epsilon}{\varepsilon}
\renewcommand{\phi}{\varphi}

\DeclareMathOperator{\diam}{diam}
\DeclareMathOperator{\rang}{rang}
\DeclareMathOperator{\const}{const}

\newenvironment{system}{%
  \begin{equation}%
    \left\{%
      \begin{aligned}%
}{%
      \end{aligned}%
    \right.%
  \end{equation}%
}
\newenvironment{system*}{%
  \begin{equation*}%
    \left\{%
      \begin{aligned}%
}{%
      \end{aligned}%
    \right.%
  \end{equation*}%
}

\makeatletter
\newcommand*{\relrelbarsep}{.386ex}
\newcommand*{\relrelbar}{%
  \mathrel{%
    \mathpalette\@relrelbar\relrelbarsep%
  }%
}
\newcommand*{\@relrelbar}[2]{%
  \raise#2\hbox to 0pt{$\m@th#1\relbar$\hss}%
  \lower#2\hbox{$\m@th#1\relbar$}%
}
\providecommand*{\rightrightarrowsfill@}{%
  \arrowfill@\relrelbar\relrelbar\rightrightarrows%
}
\providecommand*{\leftleftarrowsfill@}{%
  \arrowfill@\leftleftarrows\relrelbar\relrelbar%
}
\providecommand*{\xrightrightarrows}[2][]{%
  \ext@arrow 0359\rightrightarrowsfill@{#1}{#2}%
}
\providecommand*{\xleftleftarrows}[2][]{%
  \ext@arrow 3095\leftleftarrowsfill@{#1}{#2}%
}
\makeatother

\newcommand{\NN}{\mathbb{N}}
\newcommand{\ZZ}{\mathbb{Z}}
\newcommand{\QQ}{\mathbb{Q}}
\newcommand{\RR}{\mathbb{R}}
\newcommand{\CC}{\mathbb{C}}

\newcommand{\Max}{\displaystyle\max\limits}
\newcommand{\Sup}{\displaystyle\sup\limits}
\newcommand{\Sum}{\displaystyle\sum\limits}
\newcommand{\Int}{\displaystyle\int\limits}
\newcommand{\Iint}{\displaystyle\iint\limits}
\newcommand{\Lim}{\displaystyle\lim\limits}

\newcommand*\diff{\mathop{}\!\mathrm{d}}

\newcommand*\rfrac[2]{{}^{#1}\!/_{\!#2}}


\title{{\Huge МАТЕМАТИЧНА ФІЗИКА}}
\author{Скибицький Нікіта}
\date{\today}
 
\usepackage{amsthm}
\usepackage[dvipsnames]{xcolor}
\usepackage{thmtools}
\usepackage[framemethod=TikZ]{mdframed}

\theoremstyle{definition}
\mdfdefinestyle{mdbluebox}{%
	roundcorner = 10pt,
	linewidth=1pt,
	skipabove=12pt,
	innerbottommargin=9pt,
	skipbelow=2pt,
	nobreak=true,
	linecolor=blue,
	backgroundcolor=TealBlue!5,
}
\declaretheoremstyle[
	headfont=\sffamily\bfseries\color{MidnightBlue},
	mdframed={style=mdbluebox},
	headpunct={\\[3pt]},
	postheadspace={0pt}
]{thmbluebox}

\mdfdefinestyle{mdredbox}{%
	linewidth=0.5pt,
	skipabove=12pt,
	frametitleaboveskip=5pt,
	frametitlebelowskip=0pt,
	skipbelow=2pt,
	frametitlefont=\bfseries,
	innertopmargin=4pt,
	innerbottommargin=8pt,
	nobreak=true,
	linecolor=RawSienna,
	backgroundcolor=Salmon!5,
}
\declaretheoremstyle[
	headfont=\bfseries\color{RawSienna},
	mdframed={style=mdredbox},
	headpunct={\\[3pt]},
	postheadspace={0pt},
]{thmredbox}

\declaretheorem[style=thmbluebox,name=Теорема,numberwithin=subsubsection]{theorem}
\declaretheorem[style=thmbluebox,name=Лема,numberwithin=subsubsection]{lemma}
\declaretheorem[style=thmbluebox,name=Твердження,numberwithin=subsubsection]{proposition}
\declaretheorem[style=thmbluebox,name=Принцип,numberwithin=subsubsection]{th_principle}
\declaretheorem[style=thmbluebox,name=Закон,numberwithin=subsubsection]{law}
\declaretheorem[style=thmbluebox,name=Закон,numbered=no]{law*}
\declaretheorem[style=thmbluebox,name=Формула,numberwithin=subsubsection]{th_formula}
\declaretheorem[style=thmbluebox,name=Рівняння,numberwithin=subsubsection]{th_equation}
\declaretheorem[style=thmbluebox,name=Умова,numberwithin=subsubsection]{th_condition}
\declaretheorem[style=thmbluebox,name=Наслідок,numberwithin=subsubsection]{corollary}

\declaretheorem[style=thmredbox,name=Приклад,numberwithin=subsubsection]{example}
\declaretheorem[style=thmredbox,name=Приклади,sibling=example]{examples}

\declaretheorem[style=thmredbox,name=Властивість,numberwithin=subsubsection]{property}
\declaretheorem[style=thmredbox,name=Властивості,sibling=property]{properties}

\mdfdefinestyle{mdgreenbox}{%
	skipabove=8pt,
	linewidth=2pt,
	rightline=false,
	leftline=true,
	topline=false,
	bottomline=false,
	linecolor=ForestGreen,
	backgroundcolor=ForestGreen!5,
}
\declaretheoremstyle[
	headfont=\bfseries\sffamily\color{ForestGreen!70!black},
	bodyfont=\normalfont,
	spaceabove=2pt,
	spacebelow=1pt,
	mdframed={style=mdgreenbox},
	headpunct={ --- },
]{thmgreenbox}

\mdfdefinestyle{mdblackbox}{%
	skipabove=8pt,
	linewidth=3pt,
	rightline=false,
	leftline=true,
	topline=false,
	bottomline=false,
	linecolor=black,
	backgroundcolor=RedViolet!5!gray!5,
}
\declaretheoremstyle[
	headfont=\bfseries,
	bodyfont=\normalfont\small,
	spaceabove=0pt,
	spacebelow=0pt,
	mdframed={style=mdblackbox}
]{thmblackbox}

\declaretheorem[name=Вправа,numberwithin=subsubsection,style=thmblackbox]{exercise}
\declaretheorem[name=Зауваження,numberwithin=subsubsection,style=thmgreenbox]{remark}
\declaretheorem[name=Визначення,numberwithin=subsubsection,style=thmblackbox]{definition}

\newtheorem{problem}{Задача}[subsection]
\newtheorem{sproblem}[problem]{Задача}
\newtheorem{dproblem}[problem]{Задача}
\renewcommand{\thesproblem}{\theproblem$^{\star}$}
\renewcommand{\thedproblem}{\theproblem$^{\dagger}$}
\newcommand{\listhack}{$\empty$\vspace{-2em}} 

\theoremstyle{remark}
\newtheorem*{solution}{Розв'язок}


\begin{document}

\tableofcontents

\setcounter{section}{4}
\setcounter{subsection}{8}

\subsection{Дослідження існування розв'язків основних граничних задач рівняння Лапласа та Гельмгольца}

Будемо розглядати замкнену Ляпуновську поверхню, яка обмежує дві області $\Omega$ та $\Omega'$ --- зовнішню по відношенню до області $\Omega$. Будемо розглядати основні граничні задачі рівняння Лапласа:
\begin{longtable}{p{.45\textwidth} p{.45\textwidth} p{.45\textwidth}}
	\parbox{.45\textwidth}{\begin{equation}\label{eq:4.9.1}\left\{\begin{aligned}
		& \Delta u(x) = 0, \quad x \in \Omega, \\
		& \left. u \right|_{x \in S} = f(x).
	\end{aligned}\right.\end{equation}} & \parbox{.45\textwidth}{\begin{equation}\label{eq:4.9.1'}\left\{\begin{aligned}
		& \Delta u(x) = 0, \quad x \in \Omega', \\
		& \left. u \right|_{x \in S} = f(x), \\
		& u(x) = O(1/|x|), \quad |x| \to \infty.
	\end{aligned}\right.\end{equation}} \\
	\parbox{.45\textwidth}{\begin{equation}\label{eq:4.9.2}\left\{\begin{aligned}
		& \Delta u(x) = 0, \quad x \in \Omega, \\
		& \left. \frac{\partial u}{\partial \vec n} \right|_{x \in S} = f(x).
	\end{aligned}\right.\end{equation}} & \parbox{.45\textwidth}{\begin{equation}\label{eq:4.9.2'}\left\{\begin{aligned}
		& \Delta u(x) = 0, \quad x \in \Omega', \\
		& \left. \frac{\partial u}{\partial \vec n} \right|_{x \in S} = f(x), \\
		& u(x) = O(1/|x|), \quad |x| \to \infty.
	\end{aligned}\right.\end{equation}}
\end{longtable}

\begin{definition}
	Будемо називати граничні задачі \eqref{eq:4.9.1} та \eqref{eq:4.9.1'} \textit{внутрішньою} та \textit{зовнішньою граничними задачами Діріхле} і позначати $D_{\text{i}}$ та $D_{\text{e}}$ відповідно. Граничні задачі \eqref{eq:4.9.2} та \eqref{eq:4.9.2'} будемо називати \textit{внутрішньою} та \textit{зовнішньою граничними задачами Неймана} та позначати $N_{\text{i}}$ та $N_{\text{e}}$ відповідно.  	
\end{definition}

Функцію $f(x)$ будемо вважати неперервною на поверхні Ляпунова $S$. \medskip

Враховуючи властивості потенціалів, будемо шукати розв'язки граничних задач Діріхле у вигляді потенціалу подвійного шару, а задач Неймана у вигляді потенціалу простого шару з невідомими щільностями:
\begin{align}
	\label{eq:4.9.3}
	u(x) &= \Iint_S \sigma(y) \frac{\partial}{\partial \vec n_y} \frac{1}{4 \pi |x - y|} \diff S_y \qquad \text{для }D_{\text{i}}\text{ та }D_{\text{e}}, \\
	\label{eq:4.9.4}
	u(x) &= \Iint_S \mu(y) \frac{1}{4 \pi |x - y|} \diff S_y \qquad \text{для }N_{\text{i}}\text{ та }N_{\text{e}}.
\end{align}

В області $\Omega$ та $\Omega'$ потенціали простого шару та подвійного шару задовольняють рівняння Лапласа. \medskip

Для знаходження розв'язків відповідних граничних задач необхідно задовільними граничним умовам на поверхні $S$ та для зовнішніх задач умовам регулярності на нескінченості. \medskip

Запишемо інтегральні співвідношення, які дозволять задовольнити граничні умови та визначити невідомі щільності потенціалів. \medskip

Для запису інтегральних співвідношень використаємо теореми про граничні значення потенціалу подвійного шару та граничні значення нормальної похідної  потенціалу простого шару.
\begin{align}
	\label{eq:4.9.5}
	f(x) &= \Iint_S \sigma(y) \frac{\partial}{\partial \vec n_y} \frac{1}{4 \pi |x - y|} \diff S_y - \frac{\sigma(x)}{2}, \quad x \in S, \quad D_{\text{i}} \\
	\label{eq:4.9.6}
	f(x) &= \Iint_S \sigma(y) \frac{\partial}{\partial \vec n_y} \frac{1}{4 \pi |x - y|} \diff S_y + \frac{\sigma(x)}{2}, \quad x \in S, \quad D_{\text{e}} \\
	\label{eq:4.9.7}
	f(x) &= \Iint_S \mu(y) \frac{\partial}{\partial \vec n_x} \frac{1}{4 \pi |x - y|} \diff S_y + \frac{\mu(x)}{2}, \quad x \in S, \quad N_{\text{i}} \\
	\label{eq:4.9.8}
	f(x) &= \Iint_S \mu(y) \frac{\partial}{\partial \vec n_x} \frac{1}{4 \pi |x - y|} \diff S_y - \frac{\mu(x)}{2}, \quad x \in S, \quad N_{\text{e}}.
\end{align}

Неважко бачити, що рівняння \eqref{eq:4.9.5}--\eqref{eq:4.9.8} є інтегральними рівняннями Фредгольма другого роду і розпадаються на дві пари союзних (спряжених): $D_{\text{i}}$, $N_{\text{e}}$ та $D_{\text{e}}$, $N_{\text{i}}$. З теорем Фредгольма відомо, що існування і єдність розв'язків інтегрального рівняння і спряженого до нього виконується одночасно. Тому ми будемо досліджувати одночасно пари рівнянь $D_{\text{i}}$, $N_{\text{e}}$ та $D_{\text{e}}$, $N_{\text{i}}$. \medskip

Ядро інтегрального рівняння \eqref{eq:4.9.6} має вигляд
\begin{equation}
	\label{eq:4.9.9}
	K(x, y) = \frac{\partial}{\partial \vec n_y} \frac{1}{4 \pi |x - y|} = - \frac{\cos(y - x, \vec n_y)}{4 \pi |x - y|^2},
\end{equation}
відповідно спряженого до нього рівняння \eqref{eq:4.9.8} має ядро:
\begin{equation}
	\label{eq:4.9.10}
	K^\star(x, y) = \frac{\partial}{\partial \vec n_x} \frac{1}{4 \pi |x - y|} = - \frac{\cos(x - y, \vec n_x)}{4 \pi |x - y|^2},
\end{equation}

Враховуючи властивості поверхні Ляпунова і оцінки $\cos(y - x, \vec n_y) < a_1 |x - y|^\alpha$, можна встановити, що ядра $K(x, y), K^\star(x, y)$ є полярними і для таких інтегральних рівнянь можна застосовувати теореми Фредгольма.

\subsubsection{Дослідження внутрішньої задачі Діріхле та зовнішньої задачі Неймана}

Покажемо, що інтегральні рівняння Фредгольма \eqref{eq:4.9.5} з полярним ядром \eqref{eq:4.9.9} та спряжене до нього рівняння \eqref{eq:4.9.8} мають єдиний розв'язок для будь-якого неперервного вільного члена. \medskip

Розглянемо однорідне рівняння для рівняння \eqref{eq:4.9.8} 
\begin{equation}
	\label{eq:4.9.8'}
	\Iint_S \mu(y) \frac{\partial}{\partial \vec n_x} \frac{1}{4 \pi |x - y|} \diff S_y - \frac{\mu(x)}{2} = 0, \quad x \in S.
\end{equation}

Нехай $\mu_0(x) \in L_2(S)$ ---  розв'язок однорідного рівняння \eqref{eq:4.9.8'}. Зрозуміло, що в цьому випадку функція $\mu_0(x) \in C(S)$ і задовольняє рівняння
\begin{equation}
	\label{eq:4.9.8''}
	\Iint_S \mu_0(y) \frac{\partial}{\partial \vec n_x} \frac{1}{4 \pi |x - y|} \diff S_y - \frac{\mu_0(x)}{2} = 0, \quad x \in S.
\end{equation}

Побудуємо потенціал простого шару з щільністю $\mu_0$ і позначимо його
\begin{equation}
	\label{eq:4.9.11}	
	V_0(x) = \Iint_S \mu_0(y) \frac{1}{4 \pi |x - y|} \diff S_y.
\end{equation}

Потенціал \eqref{eq:4.9.11} має правильну нормальну похідну при підході до поверхні $S$ ззовні, а рівняння \eqref{eq:4.9.8''} означає, що
\begin{equation}
	\label{eq:4.9.12}
	\frac{\partial V_0(x)}{\partial \vec n_{\text{e}}} \equiv 0, x \in S.
\end{equation}

Зрозуміло, що для $V_0$ задовольняє граничній задачі
\begin{equation}
	\label{eq:4.9.13}
	\left\{
		\begin{aligned}
			& \Delta V_0(x) = 0, \quad x \in \Omega', \\
			& \left. \frac{\partial V_0(x)}{\partial \vec n} \right|_{x \in S} = 0,\\
			& V_0(x) = O(1/|x|), \quad |x| \to \infty
		\end{aligned}
	\right.
\end{equation}
яка має лише тривіальний розв'язок, тобто $V_0(x) \equiv 0$, $x \in \Omega'$. Оскільки потенціал простого шару є неперервною функцією в усьому евклідовому просторі, то $V_0(x) \equiv 0$, $x \in S$, таким чином $V_0(x)$ задовольняє граничній задачі Діріхле
\begin{equation}
	\left\{
		\begin{aligned}
			& \Delta V_0(x) = 0, \quad x \in \Omega, \\
			& \left .V_0(x) \right|_{x \in S} = 0,
		\end{aligned}
	\right.
\end{equation}
яка має лише тривіальний розв'язок $V_0(x) \equiv 0$, $x \in \Omega$. Таким чином маємо, що $V_0(x) \equiv 0$, $x \in \RR^3$, а це означає, враховуючи \eqref{eq:4.9.11}, що $\mu_0(x) \equiv 0$, $x \in S$. Таким чином однорідне рівняння \eqref{eq:4.9.8'} має лише тривіальний розв'язок, а відповідне неоднорідне рівняння \eqref{eq:4.9.8} і спряжене до нього неоднорідне рівняння \eqref{eq:4.9.5} має єдиний розв'язок для будь-якого неперервного вільного члену $f(x)$. З наведених міркувань має місце 

\begin{theorem}[про існування розв'язку внутрішньої задачі Діріхле та зовнішньої задачі Неймана]
	Нехай $S$ --- замкнена поверхня Ляпунова, тоді внутрішня задача Діріхле \eqref{eq:4.9.1} та зовнішня задача Неймана \eqref{eq:4.9.2'} мають єдині розв'язки для будь-яких неперервних функцій $f(x)$ і розв'язки цих задач можна представити у вигляді потенціалу подвійного шару \eqref{eq:4.9.3} та простого шару \eqref{eq:4.9.4} відповідно.
\end{theorem}

\subsubsection{Дослідження зовнішньої задачі Діріхле та внутрішньої задачі Неймана}

Розглянемо однорідне рівняння, яке відповідає рівнянню Фредгольма \eqref{eq:4.9.6}
\begin{equation}
	\label{eq:4.9.6'}
	\Iint_S \sigma_0(y) \frac{\partial}{\partial \vec n_y} \frac{1}{4 \pi |x - y|} \diff S_y + \frac{\sigma_0(x)}{2} = 0, \quad x \in S, \quad D_{\text{e}}.
\end{equation}

Згідно до властивостей інтегралу Гауса $W_0(x)$, легко перевірити, що, рівняння \eqref{eq:4.9.6'} має нетривіальний розв'язок $\sigma_0(x) \equiv 1$, можна показати, що інших нетривіальних розв'язків не існує. Це означає, що розв'язок спряженого неоднорідного рівняння \eqref{eq:4.9.7} існує тоді і лише тоді, коли його вільний член $f$ ортогональний розв'язкам спряженого однорідного рівняння, тобто функції $\sigma_0(x) \equiv 1$. Умову ортогональності можна записати у вигляді 
\begin{equation}
	\label{eq:4.9.14}
	\Iint_S f(x) \diff S = 0.
\end{equation}

Таким чином має місце

\begin{theorem}[про існування розв'язку внутрішньої задачі Неймана]
	Нехай $S$ --- замкнута поверхня Ляпунова, а $f \in C(S)$, тоді внутрішня задача Неймана \eqref{eq:4.9.2} має розв'язок тоді і лише тоді, коли виконана умова ортогональності \eqref{eq:4.9.14}, а сам розв'язок можна представити у вигляді потенціалу простого шару \eqref{eq:4.9.4}.
\end{theorem}

Розглянемо випадок зовнішньої задачі Діріхле. Згідно до того, що однорідне рівняння \eqref{eq:4.9.6'} має один нетривіальний розв'язок, то однорідне спряжене до нього рівняння
\begin{equation}
	\label{eq:4.9.7'}
	\Iint_S \mu_1(y) \frac{\partial}{\partial \vec n_x} \frac{1}{4 \pi |x - y|} \diff S_y + \frac{\mu_1(x)}{2}, \quad x \in S, \quad N_{\text{inner}}
\end{equation}
теж має один нетривіальний розв'язок $\mu_1(x)$ --- власну функцію характеристичного числа $\lambda = -2$. Згідно до третьої теореми Фредгольма, неоднорідне рівняння \eqref{eq:4.9.6} має розв'язок тоді і лише тоді, коли виконується умова ортогональності
\begin{equation}
	\Iint_S f(x) \mu_1(x) \diff S = 0
\end{equation}
і при цьому розв'язок інтегрального рівняння \eqref{eq:4.9.6} неєдиний. \medskip

В той же час відомо, що зовнішня задача Діріхле має єдиний розв'язок (друга теорема єдиності гармонічних функцій). Отже маємо протиріччя, яке викликане тим, що потенціал подвійного шару
\begin{equation}
	u(x) = \Iint_S \sigma(y) \frac{\partial}{\partial \vec n_y} \frac{1}{4 \pi |x - y|} \diff S_y
\end{equation}
у вигляді якого ми шукаємо розв'язок зовнішньої задачі Діріхле не є гармонічною функцією регулярною на нескінченості. Дійсно цей потенціал веде себе як $O(1/|x|^2)$, $|x| \to \infty$, тобто прямує до нуля швидше ніж регулярна гармонічна функція. \medskip

Для виправлення цієї ситуації будемо шукати розв'язок зовнішньої задачі Діріхле $D_{\text{e}}$ у вигляді:
\begin{equation}
	\label{eq:4.9.15}
	u(x) = \Iint_S \sigma(y) \frac{\partial}{\partial \vec n_y} \frac{1}{4 \pi |x - y|} \diff S_y + \frac{1}{|x|} \Iint_S \sigma(y) \diff S_y,
\end{equation}
де перший доданок як і раніше є потенціалом подвійного шару, а другий --- \textit{потенціал Робена}, який не порушуючи гармонічність цієї функції, забезпечує регулярність на нескінченості. При цьому ми вважаємо, що система координат обрана таким чином, що точка $x = 0 \in \Omega$. \medskip

Для знаходження щільності $\sigma$, запишемо інтегральне співвідношення використовуючи теорему про граничні значення потенціалу подвійного шару (потенціал Робена є неперервною функцією в $\RR^3$):
\begin{equation}
	\label{eq:4.9.16}
	f(x) = \Iint_S \sigma(y) \left( \frac{\partial}{\partial \vec n_y} \frac{1}{4 \pi |x - y|} + \frac{1}{|x|} \right) \diff S_y + \frac{\sigma(x)}{2}.
\end{equation}

Тут $K(x, y) = \dfrac{\partial}{\partial \vec n_y} \dfrac{1}{4 \pi |x - y|} + \dfrac{1}{|x|}$ --- ядро інтегрального рівняння. \medskip

Покажемо, що \eqref{eq:4.9.16} має єдиний розв'язок. Для цього покажемо, що однорідне рівняння
\begin{equation}
	\label{eq:4.9.16'}
	\Iint_S \sigma(y) \left( \frac{\partial}{\partial \vec n_y} \frac{1}{4 \pi |x - y|} + \frac{1}{|x|} \right) \diff S_y + \frac{\sigma(x)}{2} = 0
\end{equation}
має лише тривіальний розв'язок. Нехай $\sigma_0$ --- деякий розв'язок рівняння \eqref{eq:4.9.16'}. Побудуємо потенціал з щільністю $\sigma_0$:
\begin{equation}
	\label{eq:4.9.17}
	U_0(x) = \Iint_S \sigma_0(y) \frac{\partial}{\partial \vec n_y} \frac{1}{4 \pi |x - y|} \diff S_y + \frac{1}{|x|} \Iint_S \sigma_0(y) \diff S_y,
\end{equation}
який задовольняє рівняння Лапласа для $x \in \Omega'$. З \eqref{eq:4.9.16'} випливає, що граничне значення потенціалу $U_0$ при підході до поверхні $S$ ззовні дорівнює нулю, тобто функція $U_0$ є розв'язком граничної задачі
\begin{equation}
	\left\{
		\begin{aligned}
			& \Delta U_0(x) = 0, \quad x \in \Omega', \\
			& \left. U_0 \right|_{x \in S} = 0, \\
			& U_0(x) = O(1/|x|), \quad |x| \to \infty.
		\end{aligned}		
	\right.
\end{equation}

Ця задача має лише тривіальний розв'язок, тобто $U_0(x) \equiv 0$, $x \in \Omega'$. Помножимо \eqref{eq:4.9.17} на $|x|$, та спрямуємо $x \to \infty$. В результаті отримаємо, що
\begin{equation}
	\label{eq:4.9.18}
	\Iint_S \sigma(y) \diff S = 0.
\end{equation}

Тобто будь-який розв'язок інтегрального рівняння \eqref{eq:4.9.16'}, задовольняє співвідношення \eqref{eq:4.9.18}, але тоді рівняння \eqref{eq:4.9.16'} спрощується і приймає вигляд
\begin{equation}
	\Iint_s \sigma_0(y) \left( \frac{\partial}{\partial \vec n_y} \frac{1}{4 \pi |x - y|} \right) \diff S_y + \frac{\sigma_0(x)}{2} = 0.
\end{equation}

Відомо, що це рівняння має у якості розв'язку функцію $\sigma_0(x) \equiv \const$, але з \eqref{eq:4.9.18} випливає що $\sigma_0(x) \equiv 0$. Таким чином, однорідне рівняння має лише тривіальний розв'язок, а відповідне неоднорідне рівняння \eqref{eq:4.9.16} має єдиний розв'язок для будь-якого вільного члена $f$. Тобто нами доведена
\begin{theorem}[про існування розв'язку зовнішньої задачі Діріхле]
	Нехай $S$ --- замкнена поверхня Ляпунова, $f \in C(S)$, тоді зовнішня задача Діріхле \eqref{eq:4.9.1'} має єдиний розв'язок регулярний на нескінченості для довільної неперервної функції $f$ і цей розв'язок може бути знайдений у вигляді суми потенціалі подвійного шару і потенціалу Робена.
\end{theorem}

\subsubsection{Третя гранична задача для рівняння Лапласа}

Окрім основних граничних задач рівняння Лапласа, метод потенціалів може бути застосований до третьої граничної задачі (Ньютона).
\begin{longtable}{p{.45\textwidth} p{.45\textwidth}p{.45\textwidth}}
	\parbox{.45\textwidth}{\begin{equation}\label{eq:4.9.19}\left\{\begin{aligned}
		& \Delta u(x) = 0, \quad x \in \Omega, \\
		& \left. \left( \frac{\partial u}{\partial \vec n} + \alpha(x) u \right) \right|_{x \in S} = f(x).
	\end{aligned}\right.\end{equation}} & \parbox{.45\textwidth}{\begin{equation}\label{eq:4.9.19'}\left\{\begin{aligned}
		& \Delta u(x) = 0, \quad x \in \Omega', \\
		& \left. \left( \frac{\partial u}{\partial \vec n} + \alpha(x) u \right) \right|_{x \in S} = f(x), \\
		& u(x) = O(1/|x|), \quad |x| \to \infty.
	\end{aligned}\right.\end{equation}}
\end{longtable}

Будемо припускати, $\alpha(x) > 0$, $x \in S$. Розв'язок граничних задач \eqref{eq:4.9.19} та \eqref{eq:4.9.19'} будемо шукати у вигляді потенціалів простого шару
\begin{equation}
	V(x) = \Iint_S \mu(y) \frac{1}{4 \pi |x - y|} \diff S_y.
\end{equation}

Враховуючи теорему про граничні значення нормальної похідної потенціалу простого шару запишемо інтегральні рівняння для щільності потенціалу $\mu$ по поверхні $S$:
\begin{align}
	\label{eq:4.9.20}
	f(x) &= \Iint_S \mu(y) \left( \frac{\partial}{\partial \vec n_x} \frac{1}{4 \pi |x - y|} + \frac{\alpha(x)}{4 \pi |x - y|} \right) \diff S_y + \frac{\mu(x)}{2}, \quad x \in S, \quad H_{\text{i}}, \\
	\label{eq:4.9.20'}
	f(x) &= \Iint_S \mu(y) \left( \frac{\partial}{\partial \vec n_x} \frac{1}{4 \pi |x - y|} + \frac{\alpha(x)}{4 \pi |x - y|} \right) \diff S_y - \frac{\mu(x)}{2}, \quad x \in S, \quad H_{\text{e}}.
\end{align}

Можна показати, що у випадку, коли $\alpha(x) \ge 0$, $x \in S$, інтегральні рівняння \eqref{eq:4.9.20} та \eqref{eq:4.9.20'} мають єдині розв'язки для будь-якої неперервної функції $f(x)$, і ці розв'язки можуть бути знайдені у вигляді потенціалу простого шару. 

\subsubsection{Дослідження існування розв'язків граничних задач рівняння Гельмгольца}

Розглянемо граничні задачі для однорідного рівняння Гельмгольца
\begin{align}
	\label{eq:4.9.21}
	& \left\{
		\begin{aligned}
			& (\Delta + k^2) u(x) = 0, \quad x \in \Omega, \\
			& \left. u \right|_{x \in S} = f(x).
		\end{aligned}
	\right. \\[.5cm]
	\label{eq:4.9.21'}
	& \left\{
		\begin{aligned}
			& (\Delta + k^2) u(x) = 0, \quad x \in \Omega', \\
			& \left. u \right|_{x \in S} = f(x), \\
			& u(x) = O(1/|x|), \quad |x| \to \infty, \\
			& \frac{\partial u(x)}{\partial |x|} \pm i k u(x) = o(1/|x|), \quad |x| \to \infty.
		\end{aligned}
	\right. \\[.5cm]
	\label{eq:4.9.22}
	& \left\{
		\begin{aligned}
			& (\Delta + k^2) u(x) = 0, \quad x \in \Omega, \\
			& \left. \frac{\partial u}{\partial \vec n} \right|_{x \in S} = f(x)
		\end{aligned}
	\right. \\[.5cm]
	\label{eq:4.9.22'}
	& \left\{
		\begin{aligned}
			& (\Delta + k^2) u(x) = 0, \quad x \in \Omega', \\
			& \left. \frac{\partial u}{\partial \vec n} \right|_{x \in S} = f(x), \\
			& u(x) = O(1/|x|), \quad |x| \to \infty, \\
			& \frac{\partial u(x)}{\partial |x|} \pm i k u(x) = o(1/|x|), \quad |x| \to \infty.
		\end{aligned}
	\right.
\end{align}

Будемо шукати розв'язки граничних задач для рівняння Гельмгольца у вигляді потенціалу простого шару
\begin{equation}
	V^k(x) = \Iint_S \frac{e^{\pm i k |x - y|} \mu(y)}{4 \pi |x - y|} \diff S_y
\end{equation}
для задачі Неймана \eqref{eq:4.9.22}, \eqref{eq:4.9.22'} та у вигляді потенціалу подвійного шару
\begin{equation}
	W^k(x) = \Iint_S \sigma(y) \frac{\partial}{\partial \vec n_y} \frac{e^{\pm i k |x - y|}}{4 \pi |x - y|} \diff S_y
\end{equation}
для задачі Діріхле \eqref{eq:4.9.21}, \eqref{eq:4.9.21'}.

\begin{theorem}[про існування розв'язків граничних задач для рівняння Гельмгольца]
	Нехай число $k^2$ не є власним числом внутрішніх задач Діріхле та Неймана оператора Лапласа в області $\Omega$, тоді граничні задачі Діріхле та Неймана, внутрішня та зовнішня \eqref{eq:4.9.21}, \eqref{eq:4.9.22} та \eqref{eq:4.9.21'}, \eqref{eq:4.9.22'} мають єдиний розв'язок для будь-якої неперервної функції $f(x)$ на поверхні $S$, і ці розв'язки можна представити у вигляді потенціалів подвійного шару та простого шару відповідно.
\end{theorem}

\begin{proof}
	Будемо шукати розв'язки внутрішньої та зовнішньої задачі Діріхле у вигляді потенціалу подвійного шару
	\begin{equation}
		\label{eq:4.9.23}
		u(x) = \Iint_S \sigma(y) \frac{\partial}{\partial \vec n_y} \frac{e^{i k |x - y|}}{4 \pi |x - y|} \diff S_y,
	\end{equation}
	а внутрішньої та зовнішньої задач Неймана у вигляді потенціалу простого шару
	\begin{equation}
		\label{eq:4.9.24}
	u(x)=	\Iint_S \frac{e^{- i k |x - y|} \mu(y)}{4 \pi |x - y|} \diff S_y,
	\end{equation}

	Враховуючи властивості потенціалів та фундаментальних розв'язків оператора Гельмгольца, легко бачити, що ці функції задовольняють однорідному рівнянню Гельмгольца та умовам регулярності на нескінченості у випадку зовнішніх задач.
	Для визначення невідомої щільності запишемо граничні інтегральні рівняння:
	\begin{align}
		\label{eq:4.9.25}
		& f(x) = - \frac{\sigma(x)}{2} + \Iint_S K(x, y) \sigma(y) \diff S_y, \quad D_{\text{i}}, \\
		\label{eq:4.9.25'}
		& f(x) = \frac{\sigma(x)}{2} + \Iint_S K(x, y) \sigma(y) \diff S_y, \quad D_{\text{e}}, \\
		\label{eq:4.9.26}
		& f(x) = \frac{\mu(x)}{2} + \Iint_S K^\star(x, y) \mu(y) \diff S_y, \quad N_{\text{i}}, \\
		\label{eq:4.9.26'}
		& f(x) = - \frac{\mu(x)}{2} + \Iint_S K^\star(x, y) \mu(y) \diff S_y, \quad N_{\text{e}}.
	\end{align}

	Тут
	\begin{equation}
		\label{eq:4.9.27}
		K(x, y) = \frac{\partial}{\partial \vec n_y} \frac{e^{ik|x - y|}}{4 \pi |x - y|} = \frac{i k |x - y| - 1}{4 \pi |x - y|^2} e^{i k|x - y|} \cos(y - x, \vec n_y)
	\end{equation}
	та
	\begin{equation}
		\label{eq:4.9.27'}
		K^\star(x, y) = \frac{\partial}{\partial \vec n_x} \frac{e^{-ik|x - y|}}{4 \pi |x - y|} = \frac{- i k |x - y| - 1}{4 \pi |x - y|^2} e^{- i k|x - y|} \cos(x - y, \vec n_x)
	\end{equation}
	--- ядра основного та спряженого інтегральних рівнянь. \medskip

	Вивчимо випадок зовнішньої задачі Діріхле та внутрішньої задачі Неймана $D_{\text{e}}, N_{\text{i}}$. Розглянемо однорідне інтегральне рівняння, яке відповідає $N_{\text{i}}$:
	\begin{equation}
		\label{eq:4.9.28}
		0 = \frac{\mu(x)}{2} + \Iint_S K^\star(x, y) \mu(y) \diff S_y. 
	\end{equation}

	Припустимо, що \eqref{eq:4.9.28} має єдиний нетривіальний розв'язок $\mu_0$, тоді і однорідне рівняння для $D_{\text{e}}$
	\begin{equation}
		\label{eq:4.9.29}
		f(x) = \frac{\sigma(x)}{2} + \Iint_S K(x, y) \sigma(y) \diff S_y. 
	\end{equation}
	теж має єдиний нетривіальний розв'язок $\sigma_0$. Складемо потенціал простого шару з щільністю $\mu_0$:
	\begin{equation}
		V_0^k(x) = \Iint_S \frac{e^{-ik|x-y|}\mu_0(y)}{4\pi|x-y|}\diff S_y.	
	\end{equation}

	З теореми про розрив нормальної похідної потенціалу простого шару маємо, що цей потенціал є розв'язком граничної задачі:
	\begin{equation}
		\left\{
			\begin{aligned}
				& (\Delta + k^2) V_0^k(x) = 0, \quad x \in \Omega, \\
				& \left. \frac{\partial V_0^k(x)}{\partial \vec n_{\text{i}}} \right|_{x \in S} = 0,
			\end{aligned}
		\right.
	\end{equation}
	але оскільки $k^2$ не є власне число внутрішньої задачі Неймана, то остання гранична задача має лише тривіальний розв'язок, тобто $V_0^k(x) \equiv 0$, $x \in \Omega$. Враховуючи неперервність потенціалу простого шару в усьому евклідовому просторі можна записати, що $V_0^k(x) \equiv 0$, $x \in S$. Звідси випливає, що $V_0^k$ задовольняє граничній задачі
	\begin{equation}
		\left\{
			\begin{aligned}
				& (\Delta + k^2) V_0^k(x) = 0, \quad x \in \Omega', \\
				& \left. V_0^k(x) \right|_{x \in S} = 0,
			\end{aligned}
		\right.
	\end{equation}
	а оскільки $V_0^k(x)$ задовольняє умові Зомерфельда, то єдиним розв'язком останньої задачі Діріхле є тотожній нуль. Отже маємо, що $V_0^k(x) \equiv 0$, $x \in \RR^3$, а це в свою чергу означає, що $\mu_0 \equiv 0$, $x \in S$ і рівняння \eqref{eq:4.9.28} та \eqref{eq:4.9.29} мають лише тривіальні розв'язки. Згідно до теореми Фредгольма відповідні неоднорідні рівняння \eqref{eq:4.9.25'} та \eqref{eq:4.9.26} мають єдиний розв'язок для будь-якого вільного члена $f$.
\end{proof}

\end{document}