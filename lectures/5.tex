\documentclass[a4paper, 12pt]{article}
\usepackage[utf8]{inputenc}
\usepackage[english, ukrainian]{babel}

\usepackage{amsmath, amssymb}
\usepackage{multicol}
\usepackage{graphicx}
\usepackage{float}

\allowdisplaybreaks
\setlength\parindent{0pt}
\numberwithin{equation}{subsection}

\usepackage{hyperref}
\hypersetup{unicode=true,colorlinks=true,linktoc=all,linkcolor=red}

\numberwithin{equation}{subsection}

\renewcommand{\bf}[1]{\textbf{#1}}
\renewcommand{\it}[1]{\textit{#1}}
\newcommand{\bb}[1]{\mathbb{#1}}
\renewcommand{\cal}[1]{\mathcal{#1}}

\renewcommand{\epsilon}{\varepsilon}
\renewcommand{\phi}{\varphi}

\DeclareMathOperator{\diam}{diam}
\DeclareMathOperator{\rang}{rang}
\DeclareMathOperator{\const}{const}

\newenvironment{system}{%
  \begin{equation}%
    \left\{%
      \begin{aligned}%
}{%
      \end{aligned}%
    \right.%
  \end{equation}%
}
\newenvironment{system*}{%
  \begin{equation*}%
    \left\{%
      \begin{aligned}%
}{%
      \end{aligned}%
    \right.%
  \end{equation*}%
}

\makeatletter
\newcommand*{\relrelbarsep}{.386ex}
\newcommand*{\relrelbar}{%
  \mathrel{%
    \mathpalette\@relrelbar\relrelbarsep%
  }%
}
\newcommand*{\@relrelbar}[2]{%
  \raise#2\hbox to 0pt{$\m@th#1\relbar$\hss}%
  \lower#2\hbox{$\m@th#1\relbar$}%
}
\providecommand*{\rightrightarrowsfill@}{%
  \arrowfill@\relrelbar\relrelbar\rightrightarrows%
}
\providecommand*{\leftleftarrowsfill@}{%
  \arrowfill@\leftleftarrows\relrelbar\relrelbar%
}
\providecommand*{\xrightrightarrows}[2][]{%
  \ext@arrow 0359\rightrightarrowsfill@{#1}{#2}%
}
\providecommand*{\xleftleftarrows}[2][]{%
  \ext@arrow 3095\leftleftarrowsfill@{#1}{#2}%
}
\makeatother

\newcommand{\NN}{\mathbb{N}}
\newcommand{\ZZ}{\mathbb{Z}}
\newcommand{\QQ}{\mathbb{Q}}
\newcommand{\RR}{\mathbb{R}}
\newcommand{\CC}{\mathbb{C}}

\newcommand{\Max}{\displaystyle\max\limits}
\newcommand{\Sup}{\displaystyle\sup\limits}
\newcommand{\Sum}{\displaystyle\sum\limits}
\newcommand{\Int}{\displaystyle\int\limits}
\newcommand{\Iint}{\displaystyle\iint\limits}
\newcommand{\Lim}{\displaystyle\lim\limits}

\newcommand*\diff{\mathop{}\!\mathrm{d}}

\newcommand*\rfrac[2]{{}^{#1}\!/_{\!#2}}


\title{{\Huge МАТЕМАТИЧНА ФІЗИКА}}
\author{Скибицький Нікіта}
\date{\today}

\usepackage{amsthm}
\usepackage[dvipsnames]{xcolor}
\usepackage{thmtools}
\usepackage[framemethod=TikZ]{mdframed}

\theoremstyle{definition}
\mdfdefinestyle{mdbluebox}{%
	roundcorner = 10pt,
	linewidth=1pt,
	skipabove=12pt,
	innerbottommargin=9pt,
	skipbelow=2pt,
	nobreak=true,
	linecolor=blue,
	backgroundcolor=TealBlue!5,
}
\declaretheoremstyle[
	headfont=\sffamily\bfseries\color{MidnightBlue},
	mdframed={style=mdbluebox},
	headpunct={\\[3pt]},
	postheadspace={0pt}
]{thmbluebox}

\mdfdefinestyle{mdredbox}{%
	linewidth=0.5pt,
	skipabove=12pt,
	frametitleaboveskip=5pt,
	frametitlebelowskip=0pt,
	skipbelow=2pt,
	frametitlefont=\bfseries,
	innertopmargin=4pt,
	innerbottommargin=8pt,
	nobreak=true,
	linecolor=RawSienna,
	backgroundcolor=Salmon!5,
}
\declaretheoremstyle[
	headfont=\bfseries\color{RawSienna},
	mdframed={style=mdredbox},
	headpunct={\\[3pt]},
	postheadspace={0pt},
]{thmredbox}

\declaretheorem[style=thmbluebox,name=Теорема,numberwithin=subsubsection]{theorem}
\declaretheorem[style=thmbluebox,name=Лема,numberwithin=subsubsection]{lemma}
\declaretheorem[style=thmbluebox,name=Твердження,numberwithin=subsubsection]{proposition}
\declaretheorem[style=thmbluebox,name=Принцип,numberwithin=subsubsection]{th_principle}
\declaretheorem[style=thmbluebox,name=Закон,numberwithin=subsubsection]{law}
\declaretheorem[style=thmbluebox,name=Закон,numbered=no]{law*}
\declaretheorem[style=thmbluebox,name=Формула,numberwithin=subsubsection]{th_formula}
\declaretheorem[style=thmbluebox,name=Рівняння,numberwithin=subsubsection]{th_equation}
\declaretheorem[style=thmbluebox,name=Умова,numberwithin=subsubsection]{th_condition}
\declaretheorem[style=thmbluebox,name=Наслідок,numberwithin=subsubsection]{corollary}

\declaretheorem[style=thmredbox,name=Приклад,numberwithin=subsubsection]{example}
\declaretheorem[style=thmredbox,name=Приклади,sibling=example]{examples}

\declaretheorem[style=thmredbox,name=Властивість,numberwithin=subsubsection]{property}
\declaretheorem[style=thmredbox,name=Властивості,sibling=property]{properties}

\mdfdefinestyle{mdgreenbox}{%
	skipabove=8pt,
	linewidth=2pt,
	rightline=false,
	leftline=true,
	topline=false,
	bottomline=false,
	linecolor=ForestGreen,
	backgroundcolor=ForestGreen!5,
}
\declaretheoremstyle[
	headfont=\bfseries\sffamily\color{ForestGreen!70!black},
	bodyfont=\normalfont,
	spaceabove=2pt,
	spacebelow=1pt,
	mdframed={style=mdgreenbox},
	headpunct={ --- },
]{thmgreenbox}

\mdfdefinestyle{mdblackbox}{%
	skipabove=8pt,
	linewidth=3pt,
	rightline=false,
	leftline=true,
	topline=false,
	bottomline=false,
	linecolor=black,
	backgroundcolor=RedViolet!5!gray!5,
}
\declaretheoremstyle[
	headfont=\bfseries,
	bodyfont=\normalfont\small,
	spaceabove=0pt,
	spacebelow=0pt,
	mdframed={style=mdblackbox}
]{thmblackbox}

\declaretheorem[name=Вправа,numberwithin=subsubsection,style=thmblackbox]{exercise}
\declaretheorem[name=Зауваження,numberwithin=subsubsection,style=thmgreenbox]{remark}
\declaretheorem[name=Визначення,numberwithin=subsubsection,style=thmblackbox]{definition}

\newtheorem{problem}{Задача}[subsection]
\newtheorem{sproblem}[problem]{Задача}
\newtheorem{dproblem}[problem]{Задача}
\renewcommand{\thesproblem}{\theproblem$^{\star}$}
\renewcommand{\thedproblem}{\theproblem$^{\dagger}$}
\newcommand{\listhack}{$\empty$\vspace{-2em}} 

\theoremstyle{remark}
\newtheorem*{solution}{Розв'язок}


\begin{document}

\tableofcontents

\setcounter{section}{2}
\setcounter{subsection}{4}
\setcounter{subsubsection}{4}
\setcounter{theorem}{15}
\setcounter{equation}{41}

\subsubsection{Додатньо визначені ядра}

\begin{definition}[додатно-визначеного ядра]
    Неперервне ядро $K(x, y)$ називається \it{додатно-визначеним}, якщо $\forall f \in L_2(G)$: $(\bf{K}f, f) \ge 0$, рричому $(\bf{K}f, f) = 0 \iff \|f\|_{L_2(G)} = 0$.
\end{definition}

\begin{remark}
    Довільне додатньо визначене ядро є ермітовим (його білінійна форма $(\bf{K}f, f)$ приймає дійсні значення).
\end{remark}

\begin{lemma}
    Для того, щоб неперервне ядро було додатньо визначеним необхідно і достатньо, щоб його характеристичні числа були додатні.
\end{lemma}

\begin{proof}
    Необхідність: для власних функцій $(\bf{K} \phi_k, \phi_k) = 1 / \lambda_k > 0$. \medskip

    Достатність: Розглянемо $\bf{K} f$ як джерелувато-зображувану функцію, згідно до теореми Гілберта-Шмідта
    \begin{equation}
        \bf{K} f = \sum_{k = 1}^\infty \frac{(f, \phi_k)}{\lambda_k} \phi_k,
    \end{equation}
    тоді 
    \begin{equation}
        (\bf{K}f, f) = \Sum_{k = 1}^\infty \dfrac{(f, \phi_k)}{\lambda_k} (\phi_k, f) = \Sum_{k = 1}^\infty \dfrac{|(f, \phi_k)|^2}{\lambda_k} > 0,
    \end{equation}
    отже квадратична форма додатньо визначена. \medskip

    Таким чином додатність характеристичних чисел є критерієм додатної визначеності ядра.
\end{proof}

\begin{lemma}
    Довільне додатньо визначене неперервне ядро має характеристичні числа і для них має місце варіаційний принцип:
    \begin{equation}
        \dfrac{1}{\lambda_k} = \Sup_{\substack{f \in L_2(G) \\ (f, \phi_i) = 0, i = \overline{1, k - 1}}} \dfrac{(\bf{K}f, f)_{L_2(G)}}{\|f\|_{L_2(G)}^2}, \quad k = 1, 2, \ldots,
    \end{equation}
    де $\lambda_1 \le \lambda_2 \le \lambda_3 \le \ldots$, а $\phi_1, \phi_2, \phi_3, \ldots$ --- ортонормована система власних функцій.
\end{lemma}

\begin{proof}
    З теореми Гілберта Шмідта можна оцінити 
    \begin{equation}
        \dfrac{(\bf{K}f, f)_{L_2(G)}}{\|f\|_{L_2(G)}^2} = \Sum_{i=k}^\infty \dfrac{|(f, \phi_i)|^2}{\lambda_i\|f\|^2} \le \dfrac{1}{\lambda_k} \Sum_{i=k}^\infty \dfrac{|(f, \phi_i)|^2}{\|f\|^2} \le \dfrac{1}{\lambda_k}.
    \end{equation}
    (перша нерівність виконується оскільки $\lambda_k$ --- найменше характеристичне число в сумі, а друга випливає з нерівності Бесселя). \medskip

    З іншого боку при $f = \phi_k$ маємо
    \begin{equation}
        \frac{(\bf{K}\phi_k, \phi_k)}{\|\phi_k\|^2} = \frac{1}{\lambda_k},
    \end{equation}
    тобто існує функція на якій досягається верхня межа цієї нерівності.
\end{proof}

\begin{theorem}[Мерсера, про регулярну збіжність білінійного ряду для ермітових ядер зі скінченою кількістю від'ємних характеристичних чисел] 
    Якщо ермітове неперервне ядро $K(x, y)$ має лише скінчену кількість від'ємних характеристичних чисел, то його білінійний ряд 
    \begin{equation}
        K(x,y)=\Sum_{i=1}^\infty\dfrac{\phi_i(x)\overline\phi_i(y)}{\lambda_i}
    \end{equation}
    збігається в $\overline G\times\overline G$ абсолютно і рівномірно.
\end{theorem}

\begin{proof}
    Покажемо, що якщо ядро $K(x,y)$ --- додатньо визначене, то $\forall x\in\overline G$: $K(x,x)\ge0$. Оскільки $K(x,y)$ --- ермітове, то $K(x,x)=\overline K(x,x)$ і є дійсною функцією. Якщо існує хоча б одна точка $x_0\in\overline G$ така, що $K(x_0,x_0)<0$, то виходячи з неперервності знайдеться і деякій окіл цієї точки $U(x_0,x_0)\subset\overline G\times\overline G$ такий, що $\forall(x,y)\in U(x_0,x_0)$: $\text{Re} K(x,y)<0$. Оберемо невід'ємну неперервну функцію $\phi(x)$ яка відміна від нуля лише в $U(x_0,x_0)$ і отримаємо
    \begin{equation}
        \begin{aligned}
            (\bf{K}\phi,\phi)&=\Int_{U(x_0,x_0)}K(x,y)\phi(x)\phi(y)\diff x\diff y=\\
            &=\Int_{U(x_0,x_0)}\text{Re}K(x,y)\phi(x)\phi(y)\diff x\diff y\le0.
        \end{aligned}
    \end{equation}

    Остання нерівність вступає в протиріччя з припущенням додотньої визначеності ядра, тобто теорему достатньо довести для додатньо визначених ядер. \medskip

    Розглянемо ядро
    \begin{equation}
        K^p(x,y)=K(x,y)-\sum_{i=1}^p\frac{\overline\phi_i(y)\phi_i(x)}{\lambda_i},
    \end{equation}
    де $p$ --- номер останнього від'ємного характеристичного числа, так що усі $\lambda_i$, $i=p+1,p+2,\ldots$ є додатніми. Таким чином ядро $K^p(x,y)$ є неперервним та додатньо визначеним. А це означає, що $\forall x\in\overline G:K(x,x)\ge0$. Таким чином маємо нерівність:
    \begin{equation}
        \Sum_{i=1}^N\dfrac{|\phi_i(x)|^2}{\lambda_i}\le K(x,x)\le M,\quad x\in\overline G, N=p+1,p+2,\ldots,
    \end{equation}
    тобто ряд
    \begin{equation}
        \sum_{k=1}^\infty\frac{|\phi_k(x)|^2}{\lambda_k}
    \end{equation}
    рівномірно збіжний. \medskip

    Розглянемо білінійний ряд 
    \begin{equation}
        \sum_{i=1}^\infty\frac{\phi_i(x)\overline\phi_i(y)}{\lambda_i}
    \end{equation}
    і доведемо його абсолютну і рівномірну збіжність за критерієм Коші. Використовуючи нерівність Коші-Буняківського маємо:
    \begin{equation}
        \Sum_{k=p}^{p+q}\dfrac{|\phi_k(x)\overline\phi_k(y)|}{\lambda_k}\le\left(\Sum_{k=p}^{p+q}\dfrac{|\phi_k(x)|^2}{\lambda_k}\cdot\Sum_{k=p}^{p+q}\dfrac{|\phi_k(y)|^2}{\lambda_k}\right)^{1/2}
    \end{equation}

    Але оскільки
    \begin{equation}
        \sum_{k=1}^\infty\frac{|\phi_k(x)|^2}{\lambda_k}
    \end{equation}
    рівномырно збіжний, то білінійний ряд
    \begin{equation}
        K(x,y)=\Sum_{i=1}^\infty\dfrac{\phi_i(x)\overline\phi_i(y)}{\lambda_i}
    \end{equation}
    збігається абсолютно і рівномірно (регулярно) в $\overline G\times\overline G$.
\end{proof}

\begin{remark}
    Теорема Гільберта-Шмідта і її наслідки, встановлені для ермітового неперервного ядра, залишаються вірними і для ермітового слабо полярного ядра.
\end{remark}

\begin{remark}
    Теорема Гільберта-Шмідта і формула Шмідта у випадку полярного ядра залишаються вірними, але з заміною рівномірної збіжності на середньоквадратичну.
\end{remark}

\subsection{Задача Штурма-Ліувілля. Теорема Стеклова}

Постановка задачі Штурма-Ліувілля: нехай $\bf{L}$ --- диференціальний оператор другого порядку: задано рівняння
\begin{equation}
    \bf{L}u=(-p(x)u')'+q(x)u=\lambda u,\quad0<x<l,
\end{equation}
з крайовими умовами
\begin{align}
    l_1(u)|_{x=0}&=h_1u(0)-h_2u'(0)=0,\\
    l_2(u)|_{x=l}&=H_1u(l)-H_2u'(l)=0,
\end{align}
де функція $p\in C^{(1)}([0,l])$, $p>0$, функція $q\in C([0,l])$, $q\ge0$, виконуються наступні умови на сталі: $h_1, h_2, H_1, H_2 \ge 0$, $h_1+h_2>0$, $H_1+H_2>0$,
а також
\begin{equation}
    M_L=\{u:u\in C^{(2)}(0,l)\cap C^{(1)}([0,l]), u''\in L_2(0,l), l_1u(0)=l_2u(l)=0\}
\end{equation}
-- область визначення оператора $\bf{L}$.

\begin{definition}[власних чисел і функцій задачі Штурма-Ліувілля]
    Знайти розв'язки задачі Штурма-Ліувіля означає знайти всі ті значення параметра $\lambda$, при яких вищезгаданакрайова задача має нетривіальний розв'язок. Ці значення називаються \it{власними значеннями} задачі Штурма-Ліувіля, а самі розв'язки --- \it{власними функціями}.
\end{definition}

\subsubsection{Функція Гріна оператора $\bf{L}$}

Будемо припускати, що $\lambda = 0$ не є власним числом оператора $\bf{L}$ задачі Штурма-Ліувіля. \medskip

Розглянемо крайову задачу:
\begin{system}
    & (-p(x)u')'+q(x)u=f(x),\quad0<x<l \\
    & l_1(u)|_{x=0}=l_2(u)|_{x=l}=0
\end{system}

Припустимо що $f \in C(0,l)\cap L_2(0,l)$. \medskip

З припущення, що $\lambda = 0$ не є власним числом випливає, що задача має єдиний розв'язок. \medskip

Розглянемо функції $v_i(x)$, $i=1,2$ --- ненульові дійсні розв'язки однорідних задач Коші:
\begin{system}
    & (-p(x)v_i'(x))'+q(x)v_i'(x)=0,\quad i=1,2 \\
    & l_1v_1|_{x=0}=l_2v_2|_{x=l}=0
\end{system}

З загальної теорії задач Коші випливає, що розв'язки цих задач Коші існують, тому $v_i(x)$ --- двічі неперервно диференційовані функції.

\begin{proposition}
    $v_1(x)$, $v_2(x)$ --- лінійно незалежні.
\end{proposition}

\begin{proof}
    Припустимо що це не так і $v_1(x) = cv_2(x)$, тобто $v_1(x)$ задовольняє одночасно граничним умовам на лівому і правому краях. Тоді $v_1(x)$ --- власна функція оператора $\bf{L}$, і відповідає власному числу $\lambda = 0$ що суперечить припущенню.
\end{proof}

В цьому випадку визначник Вронського
\begin{equation}
    w(x) = \begin{vmatrix} v_1 & v_2 \\ v_1' & v_2' \end{vmatrix} \ne 0
\end{equation}

Будемо шукати розв'язок задачі методом варіації довільної сталої у вигляді:
\begin{equation}
    u(x) = c_1(x) v_1(x) + c_2(x) v_2(x).
\end{equation}

Підставимо в рівняння:
\begin{equation}
    (-p(c_1'v_1+c_2'v_2+c_1v_1'+c_2v_2')'+q(c_1v_1+c_2v_2)=f.
\end{equation}

Накладемо першу умову на коефіцієнти: $c_1'v_1+c_2'v_2=0$, маємо: 
\begin{equation}
    -p'(c_1v_1'+c_2v_2')-p(c_1'v_1'+c_2'v_2'+c_1v_1''+c_2v_2'')+q(c_1v_1+c_2v_2)=f,
\end{equation}
або
\begin{equation}
    c_1Lv_1 + c_2Lv_2 - p(c_1'v_1'+c_2'v_2')=f,
\end{equation}
оскільки $c_1\bf{L}v_1=0$, $c_2\bf{L}v_2=0$, то
\begin{equation}
    -p(c_1'v_1'+c_2'v_2')=f,
\end{equation}
отже
\begin{equation}
    c_1'v_1'+c_2'v_2'=-\frac{f}{p}.
\end{equation}

Таким чином $c_1'$ та $c_2'$ повинні задовольняти системі лінійних диференціальних рівнянь:
\begin{system}
    c_1'v_1 + c_2'v_2 &= 0, \\
    c_1'v_1' + c_2'v_2' &= - \dfrac{f}{p},
\end{system}
визначник системи
\begin{equation}
    w(x) = \begin{vmatrix} v_1 & v_2 \\ v_1' & v_2' \end{vmatrix} \ne 0.
\end{equation}

\begin{remark}
    Має місце рівність Ліувілля:
    \begin{equation}
        w(x)\cdot p(x)=w(0)\cdot p(0)=\text{const}.
    \end{equation}
\end{remark}

Розв'язавши систему рівнянь, отримаємо:
\begin{system}
    c_1'(x) &= \dfrac{1}{w(x)} \begin{vmatrix} 0 & v_2 \\ - \dfrac{f}{p} & v_2' \end{vmatrix} = \dfrac{v_2(x)f(x)}{p(0)w(0)}, \\
    c_2'(x) &= \dfrac{1}{w(x)} \begin{vmatrix} v_1 & 0 \\ v_1' & - \dfrac{f}{p} \end{vmatrix} = -\dfrac{v_1(x)f(x)}{p(0)w(0)}, 
\end{system}

Знайдемо додаткові умови для диференціальних рівнянь вище: 
\begin{multline} 
l_1u|_{x=0} = h_1(c_1(0)v_1(0) + c_2(0)v_2(0)) - \\
- h_2(c_1'(0)v_1(0) + c_2'(0)v_2(0) + c_1(0)v_1'(0) + c_2(0)v_2'(0)) = 0,
\end{multline}
а враховуючи, що $c_1'(0)v_1(0)+c_2'(0)v_2(0) = 0$ маємо
\begin{equation}
    c_1(0) (h_1v_1(0) - h_2v_1'(0)) + c_2(0)v_2'(0) = 0.
\end{equation}

Оскільки перший доданок дорівнює нулю, то остання рівність виконується коли $c_2(0) = 0$, аналогічно отримаємо, що $c_1(l) = 0$. \medskip

Проінтегруємо систему дифурів що розглядається, отримаємо
\begin{equation}
    c_1(x)=-\Int_x^l\dfrac{f(\xi)v_2(\xi)}{p(0)w(0)}\diff\xi, \quad c_2(x)=-\Int_0^x\dfrac{v_1(\xi)f(\xi)}{p(0)w(0)}\diff\xi
\end{equation}

Тоді розв'язок крайової задачі буде мати вигляд:
\begin{equation}
    u(x)=-\Int_0^x\dfrac{v_1(\xi)v_2(x)f(\xi)}{p(0)w(0)}\diff\xi-\Int_x^l\dfrac{f(\xi)v_1(x)v_2(\xi)}{p(0)w(0)}\diff\xi
\end{equation}

\begin{definition}[функції Гріна]
    \it{Функція Гріна} визначається наступним чином:
    \begin{equation}
        G(x, \xi) = - \dfrac{1}{p(0)w(0)} \begin{cases}
            v_1(\xi) v_2(x), & 0 \le \xi \le x \le l, \\
            v_1(x) v_2(\xi), & 0 \le x \le \xi \le l.
        \end{cases}
    \end{equation}
\end{definition}

Отже розв'язок крайової задачі можна записати у вигляді
\begin{equation}
    u(x) = \Int_0^l G(x, \xi) f(\xi) \diff \xi
\end{equation}

$G(x, \xi)$ називається функцією Гріна оператора Штурма-Ліувіля. Попередні міркування доводять наступну лемму.
\begin{lemma}
    \label{lemma:2.5.5}
    Якщо $\lambda = 0$ не є власним числом задачі Штурма-Ліувіля, то роз\-в'яз\-ок крайової задачі існує та єдиний і представляється за формулою
    \begin{equation}
        u(x) = \Int_0^l G(x, \xi) f(\xi) \diff \xi
    \end{equation}
    через функцію Гріна.
\end{lemma}

\newpage

\subsubsection{Властивості функції Гріна}
\begin{properties}[функції Гріна]
    Можна показати, що:
    \begin{enumerate}
        \item \begin{itemize}
                \item $G(x, \xi) \in C([0, l] \times [0, l])$;
                \item $G(x, \xi) \in C^{(2)}(0 < x < \xi < l)$;
                \item $G(x, \xi) \in C^{(2)} (0 < \xi < x < l)$.
            \end{itemize}
        \item Симетричність: $G(x, \xi) = G(\xi, x)$, $x, \xi \in [0, l] \times [0, l]$.
        \item На діагоналі $x = \xi$ має місце розрив першої похідної:
            \begin{equation}
                \frac{\partial G(\xi + 0, \xi)}{\partial x} - \frac{\partial G(\xi - 0, \xi)}{\partial x} = - \frac{1}{p(\xi)},
            \end{equation}
            да $\xi\in(0, l)$. 
        \item Поза діагоналлю $x = \xi$ функція Гріна задовольняє однорідному диференціальному рівнянню $\bf{L}_x G(x, y) = 0$.
        \item На бічних сторонах квадрату $[0,l]\times[0,l]$ функція Гріна $G(x, y)$ задовольняє граничним умовам $l_1G|_{x=0}=l_2G|_{x=l}=0$.
        \item Функція $G(x,\xi)$ є розв'язком неоднорідного рівняння:
            \begin{equation}
                \bf{L}_xG(x,\xi)=-\delta(x-\xi),
            \end{equation}
            де $\delta(x)$ --- дельта-функція Дірака.
    \end{enumerate}
\end{properties}

\subsubsection{Зведення крайової задачі з оператором Штурма-Ліувілля до інтегрального рівняння}

Розглянемо крайову задачу з параметром
\begin{system}
    & \bf{L}u = (-p(x)u')' + q(x)u = f + \lambda u, \\
    & l_1(u)|_{x=0} = 0, \\
    & l_2(u)|_{x=l} = 0, \\
    & f \in C(0, l) \cap L_2(0, l),
\end{system}
і покажемо що вона зводиться до інтегрального рівняння Фредгольма другого роду з дійсним, симетричним та неперервним ядром $G(x, \xi)$.

\begin{theorem}[про еквівалентність крайової задачі для рівняння другого порядку інтегральному рівнянню з ермітовим ядром] 
    \label{theorem:2.5.7}
    Крайова задача при умові, що $\lambda = 0$ не є власним числом оператора $\bf{L}$, еквівалентна інтегральному рівнянню Фредгольма другого роду:
    \begin{equation}
        u(x) = \lambda \Int_0^l G(x, \xi) u(\xi) \diff \xi + \Int_0^l G(x, \xi) f(\xi) \diff \xi, \quad u \in C([0, l]),
    \end{equation}
    де $G(x, \xi)$ --- функція Гріна оператора $\bf{L}$.
\end{theorem}

\begin{proof}
    Необхідність. Нехай виконуються умови крайової задачі, тоді з леми \ref{lemma:2.5.5} із заміною правої частини $f \mapsto f + \lambda u$ розв'язок крайової задачі можемо представити у вигляді:
    \begin{equation}
        u(x) = \Int_0^l G(x, \xi) (\lambda u(\xi) + f(\xi)) \diff \xi,
    \end{equation}
    тобто $u(x)$ задовольняє вищенаведеному інтегральному рівнянню. \medskip

    Достатність. Нехай має місце інтегралньа рівність і $u_0(x)$ --- її розв'язок. Розглянемо крайову задачу:
    \begin{system*}
        & \bf{L}u = f + \lambda u_0, \\
        & l_1(u)|_{x=0} = l_2(u)|_{x=l} = 0.
    \end{system*}

    За лемою \ref{lemma:2.5.5}, єдиний розв'язок цієї задачі задається формулою
    \begin{equation}
        u(x) = \lambda \Int_0^1 G(x, \xi) u_0(\xi) \diff \xi + \Int_0^1 G(x, \xi) f(\xi) \diff \xi,
    \end{equation}
    звідки випливає, що $u_0$ задовольняє рівнянню $Lu_0=f+\lambda u_0$, таким чином $u(x)=u_0(x)$ тобто $u_0$ є розв'язком вищенаведеної крайової задачі.
\end{proof}

У випадку коли $f \equiv 0$, ця крайова задача перетворюється в задачу Штурма-Ліувіля
\begin{system}
    & \bf{L}u = \lambda u, \quad 0 < x < l, \\
    & l_1(u)|_{x=0} = l_2(u)|_{x=l} = 0.
\end{system}

Задача Штурма-Ліувіля еквівалентна задачі про знаходження характеристичних чисел та власних функцій для однорідного інтегрального рівняння Фредгольма
\begin{equation}
    u(x) = \lambda \Int_0^1 G(x, \xi) u(\xi) \diff \xi
\end{equation}
при умові, що $\lambda = 0$ не є власним числом оператора $\bf{L}$.\medskip

Покажемо як позбавитись цього припущення. Нехай маємо задачу \allowbreak Штурма-Ліувілля:
\begin{system}
    & \bf{L}u = \lambda u, \quad 0 < x < l, \\
    & l_1(u)|_{x=0} = l_2(u)|_{x=l} = 0.
\end{system}

Легко бачити, що $(\bf{L}u, u) \ge 0$, тобто власні числа невід'ємні. \medskip

Розглянемо крайову задачу:
\begin{system}
    &\bf{L}_1 u \equiv (-p(x)u')'+(q(x)+1)u=\mu u,\\
    & l_1u|_{x=0}=l_2u|_{x=l}, \quad \mu=\lambda+1.
\end{system}

Ця задача з точністю до позначень співпадає з початковою задачею Штурма-Ліувіля. Очевидно, що $\mu = 0$ не є власним числом нової задачі Штурма-Ліувіля (бо тоді $\lambda = -1$ могло би бути власним числом початкової задачі Штурма-Ліувілля). \medskip

Введемо диференціальний оператор
\begin{equation}
    \bf{L}_1u = (-pu')' + q_1u = \mu u
\end{equation}

Отже, нова задача еквівалентна попередній задачі при $\mu = \lambda + 1$, та еквівалентна інтегральному рівнянню 
\begin{equation}
    u(x) = (\lambda+1)\Int_0^1 G_1(x, \xi)u(\xi)\diff \xi,
\end{equation}
де $G_1(x, \xi)$ --- функція Гріна оператора $\bf{L}_1$. \medskip

Таким чином, ввівши оператор $\bf{L}_1$ і відповідну йому нову функцію Гріна $G_1(x, \xi)$, можна позбутися припущення, що $\lambda = 0$ не є власним числом задачі Штурма-Ліувілля.

\newpage

\begin{example}
    Знайти розв'язок інтегрального рівняння
    \begin{equation*}
        \phi(x) = \lambda \Int_0^1 K(x, y) \phi(y) \diff y + x,
    \end{equation*}
    де
    \begin{equation*}
        K(x,y)=\begin{cases}x(y-1), & 0 \le x \le y \le 1 \\ y(x-1), & 0\le y\le x\le1\end{cases}.
    \end{equation*}
\end{example}

\begin{solution}
    Розв'язок будемо шукати за формулою Шмідта. Знайдемо характеристичні числа та власні функції ермітового ядра. Запишемо однорідне рівняння
    \begin{equation*}
        \phi(x) = \lambda \Int_0^x y(x-1)\phi(y)\diff y + \lambda\Int_x^1x(y-1)\phi(y)\diff y.
    \end{equation*}
    
    Продиференціюємо рівняння:
    \begin{equation*}
        \phi'(x) = \lambda \Int_0^x y\phi(y)\diff y + \lambda x(x-1)\phi(x) + \lambda\Int_x^1(y-1)\phi(y)\diff y - \lambda x(x-1)\phi(x).
    \end{equation*}
    
    Обчислимо другу похідну:
    \begin{equation*}
        \phi''(x) = \lambda x\phi(x)-\lambda(x-1)\phi(x).
    \end{equation*}

    Або після спрощення $\phi'' = \lambda \phi$. Доповнимо диференціальне рівняння другого порядку крайовими умовами: легко бачити що
    \begin{equation*}
        \phi(0) = \lambda \Int_0^0 y(0-1)\phi(y)\diff y + \lambda\Int_0^10(y-1)\phi(y)\diff y=0.
    \end{equation*}

    Аналогічно
    \begin{equation*}
        \phi(1) = \lambda\Int_0^1y(1-1)\phi(y)\diff y + \lambda\Int_1^11(y-1)\phi(y)\diff y=0.
    \end{equation*}

    Таким чином отримаємо задачу Штурма-Ліувілля:
    \begin{system*}
        & \phi'' = \lambda \phi, \quad 0 < x < 1, \\
        & \phi(0) = \phi(1) = 0.
    \end{system*}

    Для знаходження власних чисел і власних функцій розглянемо можливі значення параметру $\lambda$:
    \begin{enumerate}
        \item $\lambda > 0$, $\phi(x)=c_1\sinh(\sqrt{\lambda}x)+c_2\cosh(\sqrt{\lambda}x)$. \medskip

        Враховуючи граничні умови, маємо систему рівнянь 
        \begin{system*}
            & c_1 \cdot 0 + c_2 = 0, \\
            & c_1 \sinh(\sqrt{\lambda}) + c_2 \cosh(\sqrt{\lambda}) = 0.
        \end{system*}

        Визначник цієї системи повинен дорівнювати нулю:
        \begin{equation*}
            D(\lambda) = \begin{vmatrix} 0 & 1 \\ \sinh(\sqrt{\lambda}) & \cosh(\sqrt{\lambda}) \end{vmatrix} = - \sinh(\sqrt{\lambda}) = 0.
        \end{equation*}

        Єдиним розв'язком цього рівняння є $\lambda = 0$, яке не задовольняє, бо $\lambda > 0$. Це означає, що система рівнянь має тривіальний розв'язок і будь-яке $\lambda > 0$ не є власним числом.

        \item $\lambda = 0$, $\phi(x) = c_1 x + c_2$. З граничних умов маємо, що $c_1 = c_2 = 0$. Тобто $\lambda=0$ не є власним числом.

        \item $\lambda < 0$, $\phi(x) = c_1\sin(\sqrt{-\lambda}x)+c_2\cos(\sqrt{-\lambda}x)$. \medskip

        Враховуючи граничні умови, маємо систему рівнянь
        \begin{system*}
            & c_1 \cdot 0 + c_2 = 0, \\
            & c_1 \sin(\sqrt{-\lambda}) + c_2 \cos(\sqrt{-\lambda}) = 0.
        \end{system*}

        Визначник цієї системи прирівняємо до нуля:
        \begin{equation*}
            D(\lambda) = \begin{vmatrix} 0 & 1 \\ \sin(\sqrt{-\lambda}) & \cos(\sqrt{-\lambda}) \end{vmatrix} = -\sin(\sqrt{-\lambda}) = 0.
        \end{equation*}

        Це рівняння має зліченну множину розв'язків
        \begin{equation*}
            \lambda_k = - (\pi k)^2, \quad k = 1, 2, \ldots
        \end{equation*}
        Система лінійних алгебраїчних рівнянь має розв'язок $c_2=0$, $c_1=c_1$. \medskip

        Таким чином нормовані власні функції задачі Штурма-Ліувілля мають вигляд $\phi_k(x) = \sqrt{2} \sin(k \pi x)$. \medskip

        Порахуємо коефіцієнти Фур'є:
        \begin{equation*}
            f_n = (f, \phi_n) = \sqrt{2} \int_0^1 x \sin(\pi nx) \diff x = \sqrt{2} \frac{(-1)^n}{\pi n}
        \end{equation*}

        Згідно до формули Шмідта розв'язок рівняння при $\lambda \ne \lambda_k$ має вигляд:
        \begin{equation*}
            \phi(x) = x - 2 \lambda \Sum_{k=1}^\infty \dfrac{(-1)^{k+1}\sin(\pi k x)}{((\pi k)^2+\lambda)\pi k}
        \end{equation*}

        При $\lambda = \lambda_k$ розв'язок не існує, оскільки не виконана умова ортогональності вільного члена до власної функції.
    \end{enumerate}
\end{solution}

\end{document}