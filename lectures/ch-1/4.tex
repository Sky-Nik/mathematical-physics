 \documentclass[a4paper, 12pt]{article}
\usepackage[utf8]{inputenc}
\usepackage[english, ukrainian]{babel}

\usepackage{amsmath, amssymb}
\usepackage{multicol}
\usepackage{graphicx}
\usepackage{float}

\allowdisplaybreaks
\setlength\parindent{0pt}
\numberwithin{equation}{subsection}

\usepackage{hyperref}
\hypersetup{unicode=true,colorlinks=true,linktoc=all,linkcolor=red}

\numberwithin{equation}{subsection}

\renewcommand{\bf}[1]{\textbf{#1}}
\renewcommand{\it}[1]{\textit{#1}}
\newcommand{\bb}[1]{\mathbb{#1}}
\renewcommand{\cal}[1]{\mathcal{#1}}

\renewcommand{\epsilon}{\varepsilon}
\renewcommand{\phi}{\varphi}

\DeclareMathOperator{\diam}{diam}
\DeclareMathOperator{\rang}{rang}
\DeclareMathOperator{\const}{const}

\newenvironment{system}{%
  \begin{equation}%
    \left\{%
      \begin{aligned}%
}{%
      \end{aligned}%
    \right.%
  \end{equation}%
}
\newenvironment{system*}{%
  \begin{equation*}%
    \left\{%
      \begin{aligned}%
}{%
      \end{aligned}%
    \right.%
  \end{equation*}%
}

\makeatletter
\newcommand*{\relrelbarsep}{.386ex}
\newcommand*{\relrelbar}{%
  \mathrel{%
    \mathpalette\@relrelbar\relrelbarsep%
  }%
}
\newcommand*{\@relrelbar}[2]{%
  \raise#2\hbox to 0pt{$\m@th#1\relbar$\hss}%
  \lower#2\hbox{$\m@th#1\relbar$}%
}
\providecommand*{\rightrightarrowsfill@}{%
  \arrowfill@\relrelbar\relrelbar\rightrightarrows%
}
\providecommand*{\leftleftarrowsfill@}{%
  \arrowfill@\leftleftarrows\relrelbar\relrelbar%
}
\providecommand*{\xrightrightarrows}[2][]{%
  \ext@arrow 0359\rightrightarrowsfill@{#1}{#2}%
}
\providecommand*{\xleftleftarrows}[2][]{%
  \ext@arrow 3095\leftleftarrowsfill@{#1}{#2}%
}
\makeatother

\newcommand{\NN}{\mathbb{N}}
\newcommand{\ZZ}{\mathbb{Z}}
\newcommand{\QQ}{\mathbb{Q}}
\newcommand{\RR}{\mathbb{R}}
\newcommand{\CC}{\mathbb{C}}

\newcommand{\Max}{\displaystyle\max\limits}
\newcommand{\Sup}{\displaystyle\sup\limits}
\newcommand{\Sum}{\displaystyle\sum\limits}
\newcommand{\Int}{\displaystyle\int\limits}
\newcommand{\Iint}{\displaystyle\iint\limits}
\newcommand{\Lim}{\displaystyle\lim\limits}

\newcommand*\diff{\mathop{}\!\mathrm{d}}

\newcommand*\rfrac[2]{{}^{#1}\!/_{\!#2}}


 \title{{\Huge МАТЕМАТИЧНА ФІЗИКА}}
 \author{Скибицький Нікіта}
 \date{\today}

 \usepackage{amsthm}
\usepackage[dvipsnames]{xcolor}
\usepackage{thmtools}
\usepackage[framemethod=TikZ]{mdframed}

\theoremstyle{definition}
\mdfdefinestyle{mdbluebox}{%
	roundcorner = 10pt,
	linewidth=1pt,
	skipabove=12pt,
	innerbottommargin=9pt,
	skipbelow=2pt,
	nobreak=true,
	linecolor=blue,
	backgroundcolor=TealBlue!5,
}
\declaretheoremstyle[
	headfont=\sffamily\bfseries\color{MidnightBlue},
	mdframed={style=mdbluebox},
	headpunct={\\[3pt]},
	postheadspace={0pt}
]{thmbluebox}

\mdfdefinestyle{mdredbox}{%
	linewidth=0.5pt,
	skipabove=12pt,
	frametitleaboveskip=5pt,
	frametitlebelowskip=0pt,
	skipbelow=2pt,
	frametitlefont=\bfseries,
	innertopmargin=4pt,
	innerbottommargin=8pt,
	nobreak=true,
	linecolor=RawSienna,
	backgroundcolor=Salmon!5,
}
\declaretheoremstyle[
	headfont=\bfseries\color{RawSienna},
	mdframed={style=mdredbox},
	headpunct={\\[3pt]},
	postheadspace={0pt},
]{thmredbox}

\declaretheorem[style=thmbluebox,name=Теорема,numberwithin=subsubsection]{theorem}
\declaretheorem[style=thmbluebox,name=Лема,numberwithin=subsubsection]{lemma}
\declaretheorem[style=thmbluebox,name=Твердження,numberwithin=subsubsection]{proposition}
\declaretheorem[style=thmbluebox,name=Принцип,numberwithin=subsubsection]{th_principle}
\declaretheorem[style=thmbluebox,name=Закон,numberwithin=subsubsection]{law}
\declaretheorem[style=thmbluebox,name=Закон,numbered=no]{law*}
\declaretheorem[style=thmbluebox,name=Формула,numberwithin=subsubsection]{th_formula}
\declaretheorem[style=thmbluebox,name=Рівняння,numberwithin=subsubsection]{th_equation}
\declaretheorem[style=thmbluebox,name=Умова,numberwithin=subsubsection]{th_condition}
\declaretheorem[style=thmbluebox,name=Наслідок,numberwithin=subsubsection]{corollary}

\declaretheorem[style=thmredbox,name=Приклад,numberwithin=subsubsection]{example}
\declaretheorem[style=thmredbox,name=Приклади,sibling=example]{examples}

\declaretheorem[style=thmredbox,name=Властивість,numberwithin=subsubsection]{property}
\declaretheorem[style=thmredbox,name=Властивості,sibling=property]{properties}

\mdfdefinestyle{mdgreenbox}{%
	skipabove=8pt,
	linewidth=2pt,
	rightline=false,
	leftline=true,
	topline=false,
	bottomline=false,
	linecolor=ForestGreen,
	backgroundcolor=ForestGreen!5,
}
\declaretheoremstyle[
	headfont=\bfseries\sffamily\color{ForestGreen!70!black},
	bodyfont=\normalfont,
	spaceabove=2pt,
	spacebelow=1pt,
	mdframed={style=mdgreenbox},
	headpunct={ --- },
]{thmgreenbox}

\mdfdefinestyle{mdblackbox}{%
	skipabove=8pt,
	linewidth=3pt,
	rightline=false,
	leftline=true,
	topline=false,
	bottomline=false,
	linecolor=black,
	backgroundcolor=RedViolet!5!gray!5,
}
\declaretheoremstyle[
	headfont=\bfseries,
	bodyfont=\normalfont\small,
	spaceabove=0pt,
	spacebelow=0pt,
	mdframed={style=mdblackbox}
]{thmblackbox}

\declaretheorem[name=Вправа,numberwithin=subsubsection,style=thmblackbox]{exercise}
\declaretheorem[name=Зауваження,numberwithin=subsubsection,style=thmgreenbox]{remark}
\declaretheorem[name=Визначення,numberwithin=subsubsection,style=thmblackbox]{definition}

\newtheorem{problem}{Задача}[subsection]
\newtheorem{sproblem}[problem]{Задача}
\newtheorem{dproblem}[problem]{Задача}
\renewcommand{\thesproblem}{\theproblem$^{\star}$}
\renewcommand{\thedproblem}{\theproblem$^{\dagger}$}
\newcommand{\listhack}{$\empty$\vspace{-2em}} 

\theoremstyle{remark}
\newtheorem*{solution}{Розв'язок}


 \begin{document}

 \tableofcontents

 %% lecture 1, not compilable		
\section{Вступ}

\subsection{Предмет і методи математичної фізики}

Сучасні технології дослідження реального світу доволі інтенсивно використовують методи математичного моделювання, зокрема ці методи широко використовуються тоді, коли дослідження реального (фізичного) об’єкту є неможливими, або надто дорогими. Вже традиційними стали моделювання властивостей таких фізичних об’єктів:
\begin{itemize}
	\item температурні поля і теплові потоки;
	\item електричні, магнітні та електромагнітні поля;
	\item концентрація речовини в розчинах, розплавах або сумішах;
	\item напруження і деформації в пружних твердих тілах;
	\item параметри рідини або газу, який рухається (обтікає) деяке тіло;
	\item перенос різних субстанцій потоками рідин або газу та інші.
\end{itemize}

Характерною особливістю усіх математичних моделей, що описують перелічені та багато інших процесів є те, що параметри, які представляють інтерес для дослідника є функціями точки простору $\bf{x} = (x_1, x_2, x_3)$ та часу $t$, а самі співвідношення з яких ці характеристики обчислюються є диференціальними рівняннями в частинних похідних зі спеціальними додатковими умовами (крайовими умовами), які дозволяють виділяти однозначний розв’язок. \\

Таким чином можна сказати, що основними об’єктами дослідження предмету математична фізика є крайові задачі для рівнянь в частинних похідних, які моделюють певні фізичні процеси. \\

Процес дослідження реального об’єкту фізичного світу можна представити за наступною схемою:
\begin{enumerate}
	\item Побудова математичної моделі реального процесу у вигляді диференціального рівняння або системи диференціальних рівнянь в частинних похідних, доповнення диференціального рівняння в частинних похідних граничними умовами.
	\item Дослідження властивостей сформульованої крайової задачі з точки зору її коректності. Коректність постановки задачі передбачає виконання наступних умов:
	\begin{itemize}
		\item Розв’язок крайової задачі існує;
		\item Розв’язок єдиний;
		\item Розв’язок неперервним чином залежить від вхідних даних задачі.
	\end{itemize}
	\item Знаходження розв’язку крайової задачі: точного для найбільш простих задач, або наближеного для переважної більшості задач.
\end{enumerate}

Треба відмітити, що усі перелічені пункти дослідження окрім побудови наближених методів знаходження розв’язків відносяться до предмету дисципліни Математична фізика. \\

Для дослідження задач математичної фізики використовуються математичний апарат наступних розділів математики:
\begin{itemize}
	\item математичний аналіз;
	\item лінійна алгебра;
	\item диференціальні рівняння;
	\item теорія функцій комплексної змінної;
	\item функціональний аналіз;
\end{itemize}

При побудові математичних моделей використовуються знання з елементарної фізики. \\

Наведемо приклад доволі простої і в той же час цілком реальної математичної моделі розповсюдження тепла в стрижні. \\

Нехай ми маємо однорідний стрижень з теплоізольованою боковою поверхнею і наступними фізичними параметрами:
\begin{itemize}
	\item $\rho$ -- густина матеріалу;
	\item $S$ -- площа поперечного перерізу;
	\item $k$ -- коефіцієнт теплопровідності;
	\item $c$ -- коефіцієнт теплоємності;
	\item $L$ -- довжина стрижня.
\end{itemize}

Позначимо $u(x, t)$ -- температуру стрижня в точці $x$ в момент часу $t$, $u_0(x)$ -- температуру стрижня у точці $x$ в початковий момент часу $t = 0$. \\

Припустимо, що на лівому кінці стрижня температура змінюється за заданим законом $\phi(t)$, а правий кінець стрижня теплоізольований. \\

В таких припущеннях математична модель може бути записана у вигляді наступної граничної задачі:
\begin{equation}
	\label{eq:1.1.1}
	c \rho \cdot \dfrac{\partial u(x, t)}{\partial t} = k \cdot \dfrac{\partial^2 u(x, t)}{\partial x^2}, \quad 0 < x < L, \quad t > 0
\end{equation}
\begin{equation}
	\label{eq:1.1.2}
	u(0, u) = \phi(t), \quad  \dfrac{\partial u(L, t)}{\partial x} = 0
\end{equation}
\begin{equation}
	\label{eq:1.1.3}
	u(x, 0) = u_0(t)
\end{equation}

Математична модель містить диференціальне рівняння (\ref{eq:1.1.1}), яке виконується для вказаних значень аргументу, граничні умови на кінцях стрижня (\ref{eq:1.1.2}) та початкову умови (\ref{eq:1.1.3}).

\section{Інтегральні рівняння}

\subsubsection{Основні поняття}

Інтегральні рівняння -- рівняння, що містять невідому функцію під знаком інтегралу. \\

Багато задач математичної фізики зводяться до лінійних інтегральних рівнянь виду:

\begin{equation}
	\label{eq:2.1}
	\phi(x) = \lambda \Int_G K(x, y) \phi(y) \diff y + f(x)
\end{equation}
-- інтегральне рівняння Фредгольма II роду.

\begin{equation}
	\label{eq:2.2}
	\Int_G K(x, y) \phi(y) \diff y = f(x)
\end{equation}
-- інтегральне рівняння Фредгольма I роду. \\

$K(x, y)$ -- ядро інтегрального рівняння, $K(x, y) \in C\left(\bar G \times \bar G\right)$, $f(x)$ -- вільний член інтегрального рівняння, $f(x) \in C\left(\bar G\right)$, $\lambda$ -- комплексний параметр, $\lambda \in \CC$ (відомий або невідомий), $G$ -- область інтегрування, $G \subseteq \RR^n$, $\bar G$ -- замкнена та обмежена. \\

Інтегральне рівняння \eqref{eq:2.1} при $f(x) \equiv 0$ називається однорідним інтегральним рівнянням Фредгольма II роду
\begin{equation}
	\label{eq:2.3}
	\phi(x) = \lambda \Int_G K(x, y) \phi(y) \diff y.
\end{equation}

$\bf{K}$ -- інтегральний оператор: $(\bf{K} \phi)(x)$. Будемо записувати інтегральні рівняння \eqref{eq:2.1}, \eqref{eq:2.2} та \eqref{eq:2.3} скорочено в операторній формі:
\begin{align}
	\label{eq:2.4}
	\phi &= \lambda \bf{K} \phi + f, \\
	\label{eq:2.5}
	\bf{K} \phi &= f, \\
	\label{eq:2.6}
	\phi &= \lambda \bf{K} \phi.
\end{align}

\begin{equation}
	\label{eq:2.7}
	K^*(x, y) = \bar K(y, x)
\end{equation}
-- спряжене (союзне) ядро. Інтегральне рівняння
\begin{equation}
	\label{eq:2.8}
	\psi(x) = \bar \lambda \Int_G K^*(x, y) \psi(y) \diff y + g(x)
\end{equation}
називається спряженим (союзним) до інтегрального рівняння \eqref{eq:2.1}. Операторна форма:
\begin{align}
	\label{eq:2.9}
	\psi &= \bar \lambda \bf{K}^* \psi + g, \\
	\label{eq:2.10}
	\psi &= \bar \lambda \bf{K}^* \psi.
\end{align}

\begin{definition*}
	Комплексні значення $\lambda$, при яких однорідне інтегральне рівняння Фредгольма \eqref{eq:2.3} має нетривіальні розв’язки, називаються характеристичними числами ядра $K(x, y)$. Розв’язки, які відповідають власним числам, називаються власними функціями. Кількість лінійно-незалежних власних функцій називається кратністю характеристичного числа.
\end{definition*}

\subsection{Метод послідовних наближень}

\subsubsection{Метод послідовних наближень для неперервного ядра}

Нагадаємо означення норм в банаховому просторі неперервних функцій $C(\bar G)$ та гільбертовому просторі інтегрованих з квадратом функцій $L_2(G)$ та означення скалярного добутку в просторі $L_2(G)$:
\begin{align} 
	\label{eq:2.1.1}
	\|f\|_{C(\bar G)} &= \Max_{x \in \bar G} |f(x)|, \\
	\label{eq:2.1.2}
	\|f\|_{L_2(G)} &= \left( \Int_G |f(x)|^2 \diff x \right)^{1/2}, \\
	\label{eq:2.1.3}
	(f, g)_{L_2(G)} &= \Int_G f(x) \bar g(x) \diff x.
\end{align}

\begin{lemma} 
	Інтегральний оператор $\bf{K}$ з неперервним ядром $K(x, y)$ петворює множини функцій $C(\bar G) \xrightarrow{\bf{K}} C(\bar G)$, $L_2(G) \xrightarrow{\bf{K}} L_2(G)$, $L_2(G) \xrightarrow{\bf{K}} C(\bar G)$ обмежений та мають місце нерівності:
	\begin{align}
		\label{eq:2.1.4}
		\| \bf{K} \phi \|_{C(G)} &\le M V \| \phi \|_{C(G)}, \\
		\label{eq:2.1.5}
		\| \bf{K} \phi \|_{L_2(G)} &\le M V \| \phi \|_{L_2(G)}, \\
		\label{eq:2.1.6}
		\| \bf{K} \phi \|_{C(G)} &\le M \sqrt{V} \| \phi \|_{L_2(G)},
	\end{align}
	де
	\begin{align}
		\label{eq:2.1.7}
		M &= \Max_{x, y \in G \times G} |K(x, y)|, \\
		\label{eq:2.1.8}
		V &= \Int_G \diff y.
	\end{align}
\end{lemma}

\begin{proof}
	Нехай $\phi \in L_2(G)$. Тоді $\phi$ -- абсолютно інтегрована функція на $G$ і, оскільки ядро $K(x, y)$ неперервне на $G \times G$, функція $(\bf{K}\phi)(x)$ неперервна на $G$. Тому оператор $\bf{K}$ переводить $L_2(G)$ в $C(\bar G)$ і, з врахуванням нерівності Коші-Буняковського, обмежений. Доведемо нерівності:
	\begin{enumerate}
		\item \eqref{eq:2.1.4}:
		\begin{multline*}
			\| \bf{K} \phi \|_{C(\bar G)} = \Max_{x \in \bar G} \left| \Int_G K(x, y) \phi(y) \diff y \right| \le \Max_{x \in \bar G} \Int_G \left( |K(x, y)| \cdot |\phi(y)| \right) \diff y \le \\
			\le \Max_{x \in \bar G} \left( \Max_{y \in \bar G} |K(x, y)| \cdot \Max_{y \in \bar G} |\phi(y)| \cdot \Int_G \diff y \right) \le \\
			\le \Max_{x, y \in \bar G \times \bar G} |K(x, y)| \cdot \Max_{y \in \bar G} |\phi(y)| \cdot \Int_G \diff y = M V \|\phi\|_{C(\bar G)}.
		\end{multline*}
		\item \eqref{eq:2.1.5}:
		\begin{multline*}
			\left( \| \bf{K} \phi \|_{L_2(G)} \right)^2 = \Int_G \left| \Int_G K(x, y) \phi(y) \diff y \right|^2 \diff x \le \\
			\le \Int_G \left| \Max_{y \in \bar G} |K(x, y)| \cdot \Int_G \phi(y) \diff y \right|^2 \diff x \le \\
			\le \left( \Max_{x, y \in \bar G \times \bar G} |K(x, y)| \right)^2 \cdot \left| \Int_G \phi(y) \diff y \right|^2 \cdot \Int_G \diff x \le (M \| \phi\|_{L_2(G)} V)^2
		\end{multline*}
		\item \eqref{eq:2.1.6}:
		\begin{multline*}
			\| \bf{K} \phi \|_{C(\bar G)} = \Max_{x \in \bar G} |(\bf{K} \phi) (x)| = \Max_{x \in \bar G} \left| \Int_G K(x, y) \phi(y) \diff y \right| \le \\
			\le \Max_{x \in \bar G} \sqrt{\Int_G |K(x, y)|^2 \diff y} \cdot \sqrt{\Int_G |\phi(y)|^2 \diff y} \le M \sqrt{V} \|\phi\|_{L_2(G)}.
		\end{multline*}
	\end{enumerate}
\end{proof}

Розв’язок інтегрального рівняння другого роду \eqref{eq:2.4} будемо шукати методом послідовних наближень:
\begin{equation}
	\label{eq:2.1.9}
	\phi_0 = f, \quad \phi_1 = \lambda \bf{K} \phi_0 + f, \quad \phi_2 = \lambda \bf{K} \phi_1 + f, \quad \ldots, \quad \phi_{n + 1} = \lambda \bf{K} \phi_n + f
\end{equation}
\begin{equation}
	\label{eq:2.1.10}
	\phi_{n + 1} = \Sum_{i = 0}^{n + 1} \lambda^i \bf{K}^i f, \quad \bf{K}^{i + 1} = \bf{K} (\bf{K}^i)
\end{equation}
\begin{equation}
	\label{eq:2.1.11}
	\phi_\infty = \Lim_{n \to \infty} \phi_{n} = \Sum_{i = 0}^\infty \lambda^i \bf{K}^i f,
\end{equation}
ряд Неймана. Дослідимо збіжність ряду Неймана \eqref{eq:2.1.11}
\begin{multline}
	\label{eq:2.1.12}
	\left\| \Sum_{i = 0}^\infty \lambda^i \bf{K}^i f \right\|_{C(\bar G)} \le \Sum_{i = 0}^\infty |\lambda^i| \cdot \| \bf{K}^i f \|_{C(\bar G)} \le \\
	\le \Sum_{i = 0}^\infty |\lambda^i| \cdot (MV)^i \cdot \| f \|_{C(\bar G)} = \dfrac{\|f\|_{C(\bar G)}}{1 - |\lambda| MV}.
\end{multline}
Справдві, $\| \bf{K} \phi\|_{C(\bar G)} \le MV \|\phi\|_{C(\bar G)}$, тому $\|\bf{K}^2\phi\|_{C(\bar G)} \le (MV)^2 \|\phi\|_{C(\bar G)}$ і, взагалі кажучи, $\|\bf{K}^i\phi\|_{C(\bar G)} \le (MV)^i \|\phi\|_{C(\bar G)}$. \\

Отже, ряд Неймана збігається рівномірно при 
\begin{equation}
	\label{eq:2.1.13}
	|\lambda| < \dfrac{1}{MV},
\end{equation} 
умова збіжності методу послідовних наближень. \\

Покажемо, що при виконанні умови \eqref{eq:2.1.13} інтегральне рівняння \eqref{eq:2.1} має єдиний розв’язок. Дійсно припустимо, що їх два:

\begin{equation*}
	\begin{matrix}
		\phi^{(1)} = \lambda \bf{K} \phi^{(1)} + f \\
		\phi^{(2)} = \lambda \bf{K} \phi^{(2)} + f
	\end{matrix}
	\implies
	\begin{matrix}
		\phi^{(0)} = \phi^{(1)} - \phi^{(2)}  \\
		\phi^{(0)} = \lambda \bf{K} \phi^{(0)}
	\end{matrix}
\end{equation*}

Обчислимо норму Чебишева: 
\begin{multline*} 
	|\lambda| \cdot \|\bf{K} \phi^{(0)}\|_{C(\bar G)} = \| \phi^{(0)} \|_{C(\bar G)} \Rightarrow \\
	\Rightarrow \| \phi^{(0)} \|_{C(\bar G)} \le |\lambda| \cdot MV \cdot \|\phi^{(0)}\|_{C(\bar G)} \Rightarrow \\
	\Rightarrow (1 - |\lambda| \cdot MV) \cdot \|\phi^{(0)}\|_{C(\bar G)} \le 0.
\end{multline*}

Звідси маємо, що $\|\phi^{(0)}\|_{C(\bar G)} = 0$. Таким чином доведена теорема

\begin{theorem}[Про існування розв’язку інтегрального рівняння Фредгольма з неперервним
ядром для малих значень параметру]
	Будь-яке інтегральне рівняння Фредгольма другого роду \eqref{eq:2.1} з неперервним ядром $K(x, y)$ при умові \eqref{eq:2.1.13} має єдиний розв’язок $\phi$ в класі неперервних функцій $C(\bar G)$ для будь-якого неперервного вільного члена $f$. Цей роз\-в’я\-зок може бути знайдений у вигляді ряду Неймана \eqref{eq:2.1.11}.
\end{theorem}

\subsubsection{Повторні ядра}

$\forall f, g \in \bar G$ має місце рівність 
\begin{equation}
	\label{eq:2.1.14}
	(\bf{K}f,g)_{L_2(G)} = (f, \bf{K}^*g)_{L_2(G)}
\end{equation}
Дійсно, якщо $f, g \in L_2(G)$, то за лемою 1 $\bf{K}f, \bf{K}^*g \in L_2(G)$ тому
\begin{multline*}
	(\bf{K}f, g) = \Int_G (\bf{K}f)\bar g \diff x = \Int_G \left( \Int_G K(x, y) f(y) \diff y\right) \bar g(x) \diff x = \\
	= \Int_G f(y) \left( \Int_G K(x, y) \bar g(x) \diff x\right) \diff y = \Int_G f(y) \cdot (\bf{K}^* g)(y) \diff y = (f, \bf{K}^*g).
\end{multline*}

\begin{lemma}
	Якщо $\bf{K}_1$, $\bf{K}_2$ -- інтегральні оператори з неперервними ядрами $K_1(x, y)$, $K_2(x, y)$ відповідно, то оператор $\bf{K}_3 = \bf{K}_2 \bf{K}_1$ також інтегральний оператор з неперервним ядром
	\begin{equation}
		\label{eq:2.1.15}
		K_3(x, z) = \int_G K_2(x, y) K_1(y, z) \diff y.
	\end{equation}
	При цьому справедлива формула: $(\bf{K}_2\bf{K}_1)^* = \bf{K}_1^* \bf{K}_2^*$.
\end{lemma}
\begin{proof}
	Нехай $K_1(x, y)$, $K_2(x, y)$ -- ядра інтегральних операторів $\bf{K}_1$, $\bf{K}_2$. Розглянемо $\bf{K}_3 = \bf{K}_2 \bf{K}_1$:
	\begin{multline*}
		(\bf{K}_3 f)(x) = (\bf{K}_2\bf{K}_1f)(x) = \Int_G K_2(x, y) \left( \Int_G K_1(y, z) f(z) \diff z \right) \diff y = \\
		= \Int_G \left( \Int_G K_2(x, y) K_1(y, z) \diff y\right) f(z) \diff z = \Int_G K_3(x, z) f(z) \diff z.
	\end{multline*}
	Тобто \eqref{eq:2.1.15} -- ядро оператора $\bf{K}_2\bf{K}_1$. \\

	З рівності \eqref{eq:2.1.14} для всіх $f, g \in L_2(G)$ отримуємо $(f, \bf{K}_3^*g - \bf{K}_1^* \bf{K}_2^* g) = 0$, звідки випливає, що $\bf{K}_3^* = \bf{K}_1^* \bf{K}_2^*$.
\end{proof}

Із доведеної леми випливає, що оператори $\bf{K}^n = \bf{K} (\bf{K}^{n - 1}) = (\bf{K}^{n - 1})\bf{K}$ -- інтегральні та їх ядра $K_{(n)}(x, y)$ -- неперервні та задовольняють рекурентним співвідношенням:
\begin{equation}
	\label{eq:2.1.16}
	K_{(1)}(x, y) = K(x, y), \quad \ldots, \quad K_{(n)}(x, y) = \Int_G K(x, \xi) K_{(n - 1)}(\xi, y) \diff \xi
\end{equation}
-- повторні (ітеровані) ядра. Операторна форма:
\begin{equation}
	\label{eq:2.1.17}
	\bf{K}f = \Int_G K(x, y) f(y) \diff y, \quad \ldots, \quad \bf{K}^n f = \Int_G K_{(n)}(x, y)f(y) \diff y.
\end{equation}

\subsubsection{Резольвента інтегрального оператора}

Пригадаємо представлення розв’язку інтегрального рівняння \ref{eq:2.1} у вигляді ряду Неймана \eqref{eq:2.1.11}. Виконаємо перетворення
\begin{multline*}
	\phi(x) = f(x) + \lambda \Sum_{i = 1}^\infty \lambda^{i - 1} (\bf{K}^i f) x = f(x) + \Sum_{i = 1}^\infty \lambda^{i - 1} K_{(i)} (x, y) f(y) \diff y = \\
	= f(x) + \lambda \Int_G \left( \Sum_{i = 1}^\infty \lambda^{i - 1} K_{(i)} (x, y) \right) f(y) \diff y = f(x) + \lambda \Int_G \mathcal{R}(x, y, \lambda) f(y) \diff y,
\end{multline*}
при $|\lambda| < \frac{1}{MV}$, де
\begin{equation}
	\label{eq:2.1.18}
	\mathcal{R}(x, y, \lambda) = \Sum_{i = 1}^\infty \lambda^{i - 1} K_{(i)} (x, y)
\end{equation}
-- резольвента інтегрального оператора. Операторна форма запису розв’язку рівняння Фредгольма через резольвенту ядра має вигляд:
\begin{equation}
	\label{eq:2.1.19}
	\phi = f + \lambda \bf{R} f
\end{equation}

Мають місце операторні рівності:
\begin{equation}
	\label{eq:2.1.20}
	\phi = (E + \lambda \bf{R})f, \quad (E - \lambda \bf{K})\phi = f,\quad \phi = (E - \lambda \bf{K})^{-1}f.
\end{equation}

Таким чином маємо
\begin{equation}
	\label{eq:2.1.21}
	E + \lambda \bf{R} = (E - \lambda \bf{K})^{-1}, \quad |\lambda| < \dfrac{1}{MV}.
\end{equation}

Зважуючи на формулу \eqref{eq:2.1.19} має місце теорема
\begin{theorem}[Про існування розв’язку інтегрального рівняння Фредгольма з неперервним
ядром для малих значенням параметру]
	Будь-яке інтегральне рівняння Фредгольма другого роду \eqref{eq:2.1} з неперервним ядром $K(x, y)$ при умові \eqref{eq:2.1.13} має єдиний розв’язок $\phi$ в класі неперервних функцій $C(\bar G)$ для будь-якого неперервного вільного члена $f$. Цей розв’язок може бути знайдений у вигляді \eqref{eq:2.1.18} за допомогою резольвенти \eqref{eq:2.1.18}.
\end{theorem}

\begin{example}
	Методом послідовних наближень знайти розв’язок інтегрального рівняння \[\phi(x) = x + \lambda \Int_0^1 (xt)^2 \phi(t) \diff t.\]
\end{example}
\begin{solution*}
	$M = 1$, $V = 1$. \\

	Побудуємо повторні ядра 
	\begin{align*} 
		K_{(1)}(x, t) &= x^2t^2, \\
		K_2(x, t) &= \Int_0^1 x^2 z^4 t^2 \diff z = \dfrac{x^2t^2}{5}, \\ 
		K_{(p)}(x, t) &= \dfrac{1}{5^{p - 2}} \Int_0^1 x^2 z^4 t^2 \diff z = \dfrac{x^2t^2}{5^{p - 1}}.
	\end{align*}
	
	Резольвента має вигляд \[\mathcal{R}(x, t, \lambda) = x^2 t^2 \left(1 + \frac{\lambda}{5} + \frac{\lambda^2}{5^2} + \ldots + \frac{\lambda^p}{5^p} + \ldots \right) = \frac{5x^2t^2}{5 - \lambda}, \quad |\lambda| < 5. \]

	Розв’язок інтегрального рівняння має вигляд: \[ \phi(x) + x + \Int_0^1 \dfrac{5x^2t^3}{5 - \lambda} \diff t = x + \dfrac{5x^2}{4(5 - \lambda)}. \]
\end{solution*}
 %%{Лекція 2}

\subsubsection{Метод послідовних наближень для інтегральних рівнянь з полярним ядром}

Ядро $K(x, y)$ називається полярним, якщо воно представляється у вигляді:
\begin{equation}
	\label{eq:1.20}
	K(x, y) = \dfrac{A(x, y)}{|x - y|^\alpha}
\end{equation}
де $A \in C(\bar G \times \bar G)$, $|x - y| = \left( \sum_{i = 1}^n (x_i - y_i)^2 \right)^{1/2}$, $\alpha < n$ ($n$ -- розмірність евклідового простору). \\

Ядро називається слабо полярним, якщо $\alpha < n / 2$. \\

Метод послідовних наближень для інтегральних рівнянь з неперервним ядром мав вигляд: 
\begin{multline*}
\phi(x) = \lambda \Int_G K(x, y) \phi(x, y) \diff y + f(x), \\
\phi_0 = f, \quad \phi_1 = f + \lambda \bf{K} \phi_0, \quad \ldots, \quad \phi_{n + 1} = f + \lambda \bf{K} \phi_n.
\end{multline*}

Оцінки, що застосовувались для неперервних ядер не працюють для полярних ядер, тому що максимум полярного ядра рівний нескінченності (ядро необмежене в рівномірній метриці), отже, сформулюємо лему аналогічну лемі 1 для полярних ядер. 
\begin{lemma}
	Інтегральний оператор $\bf{K}$ з полярним ядром $K(x, y)$ переводить множину функцій $C(\bar G) \xrightarrow{\bf{K}} C (\bar G)$ і при цьому має місце оцінка: 
	\begin{equation}
		\label{eq:1.21}
		\| \bf{K} \phi \|_{C(\bar G)} \le N \| \phi\|_{C(\bar G)},
	\end{equation}
	де 
	\begin{equation}
		\label{eq:1.22}
		N = \Max_{x \in \bar G} \Int_G |K(x, y) \diff y|.
	\end{equation}
\end{lemma}
\begin{proof}
	Спочатку доведемо, що функція $\bf{K}\phi$ неперервна в точці $x_0$. \\

	Оцінимо при умові $|x - x_0| < \eta / 2$ вираз:
	\begin{multline*}
		\left| \Int_G K(x, y) \phi(y) \diff y - \Int_G K(x_0, y) \phi(y) \diff y \right| = \\
		= \left| \Int_G \dfrac{A(x, y)}{|x - y|^\alpha} \phi(y) \diff y - \Int_G \dfrac{A(x_0, y)}{|x_0 - y|^\alpha} \phi(y) \diff y \right| \le \\
		\le \Int_G \left|\dfrac{A(x, y)}{|x - y|^\alpha} - \dfrac{A(x_0, y)}{|x_0 - y|^\alpha}\right| |\phi(y)| \diff y \le (*)
	\end{multline*}
	винесемо $\max \phi(y)$ у вигляді $\|\phi\|_{C(\bar G)}$, а інтеграл розіб’ємо на два інтеграли: інтеграл по $U(x_0, \eta)$ -- кулі з центром в $x_0$ і радіусом $\eta$; інтеграл по залишку $G \setminus U(x_0, \eta)$.
	
	\begin{multline*} 
		(*) \le \|\phi\|_{C(\bar G)} \left( \Int_{U(x_0, \eta)} \left|\dfrac{A(x, y)}{|x - y|^\alpha} - \dfrac{A(x_0, y)}{|x_0 - y|^\alpha}\right| \diff y\right. + \\
		+ \left.\Int_{G \setminus U(x_0, \eta)} \left|\dfrac{A(x, y)}{|x - y|^\alpha} - \dfrac{A(x_0, y)}{|x_0 - y|^\alpha}\right| \diff y\right)
	\end{multline*}
	
	Оцінимо тепер кожний з інтегралів:
	
	\[ \Int_{U(x_0, \eta)} \left|\dfrac{A(x, y)}{|x - y|^\alpha} - \dfrac{A(x_0, y)}{|x_0 - y|^\alpha}\right| \diff y \le A_0 \Int_{U(x_0, \eta)} \left|\dfrac{\diff y}{|x - y|^\alpha} - \dfrac{\diff y}{|x_0 - y|^\alpha}\right|, \]

	де $A_0$ -- $\max$ функції $A(x, y)$ на потрібній множині. \\ 

	Введемо узагальнені сферичні координати з центром у точці $x_0$ в просторі $\RR^n$:
	\begin{align*} 
		y_1 &= x_{0, 1} + \rho \cos \nu_1 \\
		y_2 &= x_{0, 2} + \rho \sin \nu_1 \cos \nu_2 \\
		\ldots \\
		y_{n - 1} &= x_{0, n - 1} + \rho \sin \nu_1 \cdot \ldots \cdot \cos \nu_{n - 1} \\
		y_n &= x_{0, n} + \rho \sin \nu_1 \cdot \ldots \cdot \sin \nu_{n - 1}
	\end{align*}

	Якобіан переходу має вигляд:
	\[ \dfrac{D(y_1, \ldots, y_n)}{\rho, \nu_1, \ldots, \nu_{n - 1}} = \rho^{n - 1} \Phi(\sin \nu_1, \ldots, \sin \nu_{n - 1}, \cos \nu_1, \ldots, \cos \nu_{n - 1}), \]

	де $0 \le \rho \le \eta, 0 \le \nu_i \le \pi, i = \overline{1, n - 2}, 0 \le \nu_{n - 1} \le 2 \pi$. \\

	Отримаємо \[ \Int_{U(x_0, \eta)} \dfrac{\diff y}{|x_0 - y|^\alpha} = \sigma_n \Int_0^\eta \dfrac{\rho^{n - 1} \diff \rho}{\rho^\alpha} = \sigma_n \left.\dfrac{\rho^{n - \alpha}}{n - \alpha}\right|_0^\eta = \dfrac{\sigma_n \eta^{n - \alpha}}{n - \alpha} \le \dfrac{\epsilon}{4}, \]
	де $\sigma_n$ -- площа поверхні одиничної сфери в $n$-вимірному просторі $\RR^n$. \\

	Оскільки $|x - x_0| < \eta / 2$, то \[ \Int_{U(x_0, \eta)} \dfrac{\diff y}{|x - y|^\alpha} \le \Int_{U(x_0, 3\eta/2)} \dfrac{\diff y}{|x_0 - y|^\alpha} \le \dfrac{\sigma_n}{n - \alpha} \left(\dfrac{3\eta}{2}\right)^{n - \alpha} \le \dfrac{\epsilon}{4}. \] 

	Оскільки $\frac{A(x, y)}{|x - y|^\alpha} \in C\left(\overline{U (x_0, \eta/2)}\times\overline{G \setminus U (x_0, \eta)}\right)$, то
	\[ \Int_{G \setminus U(x_0, \eta)} \left|\dfrac{A(x, y)}{|x - y|^\alpha} - \dfrac{A(x_0, y)}{|x_0 - y|^\alpha}\right| \diff y \le \dfrac{\epsilon}{2}. \]

	Таким чином ми довели, що $\left| \int_G K(x, y) \phi(y) \diff y - \int_G K(x_0, y) \phi(y) \diff y \right| \le \epsilon$, тобто функція $\bf{K}\phi$ неперервна в точці $x_0$. \\

	Доведемо оцінку $\| \bf{K}\phi \|_{C(\bar G)} \le N \|\phi\_{C(\bar G)}$, де $N = \max_{x \in \bar G} \int_G |K(x, y) \diff y|$:
	\begin{multline*}
		\left| \Int_G K(x, y) \phi(y) \diff y \right| \le \Int_G |K(x, y)| |\phi(y)| \diff y \le \|\phi\|_{C(\bar G)} \Int_G |K(x, y)| \le \\
		\le |\phi\|_{C(\bar G)} \Max_{x \in \bar G} \Int_G |K(x, y)| \diff y = N \|\phi\|_{C(\bar G)},
	\end{multline*}
	отже $\| \bf{K}\phi \|_{C(\bar G)} \le N \|\phi\|_{C(\bar G)}$. \\

	Покажемо скінченність $N = \max_{x \in \bar G} \int_G |K(x, y) \diff y|$. Розглянемо \[\Int_G |K(x,y)| \diff y \le A_0 \Int_G \dfrac{\diff y}{|x - y|^\alpha} \le (*).\]

	Для будь-якої точки $x$, існує радіус (рівний максимальному діаметру області $G$) такий, що в кулю з цим радіусом попадає будь-яка точка $y$: $D = \diam G$.

	\[ (*) \le A_0 \Int_{U(x, D)} \dfrac{\diff y}{|x - y|^\alpha} = A_0 \dfrac{\sigma_n}{n - \alpha} D^{n - \alpha}. \]
\end{proof}

\begin{theorem}[про існування розв’язку інтегрального рівняння Фредгольма з полярним ядром
для малих значень параметру]
	Інтегральне рівняння Фредгольма 2-го роду з полярним ядром $K(x, y)$ має єдиний розв’язок в класі неперервних функцій для будь-якого неперервного вільного члена $f$ при умові
	\begin{equation}
		\label{eq:1.23}
		|\lambda| < \dfrac{1}{N}
	\end{equation}
	і цей розв’язок може бути представлений рядом Неймана, який збігається абсолютно і рівномірно.
\end{theorem}
\begin{proof}
	Сформулюємо умову збіжності ряду Неймана. \\

	$\phi = \sum_{i = 0}^\infty \lambda^i \bf{K}^i f$, отже $\|\phi\|_{C(\bar G)} \le \sum_{i = 1}^\infty |\lambda|^i N^i \|f\|_{C(\bar G)}$. \\

	Останній ряд – геометрична прогресія і збігається при умові $|\lambda| < \frac{1}{N}$.
\end{proof}

\begin{lemma}
	Нехай маємо два полярних ядра $K_i(x, y) = \frac{A_i(x, y)}{|x - y|^\alpha_i}$, $\alpha_i < n$, $i = 1, 2$, а область $G$ обмежена, тоді ядро $K_3(x, y) = \int_G K_2(x, \xi) K_1(\xi, y) \diff \xi$ також полярне, причому має місце співвідношення:
	\begin{equation}
		\label{eq:1.24}
		K_3(x, y) = \begin{cases}
			\dfrac{A_3(x, y)}{|x - y|^{\alpha_1 + \alpha_2 - n}}, & \alpha_1 + \alpha_2 - n > 0, \\
			A_3(x, y) |\ln|x - y|| + B_3(x, y), & \alpha_1 + \alpha_2 - n = 0, \\
			A_3(x, y), & \alpha_1 + \alpha_2 - n < 0,
		\end{cases}
	\end{equation}
	де $A_3, B_3$ неперервні функції.
\end{lemma}

З леми 4 випливає, що всі повторні ядра $K_{(p)}(x, y)$, полярного ядра $K(x, y)$ задовольняють оцінкам: \\

$\alpha_1 = \alpha_2 = \alpha$.

\begin{equation}
	\label{eq:1.25}
	K_{(2)}(x, y) = \begin{cases}
		\dfrac{A_2(x, y)}{|x - y|^{2\alpha - n}}, & 2\alpha - n > 0, \\
		A_2(x, y) |\ln|x - y|| + B_2(x, y), & 2\alpha - n = 0, \\
		A_2(x, y), & 2\alpha - n < 0,
	\end{cases}
\end{equation}
\begin{equation}
	\label{eq:1.26}
	K_{(3)}(x, y) = \begin{cases}
		\dfrac{A_3(x, y)}{|x - y|^{3\alpha - 2n}}, & 3\alpha - 2n > 0, \\
		A_3(x, y) |\ln|x - y|| + B_3(x, y), & 3\alpha - 2n = 0, \\
		A_3(x, y), & 3\alpha - 2n < 0,
	\end{cases}
\end{equation}
\begin{equation}
	\label{eq:1.27}
	K_{(p)}(x, y) = \begin{cases}
		\dfrac{A_p(x, y)}{|x - y|^{p\alpha - (p-1)n}}, & p\alpha - (p - 1)n > 0, \\
		A_p(x, y) |\ln|x - y|| + B_p(x, y), & p\alpha - (p - 1)n = 0, \\
		A_p(x, y), & p\alpha - (p - 1)n < 0.
	\end{cases}
\end{equation}

Легко бачити, що для $\forall \alpha, n$ існує $p_0$ таке, що починаючи з нього всі повторні ядра є неперервні:
\begin{equation}
	\label{eq:1.28}
	p \alpha - (p - 1) n < 0 \Rightarrow (n - \alpha) p > n \Rightarrow p > \dfrac{n}{n - \alpha} \Rightarrow p_0 = \left[ \dfrac{n}{n - \alpha} \right] + 1.
\end{equation}

Звідси маємо, що резольвента $\mathcal{R}(x, y, \lambda)$ полярного ядра $K(x, y)$ складається з двох частин полярної складової $\mathcal{R}_1(x, y, \lambda)$ і неперервної складової $\mathcal{R}_2(x, y, \lambda)$:
\begin{multline}
	\label{eq:1.29}
	\mathcal{R}(x, y, \lambda) = \mathcal{R}_1(x, y, \lambda) + \mathcal{R}_2(x, y, \lambda) = \\
	= \Sum_{i = 1}^\infty \lambda^{i - 1} K_{(i)}(x, y) = \Sum_{i = 1}^{p_0 - 1} \lambda^{i - 1} K_{(i)}(x, y) + \Sum_{i = p_0}^\infty \lambda^{i - 1} K_{(i)}(x, y).
\end{multline}

Для доведення збіжності резольвенти, потрібно дослідити збіжність нескінченного ряду $\mathcal{R}_2(x, y, \lambda)$. Він сходиться рівномірно при $x, y \in \bar G$, $|\lambda| \le \frac{1}{N} - \epsilon$, $\forall \epsilon > 0$, визначаючи неперервну функцію $\mathcal{R}$ при $x, y \in \bar G$, $|\lambda| < \frac{1}{N}$ і аналітичну по $\lambda$ в крузі 
\begin{equation}
	\label{eq:1.30}
	|\lambda| < \dfrac{1}{N}.
\end{equation}

Дійсно \[\mathcal{R}_2(x, y, \lambda) = \Sum_{i = p_0}^\infty \lambda^{i - 1} K_{(i)}(x, y).\]

У свою чергу, \[|\lambda^{p_0 + s - 1} K_{(p_0 + s)}(x, y)| \le |\lambda|^{p_0 + s - 1} M_{p_0} N^s,\] де $M_{p_0} = \max_{x, y \in \bar G \times \bar G} |K_{p_0}(x, y)|$. Таким чином ряд $\mathcal{R}_2(x, y, \lambda)$ мажорується геометричною прогресією, яка збігається при умові (\ref{eq:1.30}).

\subsection{Теореми Фредгольма}
\subsubsection{Інтегральні рівняння з виродженим ядром}

\begin{definition*}
	Неперервне ядро $K(x, y)$ називається виродженим, якщо представляється у вигляді
	\begin{equation}
		\label{eq:2.1}
		K(x, y) = \Sum_{i = 1}^N f_i(x) g_i(y),
	\end{equation}
	де $\{ f_i \}_{i = \overline{1, N}}, \{ g_i \}_{i = \overline{1, N}} \subset C(\bar G)$, і $\{ f_i \}_{i = \overline{1, N}}$ та $\{ g_i \}_{i = \overline{1, N}}$, -- лінійно незалежні системи функцій.
\end{definition*}

Розглянемо інтегральні рівняння Фредгольма з виродженим ядром 
\begin{equation}
	\label{eq:2.2}
	\phi(x) = \lambda \Int_G K(x, u) \phi(y) \diff y + f(x).
\end{equation}

Підставимо вигляд ядра з (\ref{eq:2.1}) отримаємо:
\begin{multline}
	\label{eq:2.3}
	\phi(x) = \lambda \Int_G \Sum_{i = 1}^N f_i(x) g_i(y) \phi(y) \diff y + f(x) = \\
	= \lambda \Sum_{i = 1}^N f_i(x) \Int_G g_i(y) \phi(y) \diff y + f(x) = f(x) + \lambda \Sum_{i = 1}^N c_i f_i(x),
\end{multline}\
де 
\begin{equation}
	\label{eq:2.4}
	c_j = \Int_G g_j(y) \phi(y) \diff y.
\end{equation}
В (\ref{eq:2.4}) підставимо значення $\phi(x)$ з (\ref{eq:2.3}):

\begin{multline*}
	c_j = \Int_G g_j(y) \phi(y) \diff y = \Int_G g_j(y) \left( f(y) + \lambda \Sum_{i = 1}^N c_i f_i(y) \right) \diff y = \\
	= \Int_G g_j(y) f(y) \diff y + \lambda \Sum_{i = 1}^N c_i \Int_G g_j(y) f_i(y) \diff y.
\end{multline*}
В результаті отримаємо систему лінійних алгебраїчних рівнянь
\begin{equation}
	\label{eq:2.5}
	c_j = \lambda \Sum_{i = 1}^N \alpha_{j i} c_i + a_j, \quad j = \overline{1, N},
\end{equation}
де 
\begin{equation}
	\label{eq:2.6}
	\alpha_{ji} = \Int_G g_j(y) f_i(y) \diff y, \quad a_j = \Int_G g_j(y) f(y) \diff y.
\end{equation}
Отримаємо систему рівнянь для спряженого ядра:
\begin{equation}
	\label{eq:2.1'}
	K^*(x, y) = \Sum_{i = 1}^N \bar f_i(y) \bar g_i(x),
\end{equation}
\begin{equation}
	\label{eq:2.2'}
	\psi(x) = \bar \lambda \Int_G K^*(x, y) \psi(y) \diff y + g(x),
\end{equation}
\begin{equation}
	\label{eq:2.3'}
	\psi(x) = \bar \lambda \Sum_{i = 1}^N \bar g_i(x) \Int_G \bar f_i(y) \psi(y) \diff y + g(x) = \bar \lambda \Sum_{i = 1}^N d_i \bar g_i(x) + g(x),
\end{equation}
\begin{equation}
	\label{eq:2.4'}
	d_i = \Int_G \bar f_i(y) \psi(y) \diff y, \quad d_j = \Int_G \bar f_j(y) \left( g(y) + \bar \lambda \Sum_{i = 1}^N d_i \bar g_i(y) \right) \diff y,
\end{equation}
\begin{equation}
	\label{eq:2.5'}
	d_j = \bar \lambda \Sum_{i = 1}^N \beta_{ji}d_i + b_j, \quad i = \overline{1, N}, 
\end{equation}
\begin{equation}
	\label{eq:2.6'}
	\beta_{ji} = \Int_G \bar f_j(y) \bar g_i(y) \diff y, \quad b_j = \Int_G \bar f_j(y) g(y) \diff y,
\end{equation}
\begin{equation}
	\label{eq:2.7}
	\beta_{ji} = \bar \alpha_{ij}.
\end{equation}
Тобто отримуємо системи лінійних рівнянь які в матричному вигляді запишуться так:
\begin{equation}
	\label{eq:2.8}
	\vec c = \lambda A \vec c + \vec a,
\end{equation}
\begin{equation}
	\label{eq:2.8'}
	\vec d = \lambda A^* \vec d + \vec b,
\end{equation}
з матрицями $E - \lambda A$ та $E - \bar \lambda A^*$ відповідно і визначником $D(\lambda = |E - \lambda A| = |E - \bar \lambda A^*|$. \\

Дослідимо питання існування та єдиності розв'язку СЛАР (\ref{eq:2.8}) та (\ref{eq:2.8'}). \\

Нехай $D(\lambda) \ne 0$, $\rang |E - \lambda A| = \rang |E - \bar \lambda A^*| = N$, тоді СЛАР (\ref{eq:2.8}) і (\ref{eq:2.8}) мають єдиний розв’язок для будь-яких векторів $\vec a$ і $\vec b$ відповідно, а тому інтегральні рівняння Фредгольма (\ref{eq:2.2}), (\ref{eq:2.2'}) мають єдині розв’язки при будь-яких $f$ та $g$ відповідно, і ці розв’язки записуються за формулами (\ref{eq:2.3}), (\ref{eq:2.3'}). \\

Нехай $D(\lambda) = 0$, $\rang |E - \lambda A| = \rang |E - \bar \lambda A^*| = q < N$, тоді однорідні СЛАР 
\begin{equation}
	\label{eq:2.9}
	\vec c = \lambda A \vec c,
\end{equation}
та
\begin{equation}
	\label{eq:2.9'}
	\vec d = \lambda A^* \vec d,
\end{equation}
мають $N - q$ лінійно незалежних розв’язків $\vec c_s$, $\vec d_s$, $s = \overline{1, N - q}$, де вектор визначається формулою $\vec c_s = (c_{s1}, \ldots, c_{sN})$, $\vec d_s = (d_{s1}, \ldots, d_{sN})$, таким чином відповідні однорідні інтегральні рівняння Фредгольма рівнянням (\ref{eq:2.2}), (\ref{eq:2.2'}) мають $N - q$ лінійно незалежних розв’язків які записуються за такими формулами:
\begin{equation}
	\label{eq:2.10}
	\phi_s(x) = \lambda \Sum_{i = 1}^N c_{si} f_i(x), \quad s = \overline{1, N - q},
\end{equation}
\begin{equation}
	\label{eq:2.10'}
	\psi_s(x) = \bar \lambda \Sum_{i = 1}^N d_{si} \bar g_i(x), \quad s = \overline{1, N - q},
\end{equation}
$\phi_s(x)$, $\psi_s(x)$ -- власні функції, а число $N - q$ -- кратність характеристичного числа $\lambda$ та $\bar \lambda$. Кожна з систем функцій $\phi_s$, $\psi_s$, $s = \overline{1, N - q}$ лінійно незалежна, оскільки лінійно незалежними є системи функцій $f_i$ та $g_i$ і лінійно незалежні вектори $\vec c_s$ і $\vec d_s$, $s = \overline{1, N - q}$. \\

Нагадаємо одне з формулювань теореми Кронекера-Капеллі. Для існування розв’язку системи лінійних алгебраїчних рівнянь необхідно і достатньо що би вільний член рівняння був ортогональним всім розв’язкам спряженого однорідного рівняння. \\

Для нашого випадку цю умову можна записати у вигляді
\begin{equation}
	\label{eq:2.11}
	(\vec a, \vec d_s) = \Sum_{i = 1}^N a_i \bar d_{si} = 0, \quad \forall s = \overline{1, N - q}.
\end{equation}

Покажемо, що для виконання умови $(\vec a, \vec d_s) = 0$, $s = \overline{1, N - q}$ необхідно і достатньо, щоб вільний член інтегрального рівняння Фредгольма (\ref{eq:2.2}) був ортогональним розв'язкам спряженого однорідного рівняння тобто 
\begin{equation}
	\label{eq:2.12}
	(f, \psi_s) = 0, \quad s = \overline{1, N - q}
\end{equation}
Дійсно, з (\ref{eq:2.10'}) та (\ref{eq:2.4}) маємо:
\[ (f, \psi_s) = \Int_G f(x) \bar \psi_s (x) \diff x = \lambda \Sum_{i = 1}^N \bar d_{si} \Int_G f(x) g_i(x) \diff x = \lambda \Sum_{i = 1}^N a_i \bar d_{si} = \lambda (\vec a, \vec d_s) = 0, \] для всіх $s = \overline{1, N - q}$. \\

В цьому випадку розв'язок СЛАР не єдиний, і визначається з точністю до довільного розв'язку однорідної системи рівнянь, тобто з точністю до лінійної оболонки натягнутої на систему власних векторів характеристичного числа $\lambda$:
\begin{equation}
	\label{eq:2.13}
	\vec c = \vec c_0 + \Sum_{i = 1}^{N - q} \gamma_i \vec c_i,
\end{equation}
де $\gamma_i$ -- довільні константи, $\vec c_0$ -- будь-який розв'язок неоднорідної системи рівнянь $\vec c_0 = \lambda A \vec c_0 + \vec a$, тоді розв'язок інтегрального рівняння можна записати у вигляді:
\begin{equation}
	\label{eq:2.14}
	\phi(x) = \phi_0(x) + \Sum_{i = 1}^{N - q} \gamma_i \phi_i(x),
\end{equation}
де $\phi_0$ -- довільний розв'язок неоднорідного рівняння $\phi_0 = \lambda \bf{K} \phi_0 + f$. \\

Отже доведені такі теореми:

\begin{theorem}[Перша теорема Фредгольма для вироджених ядер]
	Якщо $D(\lambda) \ne 0$, то інтегральне рівняння (\ref{eq:2.2}) та спряжене до нього (\ref{eq:2.2'}) мають єдині розв'язки для довільних вільних членів $f$ та $g$ з класу неперервних функцій.
\end{theorem}
\begin{theorem}[Друга теорема Фредгольма для вироджених ядер]
	Якщо $D(\lambda) = 0$, то однорідне рівняння Фредгольма другого роду (\ref{eq:2.2}) ($f \equiv 0$) і спряжене до нього (\ref{eq:2.2'}) ($g \equiv 0$) мають однакову кількість лінійно незалежних розв'язків рівну $N - q$, де $q = \rang(E - \lambda A)$.
\end{theorem}
\begin{theorem}[Третя теорема Фредгольма для вироджених ядер]
	Якщо $D(\lambda) = 0$, то для існування розв'язків рівняння (\ref{eq:2.2}) необхідно і достатньо, щоб вільний член $f$ був ортогональним усім розв'язкам однорідного спряженого рівняння (\ref{eq:2.12}). При виконанні цієї умови розв'язок існує та не єдиний і визначається з точністю до лінійної оболонки натягнутої на систему власних функцій характеристичного числа $\lambda$.
\end{theorem}
\begin{corollary}
	Характеристичні числа виродженого ядра $K(x, y)$ співпадають з коренями поліному $D(\lambda) = 0$, а їх кількість не перевищує $N$.
\end{corollary}
\begin{example}
	Знайти розв’язок інтегрального рівняння \[ \phi(x) = \lambda \Int_0^\pi \sin(x - y) \phi(y) \diff y + \cos(x). \]
\end{example}
\begin{solution*}
	\[ \phi(x) = \lambda \sin(x) \Int_0^\pi \cos(y) \phi(y) \diff y - \lambda \cos(x) \Int_0^\pi \sin(y) \phi(y) \diff y + \cos(x). \]
	Позначимо \[ c_1 = \Int_0^\pi \cos(y) \phi(y) \diff y, \quad c_2 = \Int_0^\pi \sin(y) \phi(y) \diff y. \]
	\[ \phi(x) = \lambda (c_1 \sin(x) - c_2 \cos(x)) + \cos(x). \]
	Підставляючи останню рівність в попередні отримаємо систем рівнянь:
	\begin{system*}
		c_1 &= \Int_0^\pi \cos(y) (\lambda c_1 \sin(y) - \lambda c_2 \cos(y) + \cos(y)) \diff y, \\
		c_2 &= \Int_0^\pi \sin(y) (\lambda c_1 \sin(y) - \lambda c_2 \cos(y) + \cos(y)) \diff y.
	\end{system*}
	Після обчислення інтегралів:
	\begin{system*}
		c_1 + \frac{\lambda \pi}{2} c_2 &= \frac{\pi}{2}, \\
		- \frac{\lambda\pi}{2} c_1 + c_2 &= 0.
	\end{system*}
	Визначник цієї системи
	\[ D(\lambda) = \begin{vmatrix} 1 & \frac{\lambda\pi}{2} \\ -\frac{\lambda\pi}{2} & 1 \end{vmatrix} = 1 + \left( \frac{\lambda \pi}{2} \right)^2 \ne 0. \]
	За правилом Крамера маємо
	\[ c_1 = \dfrac{2 \pi}{4 + (\lambda \pi)^2}, \quad c_2 = \dfrac{\lambda \pi^2}{4 + (\lambda \pi)^2}. \]
	Таким чином розв’язок має вигляд
	\[ \phi(x) = \dfrac{2 \lambda \pi \sin(x) + 4 \cos (x)}{4 + (\lambda \pi)^2}. \]
\end{solution*}
%\section*{Лекція 3}
\subsubsection*{Теореми Фредгольма для інтегральних рівнянь з неперервним ядром}

Будемо розглядати рівняння:
\begin{equation}
	\label{eq:2.2}
	\phi(x) = \lambda \Int_G K(x, y) \phi(y) \diff y + f(x),
\end{equation}
\begin{equation}
	\label{eq:2.2'}
	\psi(x) = \bar \lambda \Int_G K^*(x, y) \psi(y) \diff y + g(x),
\end{equation}

Ядро $K(x, y) \in C(\bar G \times \bar G)$, отже його можна наблизити поліномом (Теорема Вєйєрштраса). \\

Тобто, для будь-якого $\epsilon > 0$ існує 
\begin{equation}
	P_N(x, y) = \Sum_{|\alpha + \beta| \le N} a_{\alpha\beta}x^\alpha y^\beta.
\end{equation}
де $\alpha = (\alpha_1, \alpha_2, \ldots, \alpha_n)$, $x^\alpha = x_1^{\alpha_1} \cdot x_2^{\alpha_2} \cdot \ldots \cdot x_n^{\alpha_n}$, такий що $|K(x, y) - P_N(x, y)| < \epsilon$, $x, y \in \bar G \times \bar G$, тобто 
\begin{equation}
	\label{eq:2.15}
	K(x, y) = P_N(x, y) + Q_n(x, y),
\end{equation}
де $P_N(x,y)$ -- вироджене ядро (поліном), $|Q_N(x, y)| < \epsilon$, $x, y \in \bar G \times \bar G$. \\

Виходячи з (\ref{eq:2.15}), інтегральне рівняння Фредгольма приймає вигляд 
\begin{equation}
	\label{eq:2.16}
	\phi = \lambda \bf{P}_N \phi + \lambda \bf{Q}_N \phi + f,
\end{equation}
де $\bf{P}_N$ та $\bf{Q}_N$ інтегральні оператори з ядрами $P_N(x, y)$ та $Q_N(x, y)$ відповідно ($\bf{P}_N + \bf{Q}_N = \bf{K}$). \\

Для спряженого рівняння маємо:
\begin{equation}
	\label{eq:2.15'}
	K^*(x, y) = P_N^*(x, y) + Q_N^*(x, y),
\end{equation}
\begin{equation}
	\label{eq:2.16'}
	\psi = \bar \lambda \bf{P}_N^* \psi + \bar \lambda \bf{Q}_N^* \psi + g.
\end{equation}

Покажемо, що в класі $C(G)$ рівняння (\ref{eq:2.16}), (\ref{eq:2.16'}) еквівалентні рівнянням з виродженим ядром. Введемо нову функцію 
\begin{equation}
	\label{eq:2.17}
	\Phi = \phi - \lambda \bf{Q}_N \phi
\end{equation}

З (\ref{eq:2.16}) випливає що $\Phi = \lambda \bf{P}_N + f$, з (\ref{eq:1.19}) випливає що $\forall \lambda$ такого що $|\lambda| < \frac{1}{\epsilon V}$: $(E - \lambda \bf{Q}_N)^{-1} = (E + \lambda \bf{R}_N)$, де $\bf{R}_N$ -- резольвента для $\bf{Q}_N$. Отже $\phi = (E - \lambda \bf{Q}_N)^{-1} \Phi = (E + \lambda \bf{R}_N) \Phi$. \\

Отже, рівняння (\ref{eq:2.2}) перетворюється на 
\begin{equation}
	\label{eq:2.18}
	\Phi = \lambda \bf{P}_N (E + \lambda \bf{R}_N) \Phi.
\end{equation}

Для спряженого рівняння (\ref{eq:2.2'}) маємо:
\[ \psi = \bar \lambda (E + \bar \lambda \bf{R}_N^*) \bf{P}_N^* \psi + (E + \bar \lambda \bf{R}_N^*) g. \]

Позначимо $g_1 = (E + \bar \lambda \bf{R}_N^*) g$. Маємо:
\begin{equation}
	\label{eq:2.18'}
	\psi = \bar \lambda (E + \bar \lambda \bf{R}_N^*) \bf{P}_N^* \psi + g_1.
\end{equation}

Оскільки $(\bf{P}_N \bf{R}_N)^* = \bf{R}_N^* \bf{P}_N^*$, то рівняння (\ref{eq:2.18}) та (\ref{eq:2.18'}) спряжені. \\

Позначимо 
\begin{equation}
	\label{eq:2.19}
	\bf{T}_N = \bf{P}_N (E + \lambda \bf{R}_N),
\end{equation}
\begin{equation}
	\label{eq:2.19'}
	\bf{T}_N^* = (E + \bar \lambda \bf{R}_N^*) \bf{P}_N^*.
\end{equation}

Тоді рівняння Фредгольма з неперервним ядром можна записати у вигляді:
\begin{equation}
	\label{eq:2.20}
	\Phi = \lambda \bf{T}_N \Phi + f,
\end{equation}
\begin{equation}
	\label{eq:2.20'}
	\Psi = \bar \lambda \bf{T}_N^* \Psi + g_1,
\end{equation}
де $T_N(x, y, \lambda) = P_N(x, y) + \lambda \int_G P_N(x, \xi) R_N(\xi, y, \lambda) \diff \xi$ -- вироджене, оскільки є сумою двох вироджених, поліному $P_N(x, y)$, та інтегрального доданку. Покажемо що другий доданок в $T_N$ -- вироджений. Дійсно:
\[ \Int_G \Sum_{|\alpha + \beta| \le N} a_{\alpha \beta} x^\alpha \xi^\beta R_N(\xi, y) \diff \xi = \Sum_{|\alpha + \beta| \le N} a_{\alpha \beta} x^\alpha \Int_G \xi^\beta R_N(\xi, y) \diff \xi. \]

\subsubsection*{Альтернатива Фредгольма}

Сукупність теорем Фредгольма для інтегральних рівнянь з неперервним ядром називається альтернативою Фредгольма.

\begin{theorem}[Перша теорема Фредгольма для неперервних ядер]
	Якщо інтегральне рівняння (\ref{eq:2.2}) з неперервним ядром $K(x, y)$ має розв'язок $\forall f \in C(\bar G)$ то і спряжене рівняння (\ref{eq:2.2'}) має розв'язок для $\forall g \in C(\bar G)$ і ці роз'язки єдині.
\end{theorem}
\begin{theorem}[Друга теорема Фредгольма для непевних ядер]
	Якщо інтегральне рівняння (\ref{eq:2.2}) має розв'язки не для будь-якого вільного члена $f$, то однорідні рівняння $\phi = \lambda \bf{K} \phi$ $(*)$ та $\psi = \bar \lambda \bf{K}^* \psi$ $(**)$ мають однакову скінчену кількість лінійно-незалежних розв'язків.
\end{theorem}
\begin{theorem}[Третя теорема Фредгольма для неперервних ядер]
	Якщо інтегральне рівняння (\ref{eq:2.2}) має розв'язок не для $\forall$ вільного члена $f$, то для існування розв'язку інтегрального рівняння (\ref{eq:2.2}) в $C(\bar G)$ необхідно і достатньо, щоб вільний член $f$ був ортогональним всім розв'язкам спряженого однорідного рівняння $(**)$. Розв'язок не єдиний і визначається з точністю до лінійної оболонки, натягнутої на систему власних функцій оператора $\bf{K}$.
\end{theorem}

Доведення теорем: Для будь-якого фіксованого значення $\lambda$ виберемо $\epsilon$, таке щоби $|\lambda| < \frac{1}{\epsilon V}$.

\begin{proof}[Теореми 1]
	Нехай (\ref{eq:2.2}) має розв'язок в $C(\bar G)$ для $\forall$ вільного члена $f$, тоді еквівалентне йому рівняння (\ref{eq:2.20}): $\Phi = \lambda \bf{T}_N \Phi + F$ має такі ж властивості і згідно з першою теоремою Фредгольма для вироджених ядер $D(\lambda) \ne 0$, а спряжене до нього рівняння (\ref{eq:2.20'}): $\Psi = \bar \lambda \bf{T}_N^* + g_1$ теж має єдиний розв'язок $\forall$ вільного члена $g_1$, еквівалентне до нього рівняння (\ref{eq:2.2'}) має розв'язок $\forall g$.
\end{proof}
\begin{proof}[Теореми 2]
	Нехай (\ref{eq:2.2}) має розв'язок не $\forall$ вільного члена $f$, тоді, рівняння з виродженим ядром (\ref{eq:2.20}) має таку ж властивість. Згідно з теоремами Фредгольма для вироджених ядер $D(\lambda) = 0$ (для виродженого ядра $\bf{T}_N$). Однорідні рівняння які відповідають (\ref{eq:2.20}) і (\ref{eq:2.20'}) мають однакову скінчену кількість лінійно-незалежних розв'язків, еквівалентні до них однорідні рівняння $(*)$, $(**)$ теж мають однакову скінчену кількість лінійно незалежних розв'язків.
\end{proof}
\begin{proof}[Теореми 3]
	Нехай неоднорідне рівняння (\ref{eq:2.2}) має розв'язок не для будь-якого вільного члена $f$, тоді еквівалентне рівняння з виродженим ядром (\ref{eq:2.20}) має таку ж властивість, і за третьою теоремою Фредгольма для вироджених ядер $D(\lambda) = 0$ (для виродженого ядра $\bf{T}_N$). Розв'язок (\ref{eq:2.20}) існує тоді і тільки тоді коли $f$ ортогональний до розв'язків спряженого однорідного рівняння до (\ref{eq:2.20'}). Але легко бачити, що вільний член (\ref{eq:2.2}) і (\ref{eq:2.20}) співпадають, так само співпадають розв'язки однорідного рівняння (\ref{eq:2.2'}) і (\ref{eq:2.20'}).
\end{proof}

\begin{theorem}[Четверта теорема Фредгольма]
	Для будь-якого як завгодно великого числа $R > 0$ в крузі $|\lambda| < R$ лежить лише скінчена кількість характеристичних чисел неперервного ядра $K(x, y)$.
\end{theorem}

\subsubsection*{Наслідки з теорем Фредгольма}

\begin{corollary}
	З четвертої теореми Фредгольма випливає, що множина характеристичних чисел неперервного ядра не має скінчених граничних точок і не більш ніж злічена $\lim_{n \to \infty} |\lambda_n| = \infty$.
\end{corollary}
\begin{corollary}
	З другої теореми Фредгольма випливає, що кратність кожного характеристичного числа скінчена, їх можна занумерувати у порядку зростання модулів $|\lambda_1| \le |\lambda_2| \le \ldots \le |\lambda_k| \le |\lambda_{k + 1}| \le \ldots$, кожне число зустрічається стільки разів, яка його кратність. Також можна занумерувати послідовність власних функцій ядра $K(x, y)$: $\phi_1$, $\phi_2$, $\ldots$, $\phi_k$, $\phi_{k + 1}$, $\ldots$ і спряженого ядра $K^*(x, y)$: $\psi_1$, $\psi_2$, $\ldots$, $\psi_k$, $\psi_{k + 1}$, $\ldots$.
\end{corollary}
\begin{corollary}
	Власні функції неперервного ядра $K(x, y)$ неперервні в області $G$.
\end{corollary}
\begin{corollary}
	Якщо $\lambda_k \ne \lambda_j$, то $(\phi_k, \psi_j) = 0$.
\end{corollary}

\subsubsection*{Теореми Фредгольма для інтегральних рівнянь з полярним ядром}


Розповсюдимо Теореми Фредгольма для інтегральних рівнянь з полярним ядром:
\begin{equation}
	\label{eq:2.21}
	K(x, y) = \dfrac{A(x, y)}{|x - y|^\alpha}, \quad \alpha < n.
\end{equation}

Покажемо що $\forall \epsilon > 0$ існує таке вироджене ядро $P_N(x, y)$ що,
\begin{equation}
	\label{eq:2.22}
	\Max_{x \in \bar G} \Int_G |K(x, y) - P_N(x, y)| \diff y < \epsilon
\end{equation}
\begin{equation}
	\label{eq:2.22'}
	\Max_{x \in \bar G} \Int_G |K^*(x, y) - P_N^*(x, y)| \diff y < \epsilon
\end{equation}

Розглянемо неперервне ядро
\begin{equation}
	\label{eq:2.23}
	L_M(x, y) = \begin{cases}
		K(x, y), & |x - y| \ge 1 / M, \\
		A(x, y) M^\alpha, & |x - y| < 1 / M.
	\end{cases}
\end{equation}
Покажемо, що при достатньо великому $M$ має місце оцінка \[ \Int_G |K(x, y) - L_M(x, y)| \diff y \le \epsilon. \]

Дійсно:
\begin{multline*} 
	\Int_G |K(x, y) - L_M(x, y)| \diff y = \Int_{|x - y| < 1 / M} \left| \dfrac{A(x, y)}{|x - y|^\alpha} - A(x, y) M^\alpha \right| \diff y = \\
	= \Int_{|x - y| < 1 /M} |A(x, y)| \left| \dfrac{1}{|x - y|^\alpha} - M^\alpha \right| \diff y \le A_0 \Int_{|x - y| < 1 /M} \left| \dfrac{1}{|x - y|^\alpha} - M^\alpha \right| \diff y \le A_0 \Int_{|x - y| < 1 /M} \dfrac{\diff y}{|x - y|^\alpha} = \\
	= A_0 \sigma_n \Int_0^{1 / M} \xi^{n - 1 - \alpha} \diff \xi = A_0 \sigma_n \left.\dfrac{\xi^{n - \alpha}}{n - \alpha}\right|_0^{1 / M} = \dfrac{A_0\sigma_n}{(n - \alpha)M^{n - \alpha}} \le \dfrac{\epsilon}{2},
\end{multline*}
де $\sigma_n$ -- площа поверхні одиничної сфери. \\

Завжди можна підібрати вироджене ядро $P_N(x, y)$ таке що \[ |L_M(x, y) - P_N(x, y)| \le \dfrac{\epsilon}{2V}, \] де $V$ -- об'єм області $G$. 

\begin{multline*}
	\Int_G |K(x, y) - P_N(x, y)| \diff y = \Int_G | K(x, y) - L_M(x, y) + L_M(x, y) - P_N(x, y)| \diff y \le \\
	\le \Int_G | K(x, y) - L_M(x, y)| \diff y + \Int_G |L_M(x, y) - P_N(x, y)| \diff y \le \dfrac{\epsilon}{2} + \dfrac{\epsilon}{2V} \Int_G \diff y = \epsilon.
\end{multline*}

Використавши попередню техніку (для неперервного ядра) інтегральне рівняння з полярним ядром зводиться до еквівалентного рівняння з виродженим ядром. Тобто теореми Фредгольма залишаються вірними для інтегральних рівнянь з полярним ядром з тим же самим формулюванням. \\

Теореми Фредгольма залишаються вірними для інтегральних рівнянь з полярним ядром на обмеженій кусково-гладкій поверхні $S$ та контурі $C$:
\[ \phi(x) = \lambda \Int_S K(x, y) \phi(y) \diff y + f(x), \quad \dfrac{A(x, y)}{|x - y|^\alpha}, \quad \alpha < \dim(S). \]

\subsection*{\S3. Інтегральні рівняння з ермітовим ядром}

Розглядатимемо ядро $K(x, y) \in C(\bar G \times \bar G)$ таке що $K(x, y) = K^*(x, y)$. \\

Неперервне ядро будемо називати ермітовим, якщо виконується
\begin{equation}
	\label{eq:3.1}
	K(x, y) = K^*(x, y)
\end{equation}

Ермітовому ядру відповідає ермітовий оператор тобто $\bf{K} = \bf{K}^*$.

\begin{lemma}
	Для того, щоб лінійний оператор був ермітовим, необхідно і достатньо, щоб для довільної комплексно значної функції $f \in L_2(\bar G)$ білінійна форма $(\bf{K}f, f)$ приймала лише дійсні значення.
\end{lemma}
\begin{lemma}
	Характеристичні числа ермітового оператора дійсні.
\end{lemma}
\begin{definition}
	Множина функцій $M \subset C(\bar G)$ -- компактна в рівномірній метриці, якщо з будь-якої нескінченної множини функцій з $М$ можна виділити рівномірно збіжну підпослідовність.
\end{definition}
\begin{definition}
	Нескінченна множина $M \subset C(\bar G)$ -- рівномірно обмежена, якщо для будь-якого елемента $f \in M$ має місце $\|f\|_{C(\bar G)} \le a$, де $a$ єдина константа для $M$.
\end{definition}
\begin{definition}
	Множина $M \subset C(\bar G)$ -- одностайно неперервна якщо $\forall \epsilon > 0 \exists \delta(\epsilon): \forall f \in M, \forall x_1, x_2: |f(x_1) - f(x_2)| < \epsilon$ як тільки $|x_1 - x_2| < \delta(\epsilon)$.
\end{definition}
\begin{theorem}[Арчела-Асколі, критерій компактності в рівномірній метриці]
	Для того, щоб множина $M \subset C(\bar G)$ була компактною, необхідно і достатньо, щоб вона складалась з рівномірно-обмеженої і одностайно-неперервної множини функцій.
\end{theorem}
\begin{definition}
	Назвемо оператор $\bf{K}$ цілком неперервним з $L_2(G)$ у $C(\bar G)$, якщо він переводить обмежену множину в $L_2(G)$ у компактну множину в $C(\bar G)$ (в рівномірній метриці).
\end{definition}
\begin{lemma}
	Інтегральний оператор $\bf{K}$ з неперервним ядром $K(x, y)$ є цілком неперервний з $L_2(G)$ у $C(\bar G)$.
\end{lemma}
\begin{proof}
	Нехай $f \in < \subset L_2(G)$ та $\forall f \in M: \|f\|_{L_2(G)} \le A$. Але $\|\bf{K} f\|_{C(\bar G)} \le M \sqrt{V} \|f\|_{L_2(G)} \le M \sqrt{V} A$, тобто множина функцій є рівномірно обмеженою. \\

	Покажемо що множина $\{ \bf{K}f(x)\}$ -- одностайно неперервна. \\

	Ядро $K \in C(\bar G \times \bar G)$< а отже є рівномірно неперервним, бо неперервне на компакті, тобто $\forall \epsilon > 0 \exists \delta > 0: \forall x', x'' \in \bar G: \|x' - x''\| < \delta \Rightarrow |(\bf{K}f)(x') - (\bf{K}f)(x'')| \le \epsilon$. Дійсно,
	\begin{multline*}
		|(\bf{K}f)(x') - (\bf{K}f)(x'')| = \left| \Int_G K(x', y) f(y) \diff y - \Int_G K(x'', y) f(y) \diff y \right| \le \\
		\le \Int_G |K(x', y) - K(x'', y)| |f(y)| \diff y \le \dfrac{\epsilon \sqrt{V}}{A \sqrt{V}} \|f\|_{L_2(\bar G)} \le \epsilon.
	\end{multline*}
\end{proof}

\begin{example}
	Знайти характеристичні числа та власні функції інтегрального оператора \[ \phi(x0 = \lambda \Int_0^1 \left( \left( \dfrac{x}{t} \right)^{2/5} + \left( \dfrac{t}{x} \right)^{2/5} \right) \phi(t) \diff t. \]
\end{example}
\begin{solution*}
	\[ \phi(x) = \lambda x^{2 / 5} \Int_0^1 t^{-2/5} \phi(t) \diff t + \lambda x^{-2/5} \Int_0^1 t^{2/5} \phi(t) \diff t. \]
	Позначимо
	\[ c_1 = \Int_0^1 t^{-2/5} \phi(t) \diff t, \quad c_2 = \Int_0^1 t^{2/5} \phi(t) \diff t. \]
	\[ \phi(x) = \lambda c_1 x^{2/5} + \lambda c_2 x^{-2/5}. \]
	\begin{system*}
		c_1 &= \Int_0^1 t^{-2/5} (\lambda c_1 t^{2/5} + \lambda c_2 t^{-2/5}) \diff t, \\
		c_2 &= \Int_0^1 t^{2/5} (\lambda c_1 t^{2/5} + \lambda c_2 t^{-2/5}) \diff t.
	\end{system*}
	\begin{system*}
		(1 - \lambda) c_1 - 5 \lambda c_2 &= 0, \\
		-\frac{5\lambda}{9} c_1 + (1 - \lambda) c_2 &= 0.
	\end{system*}
	\[ D(\lambda) = \begin{vmatrix} 1 - \lambda & - 5 \lambda \\ - \dfrac{5\lambda}{9} & 1 - \lambda \end{vmatrix} = (1 - \lambda)^2 - \dfrac{25\lambda^2}{9} = 0. \]
	\[ \lambda_1 = \dfrac{3}{8}, \quad \lambda_2 = - \dfrac{3}{2}. \]
	З системи однорідних рівнянь при $\lambda = \lambda_1 = \frac{3}{8}$ маємо $c_1 = 3 c_2$. Тоді маємо власну функцію $\phi_1(x) = 3 x^{2 / 5} + x^{-2 / 5}$. \\

	При $\lambda = \lambda_2 = - \frac{3}{2}$ маємо $c_1 = - 3 c_2$. Маємо другу власну функцію $\phi_2(x) = - 3 x^{2 / 5} + x^{-2 / 5}$.
\end{solution*}

\begin{example}
	Знайти розв’язок інтегрального рівняння при всіх значеннях параметрів $\lambda$, $a$, $b$, $c$: \[\phi(x) = \lambda \Int_{-1}^1 (\sqrt[3]{x} + \sqrt[3]{y}) \phi(y) \diff y + ax^2 + bx + c. \]
\end{example}
\begin{solution*}
	Запишемо рівняння у вигляді: \[\phi(x) = \lambda \sqrt[3]{x} \Int_{-1}^1 \phi(y) \diff y + \lambda \Int_{-1}^1 \sqrt[3]{y} \phi(y) \diff y + ax^2 + bx + c. \]
	Введемо позначення: \[ c_1 = \Int_{-1}^1 \phi(y) \diff y, \quad c_2 = \Int_{-1}^1 \sqrt[3]{y} \phi(y) \diff y, \] та запишемо розв’язок у вигляді: $\phi(x) = \lambda \sqrt[3]{x} c_1 + \lambda c_2 + ax^2 + bx + c$. \\

	Для визначення констант отримаємо СЛАР:
	\begin{system*}
		c_1 - 2 \lambda c_2 &= \dfrac{2a}{3} + 2 c, \\
		- \dfrac{6\lambda}{5} c_1 + c_2 &= \dfrac{6b}{7}.
	\end{system*}

	Визначник системи дорівнює $\begin{vmatrix} 1 & - 2 \lambda \\ - \frac{6\lambda}{5} & 1 \end{vmatrix} = 1 - \frac{12\lambda^2}{5}$.

	Характеристичні числа ядра $\lambda_1 = \frac{1}{2} \sqrt{\frac{5}{3}}$, $\lambda_2 = - \frac{1}{2} \sqrt{\frac{5}{3}}$. \\

	Нехай $\lambda \ne \lambda_1$, $\lambda \ne \lambda_2$. Тоді розв'язок існує та єдиний для будь-якого вільного члена і має вигляд \[ \phi(x) = \dfrac{5 \lambda (14 a + 30 \lambda b + 42 c)}{21 (5 - 12 \lambda^2)} \sqrt[3]{x} + \dfrac{28 \lambda a + 84 \lambda c + 30 b}{7(5 - 12 \lambda^2)} + ax^2 + bx + c. \]

	Нехай $\lambda = \lambda_1 = \frac{1}{2} \sqrt{\frac{5}{3}}$. Тоді система рівнянь має вигляд:
	\begin{system*}
		c_1 - \sqrt{\dfrac{5}{3}} c_2 &= \dfrac{2a}{3} + 2 c, \\
		c_1 - \sqrt{\dfrac{5}{3}} c_2 &= - \sqrt{\dfrac{5}{3}} \dfrac{6b}{7}.
	\end{system*}

	Ранги розширеної і основної матриці співпадатимуть якщо має місце рівність $\frac{2a}{3} + 2c = - \sqrt{\frac{5}{3}} \frac{6}{7} b$, $(*)$. \\

	При виконанні цієї умови розв'язок існує $c_2 = c_2$, $c_1 = \sqrt{\frac{5}{3}} c_2 + \frac{2a}{3} + 2c$. \\

	Таким чином розв'язок можна записати \[ \phi(x) = \dfrac{1}{2} \sqrt{\dfrac{5}{3}} \sqrt[3]{x} \left( \sqrt{\dfrac{5}{3}} c_2 + \dfrac{2a}{3} + 2c \right) + \dfrac{1}{2} \sqrt{\dfrac{5}{3}} c_2 + ax^2 + bx + x. \]

	Якщо $\lambda = \lambda_1 = \frac{1}{2} \sqrt{\frac{5}{3}}$, а умова $(*)$ не виконується, то розв'язків не існує. \\

	Нехай $\lambda = \lambda_2 = - \frac{1}{2} \sqrt{\frac{5}{3}}$. Після підстановки цього значення отримаємо СЛАР
	\begin{system*}
		c_1 + \sqrt{\dfrac{5}{3}} c_2 &= \dfrac{2a}{3} + 2 c, \\
		c_1 + \sqrt{\dfrac{5}{3}} c_2 &= \sqrt{\dfrac{5}{3}} \dfrac{6b}{7}.
	\end{system*}

	Остання система має розв'язок при умові, $\frac{2a}{3} + 2c = \sqrt{\frac{5}{3}} \frac{6}{7} b$, $(**)$. \\

	При виконанні умови $(**)$, розв’язок існує $c_2 = c_2$, $c_1 = - \sqrt{\frac{5}{3}} c_2 + \frac{2a}{3} + 2c$. \\

	Розв'язок інтегрального рівняння можна записати: \[ \phi(x) = \dfrac{1}{2} \sqrt{\dfrac{5}{3}} \sqrt[3]{x} \left( -\sqrt{\dfrac{5}{3}} c_2 + \dfrac{2a}{3} + 2c \right) + \dfrac{1}{2} \sqrt{\dfrac{5}{3}} c_2 + ax^2 + bx + c. \]
\end{solution*}

 \setcounter{section}{2}
 \setcounter{subsection}{3}
 \setcounter{subsubsection}{0}
 \setcounter{theorem}{14}
 \setcounter{equation}{4}

\subsubsection{Характеристичні числа ермітового неперервного ядра}

\begin{theorem}[про існування характеристичного числа у ермітового неперервного ядра]
	Для будь-якого ермітового неперервного ядра, що не дорівнює тотожно нулю існує принаймні одне характеристичне число і найменше з них за модулем $\lambda_1$ задовольняє варіаційному принципу
	\begin{equation}
		\dfrac{1}{|\lambda_1|} = \Sup_{f \in L_2(G)} \dfrac{\|\bf{K} f\|_{L_2(G)}}{\|f\|_{L_2(G)}}.
	\end{equation}
\end{theorem}

\begin{proof}
	Серед усіх $f \in L_2$ оберемо такі, що $\|f\|_{L_2(G)} = 1$. Позначимо 
	\begin{equation}
		\nu = \Sup_{\substack{f \in L_2(G) \\ \|f\|_{L_2} = 1}} \|\bf{K} f\|_{L_2(G)}.
	\end{equation}
	
	Оскільки
	\begin{equation}
		\|\bf{K} f\|_{L_2(G)} \le MV \|f\|_{L_2(G)} \le MV,
	\end{equation}
    то $0 \le \nu \le MV$. \medskip

	Згідно до визначення точної верхньої межі,
	\begin{equation}
		\exists \{ f_k \}_{k = 1}^\infty \subset L_2(G):\lim_{n \to \infty} \|\bf{K} f_k\|_{L_2(G)} = \nu.
	\end{equation}

	Оцінимо 
	\begin{equation}
		\begin{aligned} 
		\left\| \bf{K}^2 f\right\|_{L_2(G)} &= \| \bf{K} (\bf{K} f)\|_{L_2(G)} = \\
		&=  \left\| \bf{K}\left( \dfrac{\bf{K}f}{\|\bf{K}f\|}\right) \right\|_{L_2(G)} \cdot \|\bf{K} f\|_{L_2(G)} \le \\
		&\le  \nu \cdot \|\bf{K} f\|_{L_2(G)} \le \nu^2.
		\end{aligned}
	\end{equation}
	
	Покажемо, що $\bf{K}^2 f_k - \nu^2 f_k \to 0$ в середньому квадратичному. Тобто що
	\begin{equation}
		\| \bf{K}^2 f_k - \nu^2 f_k \|_{L_2(G)}^2 \xrightarrow[k \to \infty]{} 0.
	\end{equation}

	Дійсно:
	\begin{equation}
		\begin{aligned}
			\| \bf{K}^2 f_k - \nu^2 f_k \|_{L_2(G)}^2 &= (\bf{K}^2 f_k - \nu^2 f_k, \bf{K}^2 f_k - \nu^2 f_k)_{L_2(G)} = \\
			&= \|\bf{K}^2 f_k\|_{L_2(G)}^2 + \nu^4 - \nu^2 (\bf{K}^2 f_k, f_k) - \nu^2 (f_k, \bf{K}^2 f_k) = \\
			&= \|\bf{K}^2 f_k\|_{L_2(G)}^2 + \nu^4 - 2 \nu^2 \|\bf{K} f_k\|_{L_2(G)}^2 \le \\
			&\le \nu^2 \left(\nu^2 - \|\bf{K}^2 f_k\|_{L_2(G)}^2\right) \xrightarrow[k\to\infty]{} 0.
		\end{aligned}
	\end{equation}

	Розглянемо послідовність $\{ \bf{K}f_k \} = \{ \phi_k\}$, яка є компактною в рівномірній метриці. \medskip

	У неї існує підпослідовність $\{\phi_{k_i}\}_{i = 1}^\infty$ збіжна в $C\left(\overline G\right)$, тобто $\exists \phi \in C\left(\overline G\right)$, така що $\| \phi_{k_i} - \phi\|_{C\left(\overline G\right)} \xrightarrow[i \to \infty]{} 0$. \medskip

	Покажемо, що $\bf{K}^2 \phi - \nu^2 \phi = 0$ в кожній точці, тобто $\| \bf{K}^2\phi - \nu^2 \phi\|_{C\left(\overline G\right)} = 0$. \medskip

	Справді,
	\begin{equation}
		\begin{aligned}
			\|\bf{K}^2\phi - \nu^2\phi\|_{C\left(\overline G\right)} &= \|\bf{K}^2\phi - \bf{K}^2\phi_{k_i} + \bf{K}^2\phi_{k_i} - \nu^2\phi_{k_i} + \nu^2\phi_{k_i} - \nu^2\phi\|_{C\left(\overline G\right)} \le \\
			&\le \|\bf{K}^2\phi-\bf{K}^2\phi_{k_i}\|_{C\left(\overline G\right)} + \|\bf{K}^2\phi_{k_i}-\nu^2\phi_{k_i}\|_{C\left(\overline G\right)}+ \\
			&\quad + \|\nu^2\phi_{k_i}-\nu^2\phi\|_{C\left(\overline G\right)} \le \\
			&\le (MV)^2 \|\phi_{k_i} - \phi\|_{C\left(\overline G\right)} + M\sqrt{V} \|\bf{K}^2f_{k_i}-\nu^2f_{k_i}\|_{L_2(\overline G)} + \\
			&\quad + \nu^2 \|\phi_{k_i} - \phi\|_{C\left(\overline G\right)} \to 0 + 0 + 0.
		\end{aligned}
	\end{equation}
	
	Таким чином має місце рівність
	\begin{equation}
		\bf{K}^2 \phi - \nu^2 \phi = 0
	\end{equation}
	
	Отже маємо: $(\bf{K} + E\nu)(\bf{K} - E \nu)\phi=0$. Ця рівність може мати місце у двох випадках:
	\begin{enumerate}
		\item $(\bf{K} - E\nu) \phi \equiv 0$. Тоді $\phi = \dfrac{1}{\nu}\bf{K}\phi$, а отже $\phi$ --- власна функція, $\dfrac{1}{\nu}$ --- характеристичне число оператора $\bf{K}$.

		\item $(\bf{K} - E\nu) \phi \equiv \Phi \ne 0$. Тоді $(\bf{K} + E \nu) \Phi \equiv 0$. Тоді $\Phi = -\dfrac{1}{\nu}\bf{K}\Phi$, а отже $\Phi$ --- власна функція, $-\dfrac{1}{\nu}$ --- характеристичне число оператора $\bf{K}$.
	\end{enumerate}
	
	Залишилось довести, що це характеристичне число є мінімальним за модулем. Припустимо супротивне. Нехай $\exists \lambda_0: |\lambda_0| < |\lambda_1|$, тоді
	\begin{equation}
		\frac{1}{|\lambda_1|} = \sup_{f \in L_2(G)} \frac{\|\bf{K}f\|}{\|f\|} \ge \frac{\|\bf{K}\phi_0\|}{\|\phi_0\|} = \frac{1}{|\lambda_0|},
	\end{equation}
	тобто $|\lambda_0| \ge |\lambda_1|$, протиріччя.
\end{proof}

\begin{remark}
	Доведена теорема є вірною і для ермітових полярних ядер,
\end{remark}

Звідси безпосередньо випливають такі 
\begin{properties}[характеристичних чисел та власних функцій ермітового ядра]
	Нескладно показати, що:
	\begin{enumerate}
		\item Множина характеристичних чисел ермітового неперервного \allowbreak ядра не порожня, є підмножиною множини дійсних чисел і не має скінчених граничних точок.
		\item Кратність будь-якого характеристичного числа скінчена.
		\item Власні функції можна вибрати так, що вони утворять ортонормовану систему, тобто $\{ \phi_k\}_{k = 1, 2, \ldots}$ такі що $(\phi_k, \phi_i)_{L_2(G)} = \delta_{ki}$.
	\end{enumerate}
\end{properties}

\begin{remark}
	Для доведення останньої властивості достатньо провести процес ортогоналізації Гілберта-Шмідта для будь-якої системи лінійно незалежних власних функцій, і пронормувати отриману систему.
\end{remark}

\subsection{Теорема Гілберта-Шмідта та її наслідки}

\subsubsection{Білінійе розвинення ермітового неперервного ядра}

Нехай $K(x, y) \in C\left(\overline G \times \overline G\right)$ --- ермітове неперервне ядро, $|\lambda_i| \le |\lambda_{i + 1}|$, $i = 1, 2, \ldots$ --- його характеристичні числа і $\{\phi_i\}_{i = 1}^\infty$ --- ортонормована система власних функцій, що відповідають власним числам. \medskip

Розглянемо послідовність ермітових неперервних ядер:
\begin{equation}
	K^p(x, y) = K(x, y) - \Sum_{i = 1}^p \dfrac{\overline \phi_i(y) \phi_i(x)}{\lambda_i}, \quad p = 1, 2, \ldots
\end{equation}

Зрозуміло що при цьому
\begin{equation}
	K^p(x, y) = (K^p)^\star (x, y) \in  C\left(\overline G \times \overline G\right).
\end{equation}

Дослідимо властивості повторних ермітових операторів.

\begin{proposition}
	Будь-яке характеристичне число $\lambda_j$, $j > p + 1$ та відповідна йому власна функція $\phi_j$ є характеристичним числом і власною функцією ядра $K^p(x,y)$.
\end{proposition}

\begin{proof}
	Справді:
	\begin{equation}
		\bf{K}^p \phi_j = \bf{K} \phi_j - \Sum_{i = 1}^p \dfrac{\phi_i(x)}{\lambda_i} (\phi_i, \phi_j) = \bf{K} \phi_j = \dfrac{\phi_j}{\lambda_j}.
	\end{equation}
\end{proof}

Нехай $\lambda_0$, $\phi_0$ --- характеристичне число та відповідна власна функція $K^p(x, y)$, тобто $\lambda_0 \bf{K}^p \phi_0 = \phi_0$.

\begin{proposition}
	$(\phi_0, \phi_j) = 0$ для $j = \overline{1, p}$.
\end{proposition}

\begin{proof}
	З того, що $\phi_0$ є власною функцією ядра $\bf{K}^p$ випливає, що
	\begin{equation}
		\phi_0 = \lambda_0 \bf{K} \phi_0 - \lambda_0 \Sum_{i = 1}^p \dfrac{\phi_i}{\lambda_i} (\phi_0, \phi_i).
	\end{equation}

	Підставляючи цей вираз для $\phi_0$ у потрібний скалярний добуток маємо
	\begin{equation}
		\begin{aligned}
			(\phi_0, \phi_j) &= \lambda_0 (\bf{K} \phi_0, \phi_j) - \lambda_0 \Sum_{i = 1}^p \dfrac{(\phi_0, \phi_i)(\phi_i, \phi_j)}{\lambda_i} = \\
			&= \dfrac{\lambda_0}{\lambda_j} (\phi_0, \phi_j) - \dfrac{\lambda_0}{\lambda_j} (\phi_0, \phi_j) = 0. 
		\end{aligned}
	\end{equation}
\end{proof}

Отже $\lambda_0$, $\phi_0$ відповідно характеристичне число і власна функція ядра $K(x, y)$. \medskip

Таким чином $\phi_0$ --- ортогональна до усіх власних функцій $\phi_1$, $\phi_2$, $\ldots$, $\phi_p$. Але тоді $\lambda_0$ співпадає з одним із характеристичних чисел $\lambda_{p + 1}$, $\lambda_{p + 2}$, $\ldots$ тобто $\phi_0 = \phi_k$ для деякого $k \ge p + 1$. \medskip

Отже у ядра $K^p(x, y)$ множина власних функцій і характеристичних чисел вичерпується множиною власних функцій і характеристичних чисел ядра $K(x, y)$ починаючи з номера $p + 1$. \medskip

Враховуючи, що $\lambda_{p + 1}$ --- найменше за модулем характерне число ядра $K^p(x, y)$, має місце нерівність
\begin{equation}
	\dfrac{\|\bf{K}^p f\|_{L_2(G)}}{\|f\|_{L_2(G)}} \le \dfrac{1}{|\lambda_{p + 1}|}.
\end{equation}

Для ядра, що має скінчену кількість характеристичних чисел, очевидно, має місце рівність
\begin{equation}
	K^N(x, y) = K(x, y) - \sum_{i = 1}^N \frac{\phi_i(x) \overline \phi_i(y)}{\lambda_i} \equiv 0.
\end{equation}

Тобто будь-яке ермітове ядро зі скінченою кількістю характеристичних чисел є виродженим і представляється у вигляді
\begin{equation}
	K(x, y) = \sum_{i = 1}^N \frac{\phi_i(x) \overline \phi_i(y)}{\lambda_i}.
\end{equation}

Враховуючи теорему про існування характеристичних чисел у ермітового оператора можемо записати:
\begin{equation}
	\| K^{(p)} f \|_{L_2(G)} = \left\| \bf{K}f - \Sum_{i = 1}^p \dfrac{(f, \phi_i)}{\lambda_i} \phi_i \right\|_{L_2(G)} \le \dfrac{\|f\|_{L_2(G)}}{|\lambda_{p + 1}|} \xrightarrow[p\to\infty]{} 0.
\end{equation}

Тобто можна вважати, що ермітове ядро в певному розумінні наближається наступним білінійним рядом:
\begin{equation}
	K(x, y) \sim \Sum_{i = 1}^\infty \dfrac{\phi_i(x) \overline \phi_i(y)}{\lambda_i}.
\end{equation}

Для виродженого ядра маємо його представлення у вигляді
\begin{equation}
	K(x, y) = \Sum_{i = 1}^N \dfrac{\phi_i(x) \overline \phi_i(y)}{\lambda_i}.
\end{equation}

\subsubsection{Ряд Фур'є функції із $L_2(G)$}

Розглянемо довільну функцію $f \in L_2(G)$ і деяку ортонормовану систему функцій $\{ u_i \}_{i = 1}^\infty$. 

\begin{definition}[ряда Фур'є]
	\it{Рядом Фур'є} функції $f$ із $L_2(G)$ називається
	\begin{equation}
		\Sum_{i = 1}^\infty (f, u_i) u_i \sim f.
	\end{equation}
\end{definition}

\begin{definition}[коефіцієнта Фур'є]
	Вираз $(f, u_i)$ називається \it{коефіцієнтом Фур'є}.
\end{definition}

\begin{theorem}[нерівність Бесселя]
	$\forall f \in L_2(G)$ виконується \it{нерівність Бесселя}: $\forall N$
	\begin{equation}
		\Sum_{i = 1}^N |(f, u_i)|^2 \le \|f\|_{L_2(G)}^2.
	\end{equation}
\end{theorem}

\begin{remark}
	Нерівність Бесселя гарантує збіжність ряду Фур'є в середньоквадратичному, але не обов'язково до функції $f$.
\end{remark}

\begin{definition}[повної (замкненої) системи функцій]
	Ортонормована система функцій $\{ u_i \}_{i = 1}^\infty$ називається \it{повною (замкненою)}, якщо ряд Фур'є для будь-якої функції $f \in L_2(G)$ по цій системі функцій збігається до цієї функції в просторі $L_2(G)$.
\end{definition}

\begin{theorem}[критерій повноти ортонормованої системи функцій]
	Для того щоб система функцій $\{ u_i \}_{i = 1}^\infty$ була повною в $L_2(G)$ необхідно і достатньо, щоби для будь-якої функції $f \in L_2(G)$ виконувалась рівність Парсеваля-Стеклова:
	\begin{equation}
		\Sum_{i = 1}^\infty |(f, u_i)|^2 = \|f\|_{L_2(G)}^2.
	\end{equation}
\end{theorem}

\subsubsection{Теорема Гільберта-Шмідта}

\begin{definition}[джерелувато-зображуваної функції]
	Функція $f(x)$ називається \it{джерелувато-зображуваною} через ермітове неперервне ядро $K(x, y) = K^\star (x, y)$, $K \in C(G \times G$, якщо існує функція $h(x) \in L_2(G)$, така що 
	\begin{equation}
		f(x) = \Int_G K(x, y) h(y) \diff y.
	\end{equation}
\end{definition}

\begin{theorem}[Гільберта-Шмідта]
	Довільна джерелувато-зображувана функція $f$ розкладається в абсолютно і рівномірно збіжний ряд Фур'є за системою власних функцій ермітового неперервного ядра $K(x, y)$
\end{theorem}

\begin{proof}
	Обчислимо коефіцієнти Фур'є:
	\begin{equation}
		(f, \phi_i) = (\bf{K}h, \phi_i) = (h, \bf{K}\phi_i) = \frac{(h,\phi_i)}{\lambda_i}.
	\end{equation}

	Отже ряд Фур'є функції $f$ має вигляд 
	\begin{equation}
		f \sim \Sum_{i = 1}^\infty \dfrac{(h, \phi_i)}{\lambda_i} \phi_i
	\end{equation}

	Якщо власних чисел скінчена кількість, то можливе точне представлення 
	\begin{equation}
		f(x) = \sum_{i=1}^N \frac{(h, \phi_i)}{\lambda_i} \phi_i(x),
	\end{equation}
	якщо ж власних чисел злічена кількість, то  можемо записати:
	\begin{equation}
		\left\| f - \Sum_{i = 1}^p \dfrac{(h, \phi_i)}{\lambda_i} \phi_i \right\|_{L_2(G)} = \left\| \bf{K}h - \Sum_{i = 1}^p \dfrac{(h, \phi_i)}{\lambda_i} \phi_i \right\|_{L_2(G)} \xrightarrow[p \to \infty]{}0.
	\end{equation}

	Покажемо, що формулу 
	\begin{equation}
		K(x, y) \sim \Sum_{i = 1}^\infty \dfrac{\phi_i(x) \overline \phi_i(y)}{\lambda_i}.
	\end{equation}
	можна розглядати як розвинення ядра $K(x, y)$ в ряд Фур'є по системі власних функцій $\phi_i(x)$. Перевіримо це обчислюючи коефіцієнт Фур'є:
	\begin{equation}
		\begin{aligned}
			(K(x, y),\phi_i)_{L_2(G)} &= \Int_G K(x, y) \overline \phi_i(x) \diff x = \\
			&= \Int_G \overline {K(y, x)} \overline \phi_i(x) \diff x = \dfrac{\overline \phi_i(y)}{\lambda_i}.
		\end{aligned}
	\end{equation}

	Доведемо рівномірну збіжність ряду Фур'є за критерієм Коші і покажемо, що при, $n, m \to \infty$, відрізок ряду прямує до нуля. За нерівністю Коші-Буняківського маємо:
	\begin{equation}
		\left| \Sum_{i = n}^m \dfrac{(h, \phi_i)}{\lambda_i} \phi_i \right| \le \Sum_{i = n}^m |(h, \phi_i)\dfrac{|\phi_i|}{|\lambda_i|} \le \left(\Sum_{i=n}^m |(h, \phi_i)|^2\right)^{1/2} \cdot \left(\Sum_{i=n}^m \dfrac{|\phi_i|^2}{\lambda_i^2}\right)^{1/2}
	\end{equation}

	Але
	\begin{equation}
		\Sum_{i=n}^m |(h, \phi_i)|^2 \le \|h\|_{L_2(G)}^2,
	\end{equation}
	тобто ряд збігається, а вказана сума прямує до 0 при $n, m \to \infty$.

	Зокрема маємо
	\begin{equation}
		\Sum_{i=n}^m \dfrac{|\phi_i|^2}{\lambda_i^2} \le \Int_G |K(x, y)|^2 \diff x \le M^2 V, 
	\end{equation}
	тобто ряд збігається. \medskip

	Отже
	\begin{equation}
		\left(\Sum_{i=n}^m |(h, \phi_i)|^2\right)^{1/2} \left(\Sum_{i=n}^m \dfrac{|\phi_i|^2}{\lambda_i^2}\right)^{1/2} \xrightarrow[n, m \to \infty]{} 0,
	\end{equation}
	а отже
	\begin{equation}
		\sum_{i=1}^\infty \frac{(h, \phi_i)}{\lambda_i} \phi_i
	\end{equation}
	збігається абсолютно і рівномірно.
\end{proof}

\begin{corollary}
	\label{corollary:2.4.10}
	Довільне повторне ядро для ермітового неперервного ядра $K(x ,y)$ розкладається в білінійний ряд по системі власних функцій ермітового неперервного ядра, який збігається абсолютно і рівномірно, а саме рядом
	\begin{equation}
		K_{(p)}(x, y) = \Sum_{i = 1}^\infty \dfrac{\phi_i(x) \overline \phi_i(y)}{\lambda_i^p},
	\end{equation}
	де $p = 2, 3, \ldots$, і коефіцієнти Фур'є  $\overline \phi_i(y) / \lambda_i^p$.
\end{corollary}

Повторне ядро $K_{(p)}(x,y) = \int_G K(x, \xi) K_{(p - 1)} (\xi, y) \diff \xi$ є джерелувато-зображувана функція і таким чином для нього має місце теорема Гільберта-Шмідта. \medskip

Доведемо деякі важливі рівності:
\begin{equation}
	\begin{aligned}
	K_{(2)} (x, x) &= \Int_G K(x, \xi) K(\xi, x) \diff \xi = \\
	&= \Int_G K(x, \xi) \overline {K(x, \xi)} \diff \xi = \\
	&= \Int_G |K(x, \xi)|^2 \diff \xi = \\
	&= \Sum_{i = 1}^\infty \dfrac{|\phi_i(x)|^2}{\lambda_i^2}.
	\end{aligned}
\end{equation}

\begin{remark}
	Останій перехід випливає з наслідку вище.
\end{remark}

Проінтегруємо отримане співвідношення, отримаємо
\begin{equation}
	\Iint_{G \times G} |K(x, y)|^2 \diff x \diff y = \Sum_{i = 1}^\infty \dfrac{1}{\lambda_i^2}.
\end{equation}

\begin{theorem}[про збіжність білінійного ряду для ермітового неперервного ядра]
	Ермітове неперервне ядро $K(x, y)$ розкладається в білінійний ряд
	\begin{equation}
		K(x, y) = \sum_{i=1}^\infty \frac{\phi_i(x) \overline \phi_i(y)}{\lambda_i}
	\end{equation}
	по своїх власних функціях, і цей ряд збігаються в нормі $L_2(G)$ по аргументу $x$ рівномірно для кожного $y \in \overline G$, тобто 
	\begin{equation}
		\left\| K(x, y) - \Sum_{i=1}^p \dfrac{\phi_i(x)\overline \phi_i(y)}{\lambda_i}\right\|_{L_2(x \in G)} \xrightrightarrows[p \to \infty]{y \in \overline G} 0.
	\end{equation}
\end{theorem}

\begin{proof}
	\begin{equation}
		\left\| K(x, y) - \Sum_{i = 1}^p \dfrac{\phi_i(x) \overline \phi_i(y)}{\lambda_i} \right\|_{L_2(G)}^2 = \Int_G |K(x, y)|^2 \diff x - \Sum_{i = 1}^p \dfrac{|\phi_i(y)|^2}{\lambda_i^2} \xrightrightarrows[p \to \infty]{y \in \overline G} 0.
	\end{equation}

	Додатково інтегруючи по аргументу $y \in G$ отримаємо збіжність вищезгаданого білінійного ряду в середньоквадратичному:
	\begin{equation}
		\Iint_{G \times G} \left( K(x, y) - \Sum_{i = 1}^p \dfrac{\phi_i(x) \overline \phi_i(y)}{\lambda_i} \right)^2 \diff y \xrightarrow[p \to \infty]{} 0.
	\end{equation}
\end{proof}

\subsubsection{Формула Шмідта для розв'язання інтегральних рівнянь з ермітовим неперервним ядром}

Розглянемо інтегральне рівняння Фредгольма 2 роду $\phi = \lambda \bf{K} \phi + f$, з ермітовим неперервним ядром 
\begin{equation} 
	K(x, y) = K^\star  (x, y).
\end{equation}
$\lambda_1, \ldots, \lambda_p, \ldots$, $\phi_1, \ldots, \phi_p, \ldots$ --- множина характеристичних чисел та ортонормована система власних функцій ядра $K(x, y)$. \medskip

Розкладемо розв'язок рівняння $\phi$ по системі власних функцій ядра $K(x, y)$:
\begin{equation}
	\begin{aligned}
		\phi &= \lambda \Sum_{i = 1}^\infty (\bf{K}\phi, \phi_i) \phi_i + f = \\
		&= \lambda \Sum_{i = 1}^\infty (\phi, \bf{K} \phi_i) \phi_i + f = \\
		&= \lambda \Sum_{i = 1}^\infty \dfrac{(\phi, \phi_i)}{\lambda_i} \phi_i + f,
	\end{aligned}
\end{equation}

Обчислимо коефіцієнти Фур'є:
\begin{equation}
	(\phi, \phi_k) = \lambda \Sum_{i = 1}^\infty \dfrac{(\phi, \phi_i)}{\lambda_i} (\phi_i, \phi_k) + (f, \phi_k) = \lambda \dfrac{(\phi, \phi_k)}{\lambda_k} + (f, \phi_k).
\end{equation}

Отже,
\begin{equation}
	(\phi, \phi_k) \left(1 - \frac{\lambda}{\lambda_k}\right) = (f, \phi_k),
\end{equation}
і тому
\begin{equation}
	(\phi, \phi_k) = (f, \phi_k) \frac{\lambda_k}{\lambda_k - \lambda}, \quad k = 1, 2, \ldots
\end{equation}

Таким чином має місце 
\begin{theorem}[формула Шмідта]
	Виконується співвідношення
	\begin{equation}
		\phi(x) = \lambda \Sum_{i = 1}^\infty \dfrac{(f, \phi_i)}{\lambda_i - \lambda} \phi_i(x) + f(x).
	\end{equation}
\end{theorem}

Розглянемо усі можливі значення $\lambda$:
\begin{enumerate}
	\item Якщо $\lambda \notin \{\lambda_i\}_{i=1}^\infty$, тоді існує єдиний розв'язок для довільного вільного члена $f$ і цей розв'язок представляється за формулою Шмідта.
	
	\item Якщо $\lambda = \lambda_k = \lambda_{k + 1} = \ldots = \lambda_{k + q - 1}$ --- співпадає з одним з характеристичних чисел кратності $q$, та при цьому виконуються умови ортогональності
	\begin{equation}
		(f, \phi_k) = (f, \phi_{k + 1}) = \ldots = (f, \phi_{k + q - 1}) = 0
	\end{equation}
	тоді розв'язок існує (не єдиний), і представляється у вигляді 
	\begin{equation}
		\phi(x) = \lambda_k \Sum_{\substack{i = 1 \\ \lambda_i \ne \lambda_k}}^\infty \dfrac{(f, \phi_i)}{\lambda_i - \lambda_k} \phi_i(x) + f(x) + \Sum_{j = k}^{k + q - 1} c_j \phi_j(x),
	\end{equation}
	де $c_j$ --- довільні константи.

	\item Якщо $\exists j: (f, \phi_j) \ne 0$, $k \le j \le k + q - 1$ то розв'язків не існує.
\end{enumerate}

\newpage

\begin{example}
	Знайти ті значення параметрів $a$, $b$ для яких інтегральне рівняння
	\begin{equation*}
		\phi(x) = \lambda \Int_{-1}^1 \left( xy - \dfrac{1}{3} \right) \phi(y) \diff y + ax^2 - bx + 1 
	\end{equation*}
	має розв'язок для будь-якого значення $\lambda$.
\end{example}

\begin{solution}
	Знайдемо характеристичні числа та власні функції спряженого однорідного рівняння (ядро ермітове).
	\begin{equation*}
		\phi(x) = \lambda x \Int_{-1}^1 y \phi(y) \diff y - \dfrac{\lambda}{3} \Int_{-1}^1 \phi(y) \diff y = \lambda x c_1 - \dfrac{\lambda}{3} c_2.
	\end{equation*}

	Маємо СЛАР:
	\begin{system*}
		c_1 &= \Int_{-1}^1 y \phi(y) \diff y = \Int_{-1}^1 y \left(\lambda y c_1 - \dfrac{\lambda}{3} c_2 \right) \diff y = \dfrac{2 \lambda}{3} c_1, \\
		c_2 &= \Int_{-1}^1 \phi(y) \diff y = \Int_{-1}^1 \left(\lambda y c_1 - \dfrac{\lambda}{3} c_2 \right) \diff y = - \dfrac{2 \lambda}{3} c_2.
	\end{system*}

	Її визначник
	\begin{equation*}
		D(\lambda) = \begin{vmatrix} 1 - \dfrac{2\lambda}{3} & 0 \\ 0 & 1 + \dfrac{2\lambda}{3} \end{vmatrix} = 0.
	\end{equation*}

	Тобто характеристичні числа
	\begin{equation*}
		\lambda_1 = \dfrac{3}{2}, \quad \lambda_2 = - \dfrac{3}{2}.
	\end{equation*}

	А відповідні власні функції
	\begin{equation*}
		\phi_1(x) = x, \quad \phi_2(x) = 1.
	\end{equation*}

	Умови ортогональності:
	\begin{system*}
		\Int_{-1}^1 (ax^2 - bx + 1) x \diff x = - \dfrac{2b}{3} &= 0, \\
		\Int_{-1}^1 (ax^2 - bx + 1) \diff x = \dfrac{2a}{3} + 2 &= 0.
	\end{system*}

	Тобто розв'язок існує для будь-якого $\lambda$ якщо
	\begin{equation*}
		a = -3, \quad b = 0. 
	\end{equation*}
\end{solution}

 \end{document}