\documentclass[a4paper, 12pt]{article}
\usepackage[utf8]{inputenc}
\usepackage[english, ukrainian]{babel}

\usepackage{amsmath, amssymb}
\usepackage{multicol}
\usepackage{graphicx}
\usepackage{float}

\allowdisplaybreaks
\setlength\parindent{0pt}
\numberwithin{equation}{subsection}

\usepackage{hyperref}
\hypersetup{unicode=true,colorlinks=true,linktoc=all,linkcolor=red}

\numberwithin{equation}{subsection}

\renewcommand{\bf}[1]{\textbf{#1}}
\renewcommand{\it}[1]{\textit{#1}}
\newcommand{\bb}[1]{\mathbb{#1}}
\renewcommand{\cal}[1]{\mathcal{#1}}

\renewcommand{\epsilon}{\varepsilon}
\renewcommand{\phi}{\varphi}

\DeclareMathOperator{\diam}{diam}
\DeclareMathOperator{\rang}{rang}
\DeclareMathOperator{\const}{const}

\newenvironment{system}{%
  \begin{equation}%
    \left\{%
      \begin{aligned}%
}{%
      \end{aligned}%
    \right.%
  \end{equation}%
}
\newenvironment{system*}{%
  \begin{equation*}%
    \left\{%
      \begin{aligned}%
}{%
      \end{aligned}%
    \right.%
  \end{equation*}%
}

\makeatletter
\newcommand*{\relrelbarsep}{.386ex}
\newcommand*{\relrelbar}{%
  \mathrel{%
    \mathpalette\@relrelbar\relrelbarsep%
  }%
}
\newcommand*{\@relrelbar}[2]{%
  \raise#2\hbox to 0pt{$\m@th#1\relbar$\hss}%
  \lower#2\hbox{$\m@th#1\relbar$}%
}
\providecommand*{\rightrightarrowsfill@}{%
  \arrowfill@\relrelbar\relrelbar\rightrightarrows%
}
\providecommand*{\leftleftarrowsfill@}{%
  \arrowfill@\leftleftarrows\relrelbar\relrelbar%
}
\providecommand*{\xrightrightarrows}[2][]{%
  \ext@arrow 0359\rightrightarrowsfill@{#1}{#2}%
}
\providecommand*{\xleftleftarrows}[2][]{%
  \ext@arrow 3095\leftleftarrowsfill@{#1}{#2}%
}
\makeatother

\newcommand{\NN}{\mathbb{N}}
\newcommand{\ZZ}{\mathbb{Z}}
\newcommand{\QQ}{\mathbb{Q}}
\newcommand{\RR}{\mathbb{R}}
\newcommand{\CC}{\mathbb{C}}

\newcommand{\Max}{\displaystyle\max\limits}
\newcommand{\Sup}{\displaystyle\sup\limits}
\newcommand{\Sum}{\displaystyle\sum\limits}
\newcommand{\Int}{\displaystyle\int\limits}
\newcommand{\Iint}{\displaystyle\iint\limits}
\newcommand{\Lim}{\displaystyle\lim\limits}

\newcommand*\diff{\mathop{}\!\mathrm{d}}

\newcommand*\rfrac[2]{{}^{#1}\!/_{\!#2}}


\title{{\Huge МАТЕМАТИЧНА ФІЗИКА}}
\author{Скибицький Нікіта}
\date{\today}

\usepackage{amsthm}
\usepackage[dvipsnames]{xcolor}
\usepackage{thmtools}
\usepackage[framemethod=TikZ]{mdframed}

\theoremstyle{definition}
\mdfdefinestyle{mdbluebox}{%
	roundcorner = 10pt,
	linewidth=1pt,
	skipabove=12pt,
	innerbottommargin=9pt,
	skipbelow=2pt,
	nobreak=true,
	linecolor=blue,
	backgroundcolor=TealBlue!5,
}
\declaretheoremstyle[
	headfont=\sffamily\bfseries\color{MidnightBlue},
	mdframed={style=mdbluebox},
	headpunct={\\[3pt]},
	postheadspace={0pt}
]{thmbluebox}

\mdfdefinestyle{mdredbox}{%
	linewidth=0.5pt,
	skipabove=12pt,
	frametitleaboveskip=5pt,
	frametitlebelowskip=0pt,
	skipbelow=2pt,
	frametitlefont=\bfseries,
	innertopmargin=4pt,
	innerbottommargin=8pt,
	nobreak=true,
	linecolor=RawSienna,
	backgroundcolor=Salmon!5,
}
\declaretheoremstyle[
	headfont=\bfseries\color{RawSienna},
	mdframed={style=mdredbox},
	headpunct={\\[3pt]},
	postheadspace={0pt},
]{thmredbox}

\declaretheorem[style=thmbluebox,name=Теорема,numberwithin=subsubsection]{theorem}
\declaretheorem[style=thmbluebox,name=Лема,numberwithin=subsubsection]{lemma}
\declaretheorem[style=thmbluebox,name=Твердження,numberwithin=subsubsection]{proposition}
\declaretheorem[style=thmbluebox,name=Принцип,numberwithin=subsubsection]{th_principle}
\declaretheorem[style=thmbluebox,name=Закон,numberwithin=subsubsection]{law}
\declaretheorem[style=thmbluebox,name=Закон,numbered=no]{law*}
\declaretheorem[style=thmbluebox,name=Формула,numberwithin=subsubsection]{th_formula}
\declaretheorem[style=thmbluebox,name=Рівняння,numberwithin=subsubsection]{th_equation}
\declaretheorem[style=thmbluebox,name=Умова,numberwithin=subsubsection]{th_condition}
\declaretheorem[style=thmbluebox,name=Наслідок,numberwithin=subsubsection]{corollary}

\declaretheorem[style=thmredbox,name=Приклад,numberwithin=subsubsection]{example}
\declaretheorem[style=thmredbox,name=Приклади,sibling=example]{examples}

\declaretheorem[style=thmredbox,name=Властивість,numberwithin=subsubsection]{property}
\declaretheorem[style=thmredbox,name=Властивості,sibling=property]{properties}

\mdfdefinestyle{mdgreenbox}{%
	skipabove=8pt,
	linewidth=2pt,
	rightline=false,
	leftline=true,
	topline=false,
	bottomline=false,
	linecolor=ForestGreen,
	backgroundcolor=ForestGreen!5,
}
\declaretheoremstyle[
	headfont=\bfseries\sffamily\color{ForestGreen!70!black},
	bodyfont=\normalfont,
	spaceabove=2pt,
	spacebelow=1pt,
	mdframed={style=mdgreenbox},
	headpunct={ --- },
]{thmgreenbox}

\mdfdefinestyle{mdblackbox}{%
	skipabove=8pt,
	linewidth=3pt,
	rightline=false,
	leftline=true,
	topline=false,
	bottomline=false,
	linecolor=black,
	backgroundcolor=RedViolet!5!gray!5,
}
\declaretheoremstyle[
	headfont=\bfseries,
	bodyfont=\normalfont\small,
	spaceabove=0pt,
	spacebelow=0pt,
	mdframed={style=mdblackbox}
]{thmblackbox}

\declaretheorem[name=Вправа,numberwithin=subsubsection,style=thmblackbox]{exercise}
\declaretheorem[name=Зауваження,numberwithin=subsubsection,style=thmgreenbox]{remark}
\declaretheorem[name=Визначення,numberwithin=subsubsection,style=thmblackbox]{definition}

\newtheorem{problem}{Задача}[subsection]
\newtheorem{sproblem}[problem]{Задача}
\newtheorem{dproblem}[problem]{Задача}
\renewcommand{\thesproblem}{\theproblem$^{\star}$}
\renewcommand{\thedproblem}{\theproblem$^{\dagger}$}
\newcommand{\listhack}{$\empty$\vspace{-2em}} 

\theoremstyle{remark}
\newtheorem*{solution}{Розв'язок}


\begin{document}

\tableofcontents

\setcounter{section}{3}
\setcounter{subsection}{6}

\subsection{Класифікація рівнянь в частинних похідних}

\subsubsection{Класифікація рівнянь з двома незалежними змінними}

Будемо розглядати загальне рівняння другого порядку з двома незалежними змінними, лінійне відносно старших похідних.
\begin{definition}[головної частини рівняння]
	Частину рівняння, яка містить похідні старшого порядку називають \it{головною} частиною рівняння:
	\begin{equation}
		A(x,y)\dfrac{\partial^2u}{\partial x^2}+2B(x,y)\dfrac{\partial^2u}{\partial x\partial y}+C(x,y)\dfrac{\partial^2u}{\partial y^2}+F\left(x,y,u,\dfrac{\partial u}{\partial x},\dfrac{\partial u}{\partial y}\right)=0.
	\end{equation}
\end{definition}

Поставимо задачу спростити вигляд головної частини рівняння. Для чого введемо заміну змінних: 
\begin{equation}
	\xi = \xi(x, y), \quad \eta = \eta(x, y).
\end{equation}

Для скорочення запису скористаємося позначеннями  $\partial u / \partial x = u_x$. \medskip

Обчислимо вирази для похідних в нових змінних $\xi$, $\eta$:
\begin{align}
	u_x &= u_\xi \xi_x + u_\eta \eta_x, \\
	u_y &= u_\xi \xi_y + u_\eta \eta_y, \\
	u_{xx} &= u_{\xi\xi} \xi_x^2 + 2 u_{\xi\eta} \xi_x \eta_x + u_{\eta\eta} \eta_x^2 + u_\xi \xi_{xx} + u_\eta \eta_{xx}, \\
	u_{yy} &= u_{\xi\xi} \xi_y^2 + 2 u_{\xi\eta} \xi_y \eta_y + u_{\eta\eta} \eta_y^2 + u_\xi \xi_{yy} + u_\eta \eta_{yy}, \\
	u_{xy} &= u_{\xi\xi} \xi_x \xi_y + u_{\xi\eta} (\xi_x \eta_y + \xi_y \eta_x) + u_{\eta\eta} \eta_x \eta_y + u_\xi \xi_{xy} + u_\eta \eta_{xy}.
\end{align}

Підставимо обчислені похідні в рівняння:
\begin{multline}
	A (u_{\xi\xi} \xi_x^2 + 2 u_{\xi\eta} \xi_x \eta_x + u_{\eta\eta} \eta_x^2 ) + \\
	+ 2 B(u_{\xi\xi} \xi_x \xi_y + u_{\xi\eta}(\xi_x\eta_y+\xi_y\eta_x)+u_{\eta\eta}\cdot\eta_x\cdot\eta_y) + \\
	+ C (u_{\xi\xi} \xi_y^2 + 2 u_{\xi\eta}\xi_y\eta_y + u_{\eta\eta}\eta_y^2) + \widetilde{F}(\xi, \eta, u, u_\xi, u_\eta) = 0.
\end{multline}

Перегрупуємо доданки і отримаємо рівняння у вигляді:
\begin{equation}
	\overline{A} u_{\xi\xi} + 2\overline{B}u_{\xi\eta} + \overline{C} u_{\eta\eta} + \widetilde{F}(\xi,\eta,u,u_\xi,u_\eta)=0,
\end{equation}
де 
\begin{align}
	\overline{A} &= A \xi_x^2 + 2 B \xi_x \xi_y + C \xi_y^2 \\
	\overline{B} &= A \xi_x \eta_x + B(\xi_x \eta_y + \xi_y \eta_x) + C \xi_y \eta_y \\
	\overline{C} &= A \eta_y^2 + 2 B \eta_x \eta_y + C \eta_y^2
\end{align}

Зробимо нульовими коефіцієнти при $u_{\xi\xi}$ та $u_{\eta\eta}$ за рахунок вибору нових змінних:
\begin{align}
	A \xi_x^2 + 2 B \xi_x \xi_y + C \xi_y^2 &= 0, \\
	A \eta_x^2 + 2 B \eta_x \eta_y + C \eta_y^2 &= 0,
\end{align}

Розв'язком цих рівнянь будуть функції $\xi(x,y)$ і $\eta(x,y)$ відповідно. \medskip

Для знаходження функцій $\xi(x, y)$, $\eta(x,y)$, зведемо рівняння в частинних похідних до звичайного диференціального рівняння. \medskip

Розглянемо перше з цих рівнянь і розділимо його на $\xi_y^2$:
\begin{equation}
	A \left( \dfrac{\xi_x}{\xi_y} \right)^2 + 2 B \dfrac{\xi_x}{\xi_y} + C = 0.
\end{equation}
Аналочічне рівняння можно отримати і для функції  $\eta(x,y)$.

Розглянемо неявно задану функцію $y = y(x)$ у вигляді $\xi(x, y) = \const$, легко бачити
\begin{equation}
	\diff \xi = \xi_x \diff x + \xi_y \diff y = 0
\end{equation}
або
\begin{equation}
	\dfrac{\xi_x}{\xi_y} = - \dfrac{\diff y}{\diff x}.
\end{equation}

Тобто рівняння в частинних похідних зводиться до звичайного диференціального характеристичного рівняння:
\begin{definition}[характеристичного рівняння]
	\it{Характеристичним} називається рівняння
	\begin{equation}
		A \left( \dfrac{\diff y}{\diff x} \right)^2 - 2 B \dfrac{\diff y}{\diff x} + C = 0.
	\end{equation}
\end{definition}

\begin{definition}[характеристик]
	Його розв'язки називаються \it{характеристиками}.
\end{definition}
Квадратне рівняння відносно похідної розпадається на два лінійних рівняння:
\begin{align}
	\dfrac{\diff y}{\diff x} &= \dfrac{B + \sqrt{B^2 - AC}}{A}, \\ 
	\dfrac{\diff y}{\diff x} &= \dfrac{B - \sqrt{B^2 - AC}}{A}.
\end{align}

Знак підкореневого виразу визначає тип рівняння і спосіб вибору нових змінних. Розглянемо можливі випадки:
\begin{enumerate}
	\item рівняння гіперболічного типу $B^2 - AC > 0$. \medskip

	Кожне з  характеристичних рівнянь має по одній дійсній характеристиці. Нехай $\phi(x,y)=\const$ та $\psi(x,y)=\const$ --- загальні інтеграли першого та другого характеристичного рівняння, тоді нові змінні вибираються у вигляді
	\begin{align}
		\xi &= \phi(x, y), \\
		\eta &= \psi(x, y).
	\end{align}

	Після застосування вказаної заміни змінних отримаємо першу канонічну форму запису рівняння гіперболічного типу 
	\begin{equation}
		u_{\xi\eta} = \Phi(\xi, \eta, u, u_\xi, u_\eta).
	\end{equation}

	\begin{remark}
		Якщо використати змінні
		\begin{align}
			\alpha &= \dfrac{\xi+\eta}{2}, \\
			\beta &= \dfrac{\xi-\eta}{2},
		\end{align}
		то можна отримати другу канонічну форму запису рівняння гіперболічного типу
		\begin{equation}
			u_{\alpha\alpha} - u_{\beta\beta} = \Phi_1(\alpha, \beta, u, u_\alpha, u_\beta).
		\end{equation}
	\end{remark}

	\item рівняння еліптичного типу $B^2 - AC < 0$. \medskip

	В цьому випадку розв'язки характеристичних рівнянь (характеристики) --- комплексно спряжені і можуть бути записані у вигляді: $\omega(x,y)=\phi(x,y)\pm i\psi(x,y)=\const$. \medskip

	Тоді для змінних
	\begin{align}
		\xi &= \phi(x,y) + i \psi(x, y), \\
		\eta &= \phi(x, y) - i \psi(x, y)
	\end{align}
	отримаємо вигляд аналогічний першій канонічній формі гіперболічного рівняння але з комплесними незалежними змінними.
	\begin{equation}
		u_{\xi\eta}=\Phi(\xi,\eta,u,u_\xi,u_\eta).
	\end{equation}

	Для того щоб позбутися комплексних змінних виберемо нові дійсні змінні
	\begin{align}
		\alpha &= \phi(x,y) = \dfrac{\xi+\eta}{2}, \\
		\beta &= \psi(x,y) = \dfrac{\xi-\eta}{2i}.
	\end{align}
	
	Тоді отримаємо канонічну форму запису рівняння еліптичного типу у вигляді:
	\begin{equation}
		u_{\alpha\alpha} + u_{\beta\beta} = \Phi(\alpha,\beta,u,u_\alpha,u_\beta).
	\end{equation}

	\item рівняння параболічного типу $B^2 - AC = 0$. \medskip

	Характеристики в цьому випадку співпадають і нові змінні обирають у вигляді: 
	\begin{align}
		\xi &= \phi(x,y), \\
		\eta &= \nu(x,y),
	\end{align}
	де $\nu(x,y)$ --- будь-яка функція незалежна від $\phi(x,y)$. \medskip

	\begin{remark}
		Необхідно, щоб визначник Вронського для нових змінних $W[\cdot]\ne0$, тобто, щоб заміна змінних була не виродженою.
	\end{remark}

	У випадку параболічного рівняння маємо $AC = B^2$ і таким чином 
	\begin{equation}
		\begin{aligned}
			\overline{A} &= (A \xi_x^2 + 2 B \xi_x \xi_y + C \xi_y^2) = \\
			&= \left( \sqrt{A} \xi_x + \sqrt{C} \xi_y \right)^2 = 0.
		\end{aligned}
	\end{equation}
	і
	\begin{equation}
		\begin{aligned}
			\overline{B} &= A \xi_x \eta_x + B(\xi_x\eta_y+\xi_y\eta_x) + C\xi_y\eta_y = \\
			&= \left(\sqrt{A}\xi_x+\sqrt{C}\xi_y\right)\left(\sqrt{A}\eta_x+\sqrt{C}\eta_y\right) = 0.
		\end{aligned}
	\end{equation}

	При цьому $\overline{C} \ne 0$. Таким чином після ділення на $\overline{C}$ отримаємо канонічну форму запису рівняння гіперболічного типу.
	\begin{equation}
		u_{\eta\eta} = \Phi(\xi,\eta,u,u_\xi,u_\eta).
	\end{equation}
\end{enumerate}

\subsubsection{Класифікація рівнянь другого порядку з багатьма незалежними змінними}

Розглянемо лінійне рівняння з дійсними коефіцієнтами
\begin{equation}
	\Sum_{i=1}^n \Sum_{j=1}^n A_{i,j} u_{x_i x_j} + \Sum_{i=1}^n B_i u_{x_i} + Cu + F = 0,
\end{equation}
де $A_{i,j} = A_{j,i}$, $A_{i,j}$, $B_i$, $C$, $F$ є функціями від $x = (x_1, x_2, \ldots, x_n)$. \medskip

Введемо нові змінні
\begin{equation}
	\xi_k = \xi_k(x_1, x_2, \ldots, x_n), \quad k = \overline{1, n}.
\end{equation}

Обчислимо похідні, що входять в рівняння
\begin{align}
	u_{x_i} &= \Sum_{k=1}^n u_{\xi_k} \alpha_{i,k}, \\
	u_{x_i x_j} &= \Sum_{k=1}^n \Sum_{l=1}^n u_{\xi_k\xi_l} \alpha_{i,k}\alpha_{j,l} + \Sum_{k=1}^n u_{\xi_k} (\xi_k)_{x_ix_j},
\end{align}
де $\alpha_{i,k}=\partial\xi_k/\partial x_i$. \medskip

Підставляючи вираз для похідних в вихідне рівняння отримаємо:
\begin{equation}
	\Sum_{k=1}^n \Sum_{l=1}^n \overline{A}_{k,l} u_{\xi_k\xi_l} + \Sum_{k=1}^n \overline{B}_k u_{\xi_k} + \overline{C} u + \overline{F} = 0,
\end{equation}
де
\begin{align}
	\label{eq:linear-coefficients-transform-a}
	\overline{A}_{k,l} &= \Sum_{i=1}^n \Sum_{j=1}^n A_{i,j}\alpha_{i,k}\alpha_{j,l}, \\
	\label{eq:linear-coefficients-transform-b}
	\overline{B}_k &= \Sum_{i=1}^n B_i \alpha_{i,k} + \Sum_{i=1}^n \Sum_{j=1}^n A_{i,j}(\xi_k)_{x_ix_j}.
\end{align}

Поставимо у відповідність головній частині диференціального рівнянню квадратичну форму
\begin{equation}
	\Sum_{i=1}^n \Sum_{j=1}^n A_{i,j}^0 y_iy_j,
\end{equation}
де $A_{i,j}^0=A_{i,j}(x_1^0,\ldots,x_n^0)$, тобто коефіцієнти квадратичної форми співпадають з коефіцієнтами рівняння в деякій точці області. \medskip

Здійснимо над змінними $y$ лінійне перетворення
\begin{equation}
	y_i = \Sum_{k=1}^n \alpha_{i,k} \eta_k.
\end{equation}

Будемо мати для квадратичної форми новий вираз:
\begin{equation}
	\Sum_{i=1}^n \Sum_{j=1}^n A_{i,j}^0 y_iy_j = \Sum_{k=1}^n\Sum_{l=1}^n \overline{A}_{k,l}^0 \eta_k \eta_l,
\end{equation}
де 
\begin{equation}
	\label{eq:linear-coefficients-transform-main-part}
	\overline{A}_{k,l}^0 = \Sum_{i=1}^n \Sum_{j=1}^n \overline{A}_{i,j}^0 \alpha_{i,k}\alpha_{j,l}.
\end{equation}

Таким чином можна бачити, що коефіцієнти головної частини рівняння \eqref{eq:linear-coefficients-transform-main-part} перетворюються аналогічно коефіцієнтам квадратичної форми при описаному лінійному перетворенні \eqref{eq:linear-coefficients-transform-a}, \eqref{eq:linear-coefficients-transform-b}. Відомо, що використовуючи лінійне перетворення можна привести матрицю $\left[A_{i,j}^0\right]_{i,j=\overline{1,n}}$ квадратичної форми до діагонального вигляду, в якому $\overline{A}_{i,j}^0 = l_i\delta_{i,j}$, де  $l_i \in \{-1,0,1\}$  . \medskip

Згідно до закону інерції квадратичних форм, кількість додатних, від'єм\-них та нульових діагональних елементів інваріантне відносно лінійного перетворення.

\begin{definition}[еліптичного рівняння]
	Будемо називати рівняння в точці $(x_1^0,\ldots,x_n^0)$ \it{еліптичним}, якщо всі $\overline{A}_{i,i}^0$, $i=\overline{1,n}$ мають один і той же знак  та жодний діагональний елемент не дорівнює нулю. 
\end{definition}

\begin{definition}[гіперболічного рівняння]
	Будемо називати рівняння в точці $(x_1^0,\ldots,x_n^0)$ \it{гіперболічним}, якщо $m < n$ елементів матриці мають один знак, а $n - m$ коефіцієнтів мають протилежний знак та жодний діагональний елемент не дорівнює нулю. 
\end{definition}

\begin{definition}[параболічного рівняння]
	Будемо називати рівняння в точці $(x_1^0,\ldots,x_n^0)$ \it{параболічним}, якщо хоча б один з діагональних елементів матриці $\overline{A}_{i,i}^0$ дорівнює нулю. 
\end{definition}

Обираючи нові незалежні змінні $\xi_i$ таким чином що б у точці $(x_1^0,\ldots,x_n^0)$ виконувалось рівність $\alpha_{i,k}= \partial \xi_k / \partial x_i = \alpha_{i,k}^0$, де $\alpha_{i,k}^0$ --- коефіцієнти перетворення, яке приводить квадратичну форму до канонічної форми запису. Зокрема, покладаючи
\begin{equation}
	\xi_k = \Sum_{i=1}^n \alpha_{i,k}^0 x_i,
\end{equation}
отримаємо у точці $(x_1^0,\ldots,x_n^0)$ канонічну форму запису рівняння в залежності від його типу.

\begin{th_equation}[канонічна форма еліптичного рівняння]
	\begin{equation}
		\Sum_{i=1}^n u_{\xi_i\xi_i} + \Phi = 0
	\end{equation}
\end{th_equation}

\begin{th_equation}[канонічна форма гіперболічного рівняння]
	\begin{equation}
		\Sum_{i=1}^m u_{\xi_i\xi_i} - \Sum_{i=m+1}^n u_{\xi_i\xi_i} + \Phi = 0
	\end{equation}
\end{th_equation}

\begin{th_equation}[канонічна форма параболічного рівняння]
	\begin{equation}
		\Sum_{i=1}^m \pm u_{\xi_i\xi_i} + \Phi = 0,
	\end{equation}
\end{th_equation}
де $\Phi=\Phi(\nabla_\xi u, u,\xi) $

\begin{example}
	Визначити тип рівняння і привести його до канонічної форми запису
	\begin{equation*}
		y^2 u_{xx} - x^2 u_{yy} = 0.
	\end{equation*}
\end{example}

\begin{solution}
	Складемо характеристичне рівняння
	\begin{equation*}
		\dfrac{\diff y}{\diff x} = \dfrac{0\pm\sqrt{0+(xy)^2}}{y^2}, \quad \dfrac{\diff y}{\diff x} = \pm\dfrac{x}{y}.
	\end{equation*}

	Оскільки обидві характеристики є дійсними, то рівняння має гіперболічний тип. Останнє рівняння можна записати у вигляді:
	\begin{equation*}
		y \diff y = \pm x \diff x.
	\end{equation*}

	Загальні інтеграли цього рівняння мають вигляд:
	\begin{equation*}
		y^2 \pm x^2 = \const.
	\end{equation*}

	Для отримання першої канонічної форми запису необхідно обрати нові змінні
	\begin{align*}
		\xi &= x^2 + y^2,\\
		\eta &= x^2 - y^2.
	\end{align*}

	Для отримання другої канонічної форми запису оберемо змінні
	\begin{align*}
		\alpha &= \dfrac{\xi + \eta}{2} = x^2, \\
		\beta &= \dfrac{\xi-\eta}{2} = y^2,
	\end{align*}

	обчислимо похідні 
	\begin{align*}
		u_x &= u_\alpha \cdot 2 x + u_\beta \cdot 0, \\
		u_y &= u_\alpha \cdot 0 + u_\beta \cdot 2 y, \\
		u_{xx} &= u_{\alpha\alpha} \cdot 4x^2 + 2 u_{\alpha\beta} \cdot 0 + u_{\beta\beta} \cdot 0 + u_\alpha \cdot 2 + u_\beta \cdot 0, \\
		u_{yy} &= u_{\alpha\alpha} \cdot 0 + 2 u_{\alpha\beta} \cdot 0 + u_{\beta\beta} \cdot 4y^2 + u_\alpha \cdot 0 + u_\beta \cdot 2.
	\end{align*}

	Підставимо знайдені похідні в рівняння:
	\begin{equation*}
		y^2(4x^2u_{\alpha\alpha} + 2u_\alpha) - x^2(4y^2u_{\beta\beta} + 2u_\beta) = 0,
	\end{equation*}
	або
	\begin{equation*}
		u_{\alpha\alpha} - u_{\beta\beta} + \dfrac{u_\alpha - u_\beta}{2x^2y^2} = 0,
	\end{equation*}
	і остаточно маємо:
	\begin{equation*}
		u_{\alpha\alpha} - u_{\beta\beta} + \dfrac{u_\alpha - u_\beta}{2\alpha\beta} = 0.
	\end{equation*}
\end{solution}

\begin{example}
	Визначити тип рівняння і привести його до канонічної форми запису
	\begin{equation*}
		u_{xx} + 2 u_{xy} - 2 u_{xz} + 2 u_{yy} + 6 u_{zz} = 0
	\end{equation*}
\end{example}

\begin{solution}
	Поставимо у відповідність головній частині рівняння квадратичну форму:
	\begin{equation*}
		\lambda_1^2+2\lambda_1\lambda_2-2\lambda_1\lambda_3+2\lambda_2^2+6\lambda_3^2. 
	\end{equation*}

	Методом виділення повних квадратів приведемо квадратичну форму до канонічної форми запису
	\begin{multline*}
		\lambda_1^2+2\lambda_1\lambda_2-2\lambda_1\lambda_3+2\lambda_2^2+6\lambda_3^2 = \\
		= (\lambda_1^2+2\lambda_1\lambda_2-2\lambda_1\lambda_3+\lambda_2^2+\lambda_3^2 - 2 \lambda_2\lambda_3) + \lambda_2^2 + 2\lambda_2\lambda_3 + 5\lambda_3^2 = \\
		= (\lambda_1+\lambda_2-\lambda_3)^2 + (\lambda_2^2 + 2\lambda_2\lambda_3+\lambda_3^2) + 4\lambda_3^2 = \\
		= (\lambda_1+\lambda_2-\lambda_3)^2 + (\lambda_2 + \lambda_3)^2 + (2\lambda_3)^2.
	\end{multline*}

	Вводимо нові незалежні змінні:
	\begin{system*}
		\mu_1 &= \lambda_1 + \lambda_2 - \lambda_3, \\
		\mu_2 &= \lambda_2 + \lambda_3, \\ 
		\mu_3 &= 2 \lambda_3.
	\end{system*}

	Отримаємо канонічну форму запису для квадратичної форми:
	\begin{equation}
		\mu_1^2 + \mu_2^2 + \mu_3^2.
	\end{equation}

	Оскільки при квадраті кожної змінної коефіцієнт дорівнює 1, то квадратична форма та рівняння мають еліптичний тип. \medskip

	Запишемо тепер заміну змінних, яка приведе рівняння до канонічної форми запису. Обчислимо матрицю оберненого лінійного перетворення:
	\begin{system*}
		\lambda_1 &= \mu_1 - \mu_2 + \mu_3, \\
		\lambda_2 &= \mu_2 - \mu_3/2, \\
		\lambda_3 &= \mu_3 / 2.
	\end{system*}

	Тобто
	\begin{equation*}
		B = \begin{pmatrix} 
			1 & -1 & 1 \\
			0 & 1 & -1/2 \\
			0 & 0 & 1/2
		\end{pmatrix}, \quad \lambda = B \mu.
	\end{equation*}

	Обчислимо транспоновану матрицю
	\begin{equation*}
		B^\intercal = \begin{pmatrix} 
			1 & 0 & 0 \\
			-1 & 1 & 0 \\
			1 & -1/2 & 1/2
		\end{pmatrix}.
	\end{equation*}

	Для диференціального рівняння маємо нові змінні: $\vec \xi = B^\intercal \vec x$. Або в координатній формі:
	\begin{system*}
		\xi &= x, \\
		\eta &= y - x, \\
		\zeta &= x - y / 2 + z / 2.
	\end{system*}

	У новій системі координат головна частина диференціального рівняння буде мати канонічну форму запису еліптичного рівняння
	\begin{equation*}
		u_{\xi\xi} + u_{\eta\eta} + u_{\zeta\zeta} = 0.
	\end{equation*}
\end{solution}

\subsubsection{Загальні принципи класифікації рівнянь довільного порядку і систем диференціальних рівнянь}

Розглянемо диференціальне рівняння $m$-го порядку відносно скалярної функції $u(x)$ змінної $x = (x_1, x_2, \ldots, x_n)$ вигляду:
\begin{equation}
	\Sum_{|\alpha|=m}a_\alpha D^\alpha u + f(D^\beta u, x) = 0.
\end{equation}

Коефіцієнти $a_\alpha$ залежать лише від $x$, функція $f$ --- від $x$, $u$ та похідних $D^\beta u$, $|\beta| < m$. 

\begin{remark}
	Позначення $D^\alpha$ треба розуміти як
	\begin{equation}
		D^\alpha = \frac{\partial^{\alpha_1+\alpha_2+\ldots+\alpha_n}}{\partial x_1^{\alpha_1}\partial x_2^{\alpha_2}\ldots\partial x_n^{\alpha_n}},
	\end{equation}
	тобто змішану похідну. 
\end{remark}

Перший доданок містить лише старші похідні рівняння і за аналогією з рівняннями другого порядку можемо дати

\begin{definition}[головної частини рівняння]
	Частину диференціального рівняння, яка містить похідні старшого порядку 
	\begin{equation}
		A_0(x, D) u = \Sum_{|\alpha|=m}a_\alpha D^\alpha u
	\end{equation}
	будемо називати \it{головною} частиною рівняння.
\end{definition}

З головною частиною зв'яжемо характеристичну форму (однорідний поліном):
\begin{equation}
	A_0(x, \xi) = \Sum_{|\alpha|=m}a_\alpha \xi^\alpha,
\end{equation}
де $\xi^\alpha = (\xi_1^{\alpha_1}, \xi_2^{\alpha_2}, \ldots, \xi_n^{\alpha_n})$. \medskip

Класифікація рівняння в частинних похідних дуже тісно зв'язана з властивостями характеристичної форми. Зрозуміло, що властивості форми залежать від точки $x$ і тому рівняння можна класифікувати по різному у різних точках $x$. \medskip

\begin{definition}[параболічного рівняння]
	Рівняння будемо називати параболічним в точці $x$, якщо існує таке афінне перетворення змінних $\xi_i = \xi_i(\eta_1, \ldots, \eta_n)$, $i = \overline{1,n}$, що в результаті застосування перетворення до характеристичної форми, вона буде містити лише $l < n$ нових змінних.
\end{definition}

\begin{definition}[еліптичного рівняння]
	Рівняння будемо називати еліптичним у точці $x$, якщо форма $A_0(x, \xi)$ знаковизначена, як функція змінної $\xi$, тобто приймає значення одного знаку для будь-яких дійсних значень $\xi \in \RR^n$, $|\xi|\ne0$, або якщо алгебраїчне рівняння $A_0(x,\xi) = 0$ не має дійсних коренів окрім $\xi = 0$.
\end{definition}

\begin{definition}
	Рівняння будемо називати гіперболічним в точці $x$, якщо можна знайти таке афінне перетворення змінних $\xi_i = \xi_i(\eta_1, \ldots, \eta_n)$, $i = \overline{1,n}$, що у просторі нових змінних $\eta_1, \ldots, \eta_n$ існує такий напрям (нехай це змінна $\eta_1$), що алгебраїчне рівняння $A_0(x, \xi(\eta)) = 0$ записане відносно цієї змінної $\eta_1$ має рівно $m$ дійсних коренів (простих або кратних) при довільному виборі останніх змінних $\eta_2, \ldots, \eta_n$.
\end{definition}

Розглянемо систему рівнянь в частинних похідних відносно $n$ невідомих функцій $u_1, u_2, \ldots, u_n$ та запишемо її у матричному вигляді:
\begin{equation}
	A(x, D) y = f,
\end{equation}
де $A(x, D)$ --- $n \times n$ матриця з елементами $A_{i,j}(x,D)$, які представляють собою диференціальні вирази деякого порядку $\mu_{i,j}$. \medskip

В матриці $A(x,D)$ можна виділити головну частину, яка містить диференціальні оператори лише старшого порядку $m$, тоді головній частині буде відповідати матриця
\begin{equation}
	[A_0(x, D)]_{i,j=1}^n = \Sum_{|\alpha|=m} a_{i,j,\alpha} (x) D^\alpha
\end{equation}

Будемо розглядати характеристичний детермінант 
\begin{equation}
	\label{eq:characteristic-determinant}
	\det \left(\Sum_{|\alpha|=m} a_{i,j,\alpha} (x) \xi^\alpha \right)
\end{equation}
який представляє собою форму порядку $n \times m$ відносно параметрів $\xi_1, \ldots, \xi_m$. \medskip

Подальша класифікація систем відбувається на основі аналізу характеристичної форми (однорідного поліному \eqref{eq:characteristic-determinant}).

\begin{example}
	Розглянемо систему стаціонарних рівнянь теорії пружності:
	\begin{equation}
		\left( \Delta \vecf U + \frac{m}{m - 2} \nabla \Big(\nabla \cdot \vecf U\Big)\right) = - \frac{\vec X}{G}.
	\end{equation}
\end{example}

\begin{solution}
	Старший порядок похідних цієї системи дорівнює двом, тоді матриця яка відповідає головній частині системи має вигляд
	\begin{equation}
		A_0(D) = \begin{pmatrix}
			\Delta + \lambda \dfrac{\partial^2}{\partial x^2} & \lambda \dfrac{\partial^2}{\partial x\partial y} & \lambda \dfrac{\partial^2}{\partial x\partial z} \\
			\\
			\lambda \dfrac{\partial^2}{\partial y\partial x} & \Delta + \lambda \dfrac{\partial^2}{\partial y^2} & \lambda \dfrac{\partial^2}{\partial y\partial z} \\
			\\
			\lambda \dfrac{\partial^2}{\partial z\partial x} & \Delta + \lambda \dfrac{\partial^2}{\partial z \partial y} & \lambda \dfrac{\partial^2}{\partial z^2}
		\end{pmatrix}
	\end{equation}
	де $\lambda = m / (m - 2)$. \medskip

	Тоді характеристична форма, що відповідає цій матриці матиме вигляд:
	\begin{equation}
		\begin{vmatrix}
			|\xi|^2 + \lambda \xi_1^2 & \lambda \xi_1 \xi_2 & \lambda \xi_1 \xi_3 \\
			\\
			\lambda \xi_2 \xi_1 & |\xi|^2 + \lambda \xi_2^2 & \lambda \xi_2 \xi_3 \\
			\\
			\lambda \xi_3 \xi_1 & \lambda \xi_3 \xi_2 & |\xi|^2 + \lambda \xi_3^2
		\end{vmatrix} = |\xi|^6 (1 + \lambda).
	\end{equation}

	Зрозуміло, що цей вираз є додатнім, що гарантує еліптичність системи статичних рівнянь теорії пружності.
\end{solution}

\begin{example}
	Розглянемо систему рівнянь гідродинаміки у випадку ізоентропічної течії 
	\begin{system}
		& \frac{\partial \rho}{\partial t} + \nabla \cdot \Big(\rho \vecf V \Big) = 0, \\
		& \frac{\partial \vecf V}{\partial t} + \langle \vecf V, \nabla \rangle \vecf V + \frac{\nabla p}{\rho} = 0, \\
		& p = p(\rho).
	\end{system}
\end{example}
 
\begin{solution}
	Порядок системи дорівнює одиниці, тому матриця системи має вигляд:
	\begin{equation}
		A_0(D) = \begin{pmatrix}
			\dfrac{\partial}{\partial t} + \langle \vecf V, \nabla \rangle & \rho \dfrac{\partial}{\partial x_1} & \rho \dfrac{\partial}{\partial x_2} & \rho \dfrac{\partial}{\partial x_3} \\
			\\
			\dfrac{c^2}{\rho} \dfrac{\partial}{\partial x_1} & \dfrac{\partial}{\partial t} + \langle \vecf V, \nabla \rangle & 0 & 0 \\
			\\
			\dfrac{c^2}{\rho} \dfrac{\partial}{\partial x_2} & 0 & \dfrac{\partial}{\partial t} + \langle \vecf V, \nabla \rangle & 0 \\
			\\
			\dfrac{c^2}{\rho} \dfrac{\partial}{\partial x_3} & 0 & 0 & \dfrac{\partial}{\partial t} + \langle \vecf V, \nabla \rangle
		\end{pmatrix}
	\end{equation}

	Відповідна характеристична форма матиме вигляд:
	\begin{equation}
		\begin{vmatrix}
			\tau + \langle \vecf V, \xi \rangle & \rho \xi_1 & \rho \xi_2 & \rho \xi_3 \\
			\\
			\dfrac{c^2}{\rho} \xi_1 & \tau + \langle \vecf V, \xi \rangle & 0 & 0 \\
			\\
			\dfrac{c^2}{\rho} \xi_2 & 0 & \tau + \langle \vecf V, \xi \rangle & 0 \\
			\\
			\dfrac{c^2}{\rho} \xi_3 & 0 & 0 & \tau + \langle \vecf V, \xi \rangle\\
		\end{vmatrix}
		= 0,
	\end{equation}
	де
	\begin{equation}
		\langle \vecf V, \xi \rangle = \Sum_{i = 1}^3 V_i \xi_i,
	\end{equation}
	а $c^2 = \diff p / \diff \rho$ --- квадрат швидкості звуку. Розкриваючи визначник отримаємо співвідношення
	\begin{equation}
		\left( \tau + \langle \vecf V, \xi \rangle \right)^2 \left(\left( \tau + \langle \vecf V, \xi \rangle \right)^2 - c^2 |\xi|^2 \right) = 0.
	\end{equation}

	Розглядаючи цей вираз як поліном відносно змінної $\tau$, яка відповідає змінній часу $t$ у системі рівнянь, отримаємо для довільних дійсних значень вектора $\xi$ чотири дійсних кореня, а саме
	\begin{align}
		\tau_{1,2} &= - \langle \vecf V, \xi \rangle, \\
		\tau_{3,4} &= - \langle \vecf V, \xi \rangle \pm c |\xi|, 
	\end{align}

	\begin{remark}
		Тут $\tau_{1,2}$ позначає кратний корінь.
	\end{remark}

	Таким чином система рівнянь гідродинаміки має гіперболічний тип.
\end{solution}

\begin{example}
	Розглянемо систему рівнянь Нав'є-Стокса для нестисливої рідини
	\begin{equation}
		\frac{\partial \vecf V}{\partial t} + \Big(\vecf V \cdot \nabla\Big) \vecf V = \frac{\eta}{\rho} \Delta \vecf V.
	\end{equation}
\end{example}

\begin{solution}
	Матриця головної частини системи має вигляд:
	\begin{equation}
		A_0(x, D) = \begin{pmatrix}
			\Delta & 0 & 0 \\
			0 & \Delta & 0 \\
			0 & 0 & \Delta
		\end{pmatrix}
	\end{equation}

	Характеристична форма запишеться у вигляді:
	\begin{equation}
		\begin{vmatrix}
			|\xi|^2 & 0 & 0 \\
			0 & |\xi|^2 & 0 \\
			0 & 0 & |\xi|^2
		\end{vmatrix} = |\xi|^6 = 0,
	\end{equation}
	де $\xi = (\xi_1, \xi_2, \xi_3)$. Враховуючи, що характеристична форма містить лише три змінних, а рівняння містить чотири, можемо зробити висновок, що система є параболічною.
\end{solution}

\end{document}