\documentclass[a4paper, 12pt]{article}
\usepackage[utf8]{inputenc}
\usepackage[english, ukrainian]{babel}

\usepackage{amsmath, amssymb}
\usepackage{multicol}
\usepackage{graphicx}
\usepackage{float}

\allowdisplaybreaks
\setlength\parindent{0pt}
\numberwithin{equation}{subsection}

\usepackage{hyperref}
\hypersetup{unicode=true,colorlinks=true,linktoc=all,linkcolor=red}

\numberwithin{equation}{subsection}

\renewcommand{\bf}[1]{\textbf{#1}}
\renewcommand{\it}[1]{\textit{#1}}
\newcommand{\bb}[1]{\mathbb{#1}}
\renewcommand{\cal}[1]{\mathcal{#1}}

\renewcommand{\epsilon}{\varepsilon}
\renewcommand{\phi}{\varphi}

\DeclareMathOperator{\diam}{diam}
\DeclareMathOperator{\rang}{rang}
\DeclareMathOperator{\const}{const}

\newenvironment{system}{%
  \begin{equation}%
    \left\{%
      \begin{aligned}%
}{%
      \end{aligned}%
    \right.%
  \end{equation}%
}
\newenvironment{system*}{%
  \begin{equation*}%
    \left\{%
      \begin{aligned}%
}{%
      \end{aligned}%
    \right.%
  \end{equation*}%
}

\makeatletter
\newcommand*{\relrelbarsep}{.386ex}
\newcommand*{\relrelbar}{%
  \mathrel{%
    \mathpalette\@relrelbar\relrelbarsep%
  }%
}
\newcommand*{\@relrelbar}[2]{%
  \raise#2\hbox to 0pt{$\m@th#1\relbar$\hss}%
  \lower#2\hbox{$\m@th#1\relbar$}%
}
\providecommand*{\rightrightarrowsfill@}{%
  \arrowfill@\relrelbar\relrelbar\rightrightarrows%
}
\providecommand*{\leftleftarrowsfill@}{%
  \arrowfill@\leftleftarrows\relrelbar\relrelbar%
}
\providecommand*{\xrightrightarrows}[2][]{%
  \ext@arrow 0359\rightrightarrowsfill@{#1}{#2}%
}
\providecommand*{\xleftleftarrows}[2][]{%
  \ext@arrow 3095\leftleftarrowsfill@{#1}{#2}%
}
\makeatother

\newcommand{\NN}{\mathbb{N}}
\newcommand{\ZZ}{\mathbb{Z}}
\newcommand{\QQ}{\mathbb{Q}}
\newcommand{\RR}{\mathbb{R}}
\newcommand{\CC}{\mathbb{C}}

\newcommand{\Max}{\displaystyle\max\limits}
\newcommand{\Sup}{\displaystyle\sup\limits}
\newcommand{\Sum}{\displaystyle\sum\limits}
\newcommand{\Int}{\displaystyle\int\limits}
\newcommand{\Iint}{\displaystyle\iint\limits}
\newcommand{\Lim}{\displaystyle\lim\limits}

\newcommand*\diff{\mathop{}\!\mathrm{d}}

\newcommand*\rfrac[2]{{}^{#1}\!/_{\!#2}}


\title{{\Huge МАТЕМАТИЧНА ФІЗИКА}}
\author{Скибицький Нікіта}
\date{\today}

\usepackage{amsthm}
\usepackage[dvipsnames]{xcolor}
\usepackage{thmtools}
\usepackage[framemethod=TikZ]{mdframed}

\theoremstyle{definition}
\mdfdefinestyle{mdbluebox}{%
	roundcorner = 10pt,
	linewidth=1pt,
	skipabove=12pt,
	innerbottommargin=9pt,
	skipbelow=2pt,
	nobreak=true,
	linecolor=blue,
	backgroundcolor=TealBlue!5,
}
\declaretheoremstyle[
	headfont=\sffamily\bfseries\color{MidnightBlue},
	mdframed={style=mdbluebox},
	headpunct={\\[3pt]},
	postheadspace={0pt}
]{thmbluebox}

\mdfdefinestyle{mdredbox}{%
	linewidth=0.5pt,
	skipabove=12pt,
	frametitleaboveskip=5pt,
	frametitlebelowskip=0pt,
	skipbelow=2pt,
	frametitlefont=\bfseries,
	innertopmargin=4pt,
	innerbottommargin=8pt,
	nobreak=true,
	linecolor=RawSienna,
	backgroundcolor=Salmon!5,
}
\declaretheoremstyle[
	headfont=\bfseries\color{RawSienna},
	mdframed={style=mdredbox},
	headpunct={\\[3pt]},
	postheadspace={0pt},
]{thmredbox}

\declaretheorem[style=thmbluebox,name=Теорема,numberwithin=subsubsection]{theorem}
\declaretheorem[style=thmbluebox,name=Лема,numberwithin=subsubsection]{lemma}
\declaretheorem[style=thmbluebox,name=Твердження,numberwithin=subsubsection]{proposition}
\declaretheorem[style=thmbluebox,name=Принцип,numberwithin=subsubsection]{th_principle}
\declaretheorem[style=thmbluebox,name=Закон,numberwithin=subsubsection]{law}
\declaretheorem[style=thmbluebox,name=Закон,numbered=no]{law*}
\declaretheorem[style=thmbluebox,name=Формула,numberwithin=subsubsection]{th_formula}
\declaretheorem[style=thmbluebox,name=Рівняння,numberwithin=subsubsection]{th_equation}
\declaretheorem[style=thmbluebox,name=Умова,numberwithin=subsubsection]{th_condition}
\declaretheorem[style=thmbluebox,name=Наслідок,numberwithin=subsubsection]{corollary}

\declaretheorem[style=thmredbox,name=Приклад,numberwithin=subsubsection]{example}
\declaretheorem[style=thmredbox,name=Приклади,sibling=example]{examples}

\declaretheorem[style=thmredbox,name=Властивість,numberwithin=subsubsection]{property}
\declaretheorem[style=thmredbox,name=Властивості,sibling=property]{properties}

\mdfdefinestyle{mdgreenbox}{%
	skipabove=8pt,
	linewidth=2pt,
	rightline=false,
	leftline=true,
	topline=false,
	bottomline=false,
	linecolor=ForestGreen,
	backgroundcolor=ForestGreen!5,
}
\declaretheoremstyle[
	headfont=\bfseries\sffamily\color{ForestGreen!70!black},
	bodyfont=\normalfont,
	spaceabove=2pt,
	spacebelow=1pt,
	mdframed={style=mdgreenbox},
	headpunct={ --- },
]{thmgreenbox}

\mdfdefinestyle{mdblackbox}{%
	skipabove=8pt,
	linewidth=3pt,
	rightline=false,
	leftline=true,
	topline=false,
	bottomline=false,
	linecolor=black,
	backgroundcolor=RedViolet!5!gray!5,
}
\declaretheoremstyle[
	headfont=\bfseries,
	bodyfont=\normalfont\small,
	spaceabove=0pt,
	spacebelow=0pt,
	mdframed={style=mdblackbox}
]{thmblackbox}

\declaretheorem[name=Вправа,numberwithin=subsubsection,style=thmblackbox]{exercise}
\declaretheorem[name=Зауваження,numberwithin=subsubsection,style=thmgreenbox]{remark}
\declaretheorem[name=Визначення,numberwithin=subsubsection,style=thmblackbox]{definition}

\newtheorem{problem}{Задача}[subsection]
\newtheorem{sproblem}[problem]{Задача}
\newtheorem{dproblem}[problem]{Задача}
\renewcommand{\thesproblem}{\theproblem$^{\star}$}
\renewcommand{\thedproblem}{\theproblem$^{\dagger}$}
\newcommand{\listhack}{$\empty$\vspace{-2em}} 

\theoremstyle{remark}
\newtheorem*{solution}{Розв'язок}


\begin{document}

\tableofcontents

\setcounter{section}{3}
\setcounter{subsection}{8}

\subsection{Загальна постановка задачі Коші для хвильового рівняння. Плоскі хвилі}

\subsubsection{Характеристичні поверхні}

В лекції 14 розглядалися питання класифікації диференціальних рівнянь другого порядку з $n > 2$ незалежними змінними. Важливу роль при визначені типу рівняння і вибору нової системи координат відіграють характеристичні поверхні, які є аналогами характеристичних кривих (характеристик) для випадку рівнянь з двома незалежними змінними. \medskip

Нехай функція $\omega(x) \in C^1$, $x = (x_1, \ldots, x_n)$, $n \ge 2$ є такою що на поверхні $\omega(x) = 0$, $\nabla \omega(x) \ne 0$ та
\begin{equation}
    \label{eq:omega-partial-system-of-linear-equations}
    \Sum_{i, j = 1}^N a_{i,j}(x) \frac{\partial \omega(x)}{\partial x_i} \frac{\partial \omega(x)}{\partial x_j} = 0.
\end{equation}

\begin{definition}[характеристичної поверхні]
    Тоді поверхню $\omega(x) = 0$ називають \textit{характеристичною поверхнею} або \textit{характеристикою} квазілінійного рівняння
    \begin{equation}
        \Sum_{i, j = 1}^n a_{i, j}(x) \frac{\partial^2 u}{\partial x_i \partial x_j} + \Phi(x, u, \nabla u) = 0.
    \end{equation}
\end{definition}

\begin{definition}[характеристичної лінії]
    При $n = 2$ характеристична поверхня називається \textit{характеристичною лінією}.
\end{definition}

Оскільки $\nabla \omega(x) \ne 0$, то сімейство характеристик $\omega(x) = \const$ заповнює область таким чином, що через кожну точку області проходить одна характеристична поверхня. \medskip

Враховуючи закон перетворення коефіцієнтів рівняння
\begin{equation}
    \overline{a_{k, l}} = \Sum_{i = 1}^n \Sum_{j = 1}^n a_{i, j} \frac{\partial \xi_k}{\partial x_i} \frac{\partial \xi_l}{\partial x_j}
\end{equation}
при виборі заміни змінних $\xi_k = \xi_k(x_1, x_2, \ldots, x_n)$, $k = 1, \ldots, n$, знання однієї чи декількох характеристичних поверхонь дозволяє спростити рівняння, зокрема, якщо $\xi_1 = \omega(x_1, x_2, \ldots, x_n)$, то
\begin{equation}
    \overline{a_{1, 1}} = \Sum_{i = 1}^n \Sum_{j = 1}^n a_{i, j} \frac{\partial \omega}{\partial x_i} \frac{\partial \omega}{\partial x_j} = 0.
\end{equation}

Для хвильового рівняння
\begin{equation}
    u_{tt}(t, x) = a^2 \Delta u(t, x) + f(t, x),
\end{equation}
де $x = (x_1, x_2, \ldots, x_n)$, характеристичне рівняння має вигляд
\begin{equation}
    \left( \frac{\partial \omega}{\partial t} \right)^2 - a^2 \Sum_{i = 1}^n \left(\frac{\partial \omega}{\partial x_i} \right)^2 = 0.
\end{equation}

Одним з розв'язків цього диференціального рівняння першого порядку є поверхня
\begin{equation}
    \omega(x, t) = a^2 (t - t_0)^2 - \Sum_{i = 1}^n (x_i - x_{0, i})^2 = 0.
\end{equation}

\begin{definition}[характеристичного конуса]
    Поверхня
    \begin{equation}
        a^2 (t - t_0)^2 - \Sum_{i = 1}^n (x_i - x_{0, i})^2 = 0.
    \end{equation}
    називається \textit{характеристичним конусом} з вершиною в точці $(x_0, t_0)$ і є характеристичною поверхнею (характеристикою) хвильного рівняння.
\end{definition}

\begin{remark}
    Характеристичний конус є границею конусів
    \begin{align}
        \Gamma^+(x_0, t_0) &= a (t - t_0) > \sqrt{ \Sum_{i = 1}^n (x_i - x_{0, i})^2 }, \\
        \Gamma^-(x_0, t_0) &= - a (t - t_0) > \sqrt{ \Sum_{i = 1}^n (x_i - x_{0, i})^2 },
    \end{align}
    які називають \textit{конусами майбутнього} та \textit{минулого} відповідно.
\end{remark}

Хвильове рівняння має також інше сімейство характеристичних поверхонь
\begin{equation}
    a (t - t_0) + \Sum_{i = 1}^n (x_i - x_{0,i}) e_i = 0,
\end{equation}
де $\{e_i\}_{i = 1}^n$ --- довільні числа такі, що $|\vec e| = 1$.

\subsubsection{Узагальнена задача Коші для рівняння коливання струни}

Довільне квазілінійне рівняння гіперболічного типу з двома незалежними змінними
\begin{equation}
    a_{1, 1} (\xi, \eta) U_{\xi, \xi} (\xi, \eta) + 2 a_{1, 2} U_{\xi \eta}(\xi, \eta) + a_{2, 2} (\xi, \eta) U_{\eta, \eta}(\xi, \eta) = F(\xi, \eta, U, U_\xi, U_\eta)
\end{equation}
може бути зведене до одного із рівнянь:
\begin{align}
    \label{eq:cauchy-equation-simplified-first-form}
    u_{xt}(t,x) &= f(x, y, u, u_t, u_x), \\
    \label{eq:cauchy-equation-simplified-second-form}
    u_{tt} - u_{xx} &= f(x, y, u, u_t, u_x).
\end{align}

\begin{remark}
    Коефіцієнти $a_{i,j}(\xi,\eta)$ ($i, j = 1, 2$) і права частина $f(\xi, \eta, U, U_\xi, U_\eta)$ вважаються неперервно диференційовними функціями у відповідних областях.
\end{remark}

При постановці задачі Коші для рівняння \eqref{eq:cauchy-equation-simplified-second-form} до тепер ми вважали, що носієм початкових умов є пряма $t = 0$. \medskip

На прикладі рівняння вільних коливань однорідної струни
\begin{equation}
    \label{eq:free-string-fluctuations}
    u_{tt} - u_{xx} = f(x, t)
\end{equation}
покажемо, що носієм початкових умов може бути крива $L$, яка є відмінною від прямої $t = 0$, причому встановимо, яким умовам повинна задовольняти крива $L$ і який вигляд повинні мати самі початкові умови, щоб одержана задача Коші була поставлена коректно. \medskip

Для цього позначимо через $D$ обмежену область фазової площини $xOt$ з кусково-гладкою жордановою межею $S$. Нехай $u(t, x) \in C^2(D)$ --- розв'язок рівняння \eqref{eq:free-string-fluctuations}, який має неперервні частинні похідні 1-го порядку в області $\overline{D} = D \cup S$. \medskip

Інтегруючи тотожність \eqref{eq:free-string-fluctuations} по області $D$ і використовуючи формулу Гріна
\begin{equation}
    \Iint_D (Q_x(x, t) - P_t(x, t)) \diff x \diff t = \Int_S P \diff x + Q \diff t,
\end{equation}
де криволінійний інтеграл в правій частині береться по контуру в напрямі проти годинникової стрілки, одержуємо
\begin{equation}
    \Iint_D (u_{xx} - u_{tt}) \diff x \diff t = \Int_S u_x \diff t + u_t \diff x = \Iint_S f(x, t) \diff x, \diff t
\end{equation}

Нехай $L$ --- розімкнута крива Жордана з неперервною кривизною, яка задовольняє умовам:
\begin{itemize}
    \item кожна пряма із двох сімей характеристик $x + t = \const$, $x - t = \const$ рівняння \eqref{eq:free-string-fluctuations} перетинає криву $L$ не більше, ніж в одній точці;
    \item напрям дотичної до кривої $L$ в жодній точці не збігається з напрямом характеристик рівняння \eqref{eq:free-string-fluctuations}.
\end{itemize}

\begin{remark}
    Іноді таку криву $L$ називають \textit{``вільною''}.
\end{remark}

Припустимо, що характеристики $x - x_1 = t - t_1$ і $x - x_1 = t_1 - t$, які виходять із точки $C$, перетинаються із кривою $L$ в точках $A$ і $B$:
\begin{figure}[H]
    \centering
    \includegraphics[]{{../img/16-1}.mps}
\end{figure}

Застосовуючи формулу \eqref{eq:cauchy-equation-simplified-second-form} в області, яка обмежена дугою $AB$ кривої $L$ і відрізками характеристик $[CA]$ і $[CB]$, одержуємо:
\begin{equation}
    \label{eq:int-ab-bc-ca}
    \Int_{AB + [BC] + [CA]} u_x \diff t + u_t \diff x = \Iint_D f(x, t) \diff x \diff t.
\end{equation}

Оскільки вздовж $[BC]$ і $[AC]$ маємо $\diff x = - \diff t$, $\diff x = \diff t$ відповідно, то \eqref{eq:free-string-fluctuations} запишеться у вигляді:
\begin{equation}
    \Int_{AB} u_x \diff t + u_t \diff x - 2 u (C) + u(A) + u(B) = \Iint_D f(x, t) \diff x \diff t,
\end{equation}
звідки знаходимо
\begin{equation}
    \label{eq:uc-via-ua-ub-and-ints}
    u(C) = \frac{u(A) + u(B)}{2} + \frac{1}{2} \Int_{AB} u_x \diff t + u_t \diff x - \frac{1}{2} \Iint_D f(x, t) \diff x \diff t.
\end{equation}

Якщо розв'язок рівняння \eqref{eq:cauchy-equation-simplified-first-form} задовольняє умовам:
\begin{equation}
    \label{eq:free-string-fluctuations-boundary-conditions}
    \left. u \right|_L = \phi(x), \quad \left. \frac{\partial u}{\partial \ell} \right|_L = \psi(x),
\end{equation}
де $\phi$ і $\psi$ --- задані дійсні відповідно два рази і один раз неперервно диференційовні функції, а $\ell$ --- заданий на $L$ достатньо гладкий вектор, що ніде не збігається з дотичною до кривої $L$. Визначимо $u_x$ і $u_t$ із рівностей:
\begin{equation}
    u_x \frac{\diff x}{\diff s} + u_t \frac{\diff t}{\diff s} = \frac{\diff \phi}{\diff s}, \quad u_x \ell_x + u_t \ell_t = \psi,
\end{equation}
де $s$ --- довжина дуги $L$, і підставляючи відомі значення $u$, $u_x$, $u_t$ в праву частину \eqref{eq:uc-via-ua-ub-and-ints}, одержуємо розв'язок задачі Коші \eqref{eq:free-string-fluctuations}, \eqref{eq:free-string-fluctuations-boundary-conditions}. \medskip

Із наведених міркувань випливає, що постановка задачі Коші \eqref{eq:cauchy-equation-simplified-second-form}, \eqref{eq:int-ab-bc-ca} є коректною, тобто вона має в розглядуваній області тільки єдиний розв'язок, і він є стійким. \medskip

Аналогічно ставиться задача Коші і у випадку рівняння \eqref{eq:cauchy-equation-simplified-first-form}. \medskip

Для рівняння \eqref{eq:cauchy-equation-simplified-first-form} характеристиками будуть прямі, паралельні осям координат ($x = \const$, $t = \const$). Отже, в цьому випадку всяка гладка крива $L$, яка перетинається не більш, ніж в одній точці з прямими, паралельними осям координат, буде ``вільною''. Нехай рівняння цієї кривої буде $t = g(x)$ (або $x = h(t)$). Вважаємо, що існують похідні $g'(x)$, $h'(t)$, відмінні від нуля. Тоді задача Коші може бути поставлена наступним чином: в області
\begin{equation}
    D = \left\{ (t, x) \middle| x_0 < x < x_0 + \alpha, g(x) < t < t_0 + \beta \right\}, \quad \alpha > 0, \quad \beta > 0
\end{equation}
знайти розв'язок диференціального рівняння \eqref{eq:omega-partial-system-of-linear-equations}, який на кривій $L$ задовольняє умови
\begin{equation}
    \label{eq:second-form-boundary-conditions}
    \left. u \right|_{t = g(x)} = \phi(x), \quad \left. u_t \right|_{t = g(x)} = \psi(x).
\end{equation}

Дані Коші \eqref{eq:uc-via-ua-ub-and-ints} дозволяють на кривій $t = g(x)$ знайти значення похідної $u_x$. Дійсно, диференціюючи по $x$ першу із умов \eqref{eq:second-form-boundary-conditions}, одержуємо
\begin{equation}
    \left. u_x \right|_{t = g(x)} + \left. u_t \right|_{t = g(x)}, \quad g'(x) = \phi'(x),
\end{equation}
або
\begin{equation}
    \left. u_x \right|_{t = g(x)} = \phi'(x) - \psi(x) \phi'(x).
\end{equation}

\subsubsection{Узагальнена задача Коші для \texorpdfstring{$n$}{n}-вимірного хвильового рівняння}

У випадку хвильового рівняння в $n$-вимірному просторі
\begin{equation}
    \label{eq:utt-equation}
    u_{tt}(t, x) = a^2 \Delta u(t, x) + f(t, x), \quad x = (x_1, x_2, \ldots, x_n)
\end{equation}
носієм початкових умов може бути будь-яка ``вільна'' поверхня $\Sigma$, тобто гіперповерхня $\Psi(x, t) = 0$, яка задовольняє умовам:
\begin{itemize}
    \item в жодній її точці $(x, t)$ не має місце рівність
    \begin{equation}
        \Sum_{i = 1}^n a^2 \left( \Psi_{x_i} \right)^2 - \left( \Psi_t \right)^2 = 0,
    \end{equation}
    тобто поверхня $\Sigma$ не є характеристичною;
    \item при $n \ge 2$:
    \begin{equation}
        \Sum_{i = 1}^n a^2 \left( \Psi_{x_i} \right)^2 - \left( \Psi_t \right)^2 < 0.
    \end{equation}
\end{itemize}

Задача Коші ставиться наступним чином: знайти два рази неперервно диференційовний розв'язок рівняння \eqref{eq:free-string-fluctuations-boundary-conditions}, який задовольняє умови
\begin{equation}
    \label{eq:utx-boundary-conditions}
    u(t, x) = \phi(x), \quad \frac{\partial u(t, x)}{\partial n} = \psi(x), \quad (x, t) \in \Sigma,
\end{equation}
де $n$ --- заданий на $\Sigma$ одиничний вектор нормалі, а $\phi(x)$ і $\psi(x)$ --- задані на $\Sigma$ досить гладкі функції. Будемо вважати, що поверхня $\Sigma$ задана рівнянням $t = \sigma(x)$. \medskip

Покажемо, що задачу Коші \eqref{eq:utt-equation}, \eqref{eq:utx-boundary-conditions} можна звести до задачі Коші з початковими умовами заданими на поверхні $\tau = 0$. \medskip

Замість змінної $t$ введемо змінну $\tau = t - \sigma(x)$. Для такої заміни змінних отримаємо хвильове рівняння для функції $\tilde u(x, \tau) = u(x, \tau + \sigma(x))$. \medskip

Підрахуємо похідні, що входять в хвильове рівняння \eqref{eq:utt-equation}:
\begin{align}
    \frac{\parital^2 u}{\partial t^2} &= \frac{\partial^2 \tilde u}{\partial \tau^2} \left( \frac{\partial \tau}{\partial t} \right)^2 =  \frac{\partial^2 \tilde u}{\partial \tau^2}, \\
    \frac{\partial^2 u}{\partial x_i^2} &= \frac{\partial^2 \tilde u}{\partial \tau^2} \left( \frac{\partial \tau}{\partial x_i} \right)^2 + \frac{\partial^2 \tilde u}{\partial x_i^2} + \frac{\partial \tilde u}{\partial \tau} \frac{\partial^2 \tau}{\partial x_i^2} = \\
    &= \frac{\partial^2 \tilde u}{\partial \tau^2} \left( \frac{\partial \sigma}{\partial \tau^2} \right)^2 + \frac{\partial^2 \tilde u}{\partial x_i^2} - \frac{\partial \tilde u}{\partial \tau} \frac{\partial^2 \sigma}{\partial x_i^2}. \nonumber
\end{align}

Таким чином хвильове рівняння буде мати вигляд:
\begin{equation}
    \frac{\partial^2 \tilde u}{\partial \tau^2} = \frac{a^2}{a_0} \Sum_{i = 1}^n \frac{\partial^2 \tilde u}{\partial x_i^2} - \frac{a^2}{a_0} \Sum_{i = 1}^n \frac{\partial \tilde u}{\partial \tau} \frac{\partial^2 \sigma}{\partial x_i^2} + \frac{f(x, \tau + \sigma(x))}{a_0},
\end{equation}
де
\begin{equation}
    a_0 = 1 - a^2 \Sum_{i = 1}^n \left( \frac{\partial \sigma}{\partial x_i} \right)^2 \ne 0.
\end{equation}

Остання нерівність випливає з того, що $\Sigma$ задана рівнянням $\tau = \sigma(x)$ не є характеристичною поверхнею. \medskip

При обраній заміні змінних поверхня $\Sigma$ переходить в площину $\tau = 0$, a умови \eqref{eq:utx-boundary-conditions} приймають вигляд:
\begin{equation}
    \label{eq:tilde-u-boundary-conditions}
    \left. \tilde u \right|_{\tau = 0} = \left. u \right|_\Sigma = \phi(x), \quad \left. \frac{\partial \tilde u}{\partial \tau} \right|_{\tau = 0} = \left. \frac{\partial u}{\partial t} \right|_\Sigma.
\end{equation}

Залишається знайти $\left. \frac{\partial u}{\partial t} \right|_\Sigma$. Для цього диференціюємо першу з умов \eqref{eq:tilde-u-boundary-conditions}. \medskip

Врахуємо, що
\begin{equation}
    \left. u \right|_\Sigma = \phi(x) = u(x, \sigma(x)), \quad \frac{\partial \phi}{\partial x_i} = \frac{\partial u}{\partial t} \frac{\partial \sigma}{\partial x_i} + \frac{\partial u}{\partial x_i}, \quad i = 1, 2, \ldots, n
\end{equation}

\end{document}