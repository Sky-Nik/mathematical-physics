\documentclass[a4paper, 12pt]{article}
\usepackage[utf8]{inputenc}
\usepackage[english, ukrainian]{babel}

\usepackage{amsmath, amssymb}
\usepackage{multicol}
\usepackage{graphicx}
\usepackage{float}

\allowdisplaybreaks
\setlength\parindent{0pt}
\numberwithin{equation}{subsection}

\usepackage{hyperref}
\hypersetup{unicode=true,colorlinks=true,linktoc=all,linkcolor=red}

\numberwithin{equation}{subsection}

\renewcommand{\bf}[1]{\textbf{#1}}
\renewcommand{\it}[1]{\textit{#1}}
\newcommand{\bb}[1]{\mathbb{#1}}
\renewcommand{\cal}[1]{\mathcal{#1}}

\renewcommand{\epsilon}{\varepsilon}
\renewcommand{\phi}{\varphi}

\DeclareMathOperator{\diam}{diam}
\DeclareMathOperator{\rang}{rang}
\DeclareMathOperator{\const}{const}

\newenvironment{system}{%
  \begin{equation}%
    \left\{%
      \begin{aligned}%
}{%
      \end{aligned}%
    \right.%
  \end{equation}%
}
\newenvironment{system*}{%
  \begin{equation*}%
    \left\{%
      \begin{aligned}%
}{%
      \end{aligned}%
    \right.%
  \end{equation*}%
}

\makeatletter
\newcommand*{\relrelbarsep}{.386ex}
\newcommand*{\relrelbar}{%
  \mathrel{%
    \mathpalette\@relrelbar\relrelbarsep%
  }%
}
\newcommand*{\@relrelbar}[2]{%
  \raise#2\hbox to 0pt{$\m@th#1\relbar$\hss}%
  \lower#2\hbox{$\m@th#1\relbar$}%
}
\providecommand*{\rightrightarrowsfill@}{%
  \arrowfill@\relrelbar\relrelbar\rightrightarrows%
}
\providecommand*{\leftleftarrowsfill@}{%
  \arrowfill@\leftleftarrows\relrelbar\relrelbar%
}
\providecommand*{\xrightrightarrows}[2][]{%
  \ext@arrow 0359\rightrightarrowsfill@{#1}{#2}%
}
\providecommand*{\xleftleftarrows}[2][]{%
  \ext@arrow 3095\leftleftarrowsfill@{#1}{#2}%
}
\makeatother

\newcommand{\NN}{\mathbb{N}}
\newcommand{\ZZ}{\mathbb{Z}}
\newcommand{\QQ}{\mathbb{Q}}
\newcommand{\RR}{\mathbb{R}}
\newcommand{\CC}{\mathbb{C}}

\newcommand{\Max}{\displaystyle\max\limits}
\newcommand{\Sup}{\displaystyle\sup\limits}
\newcommand{\Sum}{\displaystyle\sum\limits}
\newcommand{\Int}{\displaystyle\int\limits}
\newcommand{\Iint}{\displaystyle\iint\limits}
\newcommand{\Lim}{\displaystyle\lim\limits}

\newcommand*\diff{\mathop{}\!\mathrm{d}}

\newcommand*\rfrac[2]{{}^{#1}\!/_{\!#2}}


\title{{\Huge МАТЕМАТИЧНА ФІЗИКА}}
\author{Скибицький Нікіта}
\date{\today}

\usepackage{amsthm}
\usepackage[dvipsnames]{xcolor}
\usepackage{thmtools}
\usepackage[framemethod=TikZ]{mdframed}

\theoremstyle{definition}
\mdfdefinestyle{mdbluebox}{%
	roundcorner = 10pt,
	linewidth=1pt,
	skipabove=12pt,
	innerbottommargin=9pt,
	skipbelow=2pt,
	nobreak=true,
	linecolor=blue,
	backgroundcolor=TealBlue!5,
}
\declaretheoremstyle[
	headfont=\sffamily\bfseries\color{MidnightBlue},
	mdframed={style=mdbluebox},
	headpunct={\\[3pt]},
	postheadspace={0pt}
]{thmbluebox}

\mdfdefinestyle{mdredbox}{%
	linewidth=0.5pt,
	skipabove=12pt,
	frametitleaboveskip=5pt,
	frametitlebelowskip=0pt,
	skipbelow=2pt,
	frametitlefont=\bfseries,
	innertopmargin=4pt,
	innerbottommargin=8pt,
	nobreak=true,
	linecolor=RawSienna,
	backgroundcolor=Salmon!5,
}
\declaretheoremstyle[
	headfont=\bfseries\color{RawSienna},
	mdframed={style=mdredbox},
	headpunct={\\[3pt]},
	postheadspace={0pt},
]{thmredbox}

\declaretheorem[style=thmbluebox,name=Теорема,numberwithin=subsubsection]{theorem}
\declaretheorem[style=thmbluebox,name=Лема,numberwithin=subsubsection]{lemma}
\declaretheorem[style=thmbluebox,name=Твердження,numberwithin=subsubsection]{proposition}
\declaretheorem[style=thmbluebox,name=Принцип,numberwithin=subsubsection]{th_principle}
\declaretheorem[style=thmbluebox,name=Закон,numberwithin=subsubsection]{law}
\declaretheorem[style=thmbluebox,name=Закон,numbered=no]{law*}
\declaretheorem[style=thmbluebox,name=Формула,numberwithin=subsubsection]{th_formula}
\declaretheorem[style=thmbluebox,name=Рівняння,numberwithin=subsubsection]{th_equation}
\declaretheorem[style=thmbluebox,name=Умова,numberwithin=subsubsection]{th_condition}
\declaretheorem[style=thmbluebox,name=Наслідок,numberwithin=subsubsection]{corollary}

\declaretheorem[style=thmredbox,name=Приклад,numberwithin=subsubsection]{example}
\declaretheorem[style=thmredbox,name=Приклади,sibling=example]{examples}

\declaretheorem[style=thmredbox,name=Властивість,numberwithin=subsubsection]{property}
\declaretheorem[style=thmredbox,name=Властивості,sibling=property]{properties}

\mdfdefinestyle{mdgreenbox}{%
	skipabove=8pt,
	linewidth=2pt,
	rightline=false,
	leftline=true,
	topline=false,
	bottomline=false,
	linecolor=ForestGreen,
	backgroundcolor=ForestGreen!5,
}
\declaretheoremstyle[
	headfont=\bfseries\sffamily\color{ForestGreen!70!black},
	bodyfont=\normalfont,
	spaceabove=2pt,
	spacebelow=1pt,
	mdframed={style=mdgreenbox},
	headpunct={ --- },
]{thmgreenbox}

\mdfdefinestyle{mdblackbox}{%
	skipabove=8pt,
	linewidth=3pt,
	rightline=false,
	leftline=true,
	topline=false,
	bottomline=false,
	linecolor=black,
	backgroundcolor=RedViolet!5!gray!5,
}
\declaretheoremstyle[
	headfont=\bfseries,
	bodyfont=\normalfont\small,
	spaceabove=0pt,
	spacebelow=0pt,
	mdframed={style=mdblackbox}
]{thmblackbox}

\declaretheorem[name=Вправа,numberwithin=subsubsection,style=thmblackbox]{exercise}
\declaretheorem[name=Зауваження,numberwithin=subsubsection,style=thmgreenbox]{remark}
\declaretheorem[name=Визначення,numberwithin=subsubsection,style=thmblackbox]{definition}

\newtheorem{problem}{Задача}[subsection]
\newtheorem{sproblem}[problem]{Задача}
\newtheorem{dproblem}[problem]{Задача}
\renewcommand{\thesproblem}{\theproblem$^{\star}$}
\renewcommand{\thedproblem}{\theproblem$^{\dagger}$}
\newcommand{\listhack}{$\empty$\vspace{-2em}} 

\theoremstyle{remark}
\newtheorem*{solution}{Розв'язок}


\begin{document}

\tableofcontents

\setcounter{section}{3}
\setcounter{subsection}{7}

\subsection{Постановка основних граничних задач для лінійних диференційних рівнянь 2-го порядку, коректність, класичні та узагальнені розв'язки}

Серед множини математичних моделей, які були розглянуті в попередніх параграфах можна виділити найтиповіші математичні моделі, які концентрують в собі головні особливості усіх розглянутих вище. Ці моделі представляють собою граничні задачі для рівнянь трьох типів: еліптичних, параболічних та гіперболічних лінійних рівнянь другого порядку. \medskip

Розглянемо основний диференціальний оператор другого порядку:
\begin{equation}
    L u = \nabla \cdot (p(x) \nabla u) - q(x) u.
\end{equation}

Запишемо основні диференціальні рівняння:
\begin{itemize}
    \item Еліптичне рівняння:
    \begin{equation}
        \label{eq:elliptic-differential-equation}
        L u = - F(x), \quad x \in \Omega \subset \RR^n.
    \end{equation}
    \item Параболічне рівняння:
    \begin{equation}
        \label{eq:parabolic-differential-equation}
        \rho(x) \frac{\partial u}{\partial t} = L u + F(x, t), \quad x \in \Omega, \quad t > t_0.
    \end{equation}
    \item Гіперболічне рівняння:
    \begin{equation}
        \label{eq:hyperbolic-differential-equation}
        \rho(x) \frac{\partial^2 u}{\partial t^2} = L u + F(x, t), \quad x \in \Omega, \quad t > t_0.
    \end{equation}
\end{itemize}

\subsubsection{Гранична задача для еліптичного рівняння}

Будемо розділяти внутрішні і зовнішні задачі для еліптичного рівняння.

\begin{definition}[внутрішньої і зовнішньої задач]
    Якщо $x \in \Omega$, то таку задачу будемо називати \textit{внутрішньою}, якщо $x \in \Omega'$ ---  \textit{зовнішньою}.
\end{definition}

\begin{remark}
    В подальшому ми будемо розглядати класичні розв'язки граничних задач. Це означає, що рівняння і усі граничні умови виконуються в кожній точці області або границі.
\end{remark}

Введемо обмеження на коефіцієнти рівняння $p$ і $q$ та вільний член $F$. Зокрема будемо припускати, що $p > 0$, $p \in C^1 \left(\overline \Omega\right)$, $q \ge 0$, $q \in C \left(\overline \Omega\right)$, $F(x) \in C \left(\overline \Omega\right)$. \medskip

Позначимо $\partial \Omega = S$ --- поверхню на якій задаються граничні умови загального
вигляду:
\begin{equation}
    \label{eq:general-boundary-conditions}
    \left. \alpha(x) \frac{\partial u}{\partial n} + \beta(x) u \right|_S = V(x),
\end{equation}
де $\alpha, \beta \ge 0$, $\alpha, \beta, B \in C(S)$. З умови \eqref{eq:general-boundary-conditions} можна отримати умови 1, 2, 3 роду:
\begin{enumerate}
    \item Діріхле:
    \begin{equation}
        \label{eq:elliptic-dirichlet-boundary-conditions}
        \left. u \right|_S = \frac{V(x)}{\beta(x)}.
    \end{equation}
    \item Неймана:
    \begin{equation}
        \label{eq:elliptic-neumann-boundary-conditions}
        \left. \frac{\partial u}{\partial n} \right|_S = \frac{V(x)}{\alpha(x)}.
    \end{equation}
    \item Ньютона:
    \begin{equation}
        \label{eq:elliptic-newton-boundary-conditions}
        \left. \frac{\partial u}{\partial n} + \frac{\beta}{\alpha} u \right|_S = \frac{V}{\alpha}.
    \end{equation}
\end{enumerate}

Таким чином гранична задача для еліптичного рівняння може бути сформульована наступним чином: 
\begin{problem_formulation*}[граничної для еліптичного рівняння]
    Знайти функцію $u(x) \in C^2\big(\Omega\big) \cap C^1 \big( \overline \Omega \big)$, яка в кожній внутрішній точці області $\Omega$ (для внутрішньої задачі) або $\Omega'$ (для зовнішньої задачі) задовольняє рівняння \eqref{eq:elliptic-differential-equation}, а кожній точці границі $S$ виконується одна з граничних умов \eqref{eq:elliptic-dirichlet-boundary-conditions}, \eqref{eq:elliptic-neumann-boundary-conditions} або \eqref{eq:elliptic-newton-boundary-conditions}.
\end{problem_formulation*}

\begin{definition}
    У випадку зовнішньої граничної задачі в нескінченно віддаленій точці області слід задавати додаткові умови поведінки розв'язку. Такі умови називають умовами \textit{регулярності на нескінченості}.
\end{definition}

\begin{remark}
    Як правило вони полягають в завданні характеру спадання розв'язку і мають вигляд
    \begin{equation}
        u(x) = O \left( \frac{1}{|x|^\alpha} \right),
    \end{equation}
    при $|x| \to \infty$, де $\alpha$ --- деякий заданий параметр задачі.
\end{remark}

\subsubsection{Постановка змішаних задач для рівняння гіперболічного типу. Задача Коші для гіперболічного рівняння}

Для постановки граничних задач рівняння гіперболічного типу \eqref{eq:hyperbolic-differential-equation} введемо просторово-часовий циліндр, як область зміни незалежних змінних $x$, $t$:
\begin{equation}
    Z(\Omega, T) = \Omega \times (0, T].
\end{equation}

Для отримання єдиного розв'язку гіперболічного рівняння, на нижній основі просторово-часового циліндру $Z_0(\Omega, T) = \Omega \times \{t = 0\}$ треба задати початкові умови:
\begin{align}
    \label{eq:hyperbolic-starting-function-condition}
    u(x, 0) &= u_0(x), \quad x \in \Omega, \\
    \label{eq:hyperbolic-starting-derivative-condition}
    \frac{\partial u(x, 0)}{\partial t} &= v_0(x), \quad x \in \Omega.
\end{align}

На боковій поверхні просторово-часового циліндру $Z_S(\Omega, T) = S \times (0, T]$ треба задати граничні умови одного з трьох основних типів:
\begin{enumerate}
    \item Діріхле:
    \begin{equation}
        \label{eq:hyperbolic-dirichlet-boundary-conditions}
        \left. u \right|_S = \phi(x, t).
    \end{equation}
    \item Неймана:
    \begin{equation}
        \label{eq:hyperbolic-neumann-boundary-conditions}
        \left. \frac{\partial u}{\partial n} \right|_S = \phi(x, t).
    \end{equation}
    \item Ньютона:
    \begin{equation}
        \label{eq:hyperbolic-newton-boundary-conditions}
        \left. \frac{\partial u}{\partial n} + \alpha(x, t) u \right|_S = \phi(x, t).
    \end{equation}
\end{enumerate}

Таким чином постановка граничної задачі для гіперболічного рівняння має вигляд:
\begin{problem_formulation*}[граничної для гіперболічного рівняння]
    Знайти функцію $u(x, t) \in C^{(2, 2)} \Big( Z(\Omega, T) \Big) \cap C^{(1, 1)} \Big( \overline {Z(\Omega, T)} \Big)$, яка задовольняє рівнянню \eqref{eq:hyperbolic-differential-equation} для $(x, t) \in Z(\Omega, T)$, початковим умовам \eqref{eq:hyperbolic-starting-function-condition}, \eqref{eq:hyperbolic-starting-derivative-condition} для $(x, t) \in Z_0(\Omega, T)$, і в кожній точці $(x, t) \in Z_S(\Omega, T)$ одній з граничних умов \eqref{eq:hyperbolic-dirichlet-boundary-conditions}--\eqref{eq:hyperbolic-newton-boundary-conditions}.
\end{problem_formulation*}

\begin{remark}
    При цьому відносно вхідних даних будемо робити наступні припущення
    \begin{gather}
        p > 0, \quad p \in C^1 \Big( \overline \Omega \Big), \quad q \ge 0, \quad q \in C \Big( \overline \Omega \Big), \quad F(x, t) \in C \Big( \overline{Z(\Omega, T)} \Big), \\
        u_0, v_0 \in C \left( \overline{Z_0(\Omega, T)} \right), \quad \alpha, \phi \in C \left( \overline{Z_S(\Omega, T)} \right), \quad \alpha \ge 0.
    \end{gather}
\end{remark}

\subsubsection{Задача Коші}

У випадку, коли область $\Omega$ має великі розміри і впливом граничних умов можна знехтувати, область $\Omega$ ототожнюється з усім евклідовим простором, тобто $\Omega = \RR^n$. \medskip

У зв'язку з відсутністю границі, граничні умови не задаються. В цьому випадку гранична задача трансформується в задачу Коші для гіперболічного рівняння яка ставиться наступним чином: 
\begin{problem_formulation*}[Коші для гіперболічного рівняння]
    Знайти функцію $u(x, t) \in C^{(2, 2)} \Big( Z(\RR^n, T) \Big) \cap C^{(1, 1)} \Big( \overline {Z(\RR^n, T)} \Big)$, яка задовольняє рівнянню \eqref{eq:hyperbolic-differential-equation} для $(x, t) \in Z(\RR^n, T)$, початковим умовам \eqref{eq:hyperbolic-starting-function-condition}, \eqref{eq:hyperbolic-starting-derivative-condition} для $x \in \RR^n$.
\end{problem_formulation*}

\subsubsection{Постановка змішаних задач для рівняння параболічного типу}

При постановці граничної задачі і задачі Коші для рівняння параболічного типу треба враховувати, що по часовій змінній рівняння має перший порядок, що і обумовлює деякі відмінності в постановці граничних задач. \medskip

Постановка граничної задачі для рівняння параболічного типу \eqref{eq:parabolic-differential-equation} має вигляд:
\begin{problem_formulation*}[граничної для параболічного рівняння]
    Знайти функцію $u(x, t) \in C^{(2, 1)} \Big( Z(\Omega, T) \Big) \cap C^{(1, 0)} \Big( \overline {Z(\Omega, T)} \Big)$, яка задовольняє рівняння \eqref{eq:parabolic-differential-equation} для $(x, t) \in Z(\Omega, T)$, початковим умовам \eqref{eq:hyperbolic-starting-function-condition} для $(x, t) \in Z_0(\Omega, T)$ і в кожній точці $(x, t) \in Z_S(\Omega, T)$ одній з граничних умов \eqref{eq:hyperbolic-dirichlet-boundary-conditions}--\eqref{eq:hyperbolic-newton-boundary-conditions}.
\end{problem_formulation*}

Аналогічні зміни необхідно запровадити і при постановці задачі Коші для рівняння параболічного типу.

\begin{exercise}
    Записати самостійно постановку задачі Коші для параболічного рівняння \eqref{eq:parabolic-differential-equation}.
\end{exercise}

\subsubsection{Коректність задач математичної фізики}

Зважуючи на фізичну природу задач математичної фізики, до них застосовуються наступні природні вимоги.
\begin{enumerate}
    \item Існування розв'язку: задача повинна мати розв'язок (задача яка не має розв'язку не представляє інтересу як математична модель).
    \item Єдиність розв’язку: не повинно існувати декілька розв'язків задачі.
    \item Неперервна залежність від вхідних даних: розв'язок задачі повинен мало змінюватись при малій зміні вхідних даних.
\end{enumerate}

Розглянемо математичну модель у вигляді наступної граничної задачі:
\begin{equation}
    \label{eq:boundary-problem}
    \left\{
        \begin{aligned}
            & L u = f, \quad x \in \Omega, \\
            & \ell u = \phi, \quad x \in S = \partial \Omega.
        \end{aligned}
    \right.
\end{equation}

Формулювання диференціального рівняння і граничних умов ще недостатньо що б гранична задача була сформульована однозначно. Необхідно додатково вказати які аналітичні властивості вимагаються від роз\-в'яз\-ку, в якому розумінні задовольняється рівняння і граничні умови. \medskip

При аналізі граничної задачі виникають наступні питання:
\begin{itemize}
    \item Чи може існувати розв'язок з відповідними властивостями?
    \item Які аналітичні властивості треба вимагати від вхідних даних $f$, $\phi$, коефіцієнтів диференціального оператора і граничних умов?
    \item Чи існують серед умов задачі такі, що протирічать одне одном?
    \item Які умови треба накладати на гладкість границі $S$?
    \item Чи достатньо сформульованих умов для однозначного знаходження розв’язку?
    \item Чи можна гарантувати, що малі зміни $f$, $\phi$ приведуть до малих змін розв'язку?
\end{itemize}

Перелічені проблеми зручно розв'язувати звівши граничну задачу до операторного рівняння. Застосувавши загальні методи теорії операторів та операторних рівнянь. \medskip

В першу чергу виберемо два бананових простора $E$ та $F$. \medskip

Шуканий розв'язок розглядається як елемент $E$, а сукупність правих частин як елемент $F$. \medskip

Визначимо оператор $A$, як відображення $u \mapsto \{Lu, \phi\}$, тоді гранична задача
\eqref{eq:boundary-problem} зводиться до операторного рівняння
\begin{equation}
    \label{eq:boundary-operator-equation}
    A u = g, \quad g = \{f, \phi\}.
\end{equation}

Позначимо $R(A)$ та $D(A)$ --- область значень та область визначення оператора $A$. Коректність операторного рівняння визначають для пари просторів $E$ та $F$. \medskip

\begin{proposition}
    В термінах операторного рівняння \eqref{eq:boundary-operator-equation} існування розв'язку означає, що область значень оператора $R(A)$ є не порожня підмножина $F$.
\end{proposition}

\begin{proposition}
    Єдиність розв'язку означає, що відображення $A: D(A) \to R(A)$ ін'єктивне і на $R(A)$ визначений обернений оператор $A^{-1}$.
\end{proposition}

\begin{definition}[ін'єктивного відображення]
    Відображення $A$: \allowbreak $D(A) \to R(A)$ називається \textit{ін'єктивним}, якщо різні елементи множини $D(A)$ переводяться в різні елементи множини $R(A)$.    
\end{definition}

\begin{proposition}
    Вимога неперервної залежності розв'язку від правої частини або стійкості граничної задачі зводиться до неперервності або обмеженості оператора $A^{-1}$.
\end{proposition}

\subsubsection{Приклад Адамара}

\begin{example}[Адамара, некоректно поставленої задачі]
    Розглянемо рівняння Лапласа
    \begin{equation}
        \frac{\partial^2 u}{\partial t^2} = - \frac{\partial^2 u}{\partial x^2}, \quad t > 0, \quad 0 < x < \pi.
    \end{equation}

    Додаткові умови
    \begin{equation}
        \left. u \right|_{x = 0} = \left. u \right|_{x = \pi} = 0, \quad \left. u \right|_{t = 0} = 0, \quad \left. \frac{\partial u}{\partial t} \right|_{t = 0} = \frac{\sin (k x)}{k}.
    \end{equation}
\end{example}

\begin{proposition}
    Для прикладу Адамара порушена умова непевної залежності роз\-в'яз\-ку від вхідних даних.
\end{proposition}

\begin{proof}
    Розв'язок
    \begin{equation}
        u_k(x, t) = \frac{\sinh(k t) \sin (k x)}{k^2},
    \end{equation}
    причому $\forall x \in (0, \pi)$:
    \begin{equation}
        \Lim_{k \to \infty} u_k(x, 0) =  \Lim_{k \to \infty} \frac{\sin (k x)}{k} = 0,
    \end{equation}
    але $\forall t > 0$, $\forall x \in (0, \pi)$:
    \begin{equation}
        \Lim_{k \to \infty} u_k(x, t) =  \Lim_{k \to \infty} \frac{\sinh(k t) \sin (k x)}{k^2} = \infty.
    \end{equation}
\end{proof}

\subsubsection{Класичний і узагальнений розв'язки}

\begin{definition}[класичного розв'язку]
    Класичний розв'язок --- це розв'язок, який задовольняє рівнянню, початковим і граничним умовам в кожній точці, області, або границі.    
\end{definition}

Це означає, що класичний розв'язок повинен мати певну гладкість, яка визначається порядком похідних рівняння і порядком похідних граничних і початкових умов. \medskip

Розглянемо рівняння
\begin{equation}
    \label{eq:differential-equation}
    \nabla \cdot (p(x) \nabla u) - q(x) u = -F (x), \quad x \in \Omega
\end{equation}
та однорідні умови
\begin{equation}
    \label{eq:homogenuous-conditions}
    \left. u \right|_S = 0.
\end{equation}

Отримаємо інтегральне співвідношення. \medskip

Розглянемо функцію $v(x)$, таку, що $\left. v \right|_S = 0$, помножимо рівняння на $v$ та проінтегруємо по $\Omega$:
\begin{equation}
    \Iiint_\Omega v \left(\nabla \cdot (p(x) \nabla u) - u \right) \diff \Omega = - \Iiint_\Omega F v \diff \Omega.    
\end{equation}

Після інтегрування за частинами отримаємо:
\begin{equation}
    \Iiint_\Omega \left( p \langle \nabla u, \nabla v \rangle) - q u v \right) \diff \Omega + \Iint_S p v \frac{\partial u}{\partial n} \diff S = - \Iiint_\Omega F v \diff \Omega.
\end{equation}

Остаточно, після врахування граничних умов маємо:
\begin{equation}
    \label{eq:integral-equality}
    \Iiint_\Omega \left( p \langle \nabla u, \nabla v \rangle) + q u v \right) \diff \Omega = \Iiint_\Omega F v \diff \Omega.
\end{equation}

Інтегральна тотожність має зміст для більш широкого класу функцій ніж той якому належить класичний розв'язок граничної задачі і коефіцієнти рівняння. \medskip

Якщо $u, v \in C^2 \big( \Omega \big) \cap C \big( \overline \Omega \big)$, $p \in C^1 (\Omega$, $q \in C (\Omega)$ то з тотожності \eqref{eq:integral-equality}, обернений ланцюжок перетворень дозволяє отримати граничну задачу \eqref{eq:differential-equation}, \eqref{eq:homogenuous-conditions}. Але \eqref{eq:integral-equality} має зміст для функцій більш широкого класу, а саме $F, u, v, \nabla u, \nabla v \in L_2(\Omega)$, $p$, $q$ --- обмежені. Це дозволяє використовувати інтегральну тотожність \eqref{eq:homogenuous-conditions} для визначення узагальненого розв'язку граничної
задачі \eqref{eq:differential-equation}, \eqref{eq:homogenuous-conditions}. \medskip

Для цього введемо множину $N_2 = \{u | u, \nabla u \in L_2(\Omega), \left.u \right|_S = 0\}$.

\begin{definition}[узагальненого розв'язку]
    Узагальненим розв'язком граничної задачі \eqref{eq:differential-equation}, \eqref{eq:homogenuous-conditions} будемо називати довільну функцію $u \in N_2$, таку, що $\forall v \in N_2$ має місце інтегральна тотожність \eqref{eq:integral-equality}.    
\end{definition}

\subsubsection{Формально спряжені оператори. Друга формула Гріна}

Будемо розглядати лінійний диференціальний оператор
\begin{equation}
    L u = \Sum_{i, j = 1}^N \frac{\partial}{\partial x_j} \left( A_{i,j}(x) \frac{\partial u}{\partial x_i} \right) + \Sum_{k = 1}^N B_k(x) \frac{\partial u}{\partial x_k} + C(x) u.
\end{equation}

Будемо припускати, що $A_{i, j} = A_{j, i} \in C^1 \big( \overline \Omega \big)$, $B_k \in C \big( \overline \Omega \big)$, $u \in C^2 \big( \overline \Omega \big)$. \medskip

Розглянемо інтеграл:
\begin{equation}
    \Iiint_\Omega v L u \diff x = \Iiint_\Omega v \left( \Sum_{i, j = 1}^N \frac{\partial}{\partial x_j} \left( A_{i,j}(x) \frac{\partial u}{\partial x_i} \right) + \Sum_{k = 1}^N B_k(x) \frac{\partial u}{\partial x_k} + C(x) u \right) \diff x.
\end{equation}

Для перетворення першої і другої суми застосуємо формулу інтегрування за частинами:
\begin{equation}
    \Iiint_\Omega v(x) \frac{\partial u(x)}{\partial x_i} \diff x = \Iint_S v(x) u(x) \cos(n, x_i) \diff x - \Iiint_\Omega u(x) \frac{\partial v(x)}{\partial x_i} \diff x.
\end{equation}

Після однократного застосування формули інтегрування за частинами отримаємо:
\begin{multline}
    \Iiint_\Omega v L u \diff x = \Iiint_\Omega \left( - \Sum_{i, j = 1}^n A_{i, j}(x) \frac{\partial u}{\partial x_i} \frac{\partial v}{\partial x_j} - \Sum_{k = 1}^n \frac{\partial}{\partial x_k} (B_k(x) v) u + C(x) u v \right) \diff x + \\
    + \Oiint_S \left( \Sum_{i, j = 1}^N A_{i, j}(x) \frac{\partial u}{\partial x_i} \cos(n, x_j) + \Sum_{k = 1}^N B_k(x) u v \cos(n, x_k) \right) \diff S.
\end{multline}

Продовжимо інтегрування за частинами до першого інтегралу по області $\Omega$, перекидаючи похідну з функції $u$:
\begin{multline}
    \Psi = \Iiint_\Omega \left( \Sum_{i, j = 1}^N u \frac{\partial}{\partial x_i} \left( A_{i,j}(x) \frac{\partial v}{\partial x_j} \right) - \Sum_{k = 1}^n \frac{\partial}{\partial x_k} (B_k(x) v) u + C(x) u v \right) \diff x + \\
    + \Oiint_S \left( \Sum_{i, j = 1}^n A_{i,j}(x) \left( \frac{\partial u}{\partial x_i} v - \frac{\partial v}{\partial x_i} u \right) \cos(n, x_j) + \Sum_{k = 1}^N B_k(x) u v \cos(n, x_k) \right) \diff S.
\end{multline}

Введемо наступний оператор:
\begin{equation}
    M u = \Sum_{i, j = 1}^n \frac{\partial}{\partial x_j} \left( A_{i,j}(x) \frac{\partial u}{\partial x_i} \right) - \Sum_{k = 1}^N \frac{\partial}{\partial x_k} (B_k(x) u) + C(x) u.
\end{equation}

\begin{definition}[формально спряженого оператора]
    Оператор $M$ називається \textit{формально спряженим} до оператора $L$.
\end{definition}

Враховуючи це позначення, останню формулу можна записати у вигляді:
\begin{equation}
    \Iiint_\Omega (V L u - u M v) \diff x = \Oiint_S \left( \Sum_{i, j = 1}^N A_{i, j}(x) \left( \frac{\partial u}{\partial x_i} v - \frac{\partial v}{\partial x_i} u \right) \cos(n, x_j) + \Sum_{k = 1}^n B_k(x) u v \cos(n, x_k) \right) \diff S.
\end{equation}

\begin{definition}[другої формули Гріна]
    Ця формула називається другою формулою Гріна. 
\end{definition}

Розглянемо основні оператори математичної фізики другого порядку з постійними коефіцієнтами:
\begin{enumerate}
    \item Гельмгольца: $A_1 u = (\Delta + k^2) u$.
    \item Теплопровідності: $A_2 u = \left( a^2 \Delta - \frac{\partial}{\partial t} \right) u$.
    \item Хвильовий: $A_3 u = \left( a^2 \Delta - \frac{\partial^2}{\partial t^2} \right) u$.
\end{enumerate}

Оскільки оператори $A_1$, $A_3$ містять лише похідні другого порядку, то ці оператори є формально самоспряженими. Для оператора $A_2$, згідно до визначення спряженим буде оператор $A_2^\star u = \left( a^2 \Delta + \frac{\partial}{\partial t} \right) u$. \medskip

Запишемо другу формулу Гріна для кожного з основних операторів:
\begin{enumerate}
    \item для Гельмгольца: 
    \begin{equation}
        \Iiint_\Omega ( v (\Delta u + k^2 u) - u (\Delta v + k^2 v)) \diff x = \Iint_S \left( v \frac{\partial u}{\partial n} - u \frac{\partial v}{\partial n} \right) \diff S;
    \end{equation}
    \item для теплопровдіності:
    \begin{multline}
        \Int_{t_0}^T \Iiint_\Omega \left( v \left( a^2 u - \frac{\partial u}{\partial t} \right) - u \left( a^2 \Delta v + \frac{\partial v}{\partial t} \right) \right) \diff x \diff t = \\
        = \Int_{t_0}^T \Iint_S a^2 \left( v \frac{\partial u}{\partial n} - u \frac{\partial v}{\partial n} \right) \diff S \diff t- \Iiint_\Omega \left. u v \right|_{t_0}^T \diff x.
    \end{multline}
    \item для хвильового:
    \begin{multline}
        \Int_{t_0}^T \Iiint_\Omega \left( v \left( a^2 u - \frac{\partial^2 u}{\partial t^2} \right) - u \left( a^2 \Delta v - \frac{\partial^2 v}{\partial t^2} \right) \right) \diff x \diff t = \\
        = \Int_{t_0}^T \Iint_S a^2 \left( v \frac{\partial u}{\partial n} - u \frac{\partial v}{\partial n} \right) \diff S \diff t - \Iiint_\Omega \left. \left( v \frac{\partial u}{\partial t} - u \frac{\partial v}{\partial t} \right) \right|_{t_0}^T \diff x.
    \end{multline}
\end{enumerate}

\end{document}