\documentclass[a4paper, 12pt]{article}
\usepackage[utf8]{inputenc}
\usepackage[english, ukrainian]{babel}

\usepackage{amsmath, amssymb}
\usepackage{multicol}
\usepackage{graphicx}
\usepackage{float}

\allowdisplaybreaks
\setlength\parindent{0pt}
\numberwithin{equation}{subsection}

\usepackage{hyperref}
\hypersetup{unicode=true,colorlinks=true,linktoc=all,linkcolor=red}

\numberwithin{equation}{subsection}

\renewcommand{\bf}[1]{\textbf{#1}}
\renewcommand{\it}[1]{\textit{#1}}
\newcommand{\bb}[1]{\mathbb{#1}}
\renewcommand{\cal}[1]{\mathcal{#1}}

\renewcommand{\epsilon}{\varepsilon}
\renewcommand{\phi}{\varphi}

\DeclareMathOperator{\diam}{diam}
\DeclareMathOperator{\rang}{rang}
\DeclareMathOperator{\const}{const}

\newenvironment{system}{%
  \begin{equation}%
    \left\{%
      \begin{aligned}%
}{%
      \end{aligned}%
    \right.%
  \end{equation}%
}
\newenvironment{system*}{%
  \begin{equation*}%
    \left\{%
      \begin{aligned}%
}{%
      \end{aligned}%
    \right.%
  \end{equation*}%
}

\makeatletter
\newcommand*{\relrelbarsep}{.386ex}
\newcommand*{\relrelbar}{%
  \mathrel{%
    \mathpalette\@relrelbar\relrelbarsep%
  }%
}
\newcommand*{\@relrelbar}[2]{%
  \raise#2\hbox to 0pt{$\m@th#1\relbar$\hss}%
  \lower#2\hbox{$\m@th#1\relbar$}%
}
\providecommand*{\rightrightarrowsfill@}{%
  \arrowfill@\relrelbar\relrelbar\rightrightarrows%
}
\providecommand*{\leftleftarrowsfill@}{%
  \arrowfill@\leftleftarrows\relrelbar\relrelbar%
}
\providecommand*{\xrightrightarrows}[2][]{%
  \ext@arrow 0359\rightrightarrowsfill@{#1}{#2}%
}
\providecommand*{\xleftleftarrows}[2][]{%
  \ext@arrow 3095\leftleftarrowsfill@{#1}{#2}%
}
\makeatother

\newcommand{\NN}{\mathbb{N}}
\newcommand{\ZZ}{\mathbb{Z}}
\newcommand{\QQ}{\mathbb{Q}}
\newcommand{\RR}{\mathbb{R}}
\newcommand{\CC}{\mathbb{C}}

\newcommand{\Max}{\displaystyle\max\limits}
\newcommand{\Sup}{\displaystyle\sup\limits}
\newcommand{\Sum}{\displaystyle\sum\limits}
\newcommand{\Int}{\displaystyle\int\limits}
\newcommand{\Iint}{\displaystyle\iint\limits}
\newcommand{\Lim}{\displaystyle\lim\limits}

\newcommand*\diff{\mathop{}\!\mathrm{d}}

\newcommand*\rfrac[2]{{}^{#1}\!/_{\!#2}}


\title{{\Huge МАТЕМАТИЧНА ФІЗИКА}}
\author{Скибицький Нікіта}
\date{\today}

\usepackage{amsthm}
\usepackage[dvipsnames]{xcolor}
\usepackage{thmtools}
\usepackage[framemethod=TikZ]{mdframed}

\theoremstyle{definition}
\mdfdefinestyle{mdbluebox}{%
	roundcorner = 10pt,
	linewidth=1pt,
	skipabove=12pt,
	innerbottommargin=9pt,
	skipbelow=2pt,
	nobreak=true,
	linecolor=blue,
	backgroundcolor=TealBlue!5,
}
\declaretheoremstyle[
	headfont=\sffamily\bfseries\color{MidnightBlue},
	mdframed={style=mdbluebox},
	headpunct={\\[3pt]},
	postheadspace={0pt}
]{thmbluebox}

\mdfdefinestyle{mdredbox}{%
	linewidth=0.5pt,
	skipabove=12pt,
	frametitleaboveskip=5pt,
	frametitlebelowskip=0pt,
	skipbelow=2pt,
	frametitlefont=\bfseries,
	innertopmargin=4pt,
	innerbottommargin=8pt,
	nobreak=true,
	linecolor=RawSienna,
	backgroundcolor=Salmon!5,
}
\declaretheoremstyle[
	headfont=\bfseries\color{RawSienna},
	mdframed={style=mdredbox},
	headpunct={\\[3pt]},
	postheadspace={0pt},
]{thmredbox}

\declaretheorem[style=thmbluebox,name=Теорема,numberwithin=subsubsection]{theorem}
\declaretheorem[style=thmbluebox,name=Лема,numberwithin=subsubsection]{lemma}
\declaretheorem[style=thmbluebox,name=Твердження,numberwithin=subsubsection]{proposition}
\declaretheorem[style=thmbluebox,name=Принцип,numberwithin=subsubsection]{th_principle}
\declaretheorem[style=thmbluebox,name=Закон,numberwithin=subsubsection]{law}
\declaretheorem[style=thmbluebox,name=Закон,numbered=no]{law*}
\declaretheorem[style=thmbluebox,name=Формула,numberwithin=subsubsection]{th_formula}
\declaretheorem[style=thmbluebox,name=Рівняння,numberwithin=subsubsection]{th_equation}
\declaretheorem[style=thmbluebox,name=Умова,numberwithin=subsubsection]{th_condition}
\declaretheorem[style=thmbluebox,name=Наслідок,numberwithin=subsubsection]{corollary}

\declaretheorem[style=thmredbox,name=Приклад,numberwithin=subsubsection]{example}
\declaretheorem[style=thmredbox,name=Приклади,sibling=example]{examples}

\declaretheorem[style=thmredbox,name=Властивість,numberwithin=subsubsection]{property}
\declaretheorem[style=thmredbox,name=Властивості,sibling=property]{properties}

\mdfdefinestyle{mdgreenbox}{%
	skipabove=8pt,
	linewidth=2pt,
	rightline=false,
	leftline=true,
	topline=false,
	bottomline=false,
	linecolor=ForestGreen,
	backgroundcolor=ForestGreen!5,
}
\declaretheoremstyle[
	headfont=\bfseries\sffamily\color{ForestGreen!70!black},
	bodyfont=\normalfont,
	spaceabove=2pt,
	spacebelow=1pt,
	mdframed={style=mdgreenbox},
	headpunct={ --- },
]{thmgreenbox}

\mdfdefinestyle{mdblackbox}{%
	skipabove=8pt,
	linewidth=3pt,
	rightline=false,
	leftline=true,
	topline=false,
	bottomline=false,
	linecolor=black,
	backgroundcolor=RedViolet!5!gray!5,
}
\declaretheoremstyle[
	headfont=\bfseries,
	bodyfont=\normalfont\small,
	spaceabove=0pt,
	spacebelow=0pt,
	mdframed={style=mdblackbox}
]{thmblackbox}

\declaretheorem[name=Вправа,numberwithin=subsubsection,style=thmblackbox]{exercise}
\declaretheorem[name=Зауваження,numberwithin=subsubsection,style=thmgreenbox]{remark}
\declaretheorem[name=Визначення,numberwithin=subsubsection,style=thmblackbox]{definition}

\newtheorem{problem}{Задача}[subsection]
\newtheorem{sproblem}[problem]{Задача}
\newtheorem{dproblem}[problem]{Задача}
\renewcommand{\thesproblem}{\theproblem$^{\star}$}
\renewcommand{\thedproblem}{\theproblem$^{\dagger}$}
\newcommand{\listhack}{$\empty$\vspace{-2em}} 

\theoremstyle{remark}
\newtheorem*{solution}{Розв'язок}


\begin{document}

\tableofcontents

\setcounter{section}{3}
\setcounter{subsection}{7}

\subsection{Постановка основних граничних задач для лінійних диференційних рівнянь 2-го порядку, коректність, класичні та узагальнені розв'язки}

Серед множини математичних моделей, які були розглянуті в попередніх параграфах можна виділити найтиповіші математичні моделі, які концентрують в собі головні особливості усіх розглянутих вище. Ці моделі представляють собою граничні задачі для рівнянь трьох типів: еліптичних, параболічних та гіперболічних лінійних рівнянь другого порядку. \medskip

Розглянемо основний диференціальний оператор другого порядку:
\begin{equation}
    L u = \nabla \cdot (p(x) \nabla u) - q(x) u.
\end{equation}

Запишемо основні диференціальні рівняння:
\begin{itemize}
    \item Еліптичне рівняння:
    \begin{equation}
        \label{eq:elliptic-differential-equation}
        L u = - F(x), \quad x \in \Omega \subset \RR^n.
    \end{equation}
    \item Параболічне рівняння:
    \begin{equation}
        \label{eq:parabolic-differential-equation}
        \rho(x) \frac{\partial u}{\partial t} = L u + F(x, t), \quad x \in \Omega, \quad t > t_0.
    \end{equation}
    \item Гіперболічне рівняння:
    \begin{equation}
        \label{eq:hyperbolic-differential-equation}
        \rho(x) \frac{\partial^2 u}{\partial t^2} = L u + F(x, t), \quad x \in \Omega, \quad t > t_0.
    \end{equation}
\end{itemize}

\subsubsection{Гранична задача для еліптичного рівняння}

Будемо розділяти внутрішні і зовнішні задачі для еліптичного рівняння, а саме, якщо $x \in \Omega$, то таку задачу будемо називати внутрішньою, якщо $x \in \Omega'$ --- задача зовнішня. \medskip

В подальшому ми будемо розглядати класичні розв'язки граничних задач. Це означає, що рівняння і усі граничні умови виконуються в кожній точці області або границі. \medskip

Введемо обмеження на коефіцієнти рівняння $p$ і $q$ та вільний член $F$. Зокрема будемо припускати, що $p > 0$, $p \in C^1 \left(\overline \Omega\right)$, $q \ge 0$, $q \in C \left(\overline \Omega\right)$, $F(x) \in C \left(\overline \Omega\right)$. \medskip

Позначимо $\partial \Omega = S$ --- поверхню на якій задаються граничні умови загального
вигляду:
\begin{equation}
    \label{eq:general-boundary-conditions}
    \left. \alpha(x) \frac{\partial u}{\partial n} + \beta(x) u \right|_S = V(x),
\end{equation}
де $\alpha, \beta \ge 0$, $\alpha, \beta, B \in C(S)$. З умови \eqref{eq:general-boundary-conditions} можна отримати умови 1, 2, 3 роду:
\begin{enumerate}
    \item Діріхле:
    \begin{equation}
        \label{eq:elliptic-dirichlet-boundary-conditions}
        \left. u \right|_S = \frac{V(x)}{\beta(x)}.
    \end{equation}
    \item Неймана:
    \begin{equation}
        \label{eq:elliptic-neumann-boundary-conditions}
        \left. \frac{\partial u}{\partial n} \right|_S = \frac{V(x)}{\alpha(x)}.
    \end{equation}
    \item Ньютона:
    \begin{equation}
        \label{eq:elliptic-newton-boundary-conditions}
        \left. \frac{\partial u}{\partial n} + \frac{\beta}{\alpha} u \right|_S = \frac{V}{\alpha}.
    \end{equation}
\end{enumerate}

Таким чином гранична задача для еліптичного рівняння може бути сформульована наступним чином: знайти функцію $u(x) \in C^2(\Omega) \cap C^1 \left( \overline \Omega \right)$, яка в кожній внутрішній точці області $\Omega$ (для внутрішньої задачі) або $\Omega'$ (для зовнішньої задачі) задовольняє рівняння \eqref{eq:elliptic-differential-equation}, а кожній точці границі $S$ виконується одна з граничних умов \eqref{eq:elliptic-dirichlet-boundary-conditions}, \eqref{eq:elliptic-neumann-boundary-conditions} або \eqref{eq:elliptic-newton-boundary-conditions}. \medskip

У випадку зовнішньої граничної задачі в нескінченно віддаленій точці області слід задавати додаткові умови поведінки розв'язку. Такі умови називають умовами регулярності на нескінченості. Як правило вони полягають в завданні характеру спадання розв'язку і мають вигляд
\begin{equation}
    u(x) = O \left( \frac{1}{|x|^\alpha} \right),
\end{equation}
при $|x| \to \infty$, де $\alpha$ --- деякий заданий параметр задачі.

\subsubsection{Постановка змішаних задач для рівняння гіперболічного типу. Задача Коші для гіперболічного рівняння}

Для постановки граничних задач рівняння гіперболічного типу \eqref{eq:hyperbolic-differential-equation} введемо просторово-часовий циліндр, як область зміни незалежних змінних $x$, $t$:
\begin{equation}
    Z(\Omega, T) = \Omega \times (0, T].
\end{equation}

Для отримання єдиного розв'язку гіперболічного рівняння, на нижній основі просторово-часового циліндру $Z_0(\Omega, T) = \Omega \times \{t = 0\}$ треба задати початкові умови:
\begin{align}
    \label{eq:hyperbolic-starting-function-condition}
    u(x, 0) &= u_0(x), \quad x \in \Omega, \\
    \label{eq:hyperbolic-starting-derivative-condition}
    \frac{\partial u(x, 0)}{\partial t} &= v_0(x), \quad x \in \Omega.
\end{align}

На боковій поверхні просторово-часового циліндру $Z_S(\Omega, T) = S \times (0, T]$ треба задати граничні умови одного з трьох основних типів:
\begin{enumerate}
    \item Діріхле:
    \begin{equation}
        \label{eq:hyperbolic-dirichlet-boundary-conditions}
        \left. u \right|_S = \phi(x, t).
    \end{equation}
    \item Неймана:
    \begin{equation}
        \label{eq:hyperbolic-neumann-boundary-conditions}
        \left. \frac{\partial u}{\partial n} \right|_S = \phi(x, t).
    \end{equation}
    \item Ньютона:
    \begin{equation}
        \label{eq:hyperbolic-newton-boundary-conditions}
        \left. \frac{\partial u}{\partial n} + \alpha(x, t) u \right|_S = \phi(x, t).
    \end{equation}
\end{enumerate}

Таким чином постановка граничної задачі для гіперболічного рівняння має вигляд: \medskip

Знайти функцію $u(x, t) \in C^{(2, 2)} \Big( Z(\Omega, T) \Big) \cap C^{(1, 1)} \Big( \overline {Z(\Omega, T)} \Big)$, яка задовольняє рівнянню \eqref{eq:hyperbolic-differential-equation} для $(x, t) \in Z(\Omega, T)$, початковим умовам \eqref{eq:hyperbolic-starting-function-condition}, \eqref{eq:hyperbolic-starting-derivative-condition} для $(x, t) \in Z_0(\Omega, T)$, і в кожній точці $(x, t) \in Z_S(\Omega, T)$ одній з граничних умов \eqref{eq:hyperbolic-dirichlet-boundary-conditions}--\eqref{eq:hyperbolic-newton-boundary-conditions}. \medskip

При цьому відносно вхідних даних будемо робити наступні припущення
\begin{gather}
    p > 0, \quad p \in C^1 \Big( \overline \Omega \Big), \quad q \ge 0, \quad q \in C \Big( \overline \Omega \Big), \quad F(x, t) \in C \Big( \overline{Z(\Omega, T)} \Big), \\
    u_0, v_0 \in C \left( \overline{Z_0(\Omega, T)} \right), \quad \alpha, \phi \in C \left( \overline{Z_S(\Omega, T)} \right), \quad \alpha \ge 0.
\end{gather}

\subsubsection{Задача Коші}

У випадку, коли область $\Omega$ має великі розміри і впливом граничних умов можна знехтувати, область $\Omega$ ототожнюється з усім евклідовим простором, тобто $\Omega = \RR^n$. \medskip

У зв'язку з відсутністю границі, граничні умови не задаються. В цьому

\newpage 

\end{document}