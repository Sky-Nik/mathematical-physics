%{Лекція 5}

\subsubsection{Додатньо визначені ядра}

\begin{definition}
    Неперервне ядро $K(x, y)$ називається додатньо визначеним, якщо $\forall f \in L_2(G): (\bf{K}f, f) \ge 0$. Причому $(\bf{K}f, f) = 0 \Leftrightarrow \|f\|_{L_2(G)} = 0$.
\end{definition}

Довільне додатньо визначене ядро є ермітовим (його білінійна форма $(\bf{K}f, f)$ приймає дійсні значення).

\begin{lemma}
    Для того, щоб неперервне ядро було додатньо визначеним необхідно і достатньо, щоб його характеристичні числа були додатні.
\end{lemma}

\begin{proof}
    Необхідність: Для довільної власної функції $(\bf{K} \phi_k, \phi_k) = \frac{1}{\lambda_k} > 0$. \\

    Достатність: Розглянемо $\bf{K} f$ як джерелувато-зображувану функцію, згідно до теореми Гілберта-Шмідта $\bf{K} f = \sum_{k = 1}^\infty \frac{(f, \phi_k)}{\lambda_k} \phi_k$, тоді \[ (\bf{K}f, f) = \Sum_{k = 1}^\infty \dfrac{(f, \phi_k)}{\lambda_k} (\phi_k, f) = \Sum_{k = 1}^\infty \dfrac{|(f, \phi_k)|^2}{\lambda_k} > 0, \] отже квадратична форма додатньо визначена. \\

    Таким чином додатність характеристичних чисел є критерієм додатної визначеності ядра.
\end{proof}

\begin{lemma}
    Довільне додатньо визначене неперервне ядро має характеристичні числа і для них має місце варіаційний принцип:
    \begin{equation}
        \label{eq:4.16}
        \dfrac{1}{\lambda_k} = \Sup_{\substack{f \in L_2(G) \\ (f, \phi_i) = 0, i = \overline{1, k - 1}}} \dfrac{(\bf{K}f, f)_{L_2(G)}}{\|f\|_{L_2(G)}^2}, \quad k = 1, 2, \ldots,
    \end{equation}
    де $\lambda_1 \le \lambda_2 \le \lambda_3 \le \ldots$, а $\phi_1, \phi_2, \phi_3, \ldots$ -- ортонормована система власних функцій.
\end{lemma}

\begin{proof}
    З теореми Гілберта Шмідта функціонал (\ref{eq:4.16}) можна оцінити \[ \dfrac{(\bf{K}f, f)_{L_2(G)}}{\|f\|_{L_2(G)}^2} = \Sum_{i=k}^\infty \dfrac{|(f, \phi_i)|^2}{\lambda_i\|f\|^2} \le \dfrac{1}{\lambda_k} \Sum_{i=k}^\infty \dfrac{|(f, \phi_i)|^2}{\|f\|^2} \le \dfrac{1}{\lambda_k}. \] (перша нерівність виконується оскільки $\lambda_k$ -- найменше характеристичне число в сумі, а друга випливає з нерівності Бесселя). З іншого боку при $f = \phi_k$ маємо $\frac{(\bf{K}\phi_k, \phi_k)}{\|\phi_k\|^2} = \frac{1}{\lambda_k}$, тобто існує функція на якій досягається верхня межа цієї нерівності.
\end{proof}

\begin{theorem}[Мерсера, про регулярну збіжність білінійного ряду для ермітових ядер зі скінченою кількістю від’ємних характеристичних чисел] 
    Якщо ермітове неперервне ядро $K(x, y)$ має лише скінчену кількість від’ємних характеристичних чисел, то його білінійний ряд 
    \begin{equation}
        \label{eq:4.17}
        K(x,y)=\Sum_{i=1}^\infty\dfrac{\phi_i(x)\bar\phi_i(y)}{\lambda_i}
    \end{equation}
    збігається в $\bar G\times\bar G$ абсолютно і рівномірно.
\end{theorem}

\begin{proof}
    Покажемо, що якщо ядро $K(x,y)$ -- додатньо визначене, то $\forall x\in\bar G:K(x,x)\ge0$. Оскільки $K(x,y)$ -- ермітове, то $K(x,x)=\bar K(x,x)$ і є дійсною функцією. Якщо існує хоча б одна точка $x_0\in\bar G$ така, що $K(x_0,x_0)<0$, то виходячи з неперервності знайдеться і деякій окіл цієї точки $U(x_0,x_0)\subset\bar G\times\bar G$ такий, що $\forall(x,y)\in U(x_0,x_0):Re K(x,y)<0$. Оберемо невід’ємну неперервну функцію $\phi(x)$ яка відміна від нуля лише в $U(x_0,x_0)$ і отримаємо \[ (\bf{K}\phi,\phi)=\Int_{U(x_0,x_0)}K(x,y)\phi(x)\phi(y)\diff x\diff y=\Int_{U(x_0,x_0)}Re K(x,y)\phi(x)\phi(y)\diff x\diff y\le0.\] 

    Остання нерівність вступає в протиріччя з припущенням додотньої визначеності ядра, тобто теорему достатньо довести для додатньо визначених ядер. \\

    Розглянемо ядро $K^p(x,y)=K(x,y)-\sum_{i=1}^p\frac{\bar\phi_i(y)\phi_i(x)}{\lambda_i}$, де $p$ -- номер останнього від’ємного характеристичного числа, так що усі $\lambda_i$, $i=p+1,p+2,\ldots$ є додатніми. Таким чином ядро $K^p(x,y)$ є неперервним та додатньо визначеним. А це означає, що $\forall x\in\bar G:K(x,x)\ge0$. Таким чином маємо нерівність:
    \begin{equation}
        \label{eq:4.18}
        \Sum_{i=1}^N\dfrac{|\phi_i(x)|^2}{\lambda_i}\le K(x,x)\le M,\quad x\in\bar G, N=p+1,p+2,\ldots
    \end{equation}

    Розглянемо білінійний ряд $\sum_{i=1}^\infty\frac{\phi_i(x)\bar\phi_i(y)}{\lambda_i}$ і доведемо його абсолютну і рівномірну збіжність за критерієм Коші. Використовуючи нерівність Коші-Буняківського маємо:
    \begin{equation}
        \label{eq:4.19}
        \Sum_{k=p}^{p+q}\dfrac{|\phi_k(x)\bar\phi_k(y)|}{\lambda_k}\le\left(\Sum_{k=p}^{p+q}\dfrac{|\phi_k(x)|^2}{\lambda_k}\Sum_{k=p}^{p+q}\dfrac{|\phi_k(y)|^2}{\lambda_k}\right)^{1/2}
    \end{equation}

    Але оскільки має місце (\ref{eq:4.18}), яка гарантує рівномірну збіжність нескінченного ряду $\sum_{k=1}^\infty\frac{|\phi_k(x)|^2}{\lambda_k}$, то білінійний ряд (\ref{eq:4.17}) збігається абсолютно і рівномірно (регулярно) в $\bar G\times\bar G$.
\end{proof}

\begin{remark}
    Теорема Гільберта-Шмідта і її наслідки, встановлені для ермітового неперервного ядра, залишаються вірними і для ермітового слабо полярного ядра.
\end{remark}

\begin{remark}
    Теорема Гільберта-Шмідта і формула Шмідта у випадку полярного ядра залишаються вірними, але з заміною рівномірної збіжності на середньоквадратичну.
\end{remark}

\subsection{Задача Штурма-Ліувілля. Теорема Стеклова}

Постановка задачі Штурма-Ліувілля: нехай $\bf{L}$ -- диференціальний оператор другого порядку:
\begin{equation}
    \label{eq:5.1}
    \bf{L}u=(-p(x)u')'+q(x)u=\lambda u,\quad0<x<l,
    %Lu = (-p(x) u')' + q(x) u = \lambda u, \quad 0 < x < l
\end{equation}
\begin{equation}
    \label{eq:5.2}
    l_1(u)|_{x=0}=h_1u(0)-h_2u'(0)=0,
\end{equation}
\begin{equation}
    \label{eq:5.3}
    l_2(u)|_{x=l}=H_1u(l)-H_2u'(l)=0,
\end{equation}
\begin{multline}
    \label{eq:5.4}
    p\in C^{(1)}([0,l]), q\in C([0,l]), q\ge0, p>0, \\
    h_1, h_2, H_1, H_2 \ge 0, h_1+h_2>0, H_1+H_2>0,
\end{multline}
\begin{equation}
    \label{eq:5.5}
    M_L=\{u:u\in C^{(2)}(0,l)\cap C^{(1)}([0,l]), u''\in L_2(0,l), l_1u(0)=l_2u(l)=0\}
\end{equation}
-- область визначення оператора $\bf{L}$.

\begin{definition}
    Знайти розв’язки задачі Штурма-Ліувіля означає знайти всі ті значення параметра $\lambda$, при яких гранична задача (\ref{eq:5.1})-(\ref{eq:5.4}) має нетривіальний розв’язок. Ці значення називаються власними значеннями задачі Штурма-Ліувіля, а самі розв’язки -- власними функціями.
\end{definition}

\subsubsection{Функція Гріна оператора $\bf{L}$}

Будемо припускати, що $\lambda = 0$ не є власним числом оператора $\bf{L}$ задачі Штурма-Ліувіля. \\

Розглянемо граничну задачу:
\begin{system}
    \label{eq:5.6}
    & (-p(x)u')'+q(x)u=f(x),\quad0<x<l \\
    & l_1(u)|_{x=0}=l_2(u)|_{x=l}=0
\end{system}

Припустимо що $f \in C(0,l)\cap L_2(0,l)$. \\

З припущення, що $\lambda = 0$ не є власним числом випливає, що задача (\ref{eq:5.6}) має єдиний розв’язок. \\

Розглянемо функції $v_i(x)$, $i=1,2$ -- ненульові дійсні розв’язки однорідних задач Коші:
\begin{system}
    \label{eq:5.7}
    & (-p(x)v_i'(x))'+q(x)v_i'(x)=0,\quad i=1,2 \\
    & l_1v_1|_{x=0}=l_2v_2|_{x=l}=0
\end{system}

З загальної теорії задач Коші випливає, що розв’язки цих задач Коші існують, тому $v_i(x)$ -- двічі неперервно диференційовані функції. Покажемо що $v_1(x)$, $v_2(x)$ -- лінійно незалежні. \\

Припустимо що це не так і $v_1(x) = cv_2(x)$, тобто $v_1(x)$ задовольняє одночасно граничним умовам на лівому і правому краях. Тоді $v_1(x)$ -- власна функція оператора $\bf{L}$, і відповідає власному числу $\lambda = 0$ що суперечить припущенню, тому $v_1(x)$, $v_2(x)$ -- лінійно незалежні. В цьому випадку визначник Вронського $w(x) = \begin{vmatrix} v_1 & v_2 \\ v_1' & v_2' \end{vmatrix} \ne 0$. \\

Будемо шукати розв’язок задачі (\ref{eq:5.6}) методом варіації довільної сталої у вигляді: $u(x) = c_1(x) v_1(x) + c_2(x) v_2(x)$. \\

Підставимо в рівняння: $(-p(c_1'v_1+c_2'v_2+c_1v_1'+c_2v_2')'+q(c_1v_1+c_2v_2)=f$. \\

Накладемо першу умову на коефіцієнти: $c_1'v_1+c_2'v_2=0$, маємо: \[ -p'(c_1v_1'+c_2v_2')-p(c_1'v_1'+c_2'v_2'+c_1v_1''+c_2v_2'')+q(c_1v_1+c_2v_2)=f, \] або $c_1Lv_1 + c_2Lv_2 - p(c_1'v_1'+c_2'v_2')=f$, оскільки $c_1\bf{L}v_1=0$, $c_2\bf{L}v_2=0$, то $-p(c_1'v_1'+c_2'v_2')=f$, отже $c_1'v_1'+c_2'v_2'=-\frac{f}{p}$. \\ 

Таким чином $c_1'$ та $c_2'$ повинні задовольняти системі лінійних диференціальних рівнянь:
\begin{system*}
    c_1'v_1 + c_2'v_2 &= 0, \\
    c_1'v_1' + c_2'v_2' &= - \dfrac{f}{p},
\end{system*}
визначник системи $w(x) = \begin{vmatrix} v_1 & v_2 \\ v_1' & v_2' \end{vmatrix} \ne 0$. \\

Має місце рівність Ліувілля: $w(x)p(x)=w(0)p(0)=const$. \\

Розв’язавши систему рівнянь, отримаємо:
\begin{system}
    \label{eq:5.8}
    c_1'(x) &= \dfrac{1}{w(x)} \begin{vmatrix} 0 & v_2 \\ - \dfrac{f}{p} & v_2' \end{vmatrix} = \dfrac{v_2(x)f(x)}{p(0)w(0)}, \\
    c_2'(x) &= \dfrac{1}{w(x)} \begin{vmatrix} v_1 & 0 \\ v_1' & - \dfrac{f}{p} \end{vmatrix} = -\dfrac{v_1(x)f(x)}{p(0)w(0)}, 
\end{system}

Знайдемо додаткові умови для диференціальних рівнянь (\ref{eq:5.8}). 
\begin{multline} 
l_1u|_{x=0} = h_1(c_1(0)v_1(0) + c_2(0)v_2(0)) - \\
- h_2(c_1'(0)v_1(0) + c_2'(0)v_2(0) + c_1(0)v_1'(0) + c_2(0)v_2'(0)) = 0,
\end{multline}

враховуючи, що $c_1'(0)v_1(0)+c_2'(0)v_2(0) = 0$ маємо \[ c_1(0) (h_1v_1(0) - h_2v_1'(0)) + c_2(0)v_2'(0) = 0. \]

Оскільки перший доданок дорівнює нулю, то остання рівність виконується коли $c_2(0) = 0$, аналогічно отримаємо, що $c_1(l) = 0$. \\

Проінтегруємо (\ref{eq:5.8}) отримаємо
\begin{equation}
    \label{eq:5.9}
    c_1(x)=-\Int_x^l\dfrac{f(\xi)v_2(\xi)}{p(0)w(0)}\diff\xi, \quad c_2(x)=-\Int_0^x\dfrac{v_1(\xi)f(\xi)}{p(0)w(0)}\diff\xi
\end{equation}

Розв’язок граничної задачі (\ref{eq:5.6}) буде мати вигляд:
\begin{equation}
    \label{eq:5.10}
    u(x)=-\Int_0^x\dfrac{v_1(\xi)v_2(x)f(\xi)}{p(0)w(0)}\diff\xi-\Int_x^l\dfrac{f(\xi)v_1(x)v_2(\xi)}{p(0)w(0)}\diff\xi
\end{equation}

Визначимо функцію Гріна:
\begin{equation}
    \label{eq:5.11}
    G(x, \xi) = - \dfrac{1}{p(0)w(0)} \begin{cases}
        v_1(\xi) v_2(x), & 0 \le \xi \le x \le l, \\
        v_1(x) v_2(\xi), & 0 \le x \le \xi \le l.
    \end{cases}
\end{equation}

Отже розв’язок граничної задачі (\ref{eq:5.6}) можна записати у вигляді
\begin{equation}
    \label{eq:5.12}
    u(x) = \Int_0^l G(x, \xi) f(\xi) \diff \xi
\end{equation}

$G(x, \xi)$ називається функцією Гріна оператору Штурма-Ліувіля. Попередні міркування доводять наступну лемму.
\begin{lemma}
    Якщо $\lambda = 0$ не є власним числом задачі Штурма-Ліувіля (\ref{eq:5.1})--(\ref{eq:5.4}), то розв’язок граничної задачі (\ref{eq:5.6}) існує та єдиний і представляється за формулою (\ref{eq:5.12}) через функцію Гріна (\ref{eq:5.11}).
\end{lemma}

\subsubsection{Властивості функції Гріна}
\begin{enumerate}
    \item $G(x, \xi) \in C([0, l] \times [0, l])$, $G(x, \xi) \in C^{(2)}(0 < x < \xi < l)$, $G(x, \xi) \in C^{(2)} (0 < \xi < x < l)$.
    \item Симетричність: $G(x, \xi) = G(\xi, x)$, $x, \xi \in [0, l] \times [0, l]$.
    \item На діагоналі $x = \xi$ має місце розрив першої похідної: $\frac{\partial G(\xi + 0, \xi)}{\partial x} - \frac{\partial G(\xi - 0, \xi)}{\partial x} = - \frac{1}{p(\xi)}$, $\xi\in(0, l)$. 
    \item Поза діагоналлю $x = \xi$ функція Гріна задовольняє однорідному диференціальному рівнянню $\bf{L}_x G(x, y) = 0$.
    \item На бічних сторонах квадрату $[0,l]\times[0,l]$ функція Гріна $G(x, y)$ задовольняє граничним умовам $l_1G|_{x=0}=l_2G|_{x=l}=0$.
    \item Функція $G(x,\xi)$ є розв’язком неоднорідного рівняння: $\bf{L}_xG(x,\xi)=-\delta(x-\xi)$, де $\delta(x)$ -- дельта-функція Дірака.
\end{enumerate}

\subsubsection{Зведення граничної задачі з оператором Штурма-Ліувілля до інтегрального рівняння}

Розглянемо граничну задачу з параметром
\begin{system}
    \label{eq:5.13}
    & \bf{L}u = (-p(x)u')' + q(x)u = f + \lambda u, \\
    & l_1(u)|_{x=0} = 0, \\
    & l_2(u)|_{x=l} = 0, \\
    & f \in C(0, l) \cap L_2(0, l),
\end{system}
і покажемо що вона зводиться до інтегрального рівняння Фредгольма другого роду з дійсним, симетричним та неперервним ядром $G(x, \xi)$.

\begin{theorem}[Про еквівалентність граничної задачі для рівняння другого порядку інтегральному рівнянню з ермітовим ядром] 
    Гранична задача (\ref{eq:5.13}) при умові, що $\lambda = 0$ не є власним числом оператора $\bf{L}$, еквівалентна інтегральному рівнянню Фредгольма другого роду:
    \begin{equation}
        \label{eq:5.14}
        u(x) = \lambda \Int_0^l G(x, \xi) u(\xi) \diff \xi + \Int_0^l G(x, \xi) f(\xi) \diff \xi, \quad u \in C([0, l]),
    \end{equation}
    де $G(x, \xi)$ -- функція Гріна оператора $\bf{L}$.
\end{theorem}

\begin{proof}
    Необхідність. Нехай виконується (\ref{eq:5.13}), тоді з леми 3 із заміною правої частини $f \mapsto f + \lambda u$ розв’язок (\ref{eq:5.13}) можемо представити у вигляді: \[ u(x) = \Int_0^l G(x, \xi) (\lambda u(\xi) + f(\xi)) \diff \xi,\] 
    тобто $u(x)$ задовольняє інтегральному рівнянню (\ref{eq:5.14}). \\

    Достатність. Нехай має місце рівність (\ref{eq:5.14}) і $u_0(x)$ її розв’язок. Розглянемо граничну задачу:
    \begin{system*}
        & \bf{L}u = f + \lambda u_0, \\
        & l_1(u)|_{x=0} = l_2(u)|_{x=l} = 0.
    \end{system*}

    За лемою 3, єдиний розв’язок цієї задачі задається формулою \[ u(x) = \lambda \Int_0^1 G(x, \xi) u_0(\xi) \diff \xi + \Int_0^1 G(x, \xi) f(\xi) \diff \xi, \]
    звідки випливає, що $u_0$ задовольняє рівнянню $Lu_0=f+\lambda u_0$, таким чином $u(x)=u_0(x)$ тобто $u_0$ є розв’язком крайової задачі (\ref{eq:5.13}). \\

    У випадку коли $f \equiv 0$, гранична задача (\ref{eq:5.13}) перетворюється в задачу Штурма-Ліувіля
    \begin{system}
        \label{eq:5.13'}
        & \bf{L}u = \lambda u, \quad 0 < x < l, \\
        & l_1(u)|_{x=0} = l_2(u)|_{x=l} = 0.
    \end{system}

    Задача Штурма-Ліувіля еквівалентна задачі про знаходження характеристичних чисел та власних функцій для однорідного інтегрального рівняння Фредгольма
    \begin{equation}
        \label{eq:5.14'}
        u(x) = \lambda \Int_0^1 G(x, \xi) u(\xi) \diff \xi
    \end{equation}
    при умові, що $\lambda = 0$ не є власним числом оператора $\bf{L}$.\\

    Покажемо як позбавитись цього припущення. Нехай маємо задачу Штурма-Ліувілля:
    \begin{system}
        \label{eq:5.15}
        & \bf{L}u = \lambda u, \quad 0 < x < l, \\
        & l_1(u)|_{x=0} = l_2(u)|_{x=l} = 0.
    \end{system}

    Легко бачити, що $(\bf{L}u, u) \ge 0$, тобто власні числа невід’ємні. \\

    Розглянемо граничну задачу:
    \begin{system}
        \label{eq:5.16}
        &\bf{L}_1 u \equiv (-p(x)u')'+(q(x)+1)u=\mu u,\\
        & l_1u|_{x=0}=l_2u|_{x=l}, \quad \mu=\lambda+1.
    \end{system}

    Задача (\ref{eq:5.16}) з точністю до позначень співпадає з задачею Штурма-Ліувіля (\ref{eq:5.1})-(\ref{eq:5.3}). Очевидно, що $\mu = 0$ не є власним числом задачі Штурма-Ліувіля (\ref{eq:5.16}) (бо тоді $\lambda = -1$ могло би бути власним числом задачі Штурма-Ліувілля (\ref{eq:5.1})-(\ref{eq:5.4})). Введемо диференціальний оператор $\bf{L}_1u = (-pu')' + q_1u = \mu u$. \\

    Отже, задача (\ref{eq:5.16}) еквівалентна задачі (\ref{eq:5.15}) при $\mu = \lambda + 1$, та еквівалентна інтегральному рівнянню 
    \begin{equation}
        \label{eq:5.17}
        u(x) = (\lambda+1)\Int_0^1 G_1(x, \xi)u(\xi)\diff \xi,
    \end{equation}
    де $G_1(x, \xi)$ -- функція Гріна оператора $\bf{L}_1$. \\

    Таким чином, ввівши оператор $\bf{L}_1$ і відповідну йому функцію Гріна $G_1(x, \xi)$, можна позбутися припущення, що $\lambda = 0$ не є власним числом задачі Штурма-Ліувілля.
\end{proof}

\begin{example}
    Знайти розв’язок інтегрального рівняння \[ \phi(x) = \lambda \Int_0^1 K(x, y) \phi(y) \diff y + x,\] де \[ K(x,y)=\begin{cases}x(y-1), & 0 \le x \le y \le 1 \\ y(x-1), & 0\le y\le x\le1\end{cases}. \]
\end{example}

\begin{solution*}
    Розв’язок будемо шукати за формулою Шмідта. Знайдемо характеристичні числа та власні функції ермітового ядра. Запишемо однорідне рівняння \[ \phi(x) = \lambda \Int_0^x y(x-1)\phi(y)\diff y + \lambda\Int_x^1x(y-1)\phi(y)\diff y.\]
    
    Продиференціюємо рівняння: \[ \phi'(x) = \lambda \Int_0^x y\phi(y)\diff y + \lambda x(x-1)\phi(x) + \lambda\Int_x^1(y-1)\phi(y)\diff y - \lambda x(x-1)\phi(x).\]
    
    Обчислимо другу похідну: \[ \phi''(x) = \lambda x\phi(x)-\lambda(x-1)\phi(x).\]

    Або після спрощення $\phi'' = \lambda \phi$. Доповнимо диференціальне рівняння другого порядку граничними умовами вигляду (\ref{eq:5.2}), (\ref{eq:5.3}). Легко бачити що \[ \phi(0) = \lambda \Int_0^0 y(0-1)\phi(y)\diff y + \lambda\Int_0^10(y-1)\phi(y)\diff y=0. \]

    Аналогічно \[ \phi(1) = \lambda\Int_0^1y(1-1)\phi(y)\diff y + \lambda\Int_1^11(y-1)\phi(y)\diff y=0.\]

    Таким чином отримаємо задачу Штурма-Ліувілля:
    \begin{system*}
        & \phi'' = \lambda \phi, \quad 0 < x < 1, \\
        & \phi(0) = \phi(1) = 0.
    \end{system*}

    Для знаходження власних чисел і власних функцій розглянемо можливі значення параметру $\lambda$:
    \begin{enumerate}
        \item $\lambda > 0$, $\phi(x)=c_1\sinh(\sqrt{\lambda}x)+c_2\cosh(\sqrt{\lambda}x)$. \\

        Враховуючи граничні умови, маємо систему рівнянь 
        \begin{system*}
            & c_1 \cdot 0 + c_2 = 0, \\
            & c_1 \sinh(\sqrt{\lambda}) + c_2 \cosh(\sqrt{\lambda}) = 0.
        \end{system*}

        Визначник цієї системи повинен дорівнювати нулю. \[ D(\lambda) = \begin{vmatrix} 0 & 1 \\ \sinh(\sqrt{\lambda}) & \cosh(\sqrt{\lambda}) \end{vmatrix} = - \sinh(\sqrt{\lambda}) = 0. \]

        Єдиним розв’язком цього рівняння є $\lambda = 0$, яке не задовольняє, бо $\lambda > 0$. Це означає, що система рівнянь має тривіальний розв’язок і будь-яке $\lambda > 0$ не є власним числом.

        \item $\lambda = 0$, $\phi(x) = c_1 x + c_2$. З граничних умов маємо, що $c_1 = c_2 = 0$. Тобто $\lambda=0$ не є власним числом.

        \item $\lambda < 0$, $\phi(x) = c_1\sin(\sqrt{-\lambda}x)+c_2\cos(\sqrt{-\lambda}x)$. \\

        Враховуючи граничні умови, маємо систему рівнянь
        \begin{system*}
            & c_1 \cdot 0 + c_2 = 0, \\
            & c_1 \sin(\sqrt{-\lambda}) + c_2 \cos(\sqrt{-\lambda}) = 0.
        \end{system*}

        Визначник цієї системи прирівняємо до нуля \[ D(\lambda) = \begin{vmatrix} 0 & 1 \\ \sin(\sqrt{-\lambda}) & \cos(\sqrt{-\lambda}) \end{vmatrix} = -\sin(\sqrt{-\lambda}) = 0. \]

        Це рівняння має зліченну множину розв’язків $\lambda_k = - (\pi k)^2$, $k = 1, 2, \ldots$. Система лінійних алгебраїчних рівнянь має розв’язок $c_2=0$, $c_1=c_1$. \\ 

        Таким чином нормовані власні функції задачі Штурма-Ліувілля мають вигляд $\phi_k(x) = \sqrt{2} \sin(k \pi x)$. \\

        Порахуємо коефіцієнти Фур’є $f_n = (f, \phi_n) = \sqrt{2} \int_0^1 x \sin(\pi nx) \diff x = \sqrt{2} \frac{(-1)^n}{\pi n}$. \\

        Згідно до формули Шмідта розв’язок рівняння при $\lambda \ne \lambda_k$ має вигляд: \[ \phi(x) = x - 2 \lambda \Sum_{k=1}^\infty \dfrac{(-1)^{k+1}\sin(\pi k x)}{((\pi k)^2+\lambda)\pi k}.\]

        При $\lambda = \lambda_k$ розв’язок не існує, оскільки не виконана умова ортогональності вільного члена до власної функції.
    \end{enumerate}
\end{solution*}