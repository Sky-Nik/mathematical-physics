%{Лекція 14}

\subsection{Класифікація рівнянь в частинних похідних}

\subsubsection{Класифікація рівнянь з двома незалежними змінними}

Будемо розглядати загальне рівняння другого порядку з двома незалежними змінними, лінійне відносно старших похідних. Частина рівняння, яка містить похідні другого порядку називають головною частиною рівняння.

\begin{equation}
	\label{eq:7.1}
	A(x,y)\dfrac{\partial^2u}{\partial x^2}+2B(x,y)\dfrac{\partial^2u}{\partial x\partial y}+C(x,y)\dfrac{\partial^2u}{\partial y^2}+F\left(x,y,u,\dfrac{\partial u}{\partial x},\dfrac{\partial u}{\partial y}\right)=0.
\end{equation}

Поставимо задачу спростити вигляд головної частини рівняння. Для чого введемо заміну змінних: 
\begin{equation}
	\label{eq:7.2}
	\xi = \xi(x, y), \quad \eta = \eta(x, y).
\end{equation}

Для скорочення скористаємося позначеннями $\dfrac{\partial u}{\partial x} = u_x$. \\

Обчислимо вирази для похідних в нових змінних $\xi$, $\eta$:
\[ u_x = u_\xi \cdot \xi_x + u_\eta \cdot \eta_x, \quad u_y = u_\xi \cdot \xi_y + u_\eta \cdot \eta_y, \]
\[ u_{xx} = u_{\xi\xi} \cdot \xi_x^2 + 2 u_{\xi\eta} \cdot \xi_x \cdot \eta_x + u_{\eta\eta} \cdot \eta_x^2 + u_\xi \cdot \xi_{xx} + u_\eta \cdot \eta_{xx}, \]
\[ u_{yy} = u_{\xi\xi} \cdot \xi_y^2 + 2 u_{\xi\eta} \cdot \xi_y \cdot \eta_x + u_{\eta\eta} \cdot \eta_y^2 + u_\xi \cdot \xi_{yy} + u_\eta \cdot \eta_{yy}, \]
\[ u_{xy} = u_{\xi\xi} \cdot \xi_x \cdot \xi_y + u_{\xi\eta} (\xi_x \eta_y + \xi_y \eta_x) + u_{\eta\eta} \cdot \eta_x \eta_y + u_\xi \cdot \xi_{xy} + u_\eta \cdot \eta_{xy}. \]

Підставимо обчислені похідні в рівняння (\ref{eq:7.1}):

\begin{multline}
	\label{eq:7.3}
	A (u_{\xi\xi} \cdot \xi_x^2 + 2 u_{\xi\eta} \xi_x \eta_x + u_{\eta\eta} \cdot \eta_x^2 ) + \\
	+ 2 B(u_{\xi\xi} \cdot \xi_x \cdot \xi_y + u_{\xi\eta}(\xi_x\eta_y+\xi_y\eta_x)+u_{\eta\eta}\cdot\eta_x\cdot\eta_y) + \\
	+ C (u_{\xi\xi} \cdot \xi_y^2 + 2 u_{\xi\eta}\xi_y\eta_y + u_{\eta\eta}\cdot \eta_y^2) + \widetilde{F}(\xi, \eta, u, u_\xi, u_\eta) = 0.
\end{multline}

Перегрупуємо доданки і отримаємо рівняння у вигляді:
\begin{equation}
	\label{eq:7.4}
	\overline{A} \cdot u_{\xi\xi} + 2\overline{B}\cdot u_{\xi\eta} + \overline{C} \cdot u_{\eta\eta} + \widetilde{F}(\xi,\eta,u,u_\xi,u_\eta)=0,
\end{equation}
де 
\begin{multline*}
	\overline{A} = A \cdot \xi_x^2 + 2 B \xi_x \xi_y + C \cdot \xi_y^2 \\
	\overline{B} = A \cdot \xi_x \cdot \eta_x + B(\xi_x \eta_y + \xi_y \eta_x) + C \cdot \xi_y \cdot \eta_y \\
	\overline{C} = A \cdot \eta_y^2 + 2 B \eta_x \eta_y + C \cdot \eta_y^2
\end{multline*}

Зробимо нульовими коефіцієнти при $u_{\xi\xi}$ та $u_{\eta\eta}$ за рахунок вибору нових змінних:
\begin{equation}
	\label{eq:7.5}
	A \xi_x^2 + 2 B \xi_x \xi_y + C \xi_y^2 = 0,
\end{equation}
\begin{equation}
	\label{eq:7.5'}
	A \eta_x^2 + 2 B \eta_x \eta_y + C \eta_y^2 = 0,
\end{equation}

Розв’язком рівняння (\ref{eq:7.5}) буде функція $\xi(x,y)$, а рівняння (\ref{eq:7.5'}) -- $\eta(x,y)$. \\

Для знаходження функцій $\xi(x, y)$, $\eta(x,y)$, зведемо рівняння в частинних похідних до звичайного диференціального рівняння. \\

Розглянемо рівняння (\ref{eq:7.5}) і розділимо його на $\xi_y^2$
\begin{equation}
	\label{eq:7.6}
	A \left( \dfrac{\xi_x}{\xi_y} \right)^2 + 2 B \dfrac{\xi_x}{\xi_y} + C = 0.
\end{equation}

Розглянемо неявно задану функцію $y = y(x)$ у вигляді $\xi(x, y) = \text{const}$, легко бачити $\diff \xi = \xi_x \diff x + \xi_y \diff y = 0$; $\dfrac{\xi_x}{\xi_y} = - \dfrac{\diff y}{\diff x}$. Тобто рівняння в частинних похідних зводиться до звичайного диференціального рівняння:
\begin{equation}
	\label{eq:7.7}
	A \left( \dfrac{\diff y}{\diff x} \right)^2 + 2 B \dfrac{\diff y}{\diff x} + C = 0.
\end{equation}

Останнє рівняння називається характеристичним, а його розв’язки називаються характеристиками. Рівняння розпадається на два лінійних рівняння:
\begin{equation}
	\label{eq:7.8}
	\dfrac{\diff y}{\diff x} = \dfrac{B + \sqrt{B^2 - AC}}{A}, \qquad \dfrac{\diff y}{\diff x} = \dfrac{B - \sqrt{B^2 - AC}}{A}.
\end{equation}

Знак підкореневого виразу визначає тип рівняння і спосіб вибору нових змінних. Розглянемо можливі випадки:

\begin{enumerate}
	\item рівняння гіперболічного типу $B^2 - AC > 0$. \\

	Кожне з рівнянь (\ref{eq:7.8}) має по одній дійсній характеристиці. Нехай $\phi(x,y)=\const$ та $\psi(x,y)=\const$ -- загальні інтеграли першого та другого характеристичного рівняння, тоді нові змінні вибираються $\xi = \phi(x, y)$, $\eta = \psi(x, y)$. \\

	Після застосування вказаної заміни змінних отримаємо першу канонічну форму запису рівняння гіперболічного типу 
	\begin{equation}
		\label{eq:7.9}
		u_{\xi\eta} = \Phi(\xi, \eta, u, u_\xi, u_\eta).
	\end{equation}

	Якщо використати змінні $\alpha = \dfrac{\xi+\eta}{2}$, $\beta = \dfrac{\xi-\eta}{2}$, то можна отримати другу канонічну форму запису рівняння гіперболічного типу
	\begin{equation}
		\label{eq:7.10}
		u_{\alpha\alpha} - u_{\beta\beta} = \Phi_1(\alpha, \beta, u, u_\alpha, u_\beta).
	\end{equation}

	\item рівняння еліптичного типу $B^2 - AC < 0$. \\

	В цьому випадку розв’язки характеристичних рівнянь (характеристики) -- комплексно спряжені і можуть бути записані у вигляді: $\omega(x,y)=\phi(x,y)\pm i\psi(x,y)=\const$. \\

	Тоді для змінних $\xi = \phi(x,y) + i \psi(x, y)$, $\eta = \phi(x, y) - i \psi(x, y)$ отримаємо вигляд аналогічний першій канонічній формі гіперболічного рівняння $u_{\xi\eta}=\Phi(\xi,\eta,u,u_\xi,u_\eta)$. \\

	Для того щоб позбутися комплексних змінних виберемо нові дійсні змінні $\alpha = \phi(x,y) = \dfrac{\xi+\eta}{2}$, $\beta=\psi(x,y)=\dfrac{\xi-\eta}{2i}$. Тоді отримаємо канонічну форму запису рівняння еліптичного вигляду:
	\begin{equation}
		\label{eq:7.11}
		u_{\alpha\alpha} + u_{\beta\beta} = \Phi(\alpha,\beta,u,u_\alpha,u_\beta).
	\end{equation}

	\item рівняння параболічного типу $B^2 - AC = 0$. \\

	Характеристики в цьому випадку співпадають і нові змінні обирають у вигляді: $\xi=\phi(x,y)$, $\eta=\nu(x,y)$, де $\nu(x,y)$ -- будь-яка функція незалежна від $\phi(x,y)$. \\

	\textbf{Зауваження}. Необхідно, щоб визначник Вронського для нових змінних $W[]\ne0$, тобто, щоб заміна змінних $\xi = \phi(x, y)$, $\eta = \nu(x,y)$ була не виродженою. \\

	У випадку параболічного рівняння маємо $AC = B^2$ і таким чином 
	\[ \overline{A} = (A \cdot \xi_x^2 + 2 B \xi_X \xi_y + C \cdot \xi_y^2) = \left( \sqrt{A} \xi_x + \sqrt{C} \xi_y \right)^2 = 0. \]
	\[ \overline{B} = A \xi_x \eta_x + B(\xi_x\eta_y+\xi_y\eta_x) + C\xi_y\eta_y = \left(\sqrt{A}\xi_x+\sqrt{C}\xi_y\right)\left(\sqrt{A}\eta_x+\sqrt{C}\eta_y\right) = 0. \]
	При цьому $\overline{C} \ne 0$. Таким чином після ділення на $\overline{C}$ отримаємо канонічну форму запису рівняння гіперболічного типу.
	\begin{equation}
		\label{eq:7.12}
		u_{\eta\eta} = \Phi(\xi,\eta,u,u_\xi,u_\eta).
	\end{equation}
\end{enumerate}

\subsubsection{Класифікація рівнянь другого порядку з багатьма незалежними змінними}

Будемо розглядати лінійне рівняння з дійсними коефіцієнтами
\begin{equation}
	\label{eq:7.13}
	\Sum_{i=1}^n \Sum_{j=1}^n A_{i,j} u_{x_i x_j} + \Sum_{i=1}^n B_i u_{x_i} + Cu + F = 0,
\end{equation}
де $A_{i,j} = A_{j,i}$, $A_{i,j}$, $B_i$, $C$, $F$ є функціями від $x = (x_1, x_2, \ldots, x_n)$. \\

Введемо нові змінні
\begin{equation}
	\label{eq:7.14}
	\xi_k = \xi_k(x_1, x_2, \ldots, x_n), \quad k = \overline{1, n}.
\end{equation}

Обчислимо похідні, що входять в рівняння
\[ u_{x_i} = \Sum_{k=1}^n u_{\xi_k} \alpha_{i,k}, \quad u_{x_i x_j} = \Sum_{k=1}^n \Sum_{l=1}^n u_{\xi_k\xi_l} \alpha_{i,k}\alpha_{j,l} + \Sum_{k=1}^n u_{\xi_k} (\xi_k)_{x_ix_j}, \]
де $\alpha_{i,k} = \dfrac{\partial \xi_k}{\partial x_i}$. \\

Підставляючи вираз для похідних в вихідне рівняння отримаємо:
\begin{equation}
	\label{eq:7.15}
	\Sum_{k=1}^n \Sum_{l=1}^n \overline{A}_{k,l} u_{\xi_k\xi_l} + \Sum_{k=1}^n \overline{B}_k u_{\xi_k} + \overline{C} u + \overline{F} = 0,
\end{equation}
де
\begin{equation}
	\label{eq:7.16}
	\overline{A}_{k,l} = \Sum_{i=1}^n \Sum_{j=1}^n A_{i,j}\alpha_{i,k}\alpha_{j,l}, \quad \overline{B}_k = \Sum_{i=1}^n B_i \alpha_{i,k} + \Sum_{i=1}^n \Sum_{j=1}^n A_{i,j}(\xi_k)_{x_ix_j}.
\end{equation}

Поставимо у відповідність рівнянню квадратичну форму $\Sum_{i=1}^n \Sum_{j=1}^n A_{i,j}^0 y_iy_j$, де $A_{i,j}^0=A_{i,j}(x_1^0,\ldots,x_n^0)$, тобто коефіцієнти квадратичної форми співпадають з коефіцієнтами рівняння в деякій точці області. \\

Здійснимо над змінними $y$ лінійне перетворення $y_i = \Sum_{k=1}^n \alpha_{i,k} \eta_k$. \\

Будемо мати для квадратичної форми новий вираз:
\begin{equation}
	\label{eq:7.17}
	\Sum_{i=1}^n \Sum_{j=1}^n A_{i,j}^0 y_iy_j = \Sum_{k=1}^n\Sum_{l=1}^n \overline{A}_{k,l}^0 \eta_k \eta_l,
\end{equation}
де $\overline{A}_{k,l} = \Sum_{i=1}^n \Sum_{j=1}^n \overline{A}_{i,j}^0 \alpha_{i,k}\alpha_{j,l}$. \\

Таким чином можна бачити, що коефіцієнти головної частини рівняння перетворюються аналогічно коефіцієнтам квадратичної форми при лінійному перетворені (\ref{eq:7.16}), (\ref{eq:7.17}). Відомо, що використовуючи лінійне перетворення можна привести матрицю $\left[A_{i,j}^0\right]_{i,j=\overline{1,n}}$ квадратичної форми до діагонального вигляду, в якому $\overline{A}_{i,j}^0 = \delta_{i,j}$. \\

Згідно до закону інерції квадратичних форм, кількість додатних, від’ємних та нульових діагональних елементів інваріантне відносно лінійного перетворення. \\

Будемо називати рівняння в точці $(x_1^0,\ldots,x_n^0)$ еліптичним, якщо всі $\overline{A}_{i,i}^0$, $i=\overline{1,n}$ мають один і той же знак. Гіперболічним, якщо $m < n$ елементів матриці мають один знак, а $n - m$ коефіцієнтів мають протилежний знак. Параболічним, якщо хоча б один з діагональних елементів матриці $\overline{A}_{i,i}^0$ дорівнює нулю. \\

Обираючи нові незалежні змінні $\xi_i$ таким чином що б у точці $(x_1^0,\ldots,x_n^0)$ виконувалось рівність $\alpha_{i,k}=\dfrac{\partial \xi_k}{\partial xi_i} = \alpha_{i,k}^0$, де $\alpha_{i,k}^0$ -- коефіцієнти перетворення, яке приводить квадратичну форму до канонічної форми запису. Зокрема, покладаючи $\xi_k = \Sum_{i=1}^n \alpha_{i,k}^0 x_i$, отримаємо у точці $(x_1^0,\ldots,x_n^0)$ канонічну форму запису рівняння в залежності від його типу:
\begin{equation}
	\label{eq:7.18}
	\Sum_{i=1}^n u_{\xi_i\xi_i} + \Phi = 0
\end{equation} -- еліптичний тип;
\begin{equation}
	\label{eq:7.19}
	\Sum_{i=1}^m u_{\xi_i\xi_i} - \Sum_{i=m+1}^n u_{\xi_i\xi_i} + \Phi = 0
\end{equation} -- гіперболічний тип;
\begin{equation}
	\label{eq7.20}
	\Sum_{i=1}^m \pm u_{\xi_i\xi_i} + \Phi = 0
\end{equation} -- параболічний тип.
