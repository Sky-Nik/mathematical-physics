\section*{Лекція 1}

\subsection*{\S0. Предмет і методи математичної фізики}

Сучасні технології дослідження реального світу доволі інтенсивно використовують методи математичного моделювання, зокрема ці методи широко використовуються тоді, коли дослідження реального (фізичного) об’єкту є неможливими, або надто дорогими. Вже традиційними стали моделювання властивостей таких фізичних об’єктів:
\begin{itemize}
	\item температурні поля і теплові потоки;
	\item електричні, магнітні та електромагнітні поля;
	\item концентрація речовини в розчинах, розплавах або сумішах;
	\item напруження і деформації в пружних твердих тілах;
	\item параметри рідини або газу, який рухається (обтікає) деяке тіла ;
	\item перенос різних субстанцій потоками рідин або газу та інші.
\end{itemize}

Характерною особливістю усіх математичних моделей, що описують перелічені та багато інших процесів є те, що параметри, які представляють інтерес для дослідника є функціями точки простору $\bf{x} = (x_1, x_2, x_3)$ та часу $t$, а самі співвідношення з яких ці характеристики обчислюються є диференціальними рівняннями в частинних похідних зі спеціальними додатковими умовами (крайовими умовами), які дозволяють виділяти однозначний розв’язок. \\

Таким чином можна сказати, що основними об’єктами дослідження предмету математична фізика є крайові задачі для рівнянь в частинних похідних, які моделюють певні фізичні процеси. \\

Процес дослідження реального об’єкту фізичного світу можна представити за наступною схемою:
\begin{enumerate}
	\item Побудова математичної моделі реального процесу у вигляді диференціального рівняння або системи диференціальних рівнянь в частинних похідних, доповнення диференціального рівняння в частинних похідних граничними умовами.
	\item Дослідження властивостей сформульованої крайової задачі з точки зору її коректності. Коректність постановки задачі передбачає виконання наступних умов:
	\begin{itemize}
		\item Розв’язок крайової задачі існує;
		\item Розв’язок єдиний;
		\item Розв’язок неперервним чином залежить від вхідних даних задачі.
	\end{itemize}
	\item Знаходження розв’язку крайової задачі: точного для найбільш простих задач, або наближеного для переважної більшості задач.
\end{enumerate}

Треба відмітити, що усі перелічені пункти дослідження окрім побудови наближених методів знаходження розв’язків відносяться до предмету дисципліни Математична фізика. \\

Для дослідження задач математичної фізики використовуються математичний апарат наступних розділів математики:
\begin{itemize}
	\item математичний аналіз;
	\item лінійна алгебра;
	\item диференціальні рівняння;
	\item теорія функцій комплексної змінної;
	\item функціональний аналіз;
\end{itemize}

При побудові математичних моделей використовуються знання з елементарної фізики. \\

Наведемо приклад доволі простої і в той же час цілком реальної математичної моделі розповсюдження тепла в стрижні. \\

Нехай ми маємо однорідний стрижень з теплоізольованою боковою поверхнею і наступними фізичними параметрами:
\begin{itemize}
	\item $\rho$ -- густина матеріалу;
	\item $S$ -- площа поперечного перерізу;
	\item $k$ -- коефіцієнт теплопровідності;
	\item $c$ -- коефіцієнт теплоємності;
	\item $L$ -- довжина стрижня.
\end{itemize}

Позначимо $u(x, t)$ -- температуру стрижня в точці $x$ в момент часу $t$, $u_0(x)$ -- температуру стрижня у точці $x$ в початковий момент часу $t = 0$. \\

Припустимо, що на лівому кінці стрижня температура змінюється за заданим законом $\phi(t)$, а правий кінець стрижня теплоізольований. \\

В таких припущеннях математична модель може бути записана у вигляді наступної граничної задачі:
\begin{equation}
	\label{eq:1}
	c \rho \dfrac{\partial u(x, t)}{\partial t} = k \dfrac{\partial^2 u(x, t)}{\partial x^2}, \quad 0 < x < L, t > 0
\end{equation}
\begin{equation}
	\label{eq:2}
	u(0, u) = \phi(t), \quad  \dfrac{\partial u(L, t)}{\partial x} = 0
\end{equation}
\begin{equation}
	\label{eq:3}
	u(x, 0) = u_0(t)
\end{equation}

Математична модель містить диференціальне рівняння (\ref{eq:1}), яке виконується для вказаних значень аргументу, граничні умови на кінцях стрижня (\ref{eq:2}) та початкову умови (\ref{eq:3}).

\chapter*{Глава 1 Інтегральні рівняння}

\subsubsection*{Основні поняття}

Інтегральні рівняння -- рівняння, що містять невідому функцію під знаком інтегралу. \\

Багато задач математичної фізики зводяться до лінійних інтегральних рівнянь виду:

\begin{equation}
	\label{eq:*.1}
	\phi(x) = \lambda \Int_G K(x, y) \phi(y) \diff y + f(x)
\end{equation}
-- інтегральне рівняння Фредгольма II роду.

\begin{equation}
	\label{eq:*.2}
	\Int_G K(x, y) \phi(y) \diff y = f(x)
\end{equation}
-- інтегральне рівняння Фредгольма I роду. \\

$K(x, y)$ -- ядро інтегрального рівняння $K(x, y) \in C(\bar G \times \bar G)$, $f(x)$ -- вільний член інтегрального рівняння, $f(x) \in C(\bar G)$, $\lambda$ -- комплексний параметр, $\lambda \in \CC$ (відомий або невідомий), $G$ -- область інтегрування, $G \subseteq \RR^n$, $\bar G$ -- замкнена та обмежена. \\

Інтегральне рівняння (\ref{eq:*.1}) при $f(x) \equiv 0$ називається однорідним інтегральним рівнянням Фредгольма II роду
\begin{equation}
	\label{eq:*.3}
	\phi(x) = \lambda \Int_G K(x, y) \phi(y) \diff y.
\end{equation}
$\bf{K}$ -- інтегральний оператор: $(\bf{K} \phi)(x)$. \\

Будемо записувати інтегральні рівняння (\ref{eq:*.1}), (\ref{eq:*.2}) та (\ref{eq:*.3}) скорочено в операторній формі:
\begin{equation}
	\label{eq:*.1'}
	\phi = \lambda \bf{K} \phi + f
\end{equation}
\begin{equation}
	\label{eq:*.2'}
	\bf{K} \phi = f
\end{equation}
\begin{equation}
	\label{eq:*.3'}
	\phi = \lambda \bf{K} \phi
\end{equation}
\begin{equation}
	\label{eq:*.4}
	K^*(x, y) = \bar K(y, x)
\end{equation}
називають спряжене (союзне) ядро. \\

Інтегральне рівняння
\begin{equation}
	\label{eq:*.5}
	\psi(x) = \bar \lambda \Int_G K^*(x, y) \psi(y) \diff y + g(x)
\end{equation}
називається спряженим (союзним) до інтегрального рівняння (\ref{eq:*.1}). \\

Операторна форма:
\begin{equation}
	\label{eq:*.5'}
	\psi = \bar \lambda \bf{K}^* \psi + g
\end{equation}
\begin{equation}
	\label{eq:*.6}
	\psi = \bar \lambda \bf{K}^* \psi
\end{equation}

\begin{definition*}
	Комплексні значення $\lambda$, при яких однорідне інтегральне рівняння Фредгольма (\ref{eq:*.3}) має нетривіальні розв’язки, називаються характеристичними числами ядра $K(x, y)$. Розв’язки, які відповідають власним числам, називаються власними функціями. Кількість лінійно-незалежних власних функцій називається кратністю характеристичного числа.
\end{definition*}

\subsection*{\S1. Метод послідовних наближень}

\subsubsection*{Метод послідовних наближень для неперервного ядра}

Нагадаємо означення норм в банаховому просторі неперервних функцій $C(\bar G)$ та гільбертовому просторі інтегрованих з квадратом функцій $L_2(G)$ та означення скалярного добутку в просторі $L_2(G)$:
\[ \|f\|_{C(\bar G)} = \Max_{x \in \bar G} |f(x)|, \quad \|f\|_{L_2(G)} = \left( \Int_G |f(x)|^2 \diff x \right)^{1/2}, \quad (f, g)_{L_2(G)} = \Int_G f(x) \bar g(x) \diff x. \]

\begin{lemma} 
	Інтегральний оператор $\bf{K}$ з неперервним ядром $K(x, y)$ петворює $C(\bar G) \xrightarrow{\bf{K}} C(\bar G)$, $L_2(G) \xrightarrow{\bf{K}} L_2(G)$, $L_2(G) \xrightarrow{\bf{K}} C(\bar G)$ обмежений та мають місце нерівності:
	\begin{equation}
		\label{eq:1.1}
		\| \bf{K} \phi \|_{C(G)} \le M V \| \phi \|_{C(G)},
	\end{equation}
	\begin{equation}
		\label{eq:1.2}
		\| \bf{K} \phi \|_{L_2(G)} \le M V \| \phi \|_{L_2(G)},
	\end{equation}
	\begin{equation}
		\label{eq:1.3}
		\| \bf{K} \phi \|_{C(G)} \le M \sqrt{V} \| \phi \|_{L_2(G)},
	\end{equation}
	де
	\begin{equation}
		\label{eq:1.4}
		M = \Max_{x, y \in G \times G} |K(x, y)|,
	\end{equation}
	\begin{equation}
		\label{eq:1.5}
		V = \Int_G \diff y.
	\end{equation}
\end{lemma}

\begin{proof}
	Нехай $\phi \in L_2(G)$. Тоді $\phi$ -- абсолютно інтегрована функція на $G$ і, оскільки ядро $K(x, y)$ неперервне на $G \times G$, функція $(\bf{K}\phi)(x)$ неперервна на $G$. Тому оператор $\bf{K}$ переводить $L_2(G)$ в $C(\bar G)$ і, з врахуванням нерівності Коші-Буняковського, обмежений. Доведемо нерівності:
	\begin{enumerate}
		\item (\ref{eq:1.1}):
		\begin{multline*}
			\| \bf{K} \phi \|_{C(\bar G)} = \Max_{x \in \bar G} \left| \Int_G K(x, y) \phi(y) \diff y \right| \le \Max_{x \in \bar G} \Int_G |K(x, y)| |\phi(y)| \diff y \le \\
			\le \Max_{x \in \bar G} \left( \Max_{y \in \bar G} |K(x, y)| \Max_{y \in \bar G} |\phi(y)| \Int_G \diff y \right) \le \Max_{x, y \in \bar G \times \bar G} |K(x, y)| \Max_{y \in \bar G} |\phi(y)| \Int_G \diff y = M V \|\phi\|_{C(\bar G)}.
		\end{multline*}
		\item (\ref{eq:1.2}):
		\begin{multline*}
			\left( \| \bf{K} \phi \|_{L_2(G)} \right)^2 = \Int_G \left| \Int_G K(x, y) \phi(y) \diff y \right|^2 \diff x \le \Int_G \left| \Max_{y \in \bar G} |K(x, y)| \Int_G \phi(y) \diff y \right|^2 \diff x \le \\
			\le \left( \Max_{x, y \in \bar G \times \bar G} |K(x, y)| \right)^2 \left| \Int_G \phi(y) \diff y \right|^2 \Int_G \diff x \le (M \| \phi\|_{L_2(G)} V)^2
		\end{multline*}
		\item (\ref{eq:1.3}):
		\begin{multline*}
			\| \bf{K} \phi \|_{C(\bar G)} = \Max_{x \in \bar G} |(\bf{K} \phi) (x)| = \Max_{x \in \bar G} \left| \Int_G K(x, y) \phi(y) \diff y \right| \le \\
			\le \Max_{x \in \bar G} \sqrt{\Int_G |K(x, y)|^2 \diff y} \sqrt{\Int_G |\phi(y)|^2 \diff y} \le M \sqrt{V} \|\phi\|_{L_2(G)}.
		\end{multline*}
	\end{enumerate}
\end{proof}

Розв’язок інтегрального рівняння другого роду (\ref{eq:*.1'}) будемо шукати методом послідовних наближень:
\begin{equation}
	\label{eq:1.10}
	\phi_0 = f, \quad \phi_1 = \lambda \bf{K} \phi_0 + f, \quad \phi_2 = \lambda \bf{K} \phi_1 + f, \quad \ldots, \quad \phi_{n + 1} = \lambda \bf{K} \phi_n + f
\end{equation}
\begin{equation}
	\label{eq:1.11}
	\phi_{n + 1} = \Sum_{i = 0}^{n + 1} \lambda^i \bf{K}^i f, \quad \bf{K}^{i + 1} = \bf{K} (\bf{K}^i)
\end{equation}
\begin{equation}
	\label{eq:1.12}
	\phi_\infty = \Lim_{n \to \infty} \phi_{n} = \Sum_{i = 0}^\infty \lambda^i \bf{K}^i f,
\end{equation}
ряд Неймана. \\

Дослідимо збіжність ряду Неймана (\ref{eq:1.12})
\begin{equation}
	\label{eq:1.13}
	\left\| \Sum_{i = 0}^\infty \lambda^i \bf{K}^i f \right\|_{C(\bar G)} \le \Sum_{i = 0}^\infty |\lambda^i| \| \bf{K}^i f \|_{C(\bar G)} \le \Sum_{i = 0}^\infty |\lambda^i(MV)^i| \| f \|_{C(\bar G)} = \dfrac{\|f\|_{C(\bar G)}}{1 - |\lambda| MV}
\end{equation}
(оскільки: $\| \bf{K} \phi\|_{C(\bar G)} \le MV \|\phi\|_{C(\bar G)}$, $\|\bf{K}^2\phi\|_{C(\bar G)} \le (MV)^2 \|\phi\|_{C(\bar G)}\|\bf{K}^i\phi\|_{C(\bar G)} \le (MV)^i \|\phi\|_{C(\bar G)}$). \\

Отже, ряд Неймана збігається рівномірно при 
\begin{equation}
	\label{eq:1.14}
	|\lambda| < \dfrac{1}{MV},
\end{equation} 

умова збіжності методу послідовних наближень. \\

Покажемо, що при виконанні умови (\ref{eq:1.14}) інтегральне рівняння (\ref{eq:*.1}) має єдиний розв’язок. Дійсно припустимо, що їх два:

\begin{equation*}
	\begin{matrix}
		\phi^{(1)} = \lambda \bf{K} \phi^{(1)} + f \\
		\phi^{(2)} = \lambda \bf{K} \phi^{(2)} + f
	\end{matrix}
	\Rightarrow
	\begin{matrix}
		\phi^{(0)} = \phi^{(1)} - \phi^{(2)}  \\
		\phi^{(0)} = \lambda \bf{K} \phi^{(0)}
	\end{matrix}
\end{equation*}

Обчислимо норму Чебишева: \[ |\lambda| \|\bf{K} \phi^{(0)}\|_{C(\bar G)} = \| \phi^{(0)} \|_{C(\bar G)} \Rightarrow \| \phi^{(0)} \|_{C(\bar G)} \le |\lambda| MV \|\phi^{(0)}\|_{C(\bar G)} \Rightarrow (1 - |\lambda|MV)\|\phi^{(0)}\|_{C(\bar G)} \le 0. \]

Звідси маємо, що $\|\phi^{(0)}\|_{C(\bar G)} = 0$. Таким чином доведена теорема

\begin{theorem}[Про існування розв’язку інтегрального рівняння Фредгольма з неперервним
ядром для малих значень параметру]
	Будь-яке інтегральне рівняння Фредгольма другого роду $\phi(x) = \lambda \int_G K(x, y) \phi(y) \diff y + f(x)$ з неперервним ядром $K(x, y)$ при умові (\ref{eq:1.14}) має єдиний розв’язок $\phi$ в класі неперервних функцій $C(\bar G)$ для будь-якого неперервного вільного члена $f$. Цей розв’язок може бути знайдений у вигляді ряду Неймана $\phi = \lim_{n \to \infty} \phi_n = \sum_{i = 0}^\infty \lambda^i \bf{K}^i f$.
\end{theorem}

\subsubsection*{Повторні ядра}

$\forall f, g \in \bar G$ має місце рівність 
\begin{equation}
	\label{eq:1.15}
	(\bf{K}f,g)_{L_2(G)} = (f, \bf{K}^*g)_{L_2(G)}
\end{equation}
Дійсно, якщо $f, g \in L_2(G)$, то за лемою 1 $\bf{K}f, \bf{K}^*g \in L_2(G)$ тому
\begin{multline*}
	(\bf{K}f, g) = \Int_G (\bf{K}f)\bar g \diff x = \Int_G \left( \Int_G K(x, y) f(y) \diff y\right) \bar g(x) \diff x = \\
	= \Int_G f(y) \left( \Int_G K(x, y) \bar g(x) \diff x\right) \diff y = \Int_G f(y) (\bf{K}^* g)(y) \diff y = (f, \bf{K}^*g).
\end{multline*}

\begin{lemma}
	Якщо $\bf{K}_1$, $\bf{K}_2$ -- інтегральні оператори з неперервними ядрами $K_1(x, y)$, $K_2(x, y)$ відповідно, то оператор $\bf{K}_3 = \bf{K}_2 \bf{K}_1$ також інтегральний оператор з неперервним ядром $K_3(x, z) = \int_G K_2(x, y) K_1(y, z) \diff y$. При цьому справедлива формула: $(\bf{K}_2\bf{K}_1)^* = \bf{K}_1^* \bf{K}_2^*$.
\end{lemma}
\begin{proof}
	Нехай $K_1(x, y)$, $K_2(x, y)$ -- ядра інтегральних операторів $\bf{K}_1$, $\bf{K}_2$. Розглянемо $\bf{K}_3 = \bf{K}_2 \bf{K}_1$:
	\begin{multline*}
		(\bf{K}_3 f)(x) = (\bf{K}_2\bf{K}_1f)(x) = \Int_G K_2(x, y) \left( \Int_G K_1(y, z) f(z) \diff z \right) \diff y = \\
		= \Int_G \left( \Int_G K_2(x, y) K_1(y, z) \diff y\right) f(z) \diff z = \Int_G K_3(x, z) f(z) \diff z.
	\end{multline*}
	Тобто $K_3(x, z) = \int_G K_2(x, y) K_1(y, z) \diff y$ -- ядро оператора $\bf{K}_2\bf{K}_1$. \\

	З рівності (\ref{eq:1.15}) для всіх $f, g \in L_2(G)$ отримуємо $(f, \bf{K}_3^*g - \bf{K}_1^* \bf{K}_2^* g) = 0$, звідки випливає, що $\bf{K}_3^* = \bf{K}_1^* \bf{K}_2^*$.
\end{proof}

Із доведеної леми випливає, що оператори $\bf{K}^n = \bf{K} (\bf{K}^{n - 1}) = (\bf{K}^{n - 1})\bf{K}$ -- інтегральні та їх ядра $K_{(n)}(x, y)$ -- неперервні та задовольняють рекурентним співвідношенням:
\begin{equation}
	\label{eq:1.16}
	K_{(1)}(x, y) = K(x, y), \quad \ldots, \quad K_{(n)}(x, y) = \Int_G K(x, \xi) K_{(n - 1)}(\xi, y) \diff \xi
\end{equation}
-- повторні (ітеровані) ядра.
\[ \bf{K}f = \Int_G K(x, y) f(y) \diff y, \quad \ldots, \quad \bf{K}^n f = \Int_G K_{(n)}(x, y)f(y) \diff y. \]

\subsubsection*{Резольвента інтегрального оператора}

Пригадаємо представлення розв’язку інтегрального рівняння (\ref{eq:*.1}) у вигляді ряду Неймана $\phi(x) = \sum_{i = 0}^\infty \lambda^i \bf{K}^i f$. Виконаємо перетворення
\begin{multline*}
	\phi(x) = f(x) + \lambda \Sum_{i = 1}^\infty \lambda^{i - 1} (\bf{K}^i f) x = f(x) + \Sum_{i = 1}^\infty \lambda^{i - 1} K_{(i)} (x, y) f(y) \diff y = \\
	= f(x) + \lambda \Int_G \left( \Sum_{i = 1}^\infty \lambda^{i - 1} K_{(i)} (x, y) \right) f(y) \diff y = f(x) + \lambda \Int_G \mathcal{R}(x, y, \lambda) f(y) \diff y,
\end{multline*}
при $|\lambda| < \frac{1}{MV}$, де
\begin{equation}
	\label{eq:1.17}
	\mathcal{R}(x, y, \lambda) = \Sum_{i = 1}^\infty \lambda^{i - 1} K_{(i)} (x, y)
\end{equation}
-- резольвента інтегрального оператора. \\

Операторна форма запису розв’язку рівняння Фредгольма через резольвенту ядра має вигляд:
\begin{equation}
	\label{eq:1.18}
	\phi = f + \lambda \bf{R} f
\end{equation}

Мають місце операторні рівності: $\phi = (E + \lambda \bf{R})f$, $(E - \lambda \bf{K})\phi = f$, $\phi = (E - \lambda \bf{K})^{-1}f$, таким чином маємо
\begin{equation}
	\label{eq:1.19}
	E + \lambda \bf{R} = (E - \lambda \bf{K})^{-1}, \quad |\lambda| < \dfrac{1}{MV}.
\end{equation}
Зважуючи на формулу (\ref{eq:1.18}) має місце теорема
\begin{theorem}[Про існування розв’язку інтегрального рівняння Фредгольма з неперервним
ядром для малих значенням параметру]
	Будь-яке інтегральне рівняння Фредгольма другого роду $\phi(x) = \lambda \int_G K(x, y) \phi(y) \diff y + f(x)$ з неперервним ядром $K(x, y)$ при умові (\ref{eq:1.14}) має єдиний розв’язок $\phi$ в класі неперервних функцій $C(\bar G)$ для будь-якого неперервного вільного члена $f$. Цей розв’язок може бути знайдений у вигляді (\ref{eq:1.18}) за допомогою резольвенти (\ref{eq:1.17}).
\end{theorem}

\begin{example}
	Методом послідовних наближень знайти розв’язок інтегрального рівняння \[\phi(x) = x + \lambda \int_0^1 (xt)^2 \phi(t) \diff t.\]
\end{example}
\begin{solution*}
	$M = 1$, $V = 1$. \\

	Побудуємо повторні ядра \[ K_{(1)}(x, t) = x^2t^2, \quad K_2(x, t) = \int_0^1 x^2 z^4 t^2 \diff z = \frac{x^2t^2}{5}, \quad K_{(p)}(x, t) = \frac{1}{5^{p - 2}} \int_0^1 x^2 z^4 t^2 \diff z = \frac{x^2t^2}{5^{p - 1}}. \]
	
	Резольвента має вигляд \[\mathcal{R}(x, t, \lambda) = x^2 t^2 \left(1 + \frac{\lambda}{5} + \frac{\lambda^2}{5^2} + \ldots + \frac{\lambda^p}{5^p} + \ldots \right) = \frac{5x^2t^2}{5 - \lambda}, \quad |\lambda| < 5. \]

	Розв’язок інтегрального рівняння має вигляд: \[ \phi(x) + x + \Int_0^1 \dfrac{5x^2t^3}{5 - \lambda} \diff t = x + \dfrac{5x^2}{4(5 - \lambda)}. \]
\end{solution*}