\documentclass[a4paper, 12pt]{article}
\usepackage[utf8]{inputenc}
\usepackage[english, ukrainian]{babel}

\usepackage{amsmath, amssymb}
\usepackage{multicol}
\usepackage{graphicx}
\usepackage{float}

\allowdisplaybreaks
\setlength\parindent{0pt}
\numberwithin{equation}{subsection}

\usepackage{hyperref}
\hypersetup{unicode=true,colorlinks=true,linktoc=all,linkcolor=red}

\numberwithin{equation}{subsection}

\renewcommand{\bf}[1]{\textbf{#1}}
\renewcommand{\it}[1]{\textit{#1}}
\newcommand{\bb}[1]{\mathbb{#1}}
\renewcommand{\cal}[1]{\mathcal{#1}}

\renewcommand{\epsilon}{\varepsilon}
\renewcommand{\phi}{\varphi}

\DeclareMathOperator{\diam}{diam}
\DeclareMathOperator{\rang}{rang}
\DeclareMathOperator{\const}{const}

\newenvironment{system}{%
  \begin{equation}%
    \left\{%
      \begin{aligned}%
}{%
      \end{aligned}%
    \right.%
  \end{equation}%
}
\newenvironment{system*}{%
  \begin{equation*}%
    \left\{%
      \begin{aligned}%
}{%
      \end{aligned}%
    \right.%
  \end{equation*}%
}

\makeatletter
\newcommand*{\relrelbarsep}{.386ex}
\newcommand*{\relrelbar}{%
  \mathrel{%
    \mathpalette\@relrelbar\relrelbarsep%
  }%
}
\newcommand*{\@relrelbar}[2]{%
  \raise#2\hbox to 0pt{$\m@th#1\relbar$\hss}%
  \lower#2\hbox{$\m@th#1\relbar$}%
}
\providecommand*{\rightrightarrowsfill@}{%
  \arrowfill@\relrelbar\relrelbar\rightrightarrows%
}
\providecommand*{\leftleftarrowsfill@}{%
  \arrowfill@\leftleftarrows\relrelbar\relrelbar%
}
\providecommand*{\xrightrightarrows}[2][]{%
  \ext@arrow 0359\rightrightarrowsfill@{#1}{#2}%
}
\providecommand*{\xleftleftarrows}[2][]{%
  \ext@arrow 3095\leftleftarrowsfill@{#1}{#2}%
}
\makeatother

\newcommand{\NN}{\mathbb{N}}
\newcommand{\ZZ}{\mathbb{Z}}
\newcommand{\QQ}{\mathbb{Q}}
\newcommand{\RR}{\mathbb{R}}
\newcommand{\CC}{\mathbb{C}}

\newcommand{\Max}{\displaystyle\max\limits}
\newcommand{\Sup}{\displaystyle\sup\limits}
\newcommand{\Sum}{\displaystyle\sum\limits}
\newcommand{\Int}{\displaystyle\int\limits}
\newcommand{\Iint}{\displaystyle\iint\limits}
\newcommand{\Lim}{\displaystyle\lim\limits}

\newcommand*\diff{\mathop{}\!\mathrm{d}}

\newcommand*\rfrac[2]{{}^{#1}\!/_{\!#2}}


\title{{\Huge МАТЕМАТИЧНА ФІЗИКА}}
\author{Скибицький Нікіта}
\date{\today}

\usepackage{amsthm}
\usepackage[dvipsnames]{xcolor}
\usepackage{thmtools}
\usepackage[framemethod=TikZ]{mdframed}

\theoremstyle{definition}
\mdfdefinestyle{mdbluebox}{%
	roundcorner = 10pt,
	linewidth=1pt,
	skipabove=12pt,
	innerbottommargin=9pt,
	skipbelow=2pt,
	nobreak=true,
	linecolor=blue,
	backgroundcolor=TealBlue!5,
}
\declaretheoremstyle[
	headfont=\sffamily\bfseries\color{MidnightBlue},
	mdframed={style=mdbluebox},
	headpunct={\\[3pt]},
	postheadspace={0pt}
]{thmbluebox}

\mdfdefinestyle{mdredbox}{%
	linewidth=0.5pt,
	skipabove=12pt,
	frametitleaboveskip=5pt,
	frametitlebelowskip=0pt,
	skipbelow=2pt,
	frametitlefont=\bfseries,
	innertopmargin=4pt,
	innerbottommargin=8pt,
	nobreak=true,
	linecolor=RawSienna,
	backgroundcolor=Salmon!5,
}
\declaretheoremstyle[
	headfont=\bfseries\color{RawSienna},
	mdframed={style=mdredbox},
	headpunct={\\[3pt]},
	postheadspace={0pt},
]{thmredbox}

\declaretheorem[style=thmbluebox,name=Теорема,numberwithin=subsubsection]{theorem}
\declaretheorem[style=thmbluebox,name=Лема,numberwithin=subsubsection]{lemma}
\declaretheorem[style=thmbluebox,name=Твердження,numberwithin=subsubsection]{proposition}
\declaretheorem[style=thmbluebox,name=Принцип,numberwithin=subsubsection]{th_principle}
\declaretheorem[style=thmbluebox,name=Закон,numberwithin=subsubsection]{law}
\declaretheorem[style=thmbluebox,name=Закон,numbered=no]{law*}
\declaretheorem[style=thmbluebox,name=Формула,numberwithin=subsubsection]{th_formula}
\declaretheorem[style=thmbluebox,name=Рівняння,numberwithin=subsubsection]{th_equation}
\declaretheorem[style=thmbluebox,name=Умова,numberwithin=subsubsection]{th_condition}
\declaretheorem[style=thmbluebox,name=Наслідок,numberwithin=subsubsection]{corollary}

\declaretheorem[style=thmredbox,name=Приклад,numberwithin=subsubsection]{example}
\declaretheorem[style=thmredbox,name=Приклади,sibling=example]{examples}

\declaretheorem[style=thmredbox,name=Властивість,numberwithin=subsubsection]{property}
\declaretheorem[style=thmredbox,name=Властивості,sibling=property]{properties}

\mdfdefinestyle{mdgreenbox}{%
	skipabove=8pt,
	linewidth=2pt,
	rightline=false,
	leftline=true,
	topline=false,
	bottomline=false,
	linecolor=ForestGreen,
	backgroundcolor=ForestGreen!5,
}
\declaretheoremstyle[
	headfont=\bfseries\sffamily\color{ForestGreen!70!black},
	bodyfont=\normalfont,
	spaceabove=2pt,
	spacebelow=1pt,
	mdframed={style=mdgreenbox},
	headpunct={ --- },
]{thmgreenbox}

\mdfdefinestyle{mdblackbox}{%
	skipabove=8pt,
	linewidth=3pt,
	rightline=false,
	leftline=true,
	topline=false,
	bottomline=false,
	linecolor=black,
	backgroundcolor=RedViolet!5!gray!5,
}
\declaretheoremstyle[
	headfont=\bfseries,
	bodyfont=\normalfont\small,
	spaceabove=0pt,
	spacebelow=0pt,
	mdframed={style=mdblackbox}
]{thmblackbox}

\declaretheorem[name=Вправа,numberwithin=subsubsection,style=thmblackbox]{exercise}
\declaretheorem[name=Зауваження,numberwithin=subsubsection,style=thmgreenbox]{remark}
\declaretheorem[name=Визначення,numberwithin=subsubsection,style=thmblackbox]{definition}

\newtheorem{problem}{Задача}[subsection]
\newtheorem{sproblem}[problem]{Задача}
\newtheorem{dproblem}[problem]{Задача}
\renewcommand{\thesproblem}{\theproblem$^{\star}$}
\renewcommand{\thedproblem}{\theproblem$^{\dagger}$}
\newcommand{\listhack}{$\empty$\vspace{-2em}} 

\theoremstyle{remark}
\newtheorem*{solution}{Розв'язок}


\begin{document}

\tableofcontents

% lecture 1, not compilable		
\section{Вступ}

\subsection{Предмет і методи математичної фізики}

Сучасні технології дослідження реального світу доволі інтенсивно використовують методи математичного моделювання, зокрема ці методи широко використовуються тоді, коли дослідження реального (фізичного) об’єкту є неможливими, або надто дорогими. Вже традиційними стали моделювання властивостей таких фізичних об’єктів:
\begin{itemize}
	\item температурні поля і теплові потоки;
	\item електричні, магнітні та електромагнітні поля;
	\item концентрація речовини в розчинах, розплавах або сумішах;
	\item напруження і деформації в пружних твердих тілах;
	\item параметри рідини або газу, який рухається (обтікає) деяке тіло;
	\item перенос різних субстанцій потоками рідин або газу та інші.
\end{itemize}

Характерною особливістю усіх математичних моделей, що описують перелічені та багато інших процесів є те, що параметри, які представляють інтерес для дослідника є функціями точки простору $\bf{x} = (x_1, x_2, x_3)$ та часу $t$, а самі співвідношення з яких ці характеристики обчислюються є диференціальними рівняннями в частинних похідних зі спеціальними додатковими умовами (крайовими умовами), які дозволяють виділяти однозначний розв’язок. \\

Таким чином можна сказати, що основними об’єктами дослідження предмету математична фізика є крайові задачі для рівнянь в частинних похідних, які моделюють певні фізичні процеси. \\

Процес дослідження реального об’єкту фізичного світу можна представити за наступною схемою:
\begin{enumerate}
	\item Побудова математичної моделі реального процесу у вигляді диференціального рівняння або системи диференціальних рівнянь в частинних похідних, доповнення диференціального рівняння в частинних похідних граничними умовами.
	\item Дослідження властивостей сформульованої крайової задачі з точки зору її коректності. Коректність постановки задачі передбачає виконання наступних умов:
	\begin{itemize}
		\item Розв’язок крайової задачі існує;
		\item Розв’язок єдиний;
		\item Розв’язок неперервним чином залежить від вхідних даних задачі.
	\end{itemize}
	\item Знаходження розв’язку крайової задачі: точного для найбільш простих задач, або наближеного для переважної більшості задач.
\end{enumerate}

Треба відмітити, що усі перелічені пункти дослідження окрім побудови наближених методів знаходження розв’язків відносяться до предмету дисципліни Математична фізика. \\

Для дослідження задач математичної фізики використовуються математичний апарат наступних розділів математики:
\begin{itemize}
	\item математичний аналіз;
	\item лінійна алгебра;
	\item диференціальні рівняння;
	\item теорія функцій комплексної змінної;
	\item функціональний аналіз;
\end{itemize}

При побудові математичних моделей використовуються знання з елементарної фізики. \\

Наведемо приклад доволі простої і в той же час цілком реальної математичної моделі розповсюдження тепла в стрижні. \\

Нехай ми маємо однорідний стрижень з теплоізольованою боковою поверхнею і наступними фізичними параметрами:
\begin{itemize}
	\item $\rho$ -- густина матеріалу;
	\item $S$ -- площа поперечного перерізу;
	\item $k$ -- коефіцієнт теплопровідності;
	\item $c$ -- коефіцієнт теплоємності;
	\item $L$ -- довжина стрижня.
\end{itemize}

Позначимо $u(x, t)$ -- температуру стрижня в точці $x$ в момент часу $t$, $u_0(x)$ -- температуру стрижня у точці $x$ в початковий момент часу $t = 0$. \\

Припустимо, що на лівому кінці стрижня температура змінюється за заданим законом $\phi(t)$, а правий кінець стрижня теплоізольований. \\

В таких припущеннях математична модель може бути записана у вигляді наступної граничної задачі:
\begin{equation}
	\label{eq:1.1.1}
	c \rho \cdot \dfrac{\partial u(x, t)}{\partial t} = k \cdot \dfrac{\partial^2 u(x, t)}{\partial x^2}, \quad 0 < x < L, \quad t > 0
\end{equation}
\begin{equation}
	\label{eq:1.1.2}
	u(0, u) = \phi(t), \quad  \dfrac{\partial u(L, t)}{\partial x} = 0
\end{equation}
\begin{equation}
	\label{eq:1.1.3}
	u(x, 0) = u_0(t)
\end{equation}

Математична модель містить диференціальне рівняння (\ref{eq:1.1.1}), яке виконується для вказаних значень аргументу, граничні умови на кінцях стрижня (\ref{eq:1.1.2}) та початкову умови (\ref{eq:1.1.3}).

\section{Інтегральні рівняння}

\subsubsection{Основні поняття}

Інтегральні рівняння -- рівняння, що містять невідому функцію під знаком інтегралу. \\

Багато задач математичної фізики зводяться до лінійних інтегральних рівнянь виду:

\begin{equation}
	\label{eq:2.1}
	\phi(x) = \lambda \Int_G K(x, y) \phi(y) \diff y + f(x)
\end{equation}
-- інтегральне рівняння Фредгольма II роду.

\begin{equation}
	\label{eq:2.2}
	\Int_G K(x, y) \phi(y) \diff y = f(x)
\end{equation}
-- інтегральне рівняння Фредгольма I роду. \\

$K(x, y)$ -- ядро інтегрального рівняння, $K(x, y) \in C\left(\bar G \times \bar G\right)$, $f(x)$ -- вільний член інтегрального рівняння, $f(x) \in C\left(\bar G\right)$, $\lambda$ -- комплексний параметр, $\lambda \in \CC$ (відомий або невідомий), $G$ -- область інтегрування, $G \subseteq \RR^n$, $\bar G$ -- замкнена та обмежена. \\

Інтегральне рівняння \eqref{eq:2.1} при $f(x) \equiv 0$ називається однорідним інтегральним рівнянням Фредгольма II роду
\begin{equation}
	\label{eq:2.3}
	\phi(x) = \lambda \Int_G K(x, y) \phi(y) \diff y.
\end{equation}

$\bf{K}$ -- інтегральний оператор: $(\bf{K} \phi)(x)$. Будемо записувати інтегральні рівняння \eqref{eq:2.1}, \eqref{eq:2.2} та \eqref{eq:2.3} скорочено в операторній формі:
\begin{align}
	\label{eq:2.4}
	\phi &= \lambda \bf{K} \phi + f, \\
	\label{eq:2.5}
	\bf{K} \phi &= f, \\
	\label{eq:2.6}
	\phi &= \lambda \bf{K} \phi.
\end{align}

\begin{equation}
	\label{eq:2.7}
	K^*(x, y) = \bar K(y, x)
\end{equation}
-- спряжене (союзне) ядро. Інтегральне рівняння
\begin{equation}
	\label{eq:2.8}
	\psi(x) = \bar \lambda \Int_G K^*(x, y) \psi(y) \diff y + g(x)
\end{equation}
називається спряженим (союзним) до інтегрального рівняння \eqref{eq:2.1}. Операторна форма:
\begin{align}
	\label{eq:2.9}
	\psi &= \bar \lambda \bf{K}^* \psi + g, \\
	\label{eq:2.10}
	\psi &= \bar \lambda \bf{K}^* \psi.
\end{align}

\begin{definition*}
	Комплексні значення $\lambda$, при яких однорідне інтегральне рівняння Фредгольма \eqref{eq:2.3} має нетривіальні розв’язки, називаються характеристичними числами ядра $K(x, y)$. Розв’язки, які відповідають власним числам, називаються власними функціями. Кількість лінійно-незалежних власних функцій називається кратністю характеристичного числа.
\end{definition*}

\subsection{Метод послідовних наближень}

\subsubsection{Метод послідовних наближень для неперервного ядра}

Нагадаємо означення норм в банаховому просторі неперервних функцій $C(\bar G)$ та гільбертовому просторі інтегрованих з квадратом функцій $L_2(G)$ та означення скалярного добутку в просторі $L_2(G)$:
\begin{align} 
	\label{eq:2.1.1}
	\|f\|_{C(\bar G)} &= \Max_{x \in \bar G} |f(x)|, \\
	\label{eq:2.1.2}
	\|f\|_{L_2(G)} &= \left( \Int_G |f(x)|^2 \diff x \right)^{1/2}, \\
	\label{eq:2.1.3}
	(f, g)_{L_2(G)} &= \Int_G f(x) \bar g(x) \diff x.
\end{align}

\begin{lemma} 
	Інтегральний оператор $\bf{K}$ з неперервним ядром $K(x, y)$ петворює множини функцій $C(\bar G) \xrightarrow{\bf{K}} C(\bar G)$, $L_2(G) \xrightarrow{\bf{K}} L_2(G)$, $L_2(G) \xrightarrow{\bf{K}} C(\bar G)$ обмежений та мають місце нерівності:
	\begin{align}
		\label{eq:2.1.4}
		\| \bf{K} \phi \|_{C(G)} &\le M V \| \phi \|_{C(G)}, \\
		\label{eq:2.1.5}
		\| \bf{K} \phi \|_{L_2(G)} &\le M V \| \phi \|_{L_2(G)}, \\
		\label{eq:2.1.6}
		\| \bf{K} \phi \|_{C(G)} &\le M \sqrt{V} \| \phi \|_{L_2(G)},
	\end{align}
	де
	\begin{align}
		\label{eq:2.1.7}
		M &= \Max_{x, y \in G \times G} |K(x, y)|, \\
		\label{eq:2.1.8}
		V &= \Int_G \diff y.
	\end{align}
\end{lemma}

\begin{proof}
	Нехай $\phi \in L_2(G)$. Тоді $\phi$ -- абсолютно інтегрована функція на $G$ і, оскільки ядро $K(x, y)$ неперервне на $G \times G$, функція $(\bf{K}\phi)(x)$ неперервна на $G$. Тому оператор $\bf{K}$ переводить $L_2(G)$ в $C(\bar G)$ і, з врахуванням нерівності Коші-Буняковського, обмежений. Доведемо нерівності:
	\begin{enumerate}
		\item \eqref{eq:2.1.4}:
		\begin{multline*}
			\| \bf{K} \phi \|_{C(\bar G)} = \Max_{x \in \bar G} \left| \Int_G K(x, y) \phi(y) \diff y \right| \le \Max_{x \in \bar G} \Int_G \left( |K(x, y)| \cdot |\phi(y)| \right) \diff y \le \\
			\le \Max_{x \in \bar G} \left( \Max_{y \in \bar G} |K(x, y)| \cdot \Max_{y \in \bar G} |\phi(y)| \cdot \Int_G \diff y \right) \le \\
			\le \Max_{x, y \in \bar G \times \bar G} |K(x, y)| \cdot \Max_{y \in \bar G} |\phi(y)| \cdot \Int_G \diff y = M V \|\phi\|_{C(\bar G)}.
		\end{multline*}
		\item \eqref{eq:2.1.5}:
		\begin{multline*}
			\left( \| \bf{K} \phi \|_{L_2(G)} \right)^2 = \Int_G \left| \Int_G K(x, y) \phi(y) \diff y \right|^2 \diff x \le \\
			\le \Int_G \left| \Max_{y \in \bar G} |K(x, y)| \cdot \Int_G \phi(y) \diff y \right|^2 \diff x \le \\
			\le \left( \Max_{x, y \in \bar G \times \bar G} |K(x, y)| \right)^2 \cdot \left| \Int_G \phi(y) \diff y \right|^2 \cdot \Int_G \diff x \le (M \| \phi\|_{L_2(G)} V)^2
		\end{multline*}
		\item \eqref{eq:2.1.6}:
		\begin{multline*}
			\| \bf{K} \phi \|_{C(\bar G)} = \Max_{x \in \bar G} |(\bf{K} \phi) (x)| = \Max_{x \in \bar G} \left| \Int_G K(x, y) \phi(y) \diff y \right| \le \\
			\le \Max_{x \in \bar G} \sqrt{\Int_G |K(x, y)|^2 \diff y} \cdot \sqrt{\Int_G |\phi(y)|^2 \diff y} \le M \sqrt{V} \|\phi\|_{L_2(G)}.
		\end{multline*}
	\end{enumerate}
\end{proof}

Розв’язок інтегрального рівняння другого роду \eqref{eq:2.4} будемо шукати методом послідовних наближень:
\begin{equation}
	\label{eq:2.1.9}
	\phi_0 = f, \quad \phi_1 = \lambda \bf{K} \phi_0 + f, \quad \phi_2 = \lambda \bf{K} \phi_1 + f, \quad \ldots, \quad \phi_{n + 1} = \lambda \bf{K} \phi_n + f
\end{equation}
\begin{equation}
	\label{eq:2.1.10}
	\phi_{n + 1} = \Sum_{i = 0}^{n + 1} \lambda^i \bf{K}^i f, \quad \bf{K}^{i + 1} = \bf{K} (\bf{K}^i)
\end{equation}
\begin{equation}
	\label{eq:2.1.11}
	\phi_\infty = \Lim_{n \to \infty} \phi_{n} = \Sum_{i = 0}^\infty \lambda^i \bf{K}^i f,
\end{equation}
ряд Неймана. Дослідимо збіжність ряду Неймана \eqref{eq:2.1.11}
\begin{multline}
	\label{eq:2.1.12}
	\left\| \Sum_{i = 0}^\infty \lambda^i \bf{K}^i f \right\|_{C(\bar G)} \le \Sum_{i = 0}^\infty |\lambda^i| \cdot \| \bf{K}^i f \|_{C(\bar G)} \le \\
	\le \Sum_{i = 0}^\infty |\lambda^i| \cdot (MV)^i \cdot \| f \|_{C(\bar G)} = \dfrac{\|f\|_{C(\bar G)}}{1 - |\lambda| MV}.
\end{multline}
Справдві, $\| \bf{K} \phi\|_{C(\bar G)} \le MV \|\phi\|_{C(\bar G)}$, тому $\|\bf{K}^2\phi\|_{C(\bar G)} \le (MV)^2 \|\phi\|_{C(\bar G)}$ і, взагалі кажучи, $\|\bf{K}^i\phi\|_{C(\bar G)} \le (MV)^i \|\phi\|_{C(\bar G)}$. \\

Отже, ряд Неймана збігається рівномірно при 
\begin{equation}
	\label{eq:2.1.13}
	|\lambda| < \dfrac{1}{MV},
\end{equation} 
умова збіжності методу послідовних наближень. \\

Покажемо, що при виконанні умови \eqref{eq:2.1.13} інтегральне рівняння \eqref{eq:2.1} має єдиний розв’язок. Дійсно припустимо, що їх два:

\begin{equation*}
	\begin{matrix}
		\phi^{(1)} = \lambda \bf{K} \phi^{(1)} + f \\
		\phi^{(2)} = \lambda \bf{K} \phi^{(2)} + f
	\end{matrix}
	\implies
	\begin{matrix}
		\phi^{(0)} = \phi^{(1)} - \phi^{(2)}  \\
		\phi^{(0)} = \lambda \bf{K} \phi^{(0)}
	\end{matrix}
\end{equation*}

Обчислимо норму Чебишева: 
\begin{multline*} 
	|\lambda| \cdot \|\bf{K} \phi^{(0)}\|_{C(\bar G)} = \| \phi^{(0)} \|_{C(\bar G)} \Rightarrow \\
	\Rightarrow \| \phi^{(0)} \|_{C(\bar G)} \le |\lambda| \cdot MV \cdot \|\phi^{(0)}\|_{C(\bar G)} \Rightarrow \\
	\Rightarrow (1 - |\lambda| \cdot MV) \cdot \|\phi^{(0)}\|_{C(\bar G)} \le 0.
\end{multline*}

Звідси маємо, що $\|\phi^{(0)}\|_{C(\bar G)} = 0$. Таким чином доведена теорема

\begin{theorem}[Про існування розв’язку інтегрального рівняння Фредгольма з неперервним
ядром для малих значень параметру]
	Будь-яке інтегральне рівняння Фредгольма другого роду \eqref{eq:2.1} з неперервним ядром $K(x, y)$ при умові \eqref{eq:2.1.13} має єдиний розв’язок $\phi$ в класі неперервних функцій $C(\bar G)$ для будь-якого неперервного вільного члена $f$. Цей роз\-в’я\-зок може бути знайдений у вигляді ряду Неймана \eqref{eq:2.1.11}.
\end{theorem}

\subsubsection{Повторні ядра}

$\forall f, g \in \bar G$ має місце рівність 
\begin{equation}
	\label{eq:2.1.14}
	(\bf{K}f,g)_{L_2(G)} = (f, \bf{K}^*g)_{L_2(G)}
\end{equation}
Дійсно, якщо $f, g \in L_2(G)$, то за лемою 1 $\bf{K}f, \bf{K}^*g \in L_2(G)$ тому
\begin{multline*}
	(\bf{K}f, g) = \Int_G (\bf{K}f)\bar g \diff x = \Int_G \left( \Int_G K(x, y) f(y) \diff y\right) \bar g(x) \diff x = \\
	= \Int_G f(y) \left( \Int_G K(x, y) \bar g(x) \diff x\right) \diff y = \Int_G f(y) \cdot (\bf{K}^* g)(y) \diff y = (f, \bf{K}^*g).
\end{multline*}

\begin{lemma}
	Якщо $\bf{K}_1$, $\bf{K}_2$ -- інтегральні оператори з неперервними ядрами $K_1(x, y)$, $K_2(x, y)$ відповідно, то оператор $\bf{K}_3 = \bf{K}_2 \bf{K}_1$ також інтегральний оператор з неперервним ядром
	\begin{equation}
		\label{eq:2.1.15}
		K_3(x, z) = \int_G K_2(x, y) K_1(y, z) \diff y.
	\end{equation}
	При цьому справедлива формула: $(\bf{K}_2\bf{K}_1)^* = \bf{K}_1^* \bf{K}_2^*$.
\end{lemma}
\begin{proof}
	Нехай $K_1(x, y)$, $K_2(x, y)$ -- ядра інтегральних операторів $\bf{K}_1$, $\bf{K}_2$. Розглянемо $\bf{K}_3 = \bf{K}_2 \bf{K}_1$:
	\begin{multline*}
		(\bf{K}_3 f)(x) = (\bf{K}_2\bf{K}_1f)(x) = \Int_G K_2(x, y) \left( \Int_G K_1(y, z) f(z) \diff z \right) \diff y = \\
		= \Int_G \left( \Int_G K_2(x, y) K_1(y, z) \diff y\right) f(z) \diff z = \Int_G K_3(x, z) f(z) \diff z.
	\end{multline*}
	Тобто \eqref{eq:2.1.15} -- ядро оператора $\bf{K}_2\bf{K}_1$. \\

	З рівності \eqref{eq:2.1.14} для всіх $f, g \in L_2(G)$ отримуємо $(f, \bf{K}_3^*g - \bf{K}_1^* \bf{K}_2^* g) = 0$, звідки випливає, що $\bf{K}_3^* = \bf{K}_1^* \bf{K}_2^*$.
\end{proof}

Із доведеної леми випливає, що оператори $\bf{K}^n = \bf{K} (\bf{K}^{n - 1}) = (\bf{K}^{n - 1})\bf{K}$ -- інтегральні та їх ядра $K_{(n)}(x, y)$ -- неперервні та задовольняють рекурентним співвідношенням:
\begin{equation}
	\label{eq:2.1.16}
	K_{(1)}(x, y) = K(x, y), \quad \ldots, \quad K_{(n)}(x, y) = \Int_G K(x, \xi) K_{(n - 1)}(\xi, y) \diff \xi
\end{equation}
-- повторні (ітеровані) ядра. Операторна форма:
\begin{equation}
	\label{eq:2.1.17}
	\bf{K}f = \Int_G K(x, y) f(y) \diff y, \quad \ldots, \quad \bf{K}^n f = \Int_G K_{(n)}(x, y)f(y) \diff y.
\end{equation}

\subsubsection{Резольвента інтегрального оператора}

Пригадаємо представлення розв’язку інтегрального рівняння \ref{eq:2.1} у вигляді ряду Неймана \eqref{eq:2.1.11}. Виконаємо перетворення
\begin{multline*}
	\phi(x) = f(x) + \lambda \Sum_{i = 1}^\infty \lambda^{i - 1} (\bf{K}^i f) x = f(x) + \Sum_{i = 1}^\infty \lambda^{i - 1} K_{(i)} (x, y) f(y) \diff y = \\
	= f(x) + \lambda \Int_G \left( \Sum_{i = 1}^\infty \lambda^{i - 1} K_{(i)} (x, y) \right) f(y) \diff y = f(x) + \lambda \Int_G \mathcal{R}(x, y, \lambda) f(y) \diff y,
\end{multline*}
при $|\lambda| < \frac{1}{MV}$, де
\begin{equation}
	\label{eq:2.1.18}
	\mathcal{R}(x, y, \lambda) = \Sum_{i = 1}^\infty \lambda^{i - 1} K_{(i)} (x, y)
\end{equation}
-- резольвента інтегрального оператора. Операторна форма запису розв’язку рівняння Фредгольма через резольвенту ядра має вигляд:
\begin{equation}
	\label{eq:2.1.19}
	\phi = f + \lambda \bf{R} f
\end{equation}

Мають місце операторні рівності:
\begin{equation}
	\label{eq:2.1.20}
	\phi = (E + \lambda \bf{R})f, \quad (E - \lambda \bf{K})\phi = f,\quad \phi = (E - \lambda \bf{K})^{-1}f.
\end{equation}

Таким чином маємо
\begin{equation}
	\label{eq:2.1.21}
	E + \lambda \bf{R} = (E - \lambda \bf{K})^{-1}, \quad |\lambda| < \dfrac{1}{MV}.
\end{equation}

Зважуючи на формулу \eqref{eq:2.1.19} має місце теорема
\begin{theorem}[Про існування розв’язку інтегрального рівняння Фредгольма з неперервним
ядром для малих значенням параметру]
	Будь-яке інтегральне рівняння Фредгольма другого роду \eqref{eq:2.1} з неперервним ядром $K(x, y)$ при умові \eqref{eq:2.1.13} має єдиний розв’язок $\phi$ в класі неперервних функцій $C(\bar G)$ для будь-якого неперервного вільного члена $f$. Цей розв’язок може бути знайдений у вигляді \eqref{eq:2.1.18} за допомогою резольвенти \eqref{eq:2.1.18}.
\end{theorem}

\begin{example}
	Методом послідовних наближень знайти розв’язок інтегрального рівняння \[\phi(x) = x + \lambda \Int_0^1 (xt)^2 \phi(t) \diff t.\]
\end{example}
\begin{solution*}
	$M = 1$, $V = 1$. \\

	Побудуємо повторні ядра 
	\begin{align*} 
		K_{(1)}(x, t) &= x^2t^2, \\
		K_2(x, t) &= \Int_0^1 x^2 z^4 t^2 \diff z = \dfrac{x^2t^2}{5}, \\ 
		K_{(p)}(x, t) &= \dfrac{1}{5^{p - 2}} \Int_0^1 x^2 z^4 t^2 \diff z = \dfrac{x^2t^2}{5^{p - 1}}.
	\end{align*}
	
	Резольвента має вигляд \[\mathcal{R}(x, t, \lambda) = x^2 t^2 \left(1 + \frac{\lambda}{5} + \frac{\lambda^2}{5^2} + \ldots + \frac{\lambda^p}{5^p} + \ldots \right) = \frac{5x^2t^2}{5 - \lambda}, \quad |\lambda| < 5. \]

	Розв’язок інтегрального рівняння має вигляд: \[ \phi(x) + x + \Int_0^1 \dfrac{5x^2t^3}{5 - \lambda} \diff t = x + \dfrac{5x^2}{4(5 - \lambda)}. \]
\end{solution*}

% \setcounter{section}{2}
% \setcounter{subsection}{1}
% \setcounter{subsubsection}{4}

\subsubsection{Метод послідовних наближень для інтегральних рівнянь з полярним ядром}

\begin{definition}[полярного ядра]
	Ядро $K(x, y)$ називається \it{полярним}, якщо воно представляється у вигляді:
	\begin{equation}
		K(x, y) = \dfrac{A(x, y)}{|x - y|^\alpha}
	\end{equation}
	де $A \in C\left(\overline G \times \overline G\right)$, $|x - y| = \left( \sum_{i = 1}^n (x_i - y_i)^2 \right)^{1/2}$, $\alpha < n$ ($n$ --- розмірність евклідового простору).
\end{definition}

\begin{definition}[слабо полярного ядра]
	Ядро називається \it{слабо полярним}, якщо $\alpha < n / 2$.
\end{definition}

Метод послідовних наближень для інтегральних рівнянь з неперервним ядром мав вигляд: 
\begin{multline}
	\phi(x) = \lambda \Int_G K(x, y) \phi(x, y) \diff y + f(x), \\
	\phi_0 = f, \quad \phi_1 = f + \lambda \bf{K} \phi_0, \quad \ldots, \quad \phi_{n + 1} = f + \lambda \bf{K} \phi_n.
\end{multline}

Оцінки, що застосовувались для неперервних ядер не працюють для полярних ядер, тому що максимум полярного ядра рівний нескінченності (ядро необмежене в рівномірній метриці), отже, сформулюємо лему аналогічну лемі \ref{lemma:2.1.4} для полярних ядер. 

\begin{lemma}
	Інтегральний оператор $\bf{K}$ з полярним ядром $K(x, y)$ переводить множину функцій $C\left(\overline G\right) \xrightarrow{\bf{K}} C \left(\overline G\right)$ і при цьому має місце оцінка: 
	\begin{equation}
		\| \bf{K} \phi \|_{C\left(\overline G\right)} \le N \| \phi\|_{C\left(\overline G\right)},
	\end{equation}
	де 
	\begin{equation}
		N = \Max_{x \in \overline G} \Int_G |K(x, y)| \diff y.
	\end{equation}
\end{lemma}

\begin{proof}
	Спочатку доведемо, що функція $\bf{K}\phi$ неперервна в точці $x_0$. \medskip

	Оцінимо при умові $|x - x_0| < \eta / 2$ вираз:
	\begin{multline}
		\left| \Int_G K(x, y) \phi(y) \diff y - \Int_G K(x_0, y) \phi(y) \diff y \right| = \\
		= \left| \Int_G \dfrac{A(x, y)}{|x - y|^\alpha} \phi(y) \diff y - \Int_G \dfrac{A(x_0, y)}{|x_0 - y|^\alpha} \phi(y) \diff y \right| \le \\
		\le \Int_G \left|\dfrac{A(x, y)}{|x - y|^\alpha} - \dfrac{A(x_0, y)}{|x_0 - y|^\alpha}\right| |\phi(y)| \diff y \le (*)
	\end{multline}
	винесемо $\max \phi(y)$ у вигляді $\|\phi\|_{C\left(\overline G\right)}$, а інтеграл розіб'ємо на два інтеграли: 
	\begin{itemize}
		\item інтеграл по $U(x_0, \eta)$ --- кулі з центром в $x_0$ і радіусом $\eta$; 
		\item інтеграл по залишку $G \setminus U(x_0, \eta)$.
	\end{itemize}

	\begin{multline} 
		(*) \le \|\phi\|_{C\left(\overline G\right)} \left( \Int_{U(x_0, \eta)} \left|\dfrac{A(x, y)}{|x - y|^\alpha} - \dfrac{A(x_0, y)}{|x_0 - y|^\alpha}\right| \diff y\right. + \\
		+ \left.\Int_{G \setminus U(x_0, \eta)} \left|\dfrac{A(x, y)}{|x - y|^\alpha} - \dfrac{A(x_0, y)}{|x_0 - y|^\alpha}\right| \diff y\right)
	\end{multline}
	
	Оцінимо тепер кожний з інтегралів:
	
	\[ \Int_{U(x_0, \eta)} \left|\dfrac{A(x, y)}{|x - y|^\alpha} - \dfrac{A(x_0, y)}{|x_0 - y|^\alpha}\right| \diff y \le A_0 \Int_{U(x_0, \eta)} \left|\dfrac{\diff y}{|x - y|^\alpha} - \dfrac{\diff y}{|x_0 - y|^\alpha}\right|, \]

	де $A_0$ --- $\max$ функції $A(x, y)$ на потрібній множині. \medskip

	Введемо узагальнені сферичні координати з центром у точці $x_0$ в просторі $\RR^n$:
	\begin{equation}
		\begin{aligned} 
			y_1 &= x_{0, 1} + \rho \cos \nu_1 \\
			y_2 &= x_{0, 2} + \rho \sin \nu_1 \cos \nu_2 \\
			\ldots \\
			y_{n - 1} &= x_{0, n - 1} + \rho \sin \nu_1 \cdot \ldots \cdot \cos \nu_{n - 1} \\
			y_n &= x_{0, n} + \rho \sin \nu_1 \cdot \ldots \cdot \sin \nu_{n - 1}
		\end{aligned}
	\end{equation}

	Якобіан переходу має вигляд:
	\begin{equation}
		\dfrac{D(y_1, \ldots, y_n)}{\rho, \nu_1, \ldots, \nu_{n - 1}} = \rho^{n - 1} \Phi(\sin \nu_1, \ldots, \sin \nu_{n - 1}, \cos \nu_1, \ldots, \cos \nu_{n - 1}),
	\end{equation}

	де $0 \le \rho \le \eta, 0 \le \nu_i \le \pi, i = \overline{1, n - 2}, 0 \le \nu_{n - 1} \le 2 \pi$. \medskip

	Отримаємо 
	\begin{equation}
		\Int_{U(x_0, \eta)} \dfrac{\diff y}{|x_0 - y|^\alpha} = \sigma_n \Int_0^\eta \dfrac{\rho^{n - 1} \diff \rho}{\rho^\alpha} = \sigma_n \left.\dfrac{\rho^{n - \alpha}}{n - \alpha}\right|_0^\eta = \dfrac{\sigma_n \eta^{n - \alpha}}{n - \alpha} \le \dfrac{\epsilon}{4},
	\end{equation}
	де $\sigma_n$ --- площа поверхні одиничної сфери в $n$-вимірному просторі $\RR^n$. \medskip

	Оскільки $|x - x_0| < \eta / 2$, то 
	\begin{equation}
		\Int_{U(x_0, \eta)} \dfrac{\diff y}{|x - y|^\alpha} \le \Int_{U(x_0, 3\eta/2)} \dfrac{\diff y}{|x_0 - y|^\alpha} \le \dfrac{\sigma_n}{n - \alpha} \left(\dfrac{3\eta}{2}\right)^{n - \alpha} \le \dfrac{\epsilon}{4}.
	\end{equation}

	Оскільки 
	\begin{equation}
		\frac{A(x, y)}{|x - y|^\alpha} \in C\left(\overline{U (x_0, \eta/2)}\times\overline{G \setminus U (x_0, \eta)}\right),
	\end{equation}
	то
	\begin{equation}
		\Int_{G \setminus U(x_0, \eta)} \left|\dfrac{A(x, y)}{|x - y|^\alpha} - \dfrac{A(x_0, y)}{|x_0 - y|^\alpha}\right| \diff y \le \dfrac{\epsilon}{2}.
	\end{equation}

	Таким чином ми довели, що 
	\begin{equation}
		\left| \int_G K(x, y) \phi(y) \diff y - \int_G K(x_0, y) \phi(y) \diff y \right| \le \epsilon,
	\end{equation}
	тобто функція $\bf{K}\phi$ неперервна в точці $x_0$. \medskip

	Доведемо оцінку $\|\bf{K}\phi\|_{C\left(\overline G\right)} \le N \|\phi\|_{C\left(\overline G\right)}$, де $N = \max_{x \in \overline G} \int_G |K(x, y) \diff y|$:
	\begin{multline}
		\left| \Int_G K(x, y) \phi(y) \diff y \right| \le \Int_G |K(x, y)| |\phi(y)| \diff y \le \|\phi\|_{C(\overline G)} \Int_G |K(x, y)| \le \\
		\le |\phi\|_{C(\overline G)} \Max_{x \in \overline G} \Int_G |K(x, y)| \diff y = N \|\phi\|_{C(\overline G)},
	\end{multline}
	отже $\| \bf{K}\phi \|_{C(\overline G)} \le N \|\phi\|_{C(\overline G)}$. \medskip

	Покажемо скінченність $N = \max_{x \in \overline G} \int_G |K(x, y) \diff y|$. Розглянемо 
	\begin{equation}
		\Int_G |K(x,y)| \diff y \le A_0 \Int_G \dfrac{\diff y}{|x - y|^\alpha} \le (*).
	\end{equation}

	Для будь-якої точки $x$, існує радіус (рівний максимальному діаметру області $G$) такий, що в кулю з цим радіусом попадає будь-яка точка $y$: $D = \diam G$.

	\begin{equation}
		(*) \le A_0 \Int_{U(x, D)} \dfrac{\diff y}{|x - y|^\alpha} = A_0 \dfrac{\sigma_n}{n - \alpha} D^{n - \alpha}.
	\end{equation}
\end{proof}

\begin{theorem}[про існування розв'язку інтегрального рівняння Фредгольма з полярним ядром
для малих значень параметру]
	Інтегральне рівняння Фредгольма 2-го роду з полярним ядром $K(x, y)$ має єдиний розв'язок в класі неперервних функцій для будь-якого неперервного вільного члена $f$ при умові
	\begin{equation}
		|\lambda| < \dfrac{1}{N}
	\end{equation}
	і цей розв'язок може бути представлений рядом Неймана, який збігається абсолютно і рівномірно.
\end{theorem}

\begin{proof}
	Сформулюємо умову збіжності ряду Неймана. \medskip

	$\phi = \sum_{i = 0}^\infty \lambda^i \bf{K}^i f$, отже 
	\begin{equation}
		\|\phi\|_{C\left(\overline G\right)} \le \sum_{i = 1}^\infty |\lambda|^i \cdot N^i \cdot \|f\|_{C(\overline G)}.
	\end{equation}

	Останній ряд --- геометрична прогресія і збігається при умові $|\lambda| < 1 / N$.
\end{proof}

\begin{lemma}
	Нехай маємо два полярних ядра 
	\begin{equation}
		K_i(x, y) = \frac{A_i(x, y)}{|x - y|^\alpha_i}, \quad \alpha_i < n, \quad i = 1, 2,
	\end{equation}
	а область $G$ обмежена, тоді ядро 
	\begin{equation}
		K_3(x, y) = \int_G K_2(x, \xi) K_1(\xi, y) \diff \xi
	\end{equation}
	також полярне, причому має місце співвідношення:
	\begin{equation}
		K_3(x, y) = \begin{cases}
			\dfrac{A_3(x, y)}{|x - y|^{\alpha_1 + \alpha_2 - n}}, & \alpha_1 + \alpha_2 - n > 0, \\
			A_3(x, y) |\ln|x - y|| + B_3(x, y), & \alpha_1 + \alpha_2 - n = 0, \\
			A_3(x, y), & \alpha_1 + \alpha_2 - n < 0,
		\end{cases}
	\end{equation}
	де $A_3, B_3$ неперервні функції.
\end{lemma}

\begin{proof}
	З леми 4 випливає, що всі повторні ядра $K_{(p)}(x, y)$, полярного ядра $K(x, y)$ задовольняють оцінкам: \medskip

	$\alpha_1 = \alpha_2 = \alpha$.

	\begin{equation}
		\begin{aligned}
			K_{(2)}(x, y) &= \begin{cases}
				\dfrac{A_2(x, y)}{|x - y|^{2\alpha - n}}, & 2\alpha - n > 0, \\
				A_2(x, y) |\ln|x - y|| + B_2(x, y), & 2\alpha - n = 0, \\
				A_2(x, y), & 2\alpha - n < 0,
			\end{cases} \\
			K_{(3)}(x, y) &= \begin{cases}
				\dfrac{A_3(x, y)}{|x - y|^{3\alpha - 2n}}, & 3\alpha - 2n > 0, \\
				A_3(x, y) |\ln|x - y|| + B_3(x, y), & 3\alpha - 2n = 0, \\
				A_3(x, y), & 3\alpha - 2n < 0,
			\end{cases} \\
			K_{(p)}(x, y) &= \begin{cases}
				\dfrac{A_p(x, y)}{|x - y|^{p\alpha - (p-1)n}}, & p\alpha - (p - 1)n > 0, \\
				A_p(x, y) |\ln|x - y|| + B_p(x, y), & p\alpha - (p - 1)n = 0, \\
				A_p(x, y), & p\alpha - (p - 1)n < 0.
			\end{cases}
		\end{aligned}
	\end{equation}

	Легко бачити, що для $\forall \alpha, n$ існує $p_0$ таке, що починаючи з нього всі повторні ядра є неперервні:
	\begin{multline}
		p \alpha - (p - 1) n < 0 \implies (n - \alpha) p > n \implies \\
		\implies p > \dfrac{n}{n - \alpha} \implies p_0 = \left[ \dfrac{n}{n - \alpha} \right] + 1.
	\end{multline}

	Звідси маємо, що резольвента $\cal{R}(x, y, \lambda)$ полярного ядра $K(x, y)$ складається з двох частин полярної складової $\cal{R}_1(x, y, \lambda)$ і неперервної складової $\cal{R}_2(x, y, \lambda)$:
	\begin{multline}
		\cal{R}(x, y, \lambda) = \cal{R}_1(x, y, \lambda) + \cal{R}_2(x, y, \lambda) = \\
		= \Sum_{i = 1}^\infty \lambda^{i - 1} K_{(i)}(x, y) = \Sum_{i = 1}^{p_0 - 1} \lambda^{i - 1} K_{(i)}(x, y) + \Sum_{i = p_0}^\infty \lambda^{i - 1} K_{(i)}(x, y).
	\end{multline}

	Для доведення збіжності резольвенти, потрібно дослідити збіжність нескінченного ряду $\cal{R}_2(x, y, \lambda)$. Він сходиться рівномірно при $x, y \in \overline G$, $|\lambda| \le \frac{1}{N} - \epsilon$, $\forall \epsilon > 0$, визначаючи неперервну функцію $\cal{R}$ при $x, y \in \overline G$, $|\lambda| < 1 / N$ і аналітичну по $\lambda$ в крузі 
	\begin{equation}
		|\lambda| < \dfrac{1}{N}.
	\end{equation}

	Дійсно
	\begin{equation}
		\cal{R}_2(x, y, \lambda) = \Sum_{i = p_0}^\infty \lambda^{i - 1} K_{(i)}(x, y).
	\end{equation}

	У свою чергу, 
	\begin{equation}
		|\lambda^{p_0 + s - 1} K_{(p_0 + s)}(x, y)| \le |\lambda|^{p_0 + s - 1} M_{p_0} N^s,
	\end{equation}
	де
	\begin{equation}
		M_{p_0} = \max_{(x, y) \in \overline G \times \overline G} |K_{p_0}(x, y)|.
	\end{equation}

	Таким чином ряд $\cal{R}_2(x, y, \lambda)$ мажорується геометричною прогресією, яка збігається при умові $|\lambda| < 1 / N$.
\end{proof}

\subsection{Теореми Фредгольма}

\subsubsection{Інтегральні рівняння з виродженим ядром}

\begin{definition}[виродженого ядра]
	Неперервне ядро $K(x, y)$ називається \it{виродженим}, якщо представляється у вигляді
	\begin{equation}
		K(x, y) = \Sum_{i = 1}^N f_i(x) g_i(y),
	\end{equation}
	де $\{ f_i \}_{i = \overline{1, N}}, \{ g_i \}_{i = \overline{1, N}} \subset C\left(\overline G\right)$, і $\{ f_i \}_{i = \overline{1, N}}$ та $\{ g_i \}_{i = \overline{1, N}}$ --- лінійно незалежні системи функцій.
\end{definition}

\begin{definition}[інтегрального рівняння Фредгольма з виродженим ядром]
	Розглянемо інтегральні рівняння Фредгольма з виродженим ядром 
	\begin{equation}
		\phi(x) = \lambda \Int_G K(x, u) \phi(y) \diff y + f(x).
	\end{equation}
\end{definition}

Підставимо вигляд виродженого ядра і отримаємо:
\begin{multline}
	\phi(x) = \lambda \Int_G \Sum_{i = 1}^N f_i(x) g_i(y) \phi(y) \diff y + f(x) = \\
	= \lambda \Sum_{i = 1}^N f_i(x) \Int_G g_i(y) \phi(y) \diff y + f(x) = f(x) + \lambda \Sum_{i = 1}^N c_i f_i(x),
\end{multline}
де 
\begin{equation}
	c_j = \Int_G g_j(y) \phi(y) \diff y.
\end{equation}

Підставимо значення $\phi(x)$ з \eqref{eq:2.2.3}:

\begin{multline}
	c_j = \Int_G g_j(y) \phi(y) \diff y = \Int_G g_j(y) \left( f(y) + \lambda \Sum_{i = 1}^N c_i f_i(y) \right) \diff y = \\
	= \Int_G g_j(y) f(y) \diff y + \lambda \Sum_{i = 1}^N c_i \Int_G g_j(y) f_i(y) \diff y.
\end{multline}

В результаті отримаємо систему лінійних алгебраїчних рівнянь

\begin{equation}
	c_j = \lambda \Sum_{i = 1}^N \alpha_{j i} c_i + a_j, \quad j = \overline{1, N},
\end{equation}
де 
\begin{equation}
	\alpha_{ji} = \Int_G g_j(y) f_i(y) \diff y, \quad a_j = \Int_G g_j(y) f(y) \diff y.
\end{equation}

Отримаємо систему рівнянь для спряженого ядра:

\begin{equation}
	K^\star (x, y) = \Sum_{i = 1}^N \overline f_i(y) \overline g_i(x),
\end{equation}

\begin{equation}
	\psi(x) = \overline \lambda \Int_G K^\star (x, y) \psi(y) \diff y + g(x),
\end{equation}

\begin{equation}
	\psi(x) = \overline \lambda \Sum_{i = 1}^N \overline g_i(x) \Int_G \overline f_i(y) \psi(y) \diff y + g(x) = \overline \lambda \Sum_{i = 1}^N d_i \overline g_i(x) + g(x),
\end{equation}

\begin{equation}
	d_i = \Int_G \overline f_i(y) \psi(y) \diff y, \quad d_j = \Int_G \overline f_j(y) \left( g(y) + \overline \lambda \Sum_{i = 1}^N d_i \overline g_i(y) \right) \diff y,
\end{equation}

\begin{equation}
	d_j = \overline \lambda \Sum_{i = 1}^N \beta_{ji}d_i + b_j, \quad i = \overline{1, N}, 
\end{equation}

де

\begin{equation} 
	\beta_{ji} = \Int_G \overline f_j(y) \overline g_i(y) \diff y, \quad b_j = \Int_G \overline f_j(y) g(y) \diff y,
\end{equation}

причому виконується умова

\begin{equation}
	\beta_{ji} = \overline \alpha_{ij}.
\end{equation}

Тобто отримуємо системи лінійних рівнянь які в матричному вигляді запишуться так:

\begin{align}
	\vec c &= \lambda A \vec c + \vec a,
	\vec d &= \lambda A^\star  \vec d + \vec b,
\end{align}
з матрицями $E - \lambda A$ та $E - \overline \lambda A^\star $ відповідно і визначником $D(\lambda) = |E - \lambda A| = |E - \overline \lambda A^\star |$. \medskip

Дослідимо питання існування та єдиності розв'язку цих СЛАР. \medskip

Нехай $D(\lambda) \ne 0$, $\rang |E - \lambda A| = \rang |E - \overline \lambda A^\star | = N$, тоді ці СЛАР мають єдиний розв'язок для будь-яких векторів $\vec a$ і $\vec b$ відповідно, а тому інтегральні рівняння Фредгольма з полярними ядрами (як пряме так і спряжене) мають єдині розв'язки при будь-яких $f$ та $g$ відповідно, і ці розв'язки записуються за формулами \eqref{eq:2.2.3}, \eqref{eq:2.2.9}. \medskip

Нехай $D(\lambda) = 0$, $\rang |E - \lambda A| = \rang |E - \overline \lambda A^\star | = q < N$, тоді однорідні СЛАР 
\begin{equation}
	\label{eq:2.2.16}
	\vec c = \lambda A \vec c,
\end{equation}
та
\begin{equation}
	\label{eq:2.2.17}
	\vec d = \lambda A^\star  \vec d,
\end{equation}
мають $N - q$ лінійно незалежних розв'язків $\vec c_s$, $\vec d_s$, $s = \overline{1, N - q}$, де вектор визначається формулою $\vec c_s = (c_{s1}, \ldots, c_{sN})$, $\vec d_s = (d_{s1}, \ldots, d_{sN})$, таким чином відповідні однорідні інтегральні рівняння Фредгольма рівнянням \eqref{eq:2.2.2}, \eqref{eq:2.2.8} мають $N - q$ лінійно незалежних розв'язків які записуються за такими формулами:
\begin{equation}
	\label{eq:2.2.18}
	\phi_s(x) = \lambda \Sum_{i = 1}^N c_{si} f_i(x), \quad s = \overline{1, N - q},
\end{equation}

\begin{equation}
	\label{eq:2.2.19}
	\psi_s(x) = \overline \lambda \Sum_{i = 1}^N d_{si} \overline g_i(x), \quad s = \overline{1, N - q},
\end{equation}
$\phi_s(x)$, $\psi_s(x)$ --- власні функції, а число $N - q$ --- кратність характеристичного числа $\lambda$ та $\overline \lambda$. Кожна з систем функцій $\phi_s$, $\psi_s$, $s = \overline{1, N - q}$ лінійно незалежна, оскільки лінійно незалежними є системи функцій $f_i$ та $g_i$ і лінійно незалежні вектори $\vec c_s$ і $\vec d_s$, $s = \overline{1, N - q}$. \medskip

Нагадаємо одне з формулювань теореми Кронекера-Капеллі. Для існування розв'язку системи лінійних алгебраїчних рівнянь необхідно і достатньо що би вільний член рівняння був ортогональним всім розв'язкам спряженого однорідного рівняння. \medskip

Для нашого випадку цю умову можна записати у вигляді
\begin{equation}
	\label{eq:2.2.20}
	(\vec a, \vec d_s) = \Sum_{i = 1}^N a_i \overline d_{si} = 0, \quad \forall s = \overline{1, N - q}.
\end{equation}

Покажемо, що для виконання умови $(\vec a, \vec d_s) = 0$, $s = \overline{1, N - q}$ необхідно і достатньо, щоб вільний член інтегрального рівняння Фредгольма \eqref{eq:2.2.2} був ортогональним розв'язкам спряженого однорідного рівняння тобто 
\begin{equation}
	\label{eq:2.2.21}
	(f, \psi_s) = 0, \quad s = \overline{1, N - q}
\end{equation}
Дійсно, з \eqref{eq:2.2.19} та \eqref{eq:2.2.4} маємо:
\begin{multline*} (f, \psi_s) = \Int_G f(x) \overline \psi_s (x) \diff x = \lambda \Sum_{i = 1}^N \overline d_{si} \Int_G f(x) g_i(x) \diff x = \\ = \lambda \Sum_{i = 1}^N a_i \overline d_{si} = \lambda (\vec a, \vec d_s) = 0, \end{multline*} для всіх $s = \overline{1, N - q}$. \medskip

В цьому випадку розв'язок СЛАР не єдиний, і визначається з точністю до довільного розв'язку однорідної системи рівнянь, тобто з точністю до лінійної оболонки натягнутої на систему власних векторів характеристичного числа $\lambda$:
\begin{equation}
	\label{eq:2.2.22}
	\vec c = \vec c_0 + \Sum_{i = 1}^{N - q} \gamma_i \vec c_i,
\end{equation}
де $\gamma_i$ --- довільні константи, $\vec c_0$ --- будь-який розв'язок неоднорідної системи рівнянь $\vec c_0 = \lambda A \vec c_0 + \vec a$, тоді розв'язок інтегрального рівняння можна записати у вигляді:
\begin{equation}
	\label{eq:2.2.23}
	\phi(x) = \phi_0(x) + \Sum_{i = 1}^{N - q} \gamma_i \phi_i(x),
\end{equation}
де $\phi_0$ --- довільний розв'язок неоднорідного рівняння $\phi_0 = \lambda \bf{K} \phi_0 + f$. \medskip

Отже доведені такі теореми:

\begin{theorem}[Перша теорема Фредгольма для вироджених ядер]
	Якщо $D(\lambda) \ne 0$, то інтегральне рівняння \eqref{eq:2.2.2} та спряжене до нього \eqref{eq:2.2.8} мають єдині розв'язки для довільних вільних членів $f$ та $g$ з класу неперервних функцій.
\end{theorem}

\begin{theorem}[Друга теорема Фредгольма для вироджених ядер]
	Якщо $D(\lambda) = 0$, то однорідне рівняння Фредгольма другого роду \eqref{eq:2.2.2} ($f \equiv 0$) і спряжене до нього \eqref{eq:2.2.8} ($g \equiv 0$) мають однакову кількість лінійно незалежних розв'язків рівну $N - q$, де $q = \rang(E - \lambda A)$.
\end{theorem}

\begin{theorem}[Третя теорема Фредгольма для вироджених ядер]
	Якщо $D(\lambda) = 0$, то для існування розв'язків рівняння \eqref{eq:2.2.2} необхідно і достатньо, щоб вільний член $f$ був ортогональним усім розв'язкам однорідного спряженого рівняння \eqref{eq:2.2.21}. При виконанні цієї умови розв'язок існує та не єдиний і визначається з точністю до лінійної оболонки натягнутої на систему власних функцій характеристичного числа $\lambda$.
\end{theorem}

\begin{corollary}
	Характеристичні числа виродженого ядра $K(x, y)$ співпадають з коренями поліному $D(\lambda) = 0$, а їх кількість не перевищує $N$.
\end{corollary}

\subsubsection{Задачі}

\begin{example}
	Знайти розв'язок інтегрального рівняння \[ \phi(x) = \lambda \Int_0^\pi \sin(x - y) \phi(y) \diff y + \cos(x). \]
\end{example}

\begin{solution}
	\[ \phi(x) = \lambda \sin(x) \Int_0^\pi \cos(y) \phi(y) \diff y - \lambda \cos(x) \Int_0^\pi \sin(y) \phi(y) \diff y + \cos(x). \]
	Позначимо \[ c_1 = \Int_0^\pi \cos(y) \phi(y) \diff y, \quad c_2 = \Int_0^\pi \sin(y) \phi(y) \diff y. \]
	\[ \phi(x) = \lambda (c_1 \sin(x) - c_2 \cos(x)) + \cos(x). \]

	Підставляючи останню рівність в попередні отримаємо систем рівнянь:
	\begin{system*}
		c_1 &= \Int_0^\pi \cos(y) (\lambda c_1 \sin(y) - \lambda c_2 \cos(y) + \cos(y)) \diff y, \\
		c_2 &= \Int_0^\pi \sin(y) (\lambda c_1 \sin(y) - \lambda c_2 \cos(y) + \cos(y)) \diff y.
	\end{system*}
	Після обчислення інтегралів:
	\begin{system*}
		c_1 + \frac{\lambda \pi}{2} c_2 &= \frac{\pi}{2}, \\
		- \frac{\lambda\pi}{2} c_1 + c_2 &= 0.
	\end{system*}
	Визначник цієї системи
	\[ D(\lambda) = \begin{vmatrix} 1 & \frac{\lambda\pi}{2} \\ -\frac{\lambda\pi}{2} & 1 \end{vmatrix} = 1 + \left( \frac{\lambda \pi}{2} \right)^2 \ne 0. \]
	За правилом Крамера маємо
	\[ c_1 = \dfrac{2 \pi}{4 + (\lambda \pi)^2}, \quad c_2 = \dfrac{\lambda \pi^2}{4 + (\lambda \pi)^2}. \]
	Таким чином розв'язок має вигляд
	\[ \phi(x) = \dfrac{2 \lambda \pi \sin(x) + 4 \cos (x)}{4 + (\lambda \pi)^2}. \]
\end{solution}

\end{document}