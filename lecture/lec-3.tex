%{Лекція 3}
\subsubsection{Теореми Фредгольма для інтегральних рівнянь з неперервним ядром}

Будемо розглядати рівняння:
\begin{equation}
	\label{eq:2.2}
	\phi(x) = \lambda \Int_G K(x, y) \phi(y) \diff y + f(x),
\end{equation}
\begin{equation}
	\label{eq:2.2'}
	\psi(x) = \bar \lambda \Int_G K^*(x, y) \psi(y) \diff y + g(x),
\end{equation}

Ядро $K(x, y) \in C(\bar G \times \bar G)$, отже його можна наблизити поліномом (Теорема Вєйєрштраса). \\

Тобто, для будь-якого $\epsilon > 0$ існує 
\begin{equation}
	P_N(x, y) = \Sum_{|\alpha + \beta| \le N} a_{\alpha\beta}x^\alpha y^\beta.
\end{equation}
де $\alpha = (\alpha_1, \alpha_2, \ldots, \alpha_n)$, $x^\alpha = x_1^{\alpha_1} \cdot x_2^{\alpha_2} \cdot \ldots \cdot x_n^{\alpha_n}$, такий що $|K(x, y) - P_N(x, y)| < \epsilon$, $x, y \in \bar G \times \bar G$, тобто 
\begin{equation}
	\label{eq:2.15}
	K(x, y) = P_N(x, y) + Q_N(x, y),
\end{equation}
де $P_N(x,y)$ -- вироджене ядро (поліном), $|Q_N(x, y)| < \epsilon$, $x, y \in \bar G \times \bar G$. \\

Виходячи з (\ref{eq:2.15}), інтегральне рівняння Фредгольма приймає вигляд 
\begin{equation}
	\label{eq:2.16}
	\phi = \lambda \bf{P}_N \phi + \lambda \bf{Q}_N \phi + f,
\end{equation}
де $\bf{P}_N$ та $\bf{Q}_N$ інтегральні оператори з ядрами $P_N(x, y)$ та $Q_N(x, y)$ відповідно ($\bf{P}_N + \bf{Q}_N = \bf{K}$). \\

Для спряженого рівняння маємо:
\begin{equation}
	\label{eq:2.15'}
	K^*(x, y) = P_N^*(x, y) + Q_N^*(x, y),
\end{equation}
\begin{equation}
	\label{eq:2.16'}
	\psi = \bar \lambda \bf{P}_N^* \psi + \bar \lambda \bf{Q}_N^* \psi + g.
\end{equation}

Покажемо, що в класі $C(G)$ рівняння (\ref{eq:2.16}), (\ref{eq:2.16'}) еквівалентні рівнянням з виродженим ядром. Введемо нову функцію 
\begin{equation}
	\label{eq:2.17}
	\Phi = \phi - \lambda \bf{Q}_N \phi
\end{equation}

З (\ref{eq:2.16}) випливає що $\Phi = \lambda \bf{P}_N + f$, з (\ref{eq:1.19}) випливає що $\forall \lambda$ такого що $|\lambda| < \frac{1}{\epsilon V}$: $(E - \lambda \bf{Q}_N)^{-1} = (E + \lambda \bf{R}_N)$, де $\bf{R}_N$ -- резольвента для $\bf{Q}_N$. Отже $\phi = (E - \lambda \bf{Q}_N)^{-1} \Phi = (E + \lambda \bf{R}_N) \Phi$. \\

Отже, рівняння (\ref{eq:2.2}) перетворюється на 
\begin{equation}
	\label{eq:2.18}
	\Phi = \lambda \bf{P}_N (E + \lambda \bf{R}_N) \Phi.
\end{equation}

Для спряженого рівняння (\ref{eq:2.2'}) маємо:
\[ \psi = \bar \lambda (E + \bar \lambda \bf{R}_N^*) \bf{P}_N^* \psi + (E + \bar \lambda \bf{R}_N^*) g. \]

Позначимо $g_1 = (E + \bar \lambda \bf{R}_N^*) g$. Маємо:
\begin{equation}
	\label{eq:2.18'}
	\psi = \bar \lambda (E + \bar \lambda \bf{R}_N^*) \bf{P}_N^* \psi + g_1.
\end{equation}

Оскільки $(\bf{P}_N \bf{R}_N)^* = \bf{R}_N^* \bf{P}_N^*$, то рівняння (\ref{eq:2.18}) та (\ref{eq:2.18'}) спряжені. \\

Позначимо 
\begin{equation}
	\label{eq:2.19}
	\bf{T}_N = \bf{P}_N (E + \lambda \bf{R}_N),
\end{equation}
\begin{equation}
	\label{eq:2.19'}
	\bf{T}_N^* = (E + \bar \lambda \bf{R}_N^*) \bf{P}_N^*.
\end{equation}

Тоді рівняння Фредгольма з неперервним ядром можна записати у вигляді:
\begin{equation}
	\label{eq:2.20}
	\Phi = \lambda \bf{T}_N \Phi + f,
\end{equation}
\begin{equation}
	\label{eq:2.20'}
	\Psi = \bar \lambda \bf{T}_N^* \Psi + g_1,
\end{equation}
де $T_N(x, y, \lambda) = P_N(x, y) + \lambda \int_G P_N(x, \xi) R_N(\xi, y, \lambda) \diff \xi$ -- вироджене, оскільки є сумою двох вироджених, поліному $P_N(x, y)$, та інтегрального доданку. Покажемо що другий доданок в $T_N$ -- вироджений. Дійсно:
\[ \Int_G \Sum_{|\alpha + \beta| \le N} a_{\alpha \beta} x^\alpha \xi^\beta R_N(\xi, y) \diff \xi = \Sum_{|\alpha + \beta| \le N} a_{\alpha \beta} x^\alpha \Int_G \xi^\beta R_N(\xi, y) \diff \xi. \]

\subsubsection{Альтернатива Фредгольма}

Сукупність теорем Фредгольма для інтегральних рівнянь з неперервним ядром називається альтернативою Фредгольма.

\begin{theorem}[Перша теорема Фредгольма для неперервних ядер]
	Якщо інтегральне рівняння (\ref{eq:2.2}) з неперервним ядром $K(x, y)$ має розв'язок $\forall f \in C(\bar G)$ то і спряжене рівняння (\ref{eq:2.2'}) має розв'язок для $\forall g \in C(\bar G)$ і ці роз'язки єдині.
\end{theorem}
\begin{theorem}[Друга теорема Фредгольма для непевних ядер]
	Якщо інтегральне рівняння (\ref{eq:2.2}) має розв'язки не для будь-якого вільного члена $f$, то однорідні рівняння $\phi = \lambda \bf{K} \phi$ $(*)$ та $\psi = \bar \lambda \bf{K}^* \psi$ $(**)$ мають однакову скінчену кількість лінійно-незалежних розв'язків.
\end{theorem}
\begin{theorem}[Третя теорема Фредгольма для неперервних ядер]
	Якщо інтегральне рівняння (\ref{eq:2.2}) має розв'язок не для $\forall$ вільного члена $f$, то для існування розв'язку інтегрального рівняння (\ref{eq:2.2}) в $C(\bar G)$ необхідно і достатньо, щоб вільний член $f$ був ортогональним всім розв'язкам спряженого однорідного рівняння $(**)$. Розв'язок не єдиний і визначається з точністю до лінійної оболонки, натягнутої на систему власних функцій оператора $\bf{K}$.
\end{theorem}

Доведення теорем: Для будь-якого фіксованого значення $\lambda$ виберемо $\epsilon$, таке щоби $|\lambda| < \frac{1}{\epsilon V}$.

\begin{proof}[Теореми 1]
	Нехай (\ref{eq:2.2}) має розв'язок в $C(\bar G)$ для $\forall$ вільного члена $f$, тоді еквівалентне йому рівняння (\ref{eq:2.20}): $\Phi = \lambda \bf{T}_N \Phi + F$ має такі ж властивості і згідно з першою теоремою Фредгольма для вироджених ядер $D(\lambda) \ne 0$, а спряжене до нього рівняння (\ref{eq:2.20'}): $\Psi = \bar \lambda \bf{T}_N^* + g_1$ теж має єдиний розв'язок $\forall$ вільного члена $g_1$, еквівалентне до нього рівняння (\ref{eq:2.2'}) має розв'язок $\forall g$.
\end{proof}
\begin{proof}[Теореми 2]
	Нехай (\ref{eq:2.2}) має розв'язок не $\forall$ вільного члена $f$, тоді, рівняння з виродженим ядром (\ref{eq:2.20}) має таку ж властивість. Згідно з теоремами Фредгольма для вироджених ядер $D(\lambda) = 0$ (для виродженого ядра $\bf{T}_N$). Однорідні рівняння які відповідають (\ref{eq:2.20}) і (\ref{eq:2.20'}) мають однакову скінчену кількість лінійно-незалежних розв'язків, еквівалентні до них однорідні рівняння $(*)$, $(**)$ теж мають однакову скінчену кількість лінійно незалежних розв'язків.
\end{proof}
\begin{proof}[Теореми 3]
	Нехай неоднорідне рівняння (\ref{eq:2.2}) має розв'язок не для будь-якого вільного члена $f$, тоді еквівалентне рівняння з виродженим ядром (\ref{eq:2.20}) має таку ж властивість, і за третьою теоремою Фредгольма для вироджених ядер $D(\lambda) = 0$ (для виродженого ядра $\bf{T}_N$). Розв'язок (\ref{eq:2.20}) існує тоді і тільки тоді коли $f$ ортогональний до розв'язків спряженого однорідного рівняння до (\ref{eq:2.20'}). Але легко бачити, що вільний член (\ref{eq:2.2}) і (\ref{eq:2.20}) співпадають, так само співпадають розв'язки однорідного рівняння (\ref{eq:2.2'}) і (\ref{eq:2.20'}).
\end{proof}

\begin{theorem}[Четверта теорема Фредгольма]
	Для будь-якого як завгодно великого числа $R > 0$ в крузі $|\lambda| < R$ лежить лише скінчена кількість характеристичних чисел неперервного ядра $K(x, y)$.
\end{theorem}

\subsubsection{Наслідки з теорем Фредгольма}

\begin{corollary}
	З четвертої теореми Фредгольма випливає, що множина характеристичних чисел неперервного ядра не має скінчених граничних точок і не більш ніж злічена $\lim_{n \to \infty} |\lambda_n| = \infty$.
\end{corollary}
\begin{corollary}
	З другої теореми Фредгольма випливає, що кратність кожного характеристичного числа скінчена, їх можна занумерувати у порядку зростання модулів $|\lambda_1| \le |\lambda_2| \le \ldots \le |\lambda_k| \le |\lambda_{k + 1}| \le \ldots$, кожне число зустрічається стільки разів, яка його кратність. Також можна занумерувати послідовність власних функцій ядра $K(x, y)$: $\phi_1$, $\phi_2$, $\ldots$, $\phi_k$, $\phi_{k + 1}$, $\ldots$ і спряженого ядра $K^*(x, y)$: $\psi_1$, $\psi_2$, $\ldots$, $\psi_k$, $\psi_{k + 1}$, $\ldots$.
\end{corollary}
\begin{corollary}
	Власні функції неперервного ядра $K(x, y)$ неперервні в області $G$.
\end{corollary}
\begin{corollary}
	Якщо $\lambda_k \ne \lambda_j$, то $(\phi_k, \psi_j) = 0$.
\end{corollary}

\subsubsection{Теореми Фредгольма для інтегральних рівнянь з полярним ядром}

Розповсюдимо Теореми Фредгольма для інтегральних рівнянь з полярним ядром:
\begin{equation}
	\label{eq:2.21}
	K(x, y) = \dfrac{A(x, y)}{|x - y|^\alpha}, \quad \alpha < n.
\end{equation}

Покажемо що $\forall \epsilon > 0$ існує таке вироджене ядро $P_N(x, y)$ що,
\begin{equation}
	\label{eq:2.22}
	\Max_{x \in \bar G} \Int_G |K(x, y) - P_N(x, y)| \diff y < \epsilon
\end{equation}
\begin{equation}
	\label{eq:2.22'}
	\Max_{x \in \bar G} \Int_G |K^*(x, y) - P_N^*(x, y)| \diff y < \epsilon
\end{equation}

Розглянемо неперервне ядро
\begin{equation}
	\label{eq:2.23}
	L_M(x, y) = \begin{cases}
		K(x, y), & |x - y| \ge 1 / M, \\
		A(x, y) M^\alpha, & |x - y| < 1 / M.
	\end{cases}
\end{equation}
Покажемо, що при достатньо великому $M$ має місце оцінка \[ \Int_G |K(x, y) - L_M(x, y)| \diff y \le \epsilon. \]

Дійсно:
\begin{multline*} 
	\Int_G |K(x, y) - L_M(x, y)| \diff y = \Int_{|x - y| < 1 / M} \left| \dfrac{A(x, y)}{|x - y|^\alpha} - A(x, y) M^\alpha \right| \diff y = \\
	= \Int_{|x - y| < 1 /M} |A(x, y)| \left| \dfrac{1}{|x - y|^\alpha} - M^\alpha \right| \diff y \le A_0 \Int_{|x - y| < 1 /M} \left| \dfrac{1}{|x - y|^\alpha} - M^\alpha \right| \diff y \le \\
	\le A_0 \Int_{|x - y| < 1 /M} \dfrac{\diff y}{|x - y|^\alpha} = A_0 \sigma_n \Int_0^{1 / M} \xi^{n - 1 - \alpha} \diff \xi = \\
	= A_0 \sigma_n \left.\dfrac{\xi^{n - \alpha}}{n - \alpha}\right|_0^{1 / M} = \dfrac{A_0\sigma_n}{(n - \alpha)M^{n - \alpha}} \le \dfrac{\epsilon}{2},
\end{multline*}
де $\sigma_n$ -- площа поверхні одиничної сфери. \\

Завжди можна підібрати вироджене ядро $P_N(x, y)$ таке що \[ |L_M(x, y) - P_N(x, y)| \le \dfrac{\epsilon}{2V}, \] де $V$ -- об'єм області $G$. 

\begin{multline*}
	\Int_G |K(x, y) - P_N(x, y)| \diff y = \Int_G | K(x, y) - L_M(x, y) + L_M(x, y) - P_N(x, y)| \diff y \le \\
	\le \Int_G | K(x, y) - L_M(x, y)| \diff y + \Int_G |L_M(x, y) - P_N(x, y)| \diff y \le \dfrac{\epsilon}{2} + \dfrac{\epsilon}{2V} \Int_G \diff y = \epsilon.
\end{multline*}

Використавши попередню техніку (для неперервного ядра) інтегральне рівняння з полярним ядром зводиться до еквівалентного рівняння з виродженим ядром. Тобто теореми Фредгольма залишаються вірними для інтегральних рівнянь з полярним ядром з тим же самим формулюванням. \\

Теореми Фредгольма залишаються вірними для інтегральних рівнянь з полярним ядром на обмеженій кусково-гладкій поверхні $S$ та контурі $C$:
\[ \phi(x) = \lambda \Int_S K(x, y) \phi(y) \diff y + f(x), \quad \dfrac{A(x, y)}{|x - y|^\alpha}, \quad \alpha < \dim(S). \]

\subsection{Інтегральні рівняння з ермітовим ядром}

Розглядатимемо ядро $K(x, y) \in C(\bar G \times \bar G)$ таке що $K(x, y) = K^*(x, y)$. \\

Неперервне ядро будемо називати ермітовим, якщо виконується
\begin{equation}
	\label{eq:3.1}
	K(x, y) = K^*(x, y)
\end{equation}

Ермітовому ядру відповідає ермітовий оператор тобто $\bf{K} = \bf{K}^*$.

\begin{lemma}
	Для того, щоб лінійний оператор був ермітовим, необхідно і достатньо, щоб для довільної комплексно значної функції $f \in L_2(\bar G)$ білінійна форма $(\bf{K}f, f)$ приймала лише дійсні значення.
\end{lemma}
\begin{lemma}
	Характеристичні числа ермітового оператора дійсні.
\end{lemma}
\begin{definition}
	Множина функцій $M \subset C(\bar G)$ -- компактна в рівномірній метриці, якщо з будь-якої нескінченної множини функцій з $М$ можна виділити рівномірно збіжну підпослідовність.
\end{definition}
\begin{definition}
	Нескінченна множина $M \subset C(\bar G)$ -- рівномірно обмежена, якщо для будь-якого елемента $f \in M$ має місце $\|f\|_{C(\bar G)} \le a$, де $a$ єдина константа для $M$.
\end{definition}
\begin{definition}
	Множина $M \subset C(\bar G)$ -- одностайно неперервна якщо $\forall \epsilon > 0 \exists \delta(\epsilon): \forall f \in M, \forall x_1, x_2: |f(x_1) - f(x_2)| < \epsilon$ як тільки $|x_1 - x_2| < \delta(\epsilon)$.
\end{definition}
\begin{theorem}[Арчела-Асколі, критерій компактності в рівномірній метриці]
	Для того, щоб множина $M \subset C(\bar G)$ була компактною, необхідно і достатньо, щоб вона складалась з рівномірно-обмеженої і одностайно-неперервної множини функцій.
\end{theorem}
\begin{definition}
	Назвемо оператор $\bf{K}$ цілком неперервним з $L_2(G)$ у $C(\bar G)$, якщо він переводить обмежену множину в $L_2(G)$ у компактну множину в $C(\bar G)$ (в рівномірній метриці).
\end{definition}
\begin{lemma}
	Інтегральний оператор $\bf{K}$ з неперервним ядром $K(x, y)$ є цілком неперервний з $L_2(G)$ у $C(\bar G)$.
\end{lemma}
\begin{proof}
	Нехай $f \in M \subset L_2(G)$ та $\forall f \in M: \|f\|_{L_2(G)} \le A$. Але $\|\bf{K} f\|_{C(\bar G)} \le M \sqrt{V} \|f\|_{L_2(G)} \le M \sqrt{V} A$, тобто множина функцій є рівномірно обмеженою. \\

	Покажемо що множина $\{ \bf{K}f(x)\}$ -- одностайно неперервна. \\

	Ядро $K \in C(\bar G \times \bar G)$< а отже є рівномірно неперервним, бо неперервне на компакті, тобто $\forall \epsilon > 0 \exists \delta > 0: \forall x', x'' \in \bar G: \|x' - x''\| < \delta \Rightarrow |(\bf{K}f)(x') - (\bf{K}f)(x'')| \le \epsilon$. Дійсно,
	\begin{multline*}
		|(\bf{K}f)(x') - (\bf{K}f)(x'')| = \left| \Int_G K(x', y) f(y) \diff y - \Int_G K(x'', y) f(y) \diff y \right| \le \\
		\le \Int_G |K(x', y) - K(x'', y)| |f(y)| \diff y \le \dfrac{\epsilon \sqrt{V}}{A \sqrt{V}} \|f\|_{L_2(\bar G)} \le \epsilon.
	\end{multline*}
\end{proof}

\begin{example}
	Знайти характеристичні числа та власні функції інтегрального оператора \[ \phi(x0 = \lambda \Int_0^1 \left( \left( \dfrac{x}{t} \right)^{2/5} + \left( \dfrac{t}{x} \right)^{2/5} \right) \phi(t) \diff t. \]
\end{example}
\begin{solution*}
	\[ \phi(x) = \lambda x^{2 / 5} \Int_0^1 t^{-2/5} \phi(t) \diff t + \lambda x^{-2/5} \Int_0^1 t^{2/5} \phi(t) \diff t. \]
	Позначимо
	\[ c_1 = \Int_0^1 t^{-2/5} \phi(t) \diff t, \quad c_2 = \Int_0^1 t^{2/5} \phi(t) \diff t. \]
	\[ \phi(x) = \lambda c_1 x^{2/5} + \lambda c_2 x^{-2/5}. \]
	\begin{system*}
		c_1 &= \Int_0^1 t^{-2/5} (\lambda c_1 t^{2/5} + \lambda c_2 t^{-2/5}) \diff t, \\
		c_2 &= \Int_0^1 t^{2/5} (\lambda c_1 t^{2/5} + \lambda c_2 t^{-2/5}) \diff t.
	\end{system*}
	\begin{system*}
		(1 - \lambda) c_1 - 5 \lambda c_2 &= 0, \\
		-\frac{5\lambda}{9} c_1 + (1 - \lambda) c_2 &= 0.
	\end{system*}
	\[ D(\lambda) = \begin{vmatrix} 1 - \lambda & - 5 \lambda \\ - \dfrac{5\lambda}{9} & 1 - \lambda \end{vmatrix} = (1 - \lambda)^2 - \dfrac{25\lambda^2}{9} = 0. \]
	\[ \lambda_1 = \dfrac{3}{8}, \quad \lambda_2 = - \dfrac{3}{2}. \]
	З системи однорідних рівнянь при $\lambda = \lambda_1 = \frac{3}{8}$ маємо $c_1 = 3 c_2$. Тоді маємо власну функцію $\phi_1(x) = 3 x^{2 / 5} + x^{-2 / 5}$. \\

	При $\lambda = \lambda_2 = - \frac{3}{2}$ маємо $c_1 = - 3 c_2$. Маємо другу власну функцію $\phi_2(x) = - 3 x^{2 / 5} + x^{-2 / 5}$.
\end{solution*}

\begin{example}
	Знайти розв’язок інтегрального рівняння при всіх значеннях параметрів $\lambda$, $a$, $b$, $c$: \[\phi(x) = \lambda \Int_{-1}^1 (\sqrt[3]{x} + \sqrt[3]{y}) \phi(y) \diff y + ax^2 + bx + c. \]
\end{example}
\begin{solution*}
	Запишемо рівняння у вигляді: \[\phi(x) = \lambda \sqrt[3]{x} \Int_{-1}^1 \phi(y) \diff y + \lambda \Int_{-1}^1 \sqrt[3]{y} \phi(y) \diff y + ax^2 + bx + c. \]
	Введемо позначення: \[ c_1 = \Int_{-1}^1 \phi(y) \diff y, \quad c_2 = \Int_{-1}^1 \sqrt[3]{y} \phi(y) \diff y, \] та запишемо розв’язок у вигляді: $\phi(x) = \lambda \sqrt[3]{x} c_1 + \lambda c_2 + ax^2 + bx + c$. \\

	Для визначення констант отримаємо СЛАР:
	\begin{system*}
		c_1 - 2 \lambda c_2 &= \dfrac{2a}{3} + 2 c, \\
		- \dfrac{6\lambda}{5} c_1 + c_2 &= \dfrac{6b}{7}.
	\end{system*}

	Визначник системи дорівнює $\begin{vmatrix} 1 & - 2 \lambda \\ - \frac{6\lambda}{5} & 1 \end{vmatrix} = 1 - \frac{12\lambda^2}{5}$.

	Характеристичні числа ядра $\lambda_1 = \frac{1}{2} \sqrt{\frac{5}{3}}$, $\lambda_2 = - \frac{1}{2} \sqrt{\frac{5}{3}}$. \\

	Нехай $\lambda \ne \lambda_1$, $\lambda \ne \lambda_2$. Тоді розв'язок існує та єдиний для будь-якого вільного члена і має вигляд \[ \phi(x) = \dfrac{5 \lambda (14 a + 30 \lambda b + 42 c)}{21 (5 - 12 \lambda^2)} \sqrt[3]{x} + \dfrac{28 \lambda a + 84 \lambda c + 30 b}{7(5 - 12 \lambda^2)} + ax^2 + bx + c. \]

	Нехай $\lambda = \lambda_1 = \frac{1}{2} \sqrt{\frac{5}{3}}$. Тоді система рівнянь має вигляд:
	\begin{system*}
		c_1 - \sqrt{\dfrac{5}{3}} c_2 &= \dfrac{2a}{3} + 2 c, \\
		c_1 - \sqrt{\dfrac{5}{3}} c_2 &= - \sqrt{\dfrac{5}{3}} \dfrac{6b}{7}.
	\end{system*}

	Ранги розширеної і основної матриці співпадатимуть якщо має місце рівність $\frac{2a}{3} + 2c = - \sqrt{\frac{5}{3}} \frac{6}{7} b$, $(*)$. \\

	При виконанні цієї умови розв'язок існує $c_2 = c_2$, $c_1 = \sqrt{\frac{5}{3}} c_2 + \frac{2a}{3} + 2c$. \\

	Таким чином розв'язок можна записати \[ \phi(x) = \dfrac{1}{2} \sqrt{\dfrac{5}{3}} \sqrt[3]{x} \left( \sqrt{\dfrac{5}{3}} c_2 + \dfrac{2a}{3} + 2c \right) + \dfrac{1}{2} \sqrt{\dfrac{5}{3}} c_2 + ax^2 + bx + x. \]

	Якщо $\lambda = \lambda_1 = \frac{1}{2} \sqrt{\frac{5}{3}}$, а умова $(*)$ не виконується, то розв'язків не існує. \\

	Нехай $\lambda = \lambda_2 = - \frac{1}{2} \sqrt{\frac{5}{3}}$. Після підстановки цього значення отримаємо СЛАР
	\begin{system*}
		c_1 + \sqrt{\dfrac{5}{3}} c_2 &= \dfrac{2a}{3} + 2 c, \\
		c_1 + \sqrt{\dfrac{5}{3}} c_2 &= \sqrt{\dfrac{5}{3}} \dfrac{6b}{7}.
	\end{system*}

	Остання система має розв'язок при умові, $\frac{2a}{3} + 2c = \sqrt{\frac{5}{3}} \frac{6}{7} b$, $(**)$. \\

	При виконанні умови $(**)$, розв’язок існує $c_2 = c_2$, $c_1 = - \sqrt{\frac{5}{3}} c_2 + \frac{2a}{3} + 2c$. \\

	Розв'язок інтегрального рівняння можна записати: \[ \phi(x) = \dfrac{1}{2} \sqrt{\dfrac{5}{3}} \sqrt[3]{x} \left( -\sqrt{\dfrac{5}{3}} c_2 + \dfrac{2a}{3} + 2c \right) + \dfrac{1}{2} \sqrt{\dfrac{5}{3}} c_2 + ax^2 + bx + c. \]
\end{solution*}