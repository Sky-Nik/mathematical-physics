% cd ..\..\Users\NikitaSkybytskyi\Desktop\c3s2\mathematical-physics
% pdflatex 24.01.lec.tex && pdflatex 24.01.lec.tex && del 24.01.lec.out, 24.01.lec.log, 24.01.lec.aux, 24.01.lec.toc && start l24.01.lecectures.pdf

\documentclass[a4paper, 12pt]{article}
\usepackage[utf8]{inputenc}
\usepackage[english, ukrainian]{babel}

\usepackage{amsmath, amssymb}
\usepackage{multicol}
\usepackage{graphicx}
\usepackage{float}

\allowdisplaybreaks
\setlength\parindent{0pt}
\numberwithin{equation}{subsection}

\usepackage{hyperref}
\hypersetup{unicode=true,colorlinks=true,linktoc=all,linkcolor=red}

\numberwithin{equation}{subsection}

\renewcommand{\bf}[1]{\textbf{#1}}
\renewcommand{\it}[1]{\textit{#1}}
\newcommand{\bb}[1]{\mathbb{#1}}
\renewcommand{\cal}[1]{\mathcal{#1}}

\renewcommand{\epsilon}{\varepsilon}
\renewcommand{\phi}{\varphi}

\DeclareMathOperator{\diam}{diam}
\DeclareMathOperator{\rang}{rang}
\DeclareMathOperator{\const}{const}

\newenvironment{system}{%
  \begin{equation}%
    \left\{%
      \begin{aligned}%
}{%
      \end{aligned}%
    \right.%
  \end{equation}%
}
\newenvironment{system*}{%
  \begin{equation*}%
    \left\{%
      \begin{aligned}%
}{%
      \end{aligned}%
    \right.%
  \end{equation*}%
}

\makeatletter
\newcommand*{\relrelbarsep}{.386ex}
\newcommand*{\relrelbar}{%
  \mathrel{%
    \mathpalette\@relrelbar\relrelbarsep%
  }%
}
\newcommand*{\@relrelbar}[2]{%
  \raise#2\hbox to 0pt{$\m@th#1\relbar$\hss}%
  \lower#2\hbox{$\m@th#1\relbar$}%
}
\providecommand*{\rightrightarrowsfill@}{%
  \arrowfill@\relrelbar\relrelbar\rightrightarrows%
}
\providecommand*{\leftleftarrowsfill@}{%
  \arrowfill@\leftleftarrows\relrelbar\relrelbar%
}
\providecommand*{\xrightrightarrows}[2][]{%
  \ext@arrow 0359\rightrightarrowsfill@{#1}{#2}%
}
\providecommand*{\xleftleftarrows}[2][]{%
  \ext@arrow 3095\leftleftarrowsfill@{#1}{#2}%
}
\makeatother

\newcommand{\NN}{\mathbb{N}}
\newcommand{\ZZ}{\mathbb{Z}}
\newcommand{\QQ}{\mathbb{Q}}
\newcommand{\RR}{\mathbb{R}}
\newcommand{\CC}{\mathbb{C}}

\newcommand{\Max}{\displaystyle\max\limits}
\newcommand{\Sup}{\displaystyle\sup\limits}
\newcommand{\Sum}{\displaystyle\sum\limits}
\newcommand{\Int}{\displaystyle\int\limits}
\newcommand{\Iint}{\displaystyle\iint\limits}
\newcommand{\Lim}{\displaystyle\lim\limits}

\newcommand*\diff{\mathop{}\!\mathrm{d}}

\newcommand*\rfrac[2]{{}^{#1}\!/_{\!#2}}


\title{Математична фізика::лекції}
\author{Нікіта Скибицький}
\date{\today}

\begin{document}

\maketitle

\tableofcontents

\section{Постановка основних граничних задач для лінійних диференційних рівнянь 2-го порядку, коректність, класичні та узагальнені розв'язки}

Серед множини математичних моделей, які були розглянуті в попередніх параграфах можна виділити найтиповіші математичні моделі, які концентрують в собі головні особливості усіх розглянутих вище. Ці моделі представляють собою граничні задачі для рівнянь трьох типів: еліптичних, параболічних та гіперболічних лінійних рівнянь другого порядку. \\

Розглянемо основний диференціальний оператор другого порядку:
\begin{equation}
	\label{eq:8.1}
	L u = \divergence (p(x) \cdot \grad(u)) - q(x) \cdot u.
\end{equation}

Запишемо основні диференціальні рівняння:
\begin{itemize}
	\item Еліптичне рівняння:
	\begin{equation}
		\label{eq:8.2}
		L u = - F(x), \quad x \in \Omega \subset \RR^n.
	\end{equation}
	\item Параболічне рівняння:
	\begin{equation}
		\label{eq:8.3}
		\rho(x) \cdot \frac{\partial u}{\partial t} = L u + F(x, t), \quad x \in \Omega \subset \RR^n, \quad t > t_0 \in \RR.
	\end{equation}
	\item Гіперболічне рівняння:
	\begin{equation}
		\label{eq:8.4}
		\rho(x) \cdot \frac{\partial^2 u}{\partial t^2} = L u + F(x, t), \quad x \in \Omega \subset \RR^n, \quad t > t_0 \in \RR.
	\end{equation}
\end{itemize}

\subsection{Гранична задача для еліптичного рівняння}

Будемо розділяти внутрішні і зовнішні задачі для еліптичного рівняння, а саме, якщо $x \in \Omega$, то таку задачу будемо називати внутрішньою, якщо $x \in \Omega'$ -- задача зовнішня\footnote{Тут $\Omega'$ -- доповнення до $\Omega$}. \\

В подальшому ми будемо розглядати класичні розв'язки граничних задач. Це означає, що рівняння і усі граничні умови виконуються в кожній точці області або границі. \\

Введемо обмеження на коефіцієнти рівняння $p$ і $q$ та вільний член $F$. Зокрема будемо припускати, що $p > 0$, $p \in C^{(1)}(\Omega)$, $q \ge 0$, $q \in C(\Omega)$, $F(x) \in C(\Omega)$. \\

Позначимо $\partial \Omega = S$ -- поверхню на якій задаються граничні умови загального вигляду
\begin{equation}
	\label{eq:8.5}
	\alpha(x) \cdot \frac{\partial u}{\partial n} + \beta(x) \cdot u|_S = V(x),
\end{equation}
де $\alpha, \beta \ge 0$, $\{\alpha, \beta, V\} \subset C(S)$. З умови \eqref{eq:8.5} можна отримати умови 1, 2, 3 роду зокрема:
\begin{itemize}
	\item Діріхле:
	\begin{equation}
		\label{eq:8.6}
		u|_S = \frac{V(x)}{\beta(x)}.
	\end{equation} 
	\item Неймана:
	\begin{equation}
		\label{eq:8.7}
		\left. \frac{\partial u}{\partial n} \right|_S = \frac{V(x)}{\alpha(x)}.
	\end{equation}
	\item Ньютона:
	\begin{equation}
		\label{eq:8.8}
		\frac{\partial u}{\partial n} + \frac{\beta}{\alpha} \cdot u|_S = \frac{V}{\alpha}.
	\end{equation}
\end{itemize}

Таким чином гранична задача для еліптичного рівняння може бути сформульована наступним чином: знайти функцію $u(x) \in C^{(2)}{\Omega} \cap C^{(1)}(\bar \Omega)$, яка в кожній внутрішній точці області $\Omega$ (для внутрішньої задачі) або $\Omega'$ (для зовнішньої задачі) задовольняє рівняння \eqref{eq:8.2}, а кожній точці границі $S$ виконується одна з граничних умов \eqref{eq:8.6}, \eqref{eq:8.7} або \eqref{eq:8.8}. \\

У випадку зовнішньої граничної задачі в нескінченно віддаленій точці області слід задавати додаткові умови поведінки розв'язку. Такі умови називають умовами \textit{регулярності на нескінченості}. Як правило вони полягають в завданні характеру спадання розв'язку і мають вигляд: 
\begin{equation}
	\label{eq:8.9}
	u(x) = O \left( \frac{1}{|x|^\alpha} \right), \quad |x| \to \infty,
\end{equation}
де $\alpha$ -- деякий параметр задачі.

\subsection{Постановка змішаних задач для рівняння гіперболічного типу Задача Коші для гіперболічного рівняння}

Для постановки граничних задач рівняння гіперболічного типу \eqref{eq:8.4} введемо просторово-часовий циліндр, як область зміни незалежних змінних $x, t$:
\begin{equation}
	\label{eq:8.11}
	Z(\Omega, T) = \Omega \times (0, T].
\end{equation}

Для отримання єдиного розв'язку гіперболічного рівняння, на нижній основі просторово-часового циліндру $Z_0(\Omega, T) = \Omega \times \{t = 0\}$ треба задати початкові умови:
\begin{equation}
	\label{eq:8.12}
	u(x, 0) = u_0(x), \quad x \in \Omega,
\end{equation}
\begin{equation}
	\label{eq:8.13}
	\frac{\partial u(x, 0)}{\partial t} = v_0(x), \quad x \in \Omega.
\end{equation}

На боковій поверхні просторово-часового циліндру $Z_S (\Omega, T) = S \times (0, T]$ треба задати граничні умови одного з трьох основних типів:
\begin{itemize}
	\item Діріхле: 
	\begin{equation}
		\label{eq:8.14}
		u|_S = \phi(x, t).
	\end{equation}
	\item Неймана:
	\begin{equation}
		\label{eq:8.15}
		\left. \frac{\partial u}{\partial n} \right|_S = \phi(x, t).
	\end{equation}
	\item Ньютона:
	\begin{equation}
		\label{eq:8.16}
		\frac{\partial u}{\partial n} + \alpha(x, t) \cdot u|_S = \phi(x, t).
	\end{equation}
\end{itemize}

Таким чином постановка граничної задачі для гіперболічного рівняння має
вигляд: \\

\textit{
	Знайти функцію $u(x, t) \in C^{(2, 2)}{Z(\Omega, T))} \cap C^{(1, 1)} \left( \overline{Z(\Omega, T)} \right)$, яка задовольняє рівнянню \eqref{eq:8.4} для $(x, t) \in Z(\Omega, T)$, початковим умовам \eqref{eq:8.12}, \eqref{eq:8.13} для $(x, t) \in Z_0(\Omega, T)$, і в кожній точці $(x, t) \in Z_S(\Omega, T)$ одній з граничних умов \eqref{eq:8.14}-\eqref{eq:8.16}.
} \\

\textit{
	При цьому відносно вхідних даних будемо робити наступні припущення:
}
\begin{equation}
	\label{eq:8.17}
	p > 0, \quad p \in C^{(1)}(\bar \Omega), \quad q \in C(\bar \Omega), \quad F(x, t) \in C \left( \overline{Z(\Omega, T)} \right).
\end{equation}
\begin{equation}
	\label{eq:8.18}
	u_0, v_0 \in C \left( \overline{Z_0(\Omega, T)} \right), \quad \alpha, \phi \in C \left( \overline{Z_S(\Omega, T)} \right), \quad \alpha \ge 0.
\end{equation}

\subsection{Задача Коші}

У випадку, коли область $\Omega$ має великі розміри і впливом граничних умов можна знехтувати, область $\Omega$ ототожнюється з усім евклідовим простором, тобто $\Omega = \RR^n$. \\

У зв'язку з відсутністю границі, граничні умови не задаються. В цьому випадку гранична задача трансформується в задачу Коші для гіперболічного рівняння яка ставиться наступним чином: \\

\textit{
	Знайти функцію $u(x, t) \in C^{(2, 2)}{Z(\RR^n, T))} \cap C^{(1, 1)} \left( \overline{Z(\RR^n, T)} \right)$, яка задовольняє рівняння \eqref{eq:8.4} для $(x, t) \in Z(\RR^n, T)$, початковим умовам \eqref{eq:8.12}, \eqref{eq:8.13} для $x \in \RR^n$.
}

\subsection{Постановка змішаних задач для рівняння параболічного типу}

При постановці граничної задачі і задачі Коші для рівняння параболічного типу треба враховувати, що по часовій змінній рівняння має перший порядок, що і обумовлює деякі відмінності в постановці граничних задач. \\

Постановка граничної задачі для рівняння параболічного типу \eqref{eq:8.3} має вигляд: \\

\textit{
	Знайти функцію $u(x, t) \in C^{(2, 1)}{Z(\RR^n, T))} \cap C^{(1, 0)} \left( \overline{Z(\RR^n, T)} \right)$, яка задовольняє рівняння \eqref{eq:8.3} для $(x, t) \in Z(\Omega, T)$, початковим умовам \eqref{eq:8.12} для $(x, t) \in Z_0(\Omega, T)$, і в кожній точці $(x, t) \in Z_S(\Omega, T)$ одній з граничних умов \eqref{eq:8.14}-\eqref{eq:8.16}.
}

Аналогічні зміни необхідно запровадити і при постановці задачі Коші для рівняння параболічного типу (записати самостійно постановку задачі Коші для параболічного рівняння \eqref{eq:8.3}.

\subsection{Коректність задач математичної фізики}

Зважуючи на фізичну природу задач математичної фізики, до них застосовуються наступні природні вимоги.
\begin{enumerate}
	\item \textbf{Існування розв'язку}. Задача повинна мати розв'язок (задача яка не має розв'язку не представляє інтересу як математична модель).
	\item \textbf{Єдиність розв'язку}. Не повинно існувати декілька розв'язків задачі.
	\item \textbf{Неперервна залежність від вхідних даних}. Розв'язок задачі повинен мало змінюватись при малій зміні вхідних даних.
\end{enumerate}

Розглянемо математичну модель у вигляді наступної граничної задачі:
\begin{equation}
	\label{eq:8.19}
	\left\{
		\begin{aligned}
			& L u = f, \quad x \in \Omega, \\
			& h u = \phi, \quad x \in S = \partial \Omega.
		\end{aligned}
	\right.
\end{equation}

Формулювання диференціального рівняння і граничних умов ще недостатньо що б гранична задача була сформульована однозначно. Необхідно додатково вказати які аналітичні властивості вимагаються від розв'язку, в якому розумінні задовольняється рівняння і граничні умови. \\

При аналізі граничної задачі виникають наступні питання:
\begin{itemize}
	\item Чи може існувати розв'язок з відповідними властивостями?
	\item Які аналітичні властивості треба вимагати від вхідних даних $f, \phi$, коефіцієнтів диференціального оператора і граничних умов?
	\item Чи існують серед умов задачі такі, що протирічать одне одному?
	\item Які умови треба накладати на гладкість границі $S$?
	\item Чи достатньо сформульованих умов для однозначного знаходження розв'язку?
	\item Чи можна гарантувати, що малі зміни $f, \phi$ приведуть до малих змін розв'язку?
\end{itemize}

Перелічені проблеми зручно розв'язувати звівши граничну задачу до операторного рівняння. Застосувавши загальні методи теорії операторів та операторних рівнянь. \\

В першу чергу виберемо два бананових простора $E$ та $F$. \\

Шуканий розв'язок розглядається як елемент $E$, а сукупність правих частин як елемент $F$. \\

Визначимо оператор $A$, як відображення $u \to \{ Lu, \phi \}$, тоді гранична задача \eqref{eq:8.19} зводиться до операторного рівняння
\begin{equation}
	\label{eq:8.20}
	A u = g, \quad g = \{ f, \phi \}
\end{equation}

Позначимо $R(A)$ та $D(A)$ -- область значень та область визначення оператора $A$. Коректність операторного рівняння визначають для пари просторів $E$ та $F$. \\

В термінах операторного рівняння \eqref{eq:8.20} існування розв'язку означає, що область значень оператора $R(A)$ є не порожня підмножина $F$. \\

Єдиність розв'язку означає, що відображення $A: D(A) \to R(A)$ ін'єктивно і на $R(A)$ визначений обернений оператор $A^{-1}$. \\

Відображення $A: D(A) \to R(A)$ називається \textit{ін'єктивним}, якщо різні елементи множини $D(A)$ переводяться в різні елементи множини $R(A)$. \\

Вимога неперервної залежності розв'язку від правої частини або стійкості граничної задачі зводиться до неперервності або обмеженості оператора $A^{-1}$.

\subsection{Приклад Адамара некоректно поставленої задачі}

Розглянемо рівняння Лапласа
\begin{equation}
	\label{eq:8.21}
	\frac{\partial^2 u}{\partial t^2} = - \frac{\partial^2 u}{\partial x^2}, \quad t > 0, \quad 0 < x < \pi.
\end{equation}

Додаткові умови
\begin{equation}
	\label{eq:8.22}
	u|_{x = 0} = u|_{x = \pi} = 0, \quad u|_{t = 0} = 0, \quad \left. \frac{\partial u}{\partial t} \right|_{t = 0} = \frac{1}{k} \cdot \sin k x.
\end{equation}

Розв'язок $u_k(x, t) = \frac{1}{k^2} \cdot \sinh (k t) \cdot \sin (k x)$, $\forall x: \Lim_{k \to \infty} \frac{1}{k} \cdot \sin(k, x) = 0$, 
\begin{equation*}
	\forall t > 0, \quad \forall x \in (0, \pi): \quad \Lim_{k \to \infty} \frac{1}{k^2} \cdot \sinh (k t) \cdot \sin (k x) = \infty.
\end{equation*}

Для прикладу Адамара порушена умова непевної залежності розв'язку від вхідних даних.

\subsection{Класичний і узагальнений розв'язки}

Класичний розв'язок -- це розв'язок, який задовольняє рівнянню, початковим і граничним умовам в кожній точці, області, або границі. \\

Це означає, що класичний розв'язок повинен мати певну гладкість, яка визначається порядком похідних рівняння і порядком похідних граничних і початкових умов. \\

Розглянемо рівняння $\divergence (p(x) \cdot \grad u) - q(x) \cdot u = - F(x)$, $x \in \Omega$ та однорядні умови 
\begin{equation}
	\label{eq:8.23}
	u|_S = 0.
\end{equation}

Отримаємо інтегральне співвідношення. \\

Розглянемо функцію $v(x)$ таку, що $v|_S = 0$, помножимо рівняння на $v$ та проінтегруємо по $\Omega$:
\begin{equation*}
	\Iiint_\Omega v \left( \divergence (p(x) \cdot \grad u) - q \cdot u \right) \diff \Omega = - \Iiint_\Omega F \cdot v \diff \Omega.
\end{equation*}
Після інтегрування за частинами отримаємо:
\begin{equation*}
	\Iiint_\Omega \left( (p(x) \left( \grad u, \grad v \right) - q \cdot u \cdot v \right) \diff \Omega + \Iint_S p \cdot v \cdot \frac{\partial u}{\partial n} \diff S = - \Iiint_\Omega F \cdot v \diff \Omega.
\end{equation*}

Остаточно, після врахування граничних умов маємо:
\begin{equation}
	\label{eq:8.24}
	\Iiint_\Omega \left( p (\grad u, \grad v) + q \cdot u \cdot v \right) \diff \Omega = \Iiint_\Omega F \cdot v \diff \Omega.
\end{equation}

Інтегральна тотожність має зміст для більш широкого класу функцій ніж той якому належить класичний розв'язок граничної задачі і коефіцієнти рівняння. \\

Якщо $\{u, v\} \subset C^{(2)} (\Omega) \cap C(\bar \Omega)$, $p \in C^{(1)} (\Omega)$, $q \in C(\Omega)$ то з тотожності \eqref{eq:8.24}, обернений ланцюжок перетворень дозволяє отримати граничну задачу \eqref{eq:8.23}. Але \eqref{eq:8.24} має зміст для функцій більш широкого класу, а саме $\{F, u, v, \grad u, \grad v\} \subset L_2(\Omega)$, $p, q$ -- обмежені. Це дозволяє використовувати інтегральну тотожність \eqref{eq:8.24} для визначення узагальненого розв'язку граничної задачі \eqref{eq:8.23}. \\

Для цього введемо множину $N_2 = \{ u | \{u, \grad u\} \subset L_2(\Omega), u|_S = 0\} $. \\

\textit{Узагальненим розв'язком граничної задачі \eqref{eq:8.23}} будемо називати
довільну функцію $u \in N_2$, таку, що $\forall v \in N_2$ має місце інтегральна тотожність
\eqref{eq:8.24}.

\end{document}
