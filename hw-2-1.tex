\begin{problem}[Владіміров, 2.1.1]
	Звести до канонічного вигляду рівняння
	\[ u_{x x} + 2 u_{x y} - 2 u_{x z} + 2 u_{y y} + 6 u_{z z} = 0. \]
\end{problem}

\begin{solution}
    Перш за все перейменуємо змінні для зручності:
    \begin{equation}
        u_{x_1 x_1} + 2 u_{x_1 x_2} - 2 u_{x_1 x_3} + 2 u_{x_2 x_2} + 6 u_{x_3 x_3} = 0. 
    \end{equation}

    Далі запишемо згадану в теоретичній частині квадратичну форму:
    \begin{equation}
        Q(y_1, y_2, y_3) = y_1^2 + 2 y_1 y_2 - 2 y_1 y_3 + 2 y_2^2 + 6 y_3^2.
    \end{equation}

    Виділимо в ній повні квадрати:
    \begin{equation}
        \begin{aligned}
            Q(y_1, y_2, y_3) &= \left( y_1 + y_2 - y_3 \right)^2 + y_2^2 + 2 y_2 y_3 + 5 y_3^2 = \\
            &= \left( y_1 + y_2 - y_3 \right)^2 + \left( y_2 + y_3 \right)^2 + 4 y_3^2 = \\
            &= \left( y_1 + y_2 - y_3 \right)^2 + \left( y_2 + y_3 \right)^2 + \left( 2 y_3 \right)^2.
        \end{aligned}
    \end{equation}
    
    Як бачимо, рівняння має еліптичний тип. \\

    Таким чином, маємо наступну пряму заміну:
    \begin{system}
        \eta_1 &= y_1 + y_2 - y_3, \\
        \eta_2 &= y_2 + y_3, \\
        \eta_3 &= 2 y_3.
    \end{system}

    І відповідну їй обернену заміну:
    \begin{system}
        y_1 &= \eta_1 - \eta_2 + \eta_3, \\
        y_2 &= \eta_2 - \sfrac{\eta_3}{2}, \\
        y_3 &= \sfrac{\eta_3}{2}.
    \end{system}
    
    Тобто матриця з теоретичної частини має наступний вигляд:
    \begin{equation}
        B = 
        \begin{pmatrix}
            1 & -1 & 1 \\
            0 & 1 & - \sfrac12 \\
            0 & 0 & \sfrac12
        \end{pmatrix}.
    \end{equation}

    Знайдемо тепер заміну яка зводить рівняння до канонічного вигляду:
    \begin{equation}
        \begin{pmatrix}
            \xi_1 \\
            \xi_2 \\
            \xi_3
        \end{pmatrix}
        = 
        \begin{pmatrix}
            1 & 0 & 0 \\
            -1 & 1 & 0 \\
            1 & - \sfrac12 & \sfrac12
        \end{pmatrix}
        \cdot 
        \begin{pmatrix}
            x_1 \\
            x_2 \\
            x_3
        \end{pmatrix}.
    \end{equation}

    Або, що те саме,
    \begin{system}
        \xi_1 &= x_1, \\
        \xi_2 &= -x_1 + x_2, \\
        \xi_3 &= x_1 - \sfrac{x_2}{2} + \sfrac{x_3}{2}.
    \end{system}
    
    Оскільки рівняння має еліптичний тип, то канонічною формою буде
    \begin{equation}
        u_{\xi_1 \xi_1} + u_{\xi_2 \xi_2} + u_{\xi_3 \xi_3} = 0.
    \end{equation}
\end{solution}

\begin{problem}[Владіміров, 2.1.2]
	Звести до канонічного вигляду рівняння
	\[ 4 u_{x x} - 4 u_{x y} - 2 u_{y z} + u_y + u_z = 0. \]
\end{problem}

\begin{solution}
    Перш за все перейменуємо змінні для зручності:
    \begin{equation}
        4 u_{x_1 x_1} - 4 u_{x_1 x_2} - 2 u_{x_2 x_3} + u_{x_2} + u_{x_3} = 0. 
    \end{equation}

    Далі запишемо згадану в теоретичній частині квадратичну форму:
    \begin{equation}
        Q(y_1, y_2, y_3) = 4 y_1^2 - 4 y_1 y_2 - 2 y_2 y_3.
    \end{equation}

    Виділимо в ній повні квадрати:
    \begin{equation}
        \begin{aligned}
            Q(y_1, y_2, y_3) &= \left( 2 y_1 - y_2 \right)^2 - y_2^2 - 2 y_2 y_3 = \\
            &= \left( 2 y_1 - y_2 \right)^2 - \left( y_2 + y_3 \right)^2 + y_3^2.
        \end{aligned}
    \end{equation}

    Як бачимо, рівняння має гіперболічний тип. \\

    Таким чином, маємо наступну пряму заміну:
    \begin{system}
        \eta_1 &= 2 y_1 - y_2 , \\
        \eta_2 &= y_2 + y_3, \\
        \eta_3 &= y_3.
    \end{system}

    І відповідну їй обернену заміну:
    \begin{system}
        y_1 &= \sfrac{\eta_1}{2} + \sfrac{\eta_2}{2} - \sfrac{\eta_3}{2}, \\
        y_2 &= \eta_2 - \eta_3, \\
        y_3 &= \eta_3.
    \end{system}
    
    Тобто матриця з теоретичної частини має наступний вигляд:
    \begin{equation}
        B = 
        \begin{pmatrix}
            \sfrac{1}{2} & \sfrac{1}{2} & -\sfrac{1}{2} \\
            0 & 1 & - 1 \\
            0 & 0 & 1
        \end{pmatrix}.
    \end{equation}

    Знайдемо тепер заміну яка зводить рівняння до канонічного вигляду:
    \begin{equation}
        \begin{pmatrix}
            \xi_1 \\
            \xi_2 \\
            \xi_3
        \end{pmatrix}
        = 
        \begin{pmatrix}
            \sfrac{1}{2} & 0 & 0 \\
            \sfrac{1}{2} & 1 & 0 \\
            - \sfrac{1}{2} & - 1 & 1
        \end{pmatrix}
        \cdot 
        \begin{pmatrix}
            x_1 \\
            x_2 \\
            x_3
        \end{pmatrix}.
    \end{equation}
    
    Або, що те саме,
    \begin{system}
        \xi_1 &= \sfrac{x_1}{2}, \\
        \xi_2 &= \sfrac{x_1}{2} + x_2, \\
        \xi_3 &= - \sfrac{x_1}{2} - x_2 + x_3.
    \end{system}

    Оскільки рівняння має гіперболічний тип, то канонічною формою буде
    \begin{equation}
        u_{\xi_1 \xi_1} - u_{\xi_2 \xi_2} + u_{\xi_3 \xi_3} + u_{\xi_2} = 0.
    \end{equation}
\end{solution}

\begin{problem}[Владіміров, 2.1.3]
	Звести до канонічного вигляду рівняння
    \[ u_{x y} - u_{x z} + u_x + u_y - u_z = 0. \]
\end{problem}

\begin{solution}
    Перш за все перейменуємо змінні для зручності:
    \begin{equation}
        u_{x_1 x_2} - u_{x_1 x_3} + u_{x_1} + u_{x_2} - u_{x_3} = 0. 
    \end{equation}

    Далі запишемо згадану в теоретичній частині квадратичну форму:
    \begin{equation}
        Q(y_1, y_2, y_3) = y_1 y_2 - y_1 y_3.
    \end{equation}
    
    Виділимо в ній повні квадрати:
    \begin{equation}
        Q(y_1, y_2, y_3) = \left( \dfrac{y_1}{2} + \dfrac{y_2}{2} - \dfrac{y_3}{2} \right)^2 - \left( \dfrac{y_1}{2} - \dfrac{y_2}{2} + \dfrac{y_3}{2} \right)^2.
    \end{equation}
    
    Як бачимо, рівняння має параболічний тип. \\
    
    Таким чином, маємо наступну пряму заміну:
    \begin{system}
        \eta_1 &= \sfrac{y_1}{2} + \sfrac{y_2}{2} - \sfrac{y_3}{2}, \\
        \eta_2 &= \sfrac{y_1}{2} - \sfrac{y_2}{2} + \sfrac{y_3}{2}, \\
        \eta_3 &= y_3.
    \end{system}

    І відповідну їй обернену заміну:
    \begin{system}
        y_1 &= \eta_1 + \eta_2, \\
        y_2 &= \eta_1 - \eta_2 + \eta_3, \\
        y_3 &= \eta_3.
    \end{system}
    
    Тобто матриця з теоретичної частини має наступний вигляд:
    \begin{equation}
        B = 
        \begin{pmatrix}
            1 & 1 & 0 \\
            1 & -1 & 1 \\
            0 & 0 & 1
        \end{pmatrix}.
    \end{equation}

    Знайдемо тепер заміну яка зводить рівняння до канонічного вигляду:
    \begin{equation}
        \begin{pmatrix}
            \xi_1 \\
            \xi_2 \\
            \xi_3
        \end{pmatrix}
        = 
        \begin{pmatrix}
            1 & 1 & 0 \\
            1 & -1 & 0 \\
            0 & 1 & 1
        \end{pmatrix}
        \cdot 
        \begin{pmatrix}
            x_1 \\
            x_2 \\
            x_3
        \end{pmatrix}.
    \end{equation}
    
    Або, що те саме,
    \begin{system}
        \xi_1 &= x_1 + x_2, \\
        \xi_2 &= x_1 - x_2, \\
        \xi_3 &= x_2 + x_3.
    \end{system}

    Оскільки рівняння має параболічний тип, то канонічною формою буде
    \begin{equation}
        u_{\xi_1 \xi_1} - u_{\xi_2 \xi_2} + 2 u_{\xi_1} = 0.
    \end{equation}
\end{solution}


% \[ u_{x x} + 2 u_{x y} - 2 u_{x z} + 2 u_{y y} + 2 u_{z z} = 0. \]
% \[ u_{x x} + 2 u_{x y} - 4 u_{x z} - 6 u_{y z} - u_{z z} = 0. \]
% \[ u_{x x} + 2 u_{x y} + 2 u_{y y} + 2 u_{y z} + 2 u_{y t} + 2 u_{z z} + 3 \[ u_{t t} = 0. \]
% \[ u_{x y} - u_{x t} + u_{z z} - 2 u_{z t} + 2 u_{t t} = 0. \]
% \[ u_{x y} + u_{x z} + u_{x t} + u_{z t} = 0. \]
% \[ u_{x x} + 2 u_{x y} - 2 u_{x z} - 4 u_{y z} + 2 u_{y t} + u_{z z} = 0. \]
% \[ u_{x x} + 2 u_{x z} - 2 u_{x t} + u_{y y} + 2 u_{y z} + 2 u_{y t} + 2 u_{z z} + 2 u_{t t} = 0. \]
% \[ u_{x_1 x_1} + 2 \Sum_{k = 2}^n u_{x_k x_k} - 2 \Sum_{k = 1}^{n - 1} u_{x_k x_{k + 1}} = 0. \]
% \[ u_{x_1 x_1} - 2 \Sum_{k = 2}^n (-1)^k u_{x_{k - 1} x_k} = 0. \]
% \[ \Sum_{k = 1}^n k u_{x_k x_k} + 2 \Sum_{l < k} l u_{x_l x_k} = 0. \]
% \[ \Sum_{k = 1}^n u_{x_k x_k} + \Sum_{l < k} l u_{x_l x_k} = 0. \]
% \[ \Sum_{l < k} l u_{x_l x_k} = 0. \]
